\documentclass[chinese]{mathexercise}
\usepackage{bbm}
\title{Functional Analysis Solutions}
\date{\today}
\author{正寅}
\textbook{泛函分析讲义}
\textbookauthor{许全华,马涛,尹智}
\textbookversion{第一版}

\newcommand{\necessary}{$(\Rightarrow)$}
\newcommand{\sufficient}{$(\Leftarrow)$}
\newcommand{\FR}{\mathbb{R}}
\newcommand{\FK}{\mathbb{K}}
\newcommand{\FQ}{\mathbb{Q}}
\newcommand{\FN}{\mathbb{N}}
\newcommand{\FZ}{\mathbb{Z}}
\newcommand{\ii}{\mathrm{i}}
\newcommand{\FC}{\mathbb{C}}
\renewcommand{\lrangle}[2]{\langle#1,#2\rangle}
\DeclareMathOperator{\cont}{Cont}
\setCJKmainfont{FandolSong}
\bibliographystyle{plain}

\begin{document}
\maketitle
\tableofcontents

% !TeX root = main.tex
% !TeX program = xelatex

\chapter{拓扑空间简介}


\textbf{定理1.3.2的证明}
\begin{proof}
($\Rightarrow$)(反证法)假设 $\bigcap_{i\in I}F_i=\emptyset$, 则
\[\biggl(\bigcap_{i\in I}F_i\biggr)^c=\bigcup_{i\in I}F_i^c=E,\]
也即 $E$ 存在开覆盖 $(F_i^c)_{i\in I}$, 由 $E$ 为紧集可知存在有限集
$J\subset I$ 使得
\[E=\bigcup_{i\in J}F_i^c.\]
故 $\bigcap_{i\in J}F_i=\emptyset$, 这与闭集族 $(F_i)_{i\in I}$
的有限交性质相矛盾, 所以假设不成立.

$(\Leftarrow)$ 同理可证.
\end{proof}


\begin{theorem}[距离越小,拓扑越小]
    设 $(E,d_1)$ 和 $(E,d_2)$ 都是度量空间,
    且 $d_1$ 诱导的拓扑为 $\tau_1$, $d_2$ 诱导的拓扑为 $\tau_2$,
    若 $d_1\leq d_2$, 则$\tau_1\subset\tau_2$; 若存在正常数 $C_1,C_2$ 使得
    $C_1d_2(x,y)\leq d_1(x,y)\leq C_2d_2(x,y),\forall x,y\in E$, 则$\tau_1=\tau_2$.
    换言之, 若两个度量等价, 则其诱导出的拓扑是相同的.
\end{theorem}

\begin{proof}
事实上只需要证明后半部分, 因为前半部分是后半部分的推论.
\begin{align*}
    U\in\tau_1
    &\Leftrightarrow\forall x\in U,\exists r>0,\text{\ 使得\ }\{y\in E\mid d_1(x,y)<r\}\subset U\\
    &\Leftrightarrow\forall x\in U,\exists r>0, \text{\ 使得\ }\left\{y\in E\middle|\frac{1}{C_2}d_1(x,y)<\frac{r}{C_2}\right\}\subset U\\
    &\Rightarrow\forall x\in U,\exists r>0, \text{\ 使得\ }\left\{y\in E\bigg|d_2(x,y)<\frac{r}{C_2}\right\}\subset U\\
    &\Leftrightarrow U\in\tau_2. 
\end{align*}
所以 $\tau_1\subset\tau_2$, 同理可证明 $\tau_2\subset\tau_1$, 故 $\tau_1=\tau_2$.
在这里, 我们要建立起一个清醒的认识, 那就是度量越大, 其所对应的相同半径的开球越小.
\end{proof}

求证: 设 $d$ 是 $E$ 上的度量, 则 $d,\min\{1,d\},rd(r>0)$ 都诱导出 $E$ 上相同的拓扑.
\begin{proof}
记三者诱导的拓扑分别为 $\tau_1,\tau_2,\tau_3$, 要证明 $\tau_1=\tau_2=\tau_3$,
由上述性质知$\tau_1=\tau_3$显然成立, 故只需要证明$\tau_1=\tau_2$.
由 $d\geq\min\{1,d\}$ 知 $\tau_2\subset\tau_1$. 又因为
\begin{align*}
    U\in\tau_1
    &\Leftrightarrow\forall x\in U,\exists r\in(0,1),\text{\ 使得\ }\{y\in E\mid d(x,y)<r\}\subset U \\
    &\Rightarrow\forall x\in U,\exists r\in(0,1),\text{\ 使得\ }\{y\in E\mid\min\{1,d(x,y)\}<r\}\subset U \\
    &\Rightarrow U\in\tau_2,
\end{align*}
所以 $\tau_1\subset\tau_2$, 从而 $\tau_1=\tau_2$.
显然 $d$ 与 $\min\{1,d\}$ 不是等价的度量, 由此可见不等价的度量也可以诱导出相同的拓扑.
\end{proof}

\begin{exercise}
证明定理~1.1.22.
\end{exercise}

\begin{proof}
(1)显然.

(2)要证 $E\setminus\mathring{A}=\overline{E\setminus A}$,
即证 $\mathring{A}=E\setminus(\overline{E\setminus A})$,而
\begin{align*}
    x\in E\setminus(\overline{E\setminus A})&\Leftrightarrow\exists U\in\mathcal{N}(x),s.t.U\cap(E\setminus A)=\emptyset\\
    &\Leftrightarrow\exists U\in\mathcal{N}(x),s.t.U\subset A\\
    &\Leftrightarrow x\in\mathring{A}.
\end{align*}
故结论得证.

(3)因为$A\subset A\cup B$, 所以 $\overline{A}\subset\overline{A\cup B}$,
同理 $\overline{B}\subset\overline{A\cup B}$,
故 $\overline{A}\cup\overline{B}\subset\overline{A\cup B}$,
又 $\overline{A}\cup\overline{B}\supset A\cup B\Rightarrow\overline{(\overline{A}\cup\overline{B})}=\overline{A}\cup\overline{B}\supset\overline{A\cup B}$,
结合双向包含关系可得 $\overline{A\cup B}=\overline{A}\cup\overline{B}$,
同理可证$\mathring{\widehat{A\cap B}}=\mathring{A}\cap\mathring{B}$.
\end{proof}

\begin{exercise}
(a)设$(E,d)$是一个度量空间, $F\subset E$. 证明$d$在$F$上诱导的拓扑和$d$在$E$上诱导的拓扑空间在$F$上的限制一致.

(b)设$E$是一个拓扑空间, $F$是$E$的拓扑子空间, $A\subset F$.
用实例说明$A$是$F$中闭集但是在$E$中不一定是闭集, 以及$A$在$F$中是开集但在$E$中不一定是开集.
\end{exercise}

\begin{proof}
(a)记$d$在$F$上诱导的拓扑为$\tau$, 在$E$上诱导的拓扑为$\tau'$, 则我们需要证明$\tau'|_F=\tau$.
\[\begin{split}
U\in\tau'|_F&\Leftrightarrow\exists V\in\tau',s.t.U=V\cap F\\
&\Leftrightarrow\forall x\in U,\exists r>0,s.t.B(x,r)\subset V\mbox{且}U=V\cap F\\
&\Leftrightarrow\forall x\in U,\exists r>0,s.t.B(x,r)\cap F\subset U\\
&\Leftrightarrow U\in\tau
\end{split}\]
故$\tau'|_F=\tau$.

(b)取$E$为二维欧式空间, $F$为一维欧氏空间, 则$F$中的开集和闭集在$E$中分别不再是开集和闭集.
\end{proof}

\begin{exercise}
设 $E$ 是 $\mathbb{R}^*=\mathbb{R}\backslash\{0\}$ 和另外两个不同的点构成的并集,
如 $E=\mathbb{R}^*\cup\{-\infty,+\infty\}$. 并设 $\tau$ 是 $E$ 中满足如下条件的子集 $U$ 构成的集族:
\begin{enumerate}[(i)]
    \item 在 $\mathbb{R}^*$ 的拓扑下, $U\cap\mathbb{R}^*$ 在 $\mathbb{R}^*$ 中是开的.
    \item 若 $-\infty\in U$ 或 $+\infty\in U$, 则 $U$ 包含一个形如 $\mathbb{R}^*\cap V$ 的集合, 其中 $V$ 是 $\mathbb{R}$ 中零点的一个邻域.
\end{enumerate}
证明:
\begin{enumerate}[(a)]
    \item $\tau$ 是 $E$ 上的拓扑.
    \item $\tau$ 不是 Hausdorff 空间.
    \item 任一点 $x\in E$ 的所有邻域的交集为 $\{x\}$.
\end{enumerate}
\end{exercise}

\begin{proof}
(a)
\begin{itemize}
\item 显然$\emptyset,E\in\tau$
\item 任意并性质: 设$(U_i)_{i\in I}\subset\tau$, 需要证明$\bigcup_{i\in I}U_i\in\tau$.
首先验证其满足条件(i): 对于每个 $i\in I$, 因为 $U_i\cap\mathbb{R}^*$ 在 $\mathbb{R}^*$ 中是开的, 
所以存在$\mathbb{R}$中的开集$U_i^*$使得$U_i\cap\mathbb{R}^*=U_i^*\cap\mathbb{R}^*$, 故
\[\left(\bigcup_{i\in I}U_i\right)\cap\mathbb{R}^*=\bigcup_{i\in I}\left(U_i\cap\mathbb{R}^*\right)=\left(\bigcup_{i\in I}U_i^*\right)\cap\mathbb{R}^*\text{\ 在\ }\FR^*\text{\ 中是开的}.\]

再验证其满足条件(ii): 不妨设$-\infty\in\bigcup_{i\in I}U_i$, 则存在某个$i_0\in I,s.t.-\infty\in U_{i_0}$,
则 $U_{i_0}$ 包含一个形如 $\FR^*\cap V$的集合,$V$是$\FR$中零点的一个邻域,
从而 $\bigcup_{i\in I}U_i$ 必然包含此 $\FR^*\cap V$.
\item 有限交性质: 验证方法与上面方法类似.
\end{itemize}

(b)考虑 $+\infty$ 和 $-\infty$ 这两个特殊的点, 由条件 (ii) 可知 $+\infty$ 与 $-\infty$
不存在不相交的开邻域, 故 $\tau$ 不是 Hausdorff 空间.

(c) 若 $x\in\FR^*$, 则由于 $\FR^*$ 是 Housdorff 空间, 故 $x$ 的所有邻域的交集为 $\{x\}$;
若 $x\in E\setminus\FR^*$, 如 $x=-\infty$, 取 $-\infty$ 的一列邻域
$\left(-\infty\cap((-\frac{1}{n},\frac{1}{n})\cap\FR^*)\right)_{n\geq 1}$. 显然
\[\bigcap_{n=1}^{\infty}\left(-\infty\cap((-\frac{1}{n},\frac{1}{n})\cap\FR^*)\right)=-\infty.\]
因此 $-\infty$ 的所有邻域的交集必为单点集 $\{-\infty\}$.
\end{proof}

\begin{exercise}
证明: 紧空间中的任一序列均有凝聚点.
\end{exercise}

\begin{proof}
设 $E$ 是紧空间, $(x_n)_{n\geq 1}\subset E$ 为任一序列.

当 $(x_n)_{n\geq 1}$ 为有限集时, $(x_n)_{n\geq 1}$ 从某一项开始必为常值,
此常值即为 $(x_n)_{n\geq 1}$ 的凝聚点.

当 $(x_n)_{n\geq 1}$ 为无限集时, 假设 $(x_n)_{n\geq 1}$ 没有凝聚点, 则
对任意 $x\in E$, 存在 $V_x\in\mathcal{N}(x)$ 使得
\[V_x\cap (x_n)_{n\geq 1}\setminus\{x\}=\varnothing.\]
因为 $E$ 为紧空间, 所以 $E$ 的开覆盖 $\bigcup_{x\in E}V_x$ 存在有限子覆盖 $\bigcup_{i=1}^n V_{x_i}=E$.
然而 $\bigcup_{i=1}^n V_{x_i}$ 至多包含 $(x_n)_{n\geq 1}$ 中的有限个点, 这与 $(x_n)_{n\geq 1}$ 为无限集相矛盾.
\end{proof}

\begin{exercise}
证明: 有限维的赋范空间是局部紧的.
\end{exercise}

\begin{proof}
设$E$为有限维赋范空间, 教材注 3.1.12 (2) 表明在有限维空间中有界闭集为紧集.
任取 $x\in E$, $\overline{B(x,1)}$ 即为 $x$ 的紧邻域, 故有限维的赋范空间是局部紧的.
\end{proof}

\begin{exercise}
设 $(E,\tau)$ 是一个局部紧的但不是紧的 Hausdorff 空间. 我们在 $E$ 上增加一个点,
记作 $\infty$, 然后定义 $\widehat{E}=E\cup\{\infty\}$. 在 $\widehat{E}$ 上定义集族 $\widehat{\tau}$,
$U\in\widehat{\tau}$ 当且仅当 $U\in\tau$ 或者存在 $E$ 中的紧集 $K$, 使得 $U=\widehat{E}\setminus K$. 证明:
\begin{enumerate}[(a)]
    \item $\widehat{\tau}$ 是 $\widehat{E}$ 上的拓扑.
    \item $\widehat{\tau}$ 在 $E$ 上的限制等于 $\tau$, 即 $(E,\tau)$ 是 $(\widehat{E},\widehat{\tau})$ 的拓扑子空间.
    \item $(\widehat{E},\widehat{\tau})$ 是一个紧 Hausdorff 空间.
    \item $E$ 在 $\widehat{E}$ 中稠密.
\end{enumerate}
\end{exercise}

\begin{remark}
    拓扑空间 $(\widehat{E},\widehat{\tau})$ 通常被称为 $(E,\tau)$ 的 Al\-e\-x\-a\-ndorff 紧化空间 
    (Alexandorff compactification or one-point compactification).
\end{remark}

\begin{proof}
(a) 显然$\emptyset,\widehat{E}\in\widehat{\tau}$.

下面验证任意并性质:设$(U_i)_{i\in I}\subset\widehat{\tau}$,
要证明$\bigcup_{i\in I}U_i\in\widehat{\tau}$, 分三种情况讨论:

(1) 当 $(U_i)_{i\in I}\subset\tau$时, $\bigcup_{i\in I}U_i\in\tau\Rightarrow\bigcup_{i\in I}U_i\in\widehat{\tau}$.
(2) 当对任意 $i\in I$, $U_i=\widehat{E}\setminus K_i$ 时, 其中 $K_i$ 为 $E$ 中紧集.
      注意到 $\bigcap_{i\in I}K_i$ 仍为 $E$ 中紧集且
      \[\bigcup_{i\in I}U_i=\bigcup_{i\in I}\widehat{E}\setminus K_i=\widehat{E}\setminus\Bigl(\bigcap_{i\in I}K_i\Bigr),\]
      因此 $\bigcup_{i\in I}U_i\in\tau$.
(3) 存在非空真子集 $J\subset I$ 使得当 $i\in J$ 时, $U_i=\widehat{E}\setminus K_i$, 
$K_i$ 为 $E$ 中紧集; 当 $i\in I\setminus J$ 时, $U_i\in\tau$, 则
\begin{align*}
\bigcup_{i\in I}U_i
&=\bigg(\bigcup_{i\in J}\widehat{E}\setminus K_i\bigg)\bigcup\bigg(\bigcup_{i\in I\setminus J}U_i\bigg)=\biggl(\widehat{E}\setminus\bigcap_{i\in J}K_i\biggr)\bigcup\biggl(\widehat{E}\setminus\bigcap_{i\in I\setminus J}U_i^c\biggr) \\
&=\widehat{E}\setminus\biggl(\biggl(\bigcap_{i\in J}K_i\biggr)\bigcap\biggl(\bigcap_{i\in I\setminus J}U_i^c\biggr)\biggr). 
\end{align*}
而 $\left(\bigcap_{i\in J}K_i\right)\bigcap\bigl(\bigcap_{i\in I\setminus J}U_i^c\bigr)$ 是 $E$  中紧集,
故 $\bigcup_{i\in I}U_i\in\widehat{\tau}$.

再验证有限交性质: 只需要考虑两个开集 $V_1,V_2\in\widehat{\tau}$ 即可. 分三种情形:

(1) 若 $V_1,V_2\in\tau$, 则 $V_1\cap V_2\in\tau\Rightarrow V_1\cap V_2\in\widehat{\tau}$.
(2) 若 $V_1=\widehat{E}\setminus K_1$, $V_2=\widehat{E}\setminus K_2$, 其中 $K_1$ 和 $K_2$
是 $E$ 中的紧集, 则 $V_1\cap V_2=(\widehat{E}\setminus K_1)\cap(\widehat{E}\setminus K_1)=\widehat{E}\setminus (K_1\cup K_2)$.
由于 $K_1\cup K_2$ 为 $E$ 中紧集, 故 $V_1\cap V_2\in\widehat{\tau}$.
(3) 若 $V_1\in\tau$ 且 $V_2=\widehat{E}\setminus K_2$, 其中 $K_2$ 是 $E$ 中紧集, 则
$V_1\cap V_2=V_1\cap(\widehat{E}\setminus K_2)=V_1\cap(E\setminus K_2)\in\tau$.
由此可知有限交性质成立.

(b) 往证 $\widehat{\tau}|_E=\tau$.

对于任意 $U\in\tau$, 有 $U\in\widehat{\tau}$, 故 $U=U\cap E\in\widehat{\tau}|_E$, 从而 $\tau\subset\widehat{\tau}|_E$.

对于任意 $U\in\widehat{\tau}|_E$, 存在 $\widehat{U}\in\widehat{\tau}$ 使得 $U=\widehat{\tau}\cap E$.
当 $\widehat{U}\in\tau$ 时, $U=\widehat{U}\cap E=\widehat{U}\in\tau$;
当 $\widehat{U}=\widehat{E}\setminus K$ 时, $U=(\widehat{E}\setminus K)\cap E=E\setminus K\in\tau$,
从而 $\widehat{\tau}|_E\subset\tau$.

因此 $\widehat{\tau}|_E=\tau$.

(c) 首先证明 $(\widehat{E},\widehat{\tau})$ 是紧空间. 设 $\{U_i\mid i\in I\}$
为 $\widehat{E}$ 的开覆盖, 则存在 $i_0\in I$ 使得 $\infty\in U_{i_0}$ 且 $U_{i_0}=\widehat{E}\setminus K_{i_0}$,
其中 $K_{i_0}$ 为 $E$ 中紧集. 由于 $\{U_i\cap E\mid i\in I,i\neq i_0\}$ 为 $K_{i_0}$
的开覆盖, 故存在 $K_{i_0}$ 的有限子覆盖 $\{U_i\cap E\}_{i=1}^n$, 那么
$\{U_{i_0},U_1,\dots,U_n\}$ 即为 $\widehat{E}$ 的有限子覆盖.

然后证明 $(\widehat{E},\widehat{\tau})$ 为 Hausdorff 空间. 事实上,
只需要证明 $\forall x\in E$ 与 $\infty$ 存在不相交的开邻域即可.
由于 $E$ 局部紧, 所以存在开集 $V$ 使得 $x\in V\subset E$ 且 $\overline{V}$ 为紧集.
因此, 取 $x$ 的开邻域 $V$ 和 $\infty$ 的开邻域 $\widehat{E}\setminus\overline{V}$
即可.
\end{proof}

\begin{exercise}
设$E$是一个局部紧的Hausdorff空间, 则$E$中每一点都有一个紧邻域基.
\end{exercise}

\begin{proof}
对于 $E$ 中任意一点 $x$, 由 $E$ 局部紧可知 $x$ 点处存在紧邻域 $W$.
对于点 $x$ 的任意开邻域 $G$, 我们的目标是寻找紧集 $\overline{V}$
使得 $x\in\overline{V}\subset G$.

若 $W\subset G$, 取 $\overline{V}=W$ 即可;

若 $W\not\subset G$, 记 $A=W\cap G^c$, 显然 $A$ 是非空紧集. $\forall y\in A$,
由 Hausdorff 条件可知 $y$ 与 $x$ 存在不相交的开邻域 $U_y$ 和 $W_y$ 使得 $W_y\subset W$.
因为 $A$ 为紧集, 所以存在
$y_1,y_2,\cdots,y_k\in A$, 使得 $A\subset\bigcup_{i=1}^k U_{y_i}$, 令
\[U=\bigcup_{i=1}^kU_{y_i},\quad V=\bigcap_{i=1}^kW_{y_i},\]
则 $U$ 和 $V$ 都是开集并且 $U\cap V=\varnothing$, 这意味着 $\overline{V}\cap U=\varnothing$, 从而有
\[\overline{V}\cap G^c=\overline{V}\cap\left(W\cap G^c\right)=\overline{V}\cap A\subset\overline{V}\cap U=\varnothing,\]
故 $\overline{V}\subset G$, 此时的 $\overline{V}$ 就是我们要寻找的紧集.
\end{proof}

\begin{exercise}
    证明注 1.4.2 中的命题 (1), (3) 和 (4).
\end{exercise}

\begin{proof}[命题 (1) 证明]
记所有基础开集的并集构成的集合为$\tau$, 下面证明$\tau$是$E$上的拓扑.
\begin{enumerate}[(i)]
\item 显然$\varnothing,E\in\tau$
\item 任意并性质:设$(V_{\alpha})_{\alpha\in\Lambda}\subset\tau$,则每个$V_{\alpha}$可以表为:
\[V_{\alpha}=\bigcup_{\beta\in\Lambda_{\alpha}}O_{\beta}=\bigcup_{\beta\in\Lambda_{\alpha}}\left(\prod_{i\in J_{\beta}}U_i\times\prod_{i\in I\backslash J_{\beta}}E_i\right)(J_{\beta}\text{\ 有限})\]
故
\[\bigcup_{\alpha\in\Lambda}V_{\alpha}=\bigcup_{\alpha\in\Lambda}\bigcup_{\beta\in\Lambda_{\alpha}}\left(\prod_{i\in J_{\beta}}U_i\times\prod_{i\in I\backslash J_{\beta}}E_i\right)\]
上式仍为基础开集的并,故$\bigcup_{\alpha\in\Lambda}V_{\alpha}\in\tau$
\item 有限交性质:设$\left(V_{\alpha}\right)_{\alpha\in\Lambda}\subset\tau$,其中$\Lambda=\{\alpha_1,\cdots,\alpha_n\}$为有限指标集,则
\[V_{\alpha}=\bigcup_{\beta\in\Lambda_{\alpha}}O_{\beta}\]
故
\begin{align*}
    \bigcap_{\alpha\in\Lambda}V_{\alpha}
    &=\bigcap_{\alpha\in\Lambda}\bigcup_{\beta\in\Lambda_{\alpha}}O_{\beta}\\
    &=\left(\bigcup_{\beta_1\in\Lambda_{\alpha_1}}O_{\beta_1}\right)\bigcap\left(\bigcup_{\beta_2\in\Lambda_{\alpha_2}}O_{\beta_2}\right)\bigcap\cdots\bigcap\left(\bigcup_{\beta_n\in\Lambda_{\alpha_n}}O_{\beta_n}\right)\\
    &=\bigcup_{\beta_1\in\Lambda_{\alpha_1}}\bigcup_{\beta_2\in\Lambda_{\alpha_2}}\cdots\bigcup_{\beta_n\in\Lambda_{\alpha_n}}\left(O_{\beta_1}\bigcap O_{\beta_2}\bigcap\cdots\bigcap O_{\beta_n}\right)
\end{align*}
注意到有限个基础开集的交仍为基础开集, 故上式为基础开集的并, 因此$\bigcap_{\alpha\in\Lambda}V_{\alpha}\in\tau$.
\end{enumerate}
\end{proof}

\begin{proof}[命题 (3) 证明]
记 $\FR^n$ 上的自然拓扑为 $\tau_1$, $\FR\times\cdots\times\FR$ 上的乘积拓扑为 $\tau_2$, 
为叙述方便, 记 $E:=\FR^n=\FR\times\cdots\times\FR$.

首先证明 $\tau_1\subset\tau_2$. 只需证明 $(E,\tau_1)$ 中的任意开球为 $(E,\tau_2)$ 中的开集即可.
任取 $x\in E$ 和开球 $B(x,r)$, 选取 $(E,\tau_2)$ 中含 $x$ 的开集 $O=\prod_{i=1}^nB_i(x_i,\frac{r}{\sqrt{n}})$,
则对于任意 $y\in O$, 有 $(x_i-y_i)^2<\frac{r^2}{n}$, 从而
\[\biggl(\sum_{i=1}^n(x_i-y_i)^2<r\biggr)^{1/2}\Rightarrow y\in B(x,r),\]
故 $O\subset B(x,r)$, 因此 $\tau_1\subset\tau_2$.

然后证明 $\tau_2\subset\tau_1$. 只需证明 $(E,\tau_2)$ 中任意基础开集为 $(E,\tau_1)$
中的开集即可. 任取 $x\in E$ 和 $(E,\tau_2)$ 中的基础开集 $O=\prod_{i=1}^nB_i(x_i,r_i)$.
令 $r=\min\{r_1,\dots,r_n\}$, 取 $(E,\tau_1)$ 中的开球 $B(x,r)$, 则对于任意 $y\in B(x,r)$,
有 $\sum_{i=1}^n(x_i-y_i)^2<r^2$, 从而对任意 $1\leq i\leq n$ 有
\[(x_i-y_i)^2\leq\sum_{i=1}^n(x_i-y_i)^2<r^2\leq r_i^2,\]
所以 $y_i\in B_i(x_i,r_i)\Rightarrow y\in O$, 故 $B(x,r)\subset O$, 因此 $\tau_2\subset\tau_1$.
\end{proof}

\begin{proof}[命题 (4) 证明]
因为
\[\left(\prod_{i\in I}F_i\right)^c=\bigcup_{i_0\in I}\left(F_{i_0}^c\times\bigcup_{i\in I\backslash\{i_0\}}E_i\right)\text{为开集},\]
所以 $\prod_{i\in I}F_i$ 为 $E$ 中闭集.
\end{proof}


\begin{exercise}
    把定理 1.4.9 中的距离换成下面的距离
    \[\delta(x,y)=\sum_{n=1}^{\infty}\frac{1}{2^n}d_n(x_n,y_n).\]
    证明由 $\delta$ 诱导的拓扑也与乘积拓扑相同.
\end{exercise}


\begin{exercise}
    证明一列紧度量空间的乘积空间(赋予乘积拓扑)是紧的可度量化空间.
\end{exercise}


\begin{exercise}
    设 $E=\{x=(x_n)_{n=1}^{\infty}\mid \forall n\geq 1, x_n=0\text{\ 或\ }1\}=\{0,1\}^{\FN*}$.
    对每个 $x=(x_n)_{n\geq 1}\in E$, 令
    \[\phi(x)=\sum_{n=1}^{\infty}\frac{2x_n}{3^n}.\]
    在 $\{0,1\}$ 上赋予离散拓扑(这实际上对应着自然的距离 $d(0,1)=1$), 则在 $E$
    上有相应的乘积拓扑. 证明 $\phi$ 是 $E$ 到 $\FR$ 的紧子集 $C=\phi(E)$ 上的同胚.
\end{exercise}
\chapter{完备度量空间}
\thispagestyle{empty}



\begin{exercise}
    完备性不是一个拓扑概念, 我们用两个例子说明这一点.

    (a) 设有函数 $\phi(x)=\frac{x}{1+|x|}$, $x\in\FR$, 并定义
    \[d(x,y)=|\phi(x)-\phi(y)|,\quad x,y\in\FR.\]
    证明由此定义的 $d$ 是 $\FR$ 上的距离并和 $\FR$ 上通常意义下的拓扑一致, 但 $d$ 不完备.

    (b) 更一般地, 设 $O$ 是完备度量空间 $(E,d)$ 上的开子集, 且 $O\neq E$.
    映射 $\phi:O\to E\times\FR$ 定义为
    \[\phi(x)=\left(x,\frac{1}{d(x,O^c)}\right):=(x,\rho(x)),\quad\forall x\in O.\]
    证明 $\phi$ 是从 $O$ 到 $E\times\FR$ 的一个闭子集上的同胚. 并由此导出 $O$ 上存在一个完备的距离,
    由其所诱导的拓扑和 $d$ 在 $O$ 上所诱导的拓扑一致 (注意, $(O,d_O)$ 一般并不完备).
\end{exercise}

\begin{proof}
    (a)易知$\phi (x)$是严格单调递增函数且$-1<\phi(x)<1,|\phi '(x)|\leq 1$
    \begin{itemize}
    \item $d(x,y)\geq 0$且$d(x,y)=0$ 当且仅当 $x=y$
    \item $d(x,y)=|\phi(x)-\phi(y)|=|\phi(y)-\phi(x)|=d(y,x)$
    \item $d(x,y)=|\phi(x)-\phi(y)|\leq |\phi(x)-\phi(z)|+|\phi(y)-\phi(z)|=d(x,z)+d(y,z)$
    \end{itemize}
    因此, $d$ 是一个距离.

    下证两拓扑一致(距离越小,拓扑越小,下面第一个包含关系的推导是自然的), 
    记 $\tau$ 为自然拓扑, $\tau_d$ 为由 $d$ 诱导的拓扑. 一方面,
    \begin{align*}
        U\in\tau_d
        & \Leftrightarrow\forall x\in U,\exists r>0,s.t.\{y\mid|\phi(x)-\phi(y)|<r\}\subset U\\
        & \Rightarrow\forall x\in U,\exists r>0,s.t.\{y\mid|x-y|<r\}\subset U\\
        & \Rightarrow U\in\tau.
    \end{align*}
    另一方面,
    \begin{align*}
    U\in\tau
    & \Leftrightarrow\forall x\in U,\exists r>0,s.t.\{y\mid |y-x|<r\}\subset U\\
    & \Rightarrow\forall x\in U,\text{取\ }s=\min\{\phi(x)-\phi(x-r),\phi(x+r)-\phi(x)\},\text{则\ }\{y\mid |\phi(y)-\phi(x)|<s\}\subset U\\
    & \Rightarrow U\in\tau_d.
    \end{align*}
    综合两个方向知 $\tau=\tau_d$.

    最后证明 $d$ 不完备. 取集列 $\{A_n\}_{n=1}^{\infty}(A_n=[n,+\infty))$, 则
    \[\diam A_n=\sup_{x,y\in A_n}|\phi(x)-\phi(y)|=1-\frac{n}{n+1}=\frac{1}{n+1}\to 0\quad(n\to +\infty)\]
    但 $\bigcap\limits_{n\geq 1}A_n=\emptyset$, 由完备性的等价推论知 $d$ 不完备.

    (b) 由于
    \begin{itemize}
    \item $\phi$ 是连续的一一对应. $\phi_1=\id_O:x\mapsto x$ 是连续的, 且$\phi_2=\rho(x):x\mapsto\frac{1}{d(x,O^C)}$是连续的,
          故 $\phi$ 连续, 由 $\phi_1$ 是一一对应知 $\phi$ 是一一对应.
    \item $\phi(O)$是闭集. 只需证明$\phi(O)$完备,
          任取 $\phi(O)$ 中的 Cauchy 序列 $\{(x_n,\rho_n)\}$,
          设其在 $E\times\mathbb{R}$ 中收敛到 $(x,\rho)$,
          由Cauchy序列的有界性知存在 $M>0$, 使得 $0\leq\rho_n<M$, 且 $x\in\closure{O}$, 则 $x\in O$ 或者 $x\in\partial O$.
          若 $x\in\partial O$, 则 $d(x,O^C)=0$, 故存在 $n_o$, 使得 $d(x_{n_0},O^C)<\frac{1}{M}\Rightarrow\rho_{n_0}>M$,矛盾.
          所以 $x\in O$, 因此 $(x,\rho)\in\phi(O)$, 故 $\phi(O)$ 完备.
    \item $\phi^{-1}$连续. 由$(x_n,\rho_n)\rightarrow (x,\rho)$显然得到$x_n\rightarrow x$,故$\phi^{-1}$连续
    \end{itemize}
    综上得知, $\phi$是从$O$到$E\times R$上的闭子集的同胚.

    记$E\times\mathbb{R}$上的度量为$\delta$,定义$d^*$为$d^*(x_1,x_2)=\delta(\phi(x_1),\phi(x_2))$,
    容易验证$d^*$是$O$上的一个完备的距离,记$d^*$诱导的拓扑为$\tau^*$,$d$诱导的拓扑为$\tau$,则:
    由$d(x,y)\leq d^*(x,y)=\max\{d(x,y),|\rho(x)-\rho(y)|\}$知$\tau\subset\tau^*$,又:
    另一个方向待完善.
\end{proof}

% \textcolor{blue}{注:实际上有如下结果:在向量空间$E$上赋予两个距离$d_1,d_2$,若存在两个正常数$C_1,C_2$使得\[C_1d_1(x,y)\leq d_2(x,y)\leq C_2d_1(x,y)(\forall x,y\in E)\]则两距离诱导的拓扑$\tau_{d_1},\tau_{d_2}$相同}

\begin{exercise}
    证明度量空间 $(E,d)$ 是完备的充分必要条件是: 
    对 $E$ 中任意序列 $(x_n)$, 若对任一个 $n\geq 1$ 有 $d(x_n,x_{n+1})\leq 2^{-n}$, 则序列 $(x_n)$ 收敛.
\end{exercise}

\begin{proof}
    \necessary
    $\forall\varepsilon>0$, 取 $N=[1-\log_2\varepsilon]$, 对于任意 $m,n>N$ 有
    \[d(x_m,x_n)\leq \frac{1}{2^n}+\frac{1}{2^{n+1}}+\cdots+\frac{1}{2^{m-1}}=\frac{1}{2^{n-1}}-\frac{1}{2^{m-1}}<\varepsilon,\]
    所以 $(x_n)$ 是 Cauchy 序列, 因 $(E,d)$ 完备, 故 $(x_n)$ 在 $E$ 中收敛.
    
    \sufficient 任取 $(E,d)$ 中的 Cauchy 序列 $(x_n)_{n\geq 1}$.
    对于 $\varepsilon_1=\frac{1}{2}$, 存在 $N_1$, 使得对于 $\forall m,n\geq N_1$
    有 $d(x_m,x_n)<\frac{1}{2}$;
    对于 $\varepsilon_2=\frac{1}{2^2}$, 存在 $N_2>N_1$, 使得对于 $\forall m,n\geq N_2$
    有 $d(x_m,x_n)<\frac{1}{2^2}$;
    依次进行下去可得 $(x_n)_{n\geq 1}$ 的子列 $(x_{N_k})_{k\geq 1}$
    且此子列满足对于任意 $k\geq 1$ 有 $d(x_{N_k},x_{N_{k+1}})<2^{-k}$.
    由假设条件知 $(x_{N_k})_{k\geq 1}$ 收敛, 因此 $(x_n)_{n\geq 1}$ 也收敛,
    由此证明 $(E,d)$ 完备.
\end{proof}


\begin{exercise}
    设 $(E,d)$ 是度量空间, $(x_n)$ 是 $E$ 中的  Cauchy 序列, 并有 $A\subset E$.
    假设 $A$ 的闭包 $\overline{A}$ 在 $E$ 中完备并且 $\lim_{n\to\infty}d(x_n,A)=0$.
    证明 $(x_n)$ 在 $E$ 中收敛.
\end{exercise}

\begin{proof}
    先证明 $\lim_{n\to\infty}d(x_n,\overline{A})=0$. 任取 $x\in A,y\in\overline{A}$, 有
    \begin{equation*}
        d(x_n,y)\leq d(x_n,x)+d(x,y),
    \end{equation*}
    上述不等式关于 $x\in A$ 取下确界得
    \[d(x_n,y)\leq d(x_n,A)+\inf_{x\in A}d(x,y)=d(x_n,A),\]
    上述不等式再关于 $y\in\overline{A}$ 取下确界得
    \[d(x_n,\overline{A})\leq d(x_n,A).\]
    令 $n\to\infty$ 即得 $\lim_{n\to\infty}d(x_n,\bar{A})=0$.

    令 $(y_n)_{n\geq 1}$ 为 $\overline{A}$ 中满足 $d(x_n,y_n)=d(x_n,\overline{A})$ 的序列, 
    由 $\lim\limits_{n\to\infty}d(x_n,y_n)=0$ 及 $(x_n)_{n\geq 1}$ 是 Cauchy 序列有
    \[\forall\varepsilon>0,\exists N>0,\forall m,n>N,d(x_n,y_n)<\varepsilon/3,d(x_n,x_m)<\varepsilon/3.\]
    故
    \[d(y_n,y_m)\leq d(y_n,x_n)+d(x_n,x_m)+d(x_m,y_m)<\varepsilon.\]
    从而 $(y_n)_{n\geq 1}$ 是 Cauchy 序列, 由 $\overline{A}$ 的完备性知 $(y_n)_{n\geq 1}$ 收敛, 记为 $y_n\to y$, 故
    \[\forall\varepsilon>0,\exists M>0,\forall n>M,d(y_n,y)<\varepsilon/2,d(x_n,y_n)<\varepsilon/2.\]
    因此 $d(x_n,y)\leq d(x_n,y_n)+d(y_n,y)<\varepsilon$, 从而说明 $x_n\to y$.
\end{proof}


\begin{exercise}
    设 $(E,d)$ 是度量空间, $\alpha>0$. 假设 $A\subset E$ 满足对任意 $x,y\in A$
    且 $x\neq y$, 必有 $d(x,y)\geq\alpha$. 证明 $A$ 是完备的.
\end{exercise}

\begin{proof}
    任取 $A$ 中的 Cauchy 序列 $(x_n)_{n\geq 1}$, 由定义知对于题给常数 $\alpha$, 
    $\exists N>0$, 使得对于 $\forall m,n>N$, 有 $d(x_m,x_n)<\alpha$, 结合条件知
    $\forall m,n>N,x_m=x_n$, 因此序列$(x_n)_{n\geq 1}$收敛, 故 $A$ 完备.
\end{proof}


\begin{exercise}
    设 $(E,d)$ 是度量空间且 $A\subset E$. 假设 $A$ 中任一 Cauchy 序列在 $E$ 中收敛,
    证明 $A$ 的闭包 $\overline{A}$ 是完备的.
\end{exercise}

\begin{proof}
    任取 $\closure{A}$ 中的 Cauchy 序列 $(x_n)_{n\geq 1}$.
    对于 $\forall\varepsilon>0$, 存在序列 $(y_n)_{n\geq 1}\subset A$ 使得对于 $\forall n\geq 1$ 有
    \[d(x_n,y_n)<\frac{\varepsilon}{3}.\]
    因为 $(x_n)_{n\geq 1}$是 Cauchy 序列, 所以对于上述 $\varepsilon>0$,
    存在 $N\geq 1$, 使得对于 $\forall m,n\geq N$ 有
    \[d(x_m,x_n)<\frac{\varepsilon}{3},\]
    于是
    \[d(y_m,y_n)\leq d(y_m,x_m)+d(x_m,x_n)+d(x_n,y_n)<\varepsilon,\]
    所以 $(y_n)_{n\geq 1}$ 是 $A$ 中的 Cauchy 序列, 由题目条件知 $y_n\to y\in\closure{A}$,
    于是对于上述 $\varepsilon>0$, 存在 $M\geq 1$, 使得当 $n\geq M$ 时 $d(y_n,y)<\frac{2}{3}\varepsilon$,
    从而 $d(x_n,y)<d(x_n,y_n)+d(y_n,y)<\frac{\varepsilon}{3}+\frac{2}{3}\varepsilon=\varepsilon$.
    所以$x_n\to y\in\closure{A}$, 由完备性定义知 $\closure{A}$ 完备.
\end{proof}


\begin{exercise}
    设 $(E,d)$ 是度量空间, 而 $(x_n)$ 是 $E$ 中发散的 Cauchy 序列. 证明
    \begin{enumerate}[(a)]
        \item 任取 $x\in E$, 序列 $(d(x,x_n))$ 收敛于一个正数, 记为 $g(x)$.
        \item 函数 $x\mapsto\frac{1}{g(x)}$ 是从 $E$ 到 $\FR$ 的连续函数.
        \item 上面的函数无界.
    \end{enumerate}
\end{exercise}

\begin{proof}
    (a) 由 $(x_n)_{n\geq 1}$ 是 Cauchy 序列和三角不等式得
    \[|d(x,x_m)-d(x,x_n)|\leq d(x_m,x_n)\to 0,\quad m,n\to\infty,\]
    故序列 $(d(x,x_n))_{n\geq 1}$ 是 $\FR$ 中的 Cauchy 序列, 
    由 $\FR$ 的完备性知 $(d(x,x_n))_{n\geq 1}$ 收敛, 记收敛值为 $g(x)$.

    显然 $g(x)\geq 0$, 若 $g(x)=0$, 则 $\lim_{n\to\infty}d(x,x_n)=0$, 
    故 $x_n\to x$, 与 $(x_n)_{n\geq 1}$ 发散相矛盾, 因此 $g(x)>0$.

    (b)只需证明 $g(x)$ 连续即可. 任意取定 $x_0\in E$, 则
    \begin{align*}
        |g(x)-g(x_0)| & =|\lim_{n\to\infty}d(x,x_n)-\lim_{n\to\infty}d(x_0,x_n)| \\
                      & =\bigl|\lim_{n\to\infty}\bigl(d(x,x_n)-d(x_0,x_n)\bigr)\bigr| \\
                      & \leq\lim_{n\to\infty}d(x,x_0) \\
                      & =d(x,x_0),
    \end{align*}
    上述不等式表明 $g(x)$ 为连续函数.

    (c) 假设 $\frac{1}{g(x)}$ 有界, 即存在 $M>0$,
    使得 $\frac{1}{g(x)}<M\Rightarrow g(x)>\frac{1}{M}(\forall x\in E)$.
    因为 $(x_n)_{n\geq 1}$ 是 Cauchy 序列, 
    所以存在 $N\geq 1$, 当 $\forall n>N$ 时, $d(x_n,x_N)<\frac{1}{M}$,
    故 $g(x_N)=\lim_{n\to\infty}d(x_n,x_N)\leq\frac{1}{M}$, 矛盾, 因此 $\frac{1}{g(x)}$ 无界.
\end{proof}


\begin{exercise}
    设 $(E,d)$ 和 $(F,\delta)$ 都是度量空间, $f:(E,d)\to (F,\delta)$
    是一致连续的双射并且逆映射 $f^{-1}$ 也是一致连续的.
    证明对任意 $A\subset E$, $f(A)$ 完备当且仅当 $A$ 完备.
\end{exercise}

\begin{proof}
    \sufficient 假设 $A$ 完备, 要证明$f(A)$完备.
    任取 $f(A)$ 中的 Cauchy 序列 $(y_n)_{n\geq 1}$, 记 $f^{-1}(y_n)=x_n $, 
    从而得到 $A$ 中的序列 $(x_n)_{n\geq 1}$, 由 $f^{-1}$ 一致连续知
    对于 $\forall\varepsilon>0$, 存在 $\theta>0$, 使得当 $\delta(y_m,y_n)<\theta$ 时,
    有 $d(x_m,x_n)<\varepsilon$.
    对于上述的 $\theta>0$, 存在 $N\geq 1$, 当 $m,n>N$ 时,
    $\delta(y_m,y_n)<\theta$, 此时 $d(x_m,x_n)<\varepsilon$.
    从而 $(x_n)_{n\geq 1}$ 是 $A$ 中的 Cauchy 序列, 
    由 $A$ 完备知 $(x_n)_{n\geq 1}$ 收敛, 记 $x_n\to x\in A$,
    故 $y_n=f(x_n)\to f(x)\in f(A)$, 因此$f(A)$是完备的.

    \necessary 由 $f$ 的一致连续性可证, 证法同充分性.
\end{proof}


\begin{exercise}
    设 $f:\FR^n\to\FR$ 是一致连续函数. 证明存在两个非负常数 $a$ 和 $b$, 使得
    \[|f(x)|\leq a\|x\|+b,\]
    这里 $\|x\|$ 是 $x$ 的欧氏范数.
\end{exercise}

\begin{proof}
    为强调自变量为 $\FR^n$ 中向量, 下面记 $x\in\FR^n$ 为 $\vec{x}$.


    因为 $f(\vec{x})$ 一致连续, 所以对于任意 $\varepsilon>0$, 存在 $\delta>0$,
    使得当 $\|\vec{x}-\vec{y}\|<\delta$ 时, 有 $|f(\vec{x})-f(\vec{y})|<\varepsilon$.

    固定 $\varepsilon$ 和 $\delta$, 取定某 $0<\delta'<\delta$.
    则对于任意 $\vec{x}\in\FR^n$, 可将其表为
    \[\vec{x}=\delta'\frac{\vec{x}}{\|\vec{x}\|}\cdot N+\vec{x}_0,\quad\|\vec{x}_0\|<\delta',\]
    其中 $N=\frac{\|\vec{x}-\vec{x}_0\|}{\delta'}$. 可以将 $f(\vec{x})$ 进行如下和式分解:
    \[f(\vec{x})=\sum_{k=1}^N\left[f\biggl(\delta'\frac{\vec{x}}{\|\vec{x}\|}k+\vec{x}_0\biggr)-f\biggl(\delta'\frac{\vec{x}}{\|\vec{x}\|}(k-1)+\vec{x}_0\biggr)\right]+f(\vec{x}_0),\]
    并且注意到 $\|\vec{x}_0\|=\|\vec{x}_0-\vec{0}\|<\delta'<\delta$, 所以 $|f(\vec{x}_0)-f(\vec{0})|<\varepsilon$, 
    即 $f(\vec{0})-\varepsilon<f(\vec{x}_0)<f(\vec{0})+\varepsilon$, 记 $M=\max\{|f(\vec{0})-\varepsilon|,|f(\vec{0})+\varepsilon|\}$.
    从而
    \begin{align*}
        |f(\vec{x})|
        &\leq\sum_{k=1}^N \left\lvert f\biggl(\delta'\frac{\vec{x}}{\|\vec{x}\|}k+\vec{x}_0\biggr)-f\biggl(\delta'\frac{\vec{x}}{\|\vec{x}\|}(k-1)+\vec{x}_0\biggr)\right\rvert+|f(\vec{x}_0)| \\
        &\leq N\cdot\varepsilon+M \\
        &=\frac{\|\vec{x}-\vec{x}_0\|}{\delta'}\cdot\varepsilon+M \\
        &\leq\frac{\|\vec{x}\|+\|\vec{x}_0\|}{\delta'}\cdot\varepsilon+M \\
        &<\frac{\varepsilon}{\delta'}\|\vec{x}\|+(M+\varepsilon).
    \end{align*}
    记 $a=\frac{\varepsilon}{\delta'}$ 且 $b=M+\varepsilon$, 则上述不等式表明
    \[|f(\vec{x})|\leq a\|\vec{x}\|+b.\qedhere\]
\end{proof}


\begin{exercise}
    设 $f:E\to F$ 是两个度量空间之间的连续映射, 并设 $f$ 在 $E$ 的每个有界子集上一致连续.

    (a) 证明若 $(x_n)_{n\geq 1}$ 是 $E$ 中的 Cauchy 序列, 则 $(f(x_n))_{n\geq 1}$ 也是 $F$ 中的 Cauchy 序列.

    (b) 设 $E$ 在度量空间 $E'$ 中稠密并且 $F$ 是完备的, 证明 $f$ 可以唯一地拓展成从 $E'$ 到 $F$ 的连续映射.
\end{exercise}

\begin{proof}
    (a) 因为 $(x_n)_{n\geq 1}$ 是 Cauchy 序列, 所以 $(x_n)_{n\geq 1}$ 为有界序列.
    又 $f$ 在 $E$ 的有界子集上一致连续且一致连续映射将 Cauchy 序列映为 Cauchy 序列,
    故 $(f(x_n))_{n\geq 1}$ 是 $F$ 中的 Cauchy 序列.

    (b) 记 $E$ 上的度量为 $d$, $F$ 上的度量为 $\delta$.

    首先构造 $f$ 的一个扩展映射.
    由于 $E$ 在 $E'$ 中稠密, 故对于 $\forall x\in E'$,
    存在 $(x_n)_{n\geq 1}\subset E$ 使得 $(x_n)_{n\geq 1}$ 收敛于 $x$.
    显然 $(x_n)_{n\geq 1}$ 为 Cauchy 序列, 故由 (a) 知 $(f(x_n))_{n\geq 1}$
    为 $F$ 中的 Cauchy 序列, 又因 $F$ 完备, 故存在 $y\in F$,
    使得 $(f(x_n))_{n\geq 1}$ 收敛于 $y$. 定义 $\tilde{f}(x)=y$.
    由于 $E$ 中收敛于 $x$ 的序列不唯一, 故需证明这一定义不依赖 $(x_n)_{n\geq 1}$ 的选择.
    设 $(x_n')_{n\geq 1}$ 也收敛于 $x$, 相应地, 定义 $y'=\tilde{f}(x_n')$.
    由于 $(x_n)_{n\geq 1}$ 和 $(x_n')_{n\geq 1}$ 都收敛于 $x$,
    故存在 $r>0$ 使得 $(x_n)_{n\geq 1}\subset B(x,r)$ 且 $(x_n')_{n\geq 1}\subset B(x,r)$.
    由于 $f$ 在有界集 $B(x,r)$ 上一致连续,
    故对于 $\forall\varepsilon>0$, 存在 $\eta>0$, 使得当 $d(x_n,x_n')<\eta$ 时,
    有 $\delta(f(x_n),f(x_n'))<\varepsilon$.
    对于上述 $\eta>0$, 存在 $N$, 当 $n>N$ 时, 有 $d(x_n,x_n')<\eta$,
    因此 $\lim_{n\to\infty}\delta(f(x_n),f(x_n'))=0$.
    由度量的连续性即得 $\delta(y,y')=0$, 故 $y=y'$.
    显然 $\tilde{f}|_E=f$, 故 $\tilde{f}$ 为 $f$ 的一个扩展.
\end{proof}



\begin{exercise}
    构造一个反例说明: 在不动点定理中, 如果我们将映射 $f$ 满足的条件减弱为
    \[d(f(x),f(y))<d(x,y),\quad\forall x,y\in E\text{\ 且\ }x\neq y,\]
    则结论不成立.
\end{exercise}

\begin{proof}
    取函数 $f(x)=(x^2+1)^{1/2}$, 不妨设 $x>y$, 则由下面推导过程:
    \begin{align*}
        \sqrt{x^2+1}-\sqrt{y^2+1}<x-y
        &\Leftarrow 2-2\sqrt{x^2+1}\sqrt{y^2+1}<-2xy\\
        &\Leftarrow 1+xy<\sqrt{x^2+1}\sqrt{y^2+1}\\
        &\Leftarrow 1+x^2y^2+2xy<x^2y^2+x^2+y^2+1\\
        &\Leftarrow 0<(x-y)^2,
    \end{align*}
    可知 $f(x)$ 满足题给条件, 显然 $f(x)$ 没有不动点.
\end{proof}



\begin{exercise}
    设 $(E,d)$ 是一个完备的度量空间, $f$ 是其上的映射, 并满足 $f^n=f\circ\cdots\circ f$
    ($n$ 次幂) 是压缩映射. 证明 $f$ 有唯一的不动点, 并给出例子说明 $f$ 可以不连续.
\end{exercise}

\begin{proof}
    因为 $f^n$ 是压缩映射, 所以 $f^n$ 存在唯一的不动点 $x_0\in E$, 即
    $f^n(x_0)=x_0$. 那么就有
    \[f^n(f(x_0))=f(f^n(x_0))=f(x_0).\]
    这说明 $f(x_0)$ 也是 $f^n$ 的不动点, 而由不动点的唯一性知 $f(x_0)=x_0$, 即 $x_0$ 为 $f$ 的不动点.

    下证 $f$ 的不动点唯一, 假设 $f$ 存在另一个不动点 $y_0$, 即 $f(y_0)=y_0$, 则
    由归纳法可推出 $f^n(y_0)=y_0$. 由 $f^n$ 的不动点的唯一性知 $y_0=x_0$.
    
    综上可知, $f$的不动点存在且唯一.
\end{proof}

\begin{exercise}
    记区间 $I=(0,\infty)$ 上通常的拓扑为 $\tau$.

    (a) 证明 $\tau$ 可由如下完备的距离 $d$ 诱导:
    \[d(x,y)=|\log x-\log y|.\]

    (b) 设函数 $f\in C^1(I)$ 满足对某个 $\lambda<1$, 任取 $x\in I$, 都有
    $x|f'(x)|\leq\lambda f(x)$. 证明 $f$ 在 $I$ 上存在唯一的不动点.
\end{exercise}

\begin{proof}
    (a) 将距离 $d$ 诱导的拓扑记为 $\tau _d$.
    \[\begin{split}U\in\tau&\Leftrightarrow\forall x\in U,\exists r>0,s.t.\{y>0\mid|y-x|<r\}\subset U\\&\Rightarrow\forall x\in U,\exists r^{*}=\ln\left(\frac{r}{x}+1\right)>0,s.t.\{y>0\mid|\textrm{log}y-\textrm{log}x|<r^{*}\}\subset U\\&\Rightarrow U\in\tau _d\end{split}\]
    \[\begin{split}U\in\tau_d&\Leftrightarrow\forall x\in U,\exists r>0,s.t.\{y>0\mid|\textrm{log}y-\textrm{log}x|<r\}\subset U\\&\Rightarrow\forall x\in U,\exists r^{*}=x(1-e^{-r})>0,s.t.\{y>0\mid|y-x|<r^{*}\}\subset U\\&\Rightarrow U\in\tau\end{split}\]\\
    因而$\tau$可由距离$d$诱导,下面证明距离$d$是完备的:

    任取$(I,d)$中的Cauchy序列$(x_n)_{n\geq 1}$,记$y_n=\log x_n\in\mathbb{R}$,则
    \[\forall\varepsilon >0,\exists N,\forall m,n>N,|\log x_m-\log x_n|<\varepsilon\]
    也即
    \[\forall\varepsilon >0,\exists N,\forall m,n>N,|y_m-y_n|<\varepsilon\]
    故$(y_n)_{n\geq 1}$是$(\mathbb{R},d_{\mathbb{R}})$中的Cauchy序列($d_{\mathbb{R}}$表示自然距离),由$(\mathbb{R},d_{\mathbb{R}})$的完备性知\[\exists y\in\mathbb{R},s.t.y_n\rightarrow y\]
    令$x=e^y\in I$,则有$d(x_n,x)=|\log x_n-\log x|=|y_n-y|\rightarrow 0$,从而说明$(I,d)$完备.

    (b) 首先, 应该声明 $f$ 恒大于零, 否则, 取 $f\equiv 0$, 此时 $f$ 满足题目条件
    但是显然 $f$ 没有不动点.
    在度量空间 $(I,d)$ 中, $f$ 的导数为:
    \begin{align*}
        \forall x_0\in I,f^{(1)}(x_0)
        & =\lim_{x\to x_0}\frac{d(f(x),f(x_0))}{d(x,x_0)}=\lim_{x\to x_0}\frac{|\log f(x)-\log f(x_0)|}{|\log x-\log x_0|}\\
        & =\lim_{x\to x_0}\frac{\left|\frac{\log f(x)-\log f(x_0)}{x-x_0}\right|}{\left|\frac{\log x-\log x_0}{x-x_0}\right|}=\frac{x_0|f^{\prime}(x_0)|}{f(x_0)}.
    \end{align*}
    结合题目条件 $x|f^{\prime}(x)|\leq\lambda f(x)$
    知对 $\forall x\in I$, 有 $|f^{(1)}(x)|\leq\lambda<1$, 
    这表明 $f$ 在度量空间 $(I,d)$ 中为压缩映射.
    又因为 $(I,d)$ 是完备度量空间, 因此 $f$ 在 $I$ 上存在唯一的不动点.
\end{proof}



\begin{exercise}
    设 $E$ 是可数集, 其元素记为 $a_1,a_2,\cdots$. 定义
    \[d(a_p,a_p)=0\text{\ 且当\ }p\neq q\text{\ 时},\; d(a_p,a_q)=10+\frac{1}{p}+\frac{1}{q}.\]

    (a) 证明 $d$ 是 $E$ 上的距离并且 $E$ 成为一个完备的度量空间.

    (b) 设 $f:E\to E$ 定义为 $f(a_p)=a_{p+1}$. 证明当 $p\neq q$ 时, 有
    \[d(f(a_p),f(a_q))<d(a_p,a_q),\]
    但是 $f$ 没有不定点.
\end{exercise}

\begin{proof}
    (a) 由 $d$ 的定义容易验证其满足正定性、对称性以及三角不等式, 因此 $d$ 是 $E$ 上的距离.
    并且对任意 $p\neq q$, 有 $d(a_p,a_q)>10$,由第四题结论, 可知 $(E,d)$ 是完备度量空间.

    (b) 当 $p\neq q$ 时, $d(f(a_p),f(a_q))=d(a_{p+1},a_{q+1})=10+\frac{1}{p+1}+\frac{1}{q+1}<d(a_p,a_q)$.
    假设 $f$ 存在不动点 $a_k$, 则 $f(a_k)=a_k=a_{k+1}$,
    因此 $d(a_k,a_{k+1})=0$, 矛盾, 故 $f$ 没有不动点.
\end{proof}



\begin{exercise}
    本习题的目的是给不动点定理一个新的证明方法.
    设 $(E,d)$ 是非空的完备度量空间, $f:E\to E$ 是压缩映射. 任取 $R\geq 0$, 设
    \[A_R=\{x\in E\mid d(x,f(x))\leq R\}.\]
    \begin{enumerate}[(a)]
        \item 证明 $f(A_R)\subset A_{\lambda R}$.
        \item 证明当 $R>0$ 时, $A_R$ 是 $E$ 中的非空闭子集.
        \item 证明任取 $x,y\in A_R$, 有 $d(x,y)\leq 2R+d(f(x),f(y))$. 并由此导出
              \[\diam(A_R)\leq 2R/(1-\lambda).\]
        \item 证明 $A_0$ 非空.
    \end{enumerate}
\end{exercise}

\begin{proof}
    (a) 任取 $y\in f(A_R)$, 存在 $x\in A_R$, 使得 $y=f(x)$, 则
    \[d(y,f(y))=d(f(x),f(f(x)))\leq\lambda d(x,f(x))\leq\lambda R,\]
    故 $y\in  A_{\lambda R}$, 因此 $f(A_R)\subset A_{\lambda R}$.

    (b) 先证明 $A_R$ 非空. 取定某 $x_0\in E$, 若 $x_0=f(x_0)$, 则 $x_0\in A_R$;
    若 $x_0\neq f(x_0)$, 则 $d(x_0,f(x_0))>0$, 取正整数 $N\geq\log_{\lambda}\frac{R}{d(x_0,f(x_0))}$.
    通过 $x_{n+1}=f(x_n)$ ($n\geq 0$) 构造序列 $(x_n)_{n\geq 0}$, 则
    \[d(x_1,f(x_1))=d(f(x_0),f(f(x_0)))\leq\lambda d(x_0,f(x_0)),\]
    \[d(x_2,f(x_2))=d(f(x_1),f(f(x_1)))\leq\lambda^2 d(x_0,f(x_0)),\]
    由归纳法可得
    \[d(x_n,f(x_n))\leq\lambda^n d(x_0,f(x_0)).\]
    当 $n\geq N$ 时, 有 $d(x_n,f(x_n))\leq R$, 因此 $A_R$ 非空.

    再证明 $A_R$ 为闭集. 任取 $A_R$ 中的收敛序列 $(x_n)_{n\geq 1}$,
    记其收敛值为 $x$. 则对任意 $n\geq 1$ 有 $d(x_n,f(x_n))\leq R$,
    令 $n\to\infty$, 由度量的连续性以及 $f$ 的连续性得 $d(x,f(x))\leq R$,
    即得 $x\in A_R$, 因此 $A_R$ 为闭集.

    (c) 任取 $x,y\in A_R$, 由度量的三角不等式得
    \begin{align*}
        d(x,y)
        &\leq d(x,f(x))+d(f(x),f(y))+d(f(y),y) \\
        &\leq 2R+d(f(x),f(y)).
    \end{align*}
    于是 $d(x,y)\leq 2R+\lambda d(x,y)$, 即 $d(x,y)\leq 2R/(1-\lambda)$.
    关于 $x,y\in A_R$ 取上确界即得 $\diam(A_R)\leq 2R/(1-\lambda)$.

    (d) 取 $R_n=\frac{1}{n}$, 则 $(A_{R_n})_{n\geq 1}$ 为单调下降的非空闭集列且
    $\lim\limits_{n\to\infty}\diam(A_{R_n})=0$, 由\textbf{定理 2.2.6} 知 $A_0=\bigcap\limits_{n\geq 1}A_{R_n}$
    为单点集.
\end{proof}



\begin{exercise}
    设 $(E,d)$ 是完备度量空间, $f$ 和 $g$ 是 $E$ 上两个可交换的压缩映射 (即 $f\circ g=g\circ f$).
    证明 $f$ 和 $g$ 有唯一的、共同的不动点.
\end{exercise}

\begin{proof}
    因 $f$ 是压缩映射, 故 $f$ 有唯一的不动点 $x$, 即 $f(x)=x$.
    因为 $f\circ g=g\circ f$, 所以
    \[f\circ g(x)=\textcolor{red}{f(g(x))}=g\circ f(x)=\textcolor{red}{g(x)},\]
    从而 $g(x)$ 也是 $f$ 的不动点, 而由不动点唯一性知 $g(x)=x$,
    这说明 $x$ 也为 $g$ 的不动点.
\end{proof}



\begin{exercise}
    本习题的目的是把上一习题的结论推广到更一般的情形, 在某种意义上说是非交换
    的压缩映射不动点定理. 设 $(E,d)$ 是完备的度量空间.
    定义联系于集合 $A\subset E$ 的距离函数 $d_A$ 如下:
    \[d_A(x):=d(x,A)=\inf\{d(x,a)\mid a\in A\}.\]
    并设 $\mathcal{C}$ 表示 $E$ 的所有紧子集构成的集族.
    对任意的 $A,B\in\mathcal{C}$, 定义
    \[h(A,B)=\sup_{x\in E}|d_A(x)-d_B(x)|.\]
    \begin{enumerate}[(a)]
        \item 证明 $h$ 是 $\mathcal{C}$ 上的一个距离.
        \item 任取 $F\subset E$, 令 $F_{\varepsilon}=\{x\in E\mid d_F(x)\leq\varepsilon\}$. 证明
              \[h(A,B)=\inf\{\varepsilon\geq 0\mid A\subset B_{\varepsilon}, B\subset A_{\varepsilon}\}.\]
        \item 证明 $(\mathcal{C},h)$ 完备.
        \item 现在令 $f_1,\cdots,f_n$ 是 $E$ 上的 $n$ 个压缩映射.
              定义 $(\mathcal{C},h)$ 上的映射 $T$ 为
              \[T(A)=\bigcup_{k=1}^n f_k(A),\quad A\in\mathcal{C}.\]
              证明 $T$ 是压缩映射. 并由此导出存在唯一的一个紧子集 $K$,
              使得 $T(K)=K$.
    \end{enumerate}
\end{exercise}

\begin{proof}
    (a) 由 Housdorff 空间的紧子集是闭集知 $\mathcal{C}$里面的任意元素都是闭集,
    当 $h(A,B)=0$ 时, 我们有
    \[\forall x\in E,d_A(x)=d_B(x).\]
    故当 $x\in A$ 时, 有 $d_B(x)=d_A(x)=0\Rightarrow x\in B\Rightarrow A\subset B$, 
    同理可得 $B\subset A$, 因此 $h(A,B)=0\Rightarrow A=B$.
    又显然 $A=B\Rightarrow h(A,B)=0$, 因此$h(A,B)=0\Leftrightarrow A=B$,
    故 $h$ 满足正定性, 并且容易验证 $d$ 满足对称性和三角不等式,
    所以 $h$ 是 $\mathcal{C}$ 上的一个距离(实际上, $h$称为Housdorff度量).

    (b) 记 $Q=\{\varepsilon\geq 0\mid A\subset B_\varepsilon, B\subset A_\varepsilon\}$.

    任取 $\varepsilon\in Q$, 下面用反证法证明 $\varepsilon\geq h(A,B)$.

    假设 $\varepsilon <h(A,B)=\sup_{x\in E}|d_A(x)-d_B(x)|$,
    则存在 $x\in E$, 使得 $|d_A(x)-d_B(x)|>\varepsilon$,
    不妨设 $d_A(x)-d_B(x)>\varepsilon$,
    由 $A,B$ 为闭集知存在 $a\in A$ 和 $b\in B$, 使得 $d_A(x)=d(a,x),d_B(x)=d(b,x)$,
    且存在 $a'\in A$, 使得 $d_A(b)=d(a',b)\leq\varepsilon$,
    因此
    \begin{align*}
        d(x,b)+\varepsilon 
        &=d_B(x)+\varepsilon <d_A(x)=d(x,a) \\
        &\leq d(x,a')\leq d(x,b)+d(b,a')\leq d(x,b)+\varepsilon.
    \end{align*}
    矛盾, 故 $\varepsilon\geq h(A,B)$.

    任取 $r>h(A,B)$,下证 $r$ 不是集合 $Q$ 的下界.
    事实上, 存在 $s$, 使得 $r>s>h(A,B)$,故 $\forall x\in E,|d_A(x)-d_B(x)|<s$,
    因此 $\forall x\in A,d_B(x)<s$ 且 $\forall x\in B,d_A(x)<s$,
    从而 $A\subset B_s,B\subset A_s$, 这说明 $s\in Q$, 从而 $r$ 不是集合 $Q$ 的下界.

    综合两点知 $h(A,B)=\inf Q=\inf\{\varepsilon\geq 0\mid A\subset B_{\varepsilon}, B\subset A_{\varepsilon}\}$.

    (c) 任取 $\mathcal{C}$ 中的 Cauchy 序列 $(A_n)_{n\geq 1}$,
    即 $\forall\varepsilon >0,\exists N>0,s.t.\forall m,n>N_1,h(A_m,A_n)<\varepsilon /2$.

    定义集合 $A$ 为:
    \[A=\{x\mid\text{存在序列\ }(x_k) s.t. x_k\in A_k\text{\ 且\ }x_k\rightarrow x\}.\]
    $\forall x\in A,\exists (x_k)(x_k\in A_k),s.t.x_k\rightarrow x$故$\exists N_2>0,\forall k>N_2,d(x_k,x)<\varepsilon /2$\\
    若$k>max\{N_1,N_2\}$,则$h(A_k,A_n)<\varepsilon /2$,故$\exists y\in A_n,s.t.d(x_k,y)<\varepsilon /2$,故$d(y,x)\leq d(x_k,y)+d(x_k,x)<\varepsilon\Rightarrow x\in (A_n)_\varepsilon\Rightarrow A\subset (A_n)_\varepsilon$\\
    另一方面,$\forall y\in A_n$,选取一列整数$n=k_1<k_2<\cdots$使得\[h(A_{k_j},A_m)<2^{-j}\varepsilon (\forall m\geq k_j)\]
    然后我们如下定义序列$(y_k)_{k\geq 1}(y_k\in A_k)$:
    $k<n$时,$y_k$任意选取,选择$y_n=y$,如果$y_{k_j}$已经选择了,且$k_j<k\leq k_{j+1}$,
    选择$y_k\in A_k,s.t.d(y_{k_j},y_k)<2^{-j}\varepsilon$,则$(y_k)_{k\geq 1}$是Cauchy序列,
    故$y_k\rightarrow x\in A$
    由
    \[d(y,x)=\lim_{k\to\infty}d(y,y_k)=\lim_{j\to\infty}d(y,y_{k_j})\leq\lim_{j\to\infty}(2^{-1}\varepsilon +\cdots +2^{-j+1}\varepsilon)=\varepsilon\]
    知$ y\in (A)_\varepsilon\Rightarrow A_n\subset (A)_\varepsilon$,
    所以$h(A,A_n)<\varepsilon$,这就证明了$A_n\xrightarrow{h}A$
    
    下面还需证明$A$是紧的,为此,需要证明$A$是闭集且完全有界:

    i)假设$x\in\bar{A}$,则$\forall n\geq 1,\exists y_n\in A,s.t.d(x,y_n)<2^{-n}$,
    又因为$\forall n\geq 1,\exists z_n\in A_n,s.t.d(z_n,y_n)\leq h(A_n,A)$,
    故\[d(z_n,x)\leq d(z_n,y_n)+d(x,y_n)<h(A_n,A)+2^{-n}\rightarrow 0\]
    所以$z_n\rightarrow x$,故$x\in A$,因而$A$是闭集.

    ii)$\forall\varepsilon >0,\exists n\geq 1,s.t.h(A_n,A)<\varepsilon/3$,
    由于$A_n$紧,故$A_n$存在有限的$\varepsilon/3$网,
    即存在$\{y_1,y_2,\cdots,y_m\}\subset A_n,s.t.A_n\subset\bigcup_{i=1}^mB(y_i,\varepsilon/3),\forall y_i,\exists x_i\in A,s.t.d(x_i,y_i)<\varepsilon/3$,
    我们断言$\{x_1,x_2,\cdots,x_m\}$构成了$A$的一个有限$\varepsilon$网
    (反证法:假设$\exists x_0\in A,s.t.d(x_0,x_i)\geq\varepsilon(\forall i=1,2,\cdots,m)$,
    设$x_0$与$A_n$中的$y_0$距离最近,且$y_0$所在的开球球心为$y_i$,
    则$d(x_0,x_i)\leq d(x_0,y_0)+d(y_0,y_i)+d(y_i,x_i)<\varepsilon$,矛盾)因此$A$是完全有界的.

    (d)将$\{f_i\}$的压缩系数分别记为$\lambda_1,\cdots,\lambda_n$,
    令$\lambda=\max\{\lambda_1,\cdots,\lambda_n\}$,下面证明$T$是以$\lambda$为压缩系数的压缩映射,
    即证:$\forall A,B\in C,h(T(A),T(B))\leq\lambda h(A,B)$

    任取$r>h(A,B),\forall x\in T(A),\exists 1\leq i\leq n,a\in A,s.t.x=f_i(a)$,
    因为$h(A,B)<r$,所以$\exists b\in B,s.t.d(a,b)<r$,令$y=f_i(b)\in T(B)$,
    我们有\[d(x,y)=d(f_i(a),f_i(b))\leq\lambda_i d(a,b)<\lambda r\]
    所以\[d(x,T(B))<\lambda r\Rightarrow\sup_{x\in T(A)}d(x,T(B))\leq\lambda r\]
    同理可得
    \[\sup_{y\in T(B)}d(y,T(A))\leq\lambda r\]
    因此$h(T(A),T(B))\leq\lambda r$,令$r\to h(A,B)$,即得$h(T(A),T(B))\leq\lambda h(A,B)$,从而说明$T$是压缩映射.
\end{proof}
% !TeX root = main.tex
% !TeX encoding = UTF8
% !TeX program = xelatex
\setcounter{chapter}{2}
\chapter{赋范空间和连续线性映射}
\thispagestyle{empty}

\begin{exercise}
     设 $C([0,1],\FR)$ 表示 $[0,1]$ 上的所有连续实函数构成的空间. 定义
    \[\|f\|_{\infty}=\sup_{0\leq x\leq 1}|f(t)|\quad\text{且}\quad \|f\|_1=\int_0^1|f(t)|\diff t\]
    \begin{enumerate}[(a)]
        \item 证明 $\|\cdot\|_{\infty}$ 和 $\|\cdot\|_1$ 都是 $C([0,1],\FR)$ 上的范数.
        \item 证明 $C([0,1],\FR)$ 关于范数 $\|\cdot\|_{\infty}$ 是完备的.
        \item 证明 $C([0,1],\FR)$ 关于范数 $\|\cdot\|_1$ 不完备.
    \end{enumerate}
\end{exercise}

\begin{proof}
(a)由\begin{itemize}
\item $\|f\|_{\infty}=\sup\limits_{0\leq t\leq 1}|f(t)|\geq 0$, 且$\|f\|_{\infty}=0$当且仅当$f\equiv 0$
\item $\|\lambda f\|_{\infty}=\sup\limits_{0\leq t\leq 1}|\lambda f(t)|=|\lambda|\sup\limits_{0\leq t\leq 1}|f(t)|=|\lambda|\cdot\|f\|_{\infty}$
\item $\|f+g\|_{\infty}=\sup\limits_{0\leq t\leq 1}|f(t)+g(t)|\leq \sup\limits_{0\leq t\leq 1}(|f(t)|+|g(t)|)=\|f\|_{\infty}+\|g\|_{\infty}$
\end{itemize}
和
\begin{itemize}
\item $\|f\|_1=\int_0^1|f(t)|\diff t\geq 0$ 且 $\|f\|_1=0$ 当且仅当 $f\equiv 0$
\item $\|\lambda f\|_1=\int_0^1 |\lambda f(t)|\diff t=|\lambda|\cdot\|f\|_1$
\item $\|f+g\|_1=\int_0^1 |f(t)+g(t)|\diff t\leq\int_0^1|f(t)|\diff t+\int_0^1|g(t)|\diff t=\|f\|_1+\|g\|_1$
\end{itemize}
知 $\|\cdot\|_{\infty}$ 和 $\|\cdot\|_1$ 都是 $C([0,1],\FR)$ 上的范数.

(b)任取 $C([0,1],\FR)$ 中的 Cauchy 序列 $(f_n)_{n\geq 1}$, 
即对于 $\forall\epsilon>0$, 存在$N>0$, 对于 $\forall m,n>N$, 有
\[\|f_m-f_n\|_{\infty}=\max_{0\leq x\leq 1}|f_m(x)-f_n(x)|<\epsilon,\]
所以对任意 $t\in[0,1]$, 序列 $(f_n(t))_{n\geq 1}$ 为 Cauchy 序列, 其必收敛. 令
\[f(t)=\lim_{n\to\infty}f_n(t).\]
这样就定义了一个 $[0,1]$ 上的实值函数. 

下面证明 $f$ 是连续函数且 $\|f_n-f\|_{\infty}\to 0$ (即 $(f_n)_{n\geq 1}$ 一致收敛到$f$). 
而我们只需要证明 $(f_n)_{n\geq 1}$ 一致收敛到 $f$ 即可,
事实上, 由一致收敛级数的连续性定理可知, 如果 $(f_n)_{n\geq 1}$ 一致收敛到 $f$, 则
$f$ 必为连续函数.

任意给定 $\varepsilon>0$, 存在 $N=N(\varepsilon)>0$, 
使得对于 $\forall m,n>N$ 和 $\forall t\in[0,1]$ 都有 
\[|f_m(t)-f_n(t)|<\varepsilon.\]
任意固定 $n>N$ 并令 $m\to\infty$ 可得对于 $\forall n>N$ 和 $\forall t\in[0,1]$ 有
\[|f_n(t)-f(t)|<\varepsilon.\]
所以 $(f_n)_{n\geq 1}$ 一致收敛到 $f$.

(c)我们只需寻找范数 $\|\cdot\|_1$ 意义下的柯西列使其不收敛即可. 定义折线段:
\[f_n(x)=\begin{cases}
0, & 0\leq x\leq\frac{1}{2}-\frac{1}{n} \\
n\left(x-\frac{1}{2}+\frac{1}{n}\right), & \frac{1}{2}-\frac{1}{n}\leq x\leq\frac{1}{2} \\
1, & \frac{1}{2}\leq x\leq 1.
\end{cases}\]
则
\[\|f_m-f_n\|_1=\int_0^1|f_m(x)-f_n(x)|\diff x=\frac{1}{2}\left\lvert\frac{1}{m}-\frac{1}{n}\right\rvert\to 0(m,n\to\infty)\]
故 $(f_n)_{n\geq 1}$ 是 Cauchy 序列, 但是其没有极限.
\end{proof}

\begin{remark}
证明度量空间的完备性基本都是转化为基本的完备空间(如($\FR,d_{\FR}$))来考虑.
\end{remark}


\begin{exercise}
     设 $E$ 是 $\FR$ 上所有的实系数多项式构成的向量空间. 对任意 $P\in E$, 定义
    \[\|P\|_{\infty}=\max_{x\in[0,1]}|P(x)|.\]
    \begin{enumerate}[(a)]
    \item 证明 $\|\cdot\|_{\infty}$ 是 $E$ 上的范数.
    \item 任取一个 $a\in\FR$ 定义线性映射 $L_a:E\to\FR$ 满足 $L_a(P)=P(a)$. 证明 $L_a$ 连续当且仅当 $a\in[0,1]$, 并且给出该连续线性映射的范数.
    \item 设 $a<b$ 并定义 $L_{a,b}:E\to\FR$ 满足
    \[L_{a,b}(P)=\int_a^bP(x)\diff x.\]
    给出 $a,b$ 的范围, 使其成为 $L_{a,b}$ 连续的充分必要条件, 然后确定 $L_{a,b}$ 的范数.
    \end{enumerate}
\end{exercise}

\begin{proof}
(a)由
\begin{itemize}
    \item $\|P\|_\infty =0\Leftrightarrow\max\limits_{x\in [0,1]}|P(x)|=0\Leftrightarrow P(x)=0(\forall x\in [0,1])\Leftrightarrow P=0$
    \item $\|\lambda P\|_\infty =\max_{x\in [0,1]}|\lambda P(x)|=|\lambda |\max_{x\in [0,1]}|P(x)|=|\lambda|\|P\|_{\infty}$
    \item $\|P+Q\|_{\infty}=\max\limits_{x\in [0,1]}|(P+Q)(x)|\leq\max\limits_{x\in [0,1]}(|P(x)|+|Q(x)|)=\|P\|_{\infty}+\|Q\|_{\infty}$
\end{itemize}
知 $\|\cdot\|_{\infty}$ 是 $E$ 上的范数.

(b)\sufficient 
当 $a\in [0,1]$ 时, 对于任意 $P\in E$, 有
\[|L_a(P)|=|P(a)|\leq\max_{0\leq x\leq 1}|P(x)|=\|P\|_{\infty},\]
故 $L_a$ 为连续线性映射.

\necessary 
(直接法) 由 $L_a$ 为连续线性映射知, 存在常数 $C\geq 0$, 使得
对于 $\forall P\in E$, 有
\[|L_a(P)|=|P(a)|\leq C\|P\|_{\infty}=C\max_{0\leq x\leq 1}|P(x)|.\]
取 $P(x)=x^{2n}$, 则 $a^{2n}\leq C$, 故 $-1\leq a\leq 1$.
再取 $P(x)=(1-x)^{2n}$, 则 $(1-a)^{2n}\leq C$, 故 $0\leq a\leq 2$.
因此 $0\leq a\leq 1$.

综上得证: $L_a$ 连续 $\Leftrightarrow a\in [0,1]$, 且
\[\|L_a\|=\sup\limits_{p\in E,P\neq 0}\frac{|P(a)|}{\max_{0\leq x\leq 1}|P(x)|}=1(P\equiv 1\;\text{时可取到最大值}).\]

(c) $L_{a,b}$连续的充要条件是$0\leq a<b\leq 1$, 理由如下:

\sufficient
当 $0\leq a<b\leq 1$时, 对于任意 $P\in E$, 有
\begin{align*}
    |L_{a,b}(P)|
    &=\left\lvert\int_a^b P(x)\diff x\right\rvert\leq\int_a^b |P(x)|\diff x \\
    &\leq\int_0^1 |P(x)|\diff x\leq\max_{0\leq x\leq 1}|P(x)|=\|P\|_{\infty}.
\end{align*}
故 $L_{a,b}$ 为连续线性映射且 $\|L_{a,b}\|\leq 1$.

\necessary
先给出一个结论($\star$):
设 $b>1$ 且 $a<b$, 则数列 $\left(\frac{b^n-a^n}{n}\right)_{n\geq 1}$
必有子列为正无穷大量.
事实上, 当 $1<a<b$ 时, 由 Stolz 定理可得该数列为正无穷大量;
当 $-1\leq a\leq 1$ 时, 该数列显然为正无穷大量;
当 $a<-1$ 时, 子列 $\left(\frac{b^{2n+1}-a^{2n+1}}{2n+1}\right)_{n\geq 1}$
为正无穷大量.

因为$L_{a,b}$连续, 所以存在常数 $C\geq 0$ 使得对于任意 $P\in E$, 有
\[|L_{a,b}(P)|=\left\lvert\int_a^b P(x)\diff x\right\rvert\leq C\max_{0\leq x\leq 1}|P(x)|.\]

取 $P(x)=x^n$, 则 $\left\lvert\frac{b^{n+1}-a^{n+1}}{n+1}\right\rvert\leq C$, 
即数列 $\left(\frac{b^{n+1}-a^{n+1}}{n+1}\right)_{n\geq 1}$ 有界.
假设 $b>1$, 则由上述结论 ($\star$) 知 $\left(\frac{b^{n+1}-a^{n+1}}{n+1}\right)_{n\geq 1}$
存在子列为正无穷大量, 矛盾, 因此 $b\leq 1$.

再取 $P(x)=(1-x)^n$, 则 $\left\lvert\frac{(1-a)^{n+1}-(1-b)^{n+1}}{n+1}\right\rvert\leq C$, 
同理可知 $1-a\leq 1$, 即 $a\geq 0$. 从而当 $L_{a,b}$ 连续时, 有 $0\leq a<b\leq 1$.

综上得知 $L_{a,b}$ 连续 $\Leftrightarrow 0\leq a<b\leq 1$, 且
\[\|L_{a,b}\|=\sup\limits_{P\in E,P\neq 0}\frac{|\int_a^bP(x)\diff x|}{\max_{0\leq x\leq 1}|P(x)|}=b-a(P\equiv 1\;\text{时可取到最大值}).\qedhere\]
\end{proof}



\begin{exercise}
     设 $(E,\|\cdot\|_{\infty})$ 是习题~2 中定义的赋范空间. 
     设 $E_0$ 是 $E$ 中没有常数项的多项式构成的向量子空间(即多项式 $P\in E_0$ 等价于 $P(0)=0$).
    \begin{enumerate}[(a)]
        \item 证明 $N(P)=\|P'\|_{\infty}$ 定义了 $E_0$ 上的一个范数, 并且对任意 $P\in E_0$, 有 $\|P\|_{\infty}\leq N(P)$.
        \item 证明 $L(P)=\int_0^1\frac{P(x)}{x}\diff x$ 定义了 $E_0$ 关于 $N$ 的连续线性泛函, 并求它的范数.
        \item 上面定义的 $L$ 是否关于范数 $\|\cdot\|_{\infty}$ 连续?
        \item 范数 $\|\cdot\|_{\infty}$ 和 $N$ 在 $E_0$ 上是否等价?
    \end{enumerate}
\end{exercise}

\begin{proof}
(a) 由
\begin{itemize}
\item $N(P)=\|P'\|_{\infty}=\max\limits_{0\leq x\leq 1}|P'(x)|\geq 0$ 且 $N(P)=0$ 当且仅当 $P\equiv 0$
\item $N(\lambda P)=\max_{0\leq x\leq 1}|\lambda P'(x)|=|\lambda|\max_{0\leq x\leq 1}|P'(x)|=|\lambda|N(P)$
\item $N(P+Q)=\max\limits_{0\leq x\leq 1}|P'(x)+Q'(x)|\leq\max\limits_{0\leq x\leq 1}(|P'(x)|+|Q'(x)|)=N(P)+N(Q)$
\end{itemize}
知 $N(\cdot)$ 是 $E_0$ 上的范数.

由中值定理知: $P(x)-P(0)=P(x)=xP'(\theta),\forall x\in (0,1],\exists\theta\in (0,x)$. 故
\[|P(x)|\leq |P'(\theta)|\Rightarrow\max_{0\leq x\leq 1}|P(x)|\leq\max_{0\leq x\leq 1}|P'(x)|\Rightarrow \|P\|_{\infty}\leq N(P).\]

(b)由
\begin{align*}
    L(\lambda P+Q)
    & =\int_0^1\frac{\lambda P(x)+Q(x)}{x}\diff x \\
    & =\lambda\int_0^1\frac{P(x)}{x}\diff x+\int_0^1\frac{Q(x)}{x}\diff x=\lambda L(P)+L(Q)
\end{align*}
知 $L$ 是线性映射. 又因为
\begin{align*}
|L(P)|
&=\left|\int_0^1\frac{P(x)}{x}\diff x\right|\leq\int_0^1\left|\frac{P(x)}{x}\right|\diff x\leq\left\|\frac{P(x)}{x}\right\|_{\infty}=\left\lvert\frac{P(x_0)}{x_0}\right\rvert\quad(\exists x_0\in [0,1])\\
&=\left\lvert\frac{P(x_0)-P(0)}{x_0-0}\right\rvert=|P'(\theta)|\leq\|P'\|_{\infty}=N(P),
\end{align*}
即 $|L(P)|\leq N(P)$.
故 $L$ 是 $E_0$ 关于 $N$ 的连续线性泛函, 且 $\|L\|=1$.

(c) 取 $P_n(x)=-x^2+\frac{2}{n}x$, 则
\[\|P_n\|_{\infty}=\max_{x\in [0,1]}|P(x)|=\frac{1}{n^2}\to 0,\quad n\to\infty.\]
但是
\[L(P_n)=\int_0^1 \frac{P_n(x)}{x}\diff x=\int_0^1 \biggl(-x+\frac{2}{n}\biggr)\diff x=-1+\frac{2}{n}\to -1,\quad n\to\infty.\]
故 $L$ 关于范数 $\|\cdot\|_{\infty}$ 不连续.

(d) $\|\cdot\|_{\infty}$ 与 $N$ 在 $E_0$ 上不等价.
反证法证明: 
假设存在常数 $C_1 $和 $C_2$ 使得对于 $\forall P\in E_0$ 
有 $C_1N(p)\leq\|p\|_{\infty}\leq C_2N(p)$. 取 $n>\frac{1}{C_1}$ 且 $P(x)=x^n$, 则
\[N(P)=\max_{0\leq x\leq 1}|nx^{n-1}|=n>\frac{1}{C_1}=\frac{1}{C_1}\|P\|_{\infty}.\]
矛盾, 证毕.
\end{proof}


\begin{exercise}
     设 $E$ 是由 $[0,1]$ 上所有连续函数构成的向量空间. 
    定义 $E$ 上的两个范数分别为 $\displaystyle\|f\|_1=\int_0^1|f(x)|\diff x$ 和 $\displaystyle N(f)=\int_0^1x|f(x)|\diff x$.
    \begin{enumerate}[(a)]
        \item 验证 $N$ 的确是 $E$ 上的范数并且 $N\leq\|\cdot\|_1$.
        \item 设函数 $f_n(x)=n-n^2x$, 若 $x\leq\frac{1}{n}$; $f_n(x)=0$, 其它. 
              证明函数列 $(f_n)_{n\geq 1}$ 在 $(E,N)$ 上收敛到 0. 它在 $(E,\|\cdot\|_1)$ 中是否收敛? 由这两个范数在 $E$ 上诱导的拓扑是否相同?
        \item 设 $\alpha\in(0,1]$, 并令 $B=\{f\in E:f(x)=0,\forall x\in[0,\alpha]\}$. 证明这两个范数在 $B$ 上诱导相同的拓扑.
    \end{enumerate}
\end{exercise}

\begin{proof}
(a) 由
\begin{itemize}
\item $N(f)=\int_0^1x|f(x)|dx\geq 0$ 且 $N(f)=0\Leftrightarrow x|f(x)|\equiv 0\Leftrightarrow f(x)\equiv 0$ (这里利用了 $f(x)$ 的连续性)
\item $N(\lambda f)=\int_0^1x|\lambda f(x)|dx=|\lambda|\int_0^1x|f(x)|dx=|\lambda|N(f)$
\item $N(f+g)=\int_0^1x|f(x)+g(x)|dx\leq \int_0^1x(|f(x)|+|g(x)|)dx=N(f)+N(g)$
\end{itemize}
知 $N$ 是 $E$ 上的范数, $N\leq\|\cdot\|_1$是显然的.

(b) 因
\[N(f_n)=\int_0^{\frac{1}{n}}x(n-n^2x)\diff x=\frac{1}{6n}\rightarrow 0(n\to\infty),\]
故 $(f_n)_{n\geq 1}$ 在 $(E,N)$ 中收敛到 $0$. 
假设函数列在 $(E,\|\cdot\|_1)$ 中收敛, 即存在 $g(x)\in E$ 使得
\[\lim_{n\to\infty}\int_0^1|f_n(x)-g(x)|\diff x=0\Rightarrow f_n(x)-g(x)=0\almosteverywhere(n\to\infty).\]
又 $f_n(x)=0\almosteverywhere (n\to\infty)$, 故$g(x)=0$, 也就是说如果收敛只能收敛到 0, 但
\[\lim_{n\to\infty}\|f_n(x)-0\|_1=\lim_{n\to\infty}\int_0^{\frac{1}{n}}(n-n^2x)\diff x=\frac{1}{2}\neq 0.\]
矛盾, 故 $(f_n)_{n\geq 1}$ 在 $(E,\|\cdot\|_1)$ 中不收敛.

两范数在 $E$ 上诱导的拓扑不同, 理由如下:

记$\|\cdot\|_1$诱导的拓扑为$\tau_1$, $N$诱导的拓扑为$\tau_2$, 
相应的距离分别记为 $d_1,d_2$. 
由 $N\leq \|\cdot\|_1$ 知 $\tau_2\subset\tau_1$, 
故我们实际需要证明 $\tau_2$ 是 $\tau_1$ 的真子集, 即
\[\exists V\in\tau_1,\text{但}\;V\notin\tau_2.\]
取 $\tau_1$ 中开球 $B_{d_1}(0,\frac{1}{3})\in\tau_1$,
假设$B_{d_1}(0,\frac{1}{3})\in\tau_2$. 因为 $0\in B_{d_1}(0,\frac{1}{3})$, 
所以 $\exists\delta>0,s.t.B_{d_2}(0,\delta)\subset B_{d_1}(0,\frac{1}{3})$.
取前面给出的 $(f_n)_{n\geq 1}$, 由$d_2(f_n,0)\to 0(n\to\infty)$ 知
\[\exists M>0,s.t.f_M\in B_{d_2}(0,\delta)\subset B_{d_1}(0,\frac{1}{3})\]
但是 $d_1(f_M,0)=\frac{1}{2}>\frac{1}{3}$, 矛盾, 故假设不成立, 即$B_{d_1}(0,\frac{1}{3})\notin\tau_2$.

(c) 对于 $\forall f\in B$, 有 
\begin{align*}
    N(f) & =\int_0^1x|f(x)|\diff x=\int_a^1 x|f(x)|\diff x \\
         & \geq a\int_a^1|f(x)|\diff x=a\int_0^1|f(x)|\diff x=a\|f\|_1,
\end{align*} 
结合 (a) 中给出的 $N\leq\|\cdot\|_1$ 知两范数等价, 因此必在 $B$ 上诱导相同的拓扑.
\end{proof}


\begin{exercise}
     设 $\varphi:[0,1]\to[0,1]$ 是连续函数并且不恒等于 1. 设 $\alpha\in\FR$, 定义 $C([0,1],\FR)$ 上的映射 $T$ 为
    \[T(f)(x)=\alpha+\int_0^xf(\varphi(t))\diff t.\]
    证明 $T$ 是压缩映射.

    根据以上结论证明下面的方程存在唯一解:
    \[f(0)=\alpha,\quad f'(x)=f(\varphi(x)),\quad x\in[0,1].\]
\end{exercise}


\begin{proof}
取 $C([0,1])$ 上的范数 $\|\cdot\|$, 定义为
\[\|f\|=\sup\limits_{x\in [0,1]}|f(x)|\e^{-Mx}.\]
由 $\varphi([0,1])\subset [0,1]$ 知
\[\sup\limits_{t\in[0,1]}|f(\varphi(t))-g(\varphi(t))|\e^{-M\varphi(t)}\leq\sup\limits_{x\in[0,1]}|f(x)-g(x)|\e^{-x}=\|f-g\|.\]
因此
\begin{align*}
    \|T(f)-T(g)\| & =\sup\limits_{x\in[0,1]}\left|\int_0^x(f(\varphi(t))-g(\varphi(t)))\diff t\right|\e^{-Mx}\\
                  & =\sup\limits_{x\in [0,1]}\left|\int_0^x(f(\varphi(t))-g(\varphi(t)))\e^{-M\varphi(t)}\e^{M\varphi(t)}\diff t\right|\e^{-Mx}\\
                  & \leq\|f-g\|\cdot\sup\limits_{x\in [0,1]}\int_0^x\e^{M\varphi(t)}\diff t\cdot\e^{-Mx}.
\end{align*}

下面我们说明通过选取合适的 $M>0$, 可以使得 
\[\lambda\colon=\sup_{0\leq x\leq 1}\int_0^x \e^{M\varphi(t)}\diff t\cdot\e^{-Mx}<1.\]
令函数
\[h(x):=\int_0^x \e^{M\varphi(t)}\diff t\cdot\e^{-Mx}.\]
因 $h(x)$ 在 $[0,1]$ 上连续, 故 $h(x)$ 在 $[0,1]$ 上存在最大值点, 记之为 $x_0$.

若 $x_0<1$, 则
\[h(x_0)=\int_0^{x_0} \e^{M\varphi(t)}\diff t\cdot\e^{-Mx_0}\leq x_0\e^{M(1-x_0)},\]
取 $0<M<\frac{-\ln x_0}{1-x_0}$, 则 $h(x_0)<1$.

若 $x_0=1$, 注意到 $\varphi$ 不恒等于 $1$, 则
\[h(x_0)=\int_0^1 \e^{M\varphi(t)}\diff t\cdot\e^{-M}<\e^M\cdot\e^{-M}=1.\]

综上得知, 通过选择合适的 $M>0$, 可以使得映射 $T$ 为 $C([0,1])$ 上的压缩映射.
且由 $\e^{-M}\|f\|_{\infty}\leq\|f\|\leq\|f\|_{\infty}$ 
知 $(C([0,1]),\|\cdot\|)$ 是 Banach 空间, 故根据不动点定理知存在唯一$f\in C([0,1])$ 使得 $T(f)=f$, 即:
\[\alpha+\int_0^xf(\varphi(t))\diff t=f(x)\Leftrightarrow f(0)=\alpha\text{\ 且\ }f'(x)=f(\varphi(x)).\qedhere\]
\end{proof}


\begin{exercise}
    设 $\alpha\in\FR,a>0,b>1$. 考察下面的微分方程
    \begin{equation}
    f(0)=\alpha,\quad f'(x)=af(x^b),\quad 0\leq x\leq 1.\tag{$*$}
    \end{equation}
    \begin{enumerate}[(a)]
    \item 令 $M>0$. 验证 $E=C([0,1],\FR)$ 上赋予范数
    \[\|f\|=\sup_{0\leq x\leq 1}|f(x)|\e^{-Mx}\]
    后成为一个 Banach 空间.
    \item 设 $g(x)=\alpha+\int_0^x af(t^b)\diff t$, 定义映射 $T:E\to E$ 为 $T(f)=g$. 证明选择合适的 $M$, 可使 $T$ 为压缩映射.
    \item 证明方程~($*$) 有唯一解.
    \end{enumerate}
\end{exercise}

\begin{proof}
(a)容易验证 $\|\cdot\|$ 是$C([0,1],\FR)$ 上的范数, 并且
\[\e^{-M}\sup\limits_{0\leq x\leq1}|f(x)|\leq\sup\limits_{0\leq x\leq1}|f(x)|\e^{-Mx}\leq\sup\limits_{0\leq x\leq1}|f(x)|,\]
即
\[e^{-M}\|f\|_{\infty}\leq\|f\|\leq\|f\|_{\infty}.\]
因此 $E=C([0,1],\FR)$ 赋予范数 $\|\cdot\|$ 是 Banach 空间.

(b) 因为
\[(T(f_1)-T(f_2))(x)=\int_0^xa\left(f_1(t^b)-f_2(t^b)\right)\diff t,\]
所以
\begin{align*}
    \|T(f_1)-T(f_2)\| & =\sup\limits_{0\leq x\leq 1}\left|\int_0^x a\left(f_1(t^b)-f_2(t^b)\right)\diff t\right|\cdot \e^{-Mx}\\
                      & \leq a\sup\limits_{0\leq x\leq 1}\int_0^x|f_1(t^b)-f_2(t^b)|\e^{-Mt^b}\cdot \e^{Mt^b}\diff t\cdot \e^{-Mx}\\
                      & \leq\|f_1-f_2\|\cdot a\sup\limits_{0\leq x\leq 1}\int_0^x \e^{Mt^b}\diff t\cdot \e^{-Mx}\\
                      & \leq\|f_1-f_2\|\cdot a\sup\limits_{0\leq x\leq 1}\int_0^x \e^{Mt}\diff t\cdot \e^{-Mx}\\
                      & =\|f_1-f_2\|\cdot\sup\limits_{0\leq x\leq 1}\frac{a\left(1-\e^{-Mx}\right)}{M}\leq\frac{a}{M}\|f_1-f_2\|.
\end{align*}
故当 $M>a$ 时, $\|T(f_1)-T(f_2)\|<\|f_1-f_2\|$, 也就是此时$T$是压缩映射.

(c)压缩映射有唯一不动点, 即存在唯一 $f\in C([0,1],\FR)$ 使得
\[\alpha+\int_0^xaf\left(t^b\right)\diff t=f(x),\]
而上述方程等价于方程 $(*)$, 证毕.
\end{proof}


\begin{exercise}
    设 $E$ 是数域 $\FK$ 上的无限维向量空间. 设 $(e_i)_{i\in I}$
    是 $E$ 中的一组向量, 若 $E$ 中任一向量可用 $(e_i)_{i\in I}$
    中的有限个向量唯一线性表示, 即对任意 $x\in E$, 存在唯一一组 $(\alpha_i)_{i\in I}\subset\FK$,
    使得仅有有限多个 $\alpha_i$ 不等于零且 $x=\sum_{i\in I}\alpha_ie_i$,
    则称 $(e_i)_{i\in I}$ 是 $E$ 中的 Hamel 基.
    \begin{enumerate}[(a)]
        \item 由 Zorn 引理证明 $E$ 有一组 Hamel 基.
        \item 假设 $E$ 还是一个赋范空间, 证明 $E$ 上必存在不连续的线性泛函.
        \item 证明在任一无限维赋范空间上, 一定存在一个比原来的范数严格强的范数
              (即新范数诱导的拓扑一定比原来的范数诱导的拓扑强且不相同).
    \end{enumerate}
\end{exercise}

\begin{proof}
(a)首先构造一个偏序集$(\mathcal{F},\subset)$, 这里的$\mathcal{F}$是$E$中一些子集构成的集族,
满足若 $F\in\mathcal{F}$, 则 $F$ 中任意有限多个向量都线性无关, $\subset$表示集合间的包含关系.

任取$\mathcal{F}$的一个链$\mathcal{A}$,
令 $G=\bigcup_{A\in\mathcal{A}}A$, 则 $G\in\mathcal{F}$,
即 $G$ 是 $\mathcal{A}$ 的上界. 由 Zorn 引理知$\mathcal{F}$有极大元,记为 $B$.
如果存在 $x\in E$, $x$ 不能由 $B$ 中任意有限多个向量线性表达,
则 $B\bigcup\{x\}\in\mathcal{F}$, 这与 $B$ 是极大元矛盾, 故这个极大元 $B$ 就是 $E$ 的 Hamel 基.

(b) 设 $B$ 是 $E$ 上的一个 Hamel 基, 若 $E$ 还是一个赋范空间,
则不妨设 Hamel 基中的任一向量 $e$ 的范数为 1,
由于 $E$ 中的任意向量关于Hamel基的线性表达是唯一的,
故 $E$ 上的线性泛函 $f$ 由其在 Hamel 基中每一个元素上的取值 $f(e)$ 决定,
显然 Hamel 基是无限集, 故我们可以选取一个序列 $(e_n)\in B$,
令 $f(e_n)=n$; 当$e\in B\backslash(e_n)$时,$f(e)=1$,则线性泛函$f$在$E$上不连续.

\textcolor{blue}{注:和课本定理3.2.9对比体会有限维和无限维的区别}.

(c)仍考虑(b)中约定的Hamel基$B$, 并记$E$上原有的范数为$\|\cdot\|$,
接下来定义$E$上的新范数$\|\cdot\|_1$,取$(e_n)\in B$,令$\|e_n\|_1=n,n\geq 1$;
当$e\in B\backslash(e_n)$时,令$\|e\|_1=\|e\|$,任取$x\in E$,
则$x=\sum_{j\in J}\lambda_je_j,J\subset I$是有限集,注意这种表达式唯一,
令
\[\|x\|_1=\sum_{j\in J}|\lambda_j|\;\|e_j\|_1\]
容易验证 $\|\cdot\|_1$确实是 $E$ 上的范数, 并且由三角不等式有
\[\|x\|=\|\sum_{j\in J}\lambda_je_j\|\leq\sum_{j\in J}|\lambda_j|\,\|e_j\|\leq\sum_{j\in J}|\lambda_j|\,\|e_j\|_1=\|x\|_1\]
故$\|\cdot\|_1$是在$E$上比$\|\cdot\|$强的范数.另一方面(b)约定的线性泛函$f$满足
\[|f(x)|\leq\sum_{j\in J}|\lambda_j|\,|f(e_j)|=\sum_{j\in J}|\lambda_j|\,\|e_j\|_1=\|x\|_1\]
但是$f$关于原来的范数$\|\cdot\|$不连续,这意味着$\|\cdot\|$一定是比$\|\cdot\|$严格强的范数.
\end{proof}



\begin{exercise}
    设 $E$ 为数域 $\FK$ 上有限维向量空间, 其维数 $\dim E=n$.
    $\{e_{1},\cdots,e_{n}\}$ 表示 $E$ 上的一组基, 任取 $u\in\mathcal{L}(E)$, 令 $[u]$ 表示 $u$ 在这组基下对应的矩阵.

    (a) 证明映射 $u \mapsto[u]$ 建立了从 $\mathcal{L}(E)$ 到所有 $n \times n$ 矩阵构成的向量空间 
    $\mathbb{M}_{n}(\FK)$ 之间的同构映射.

    (b) 假设 $E=\mathbb{K}^{n}$ 且 $\{e_1,\cdots,e_n\}$ 是经典基 
    (即 $e_{k}=(0, \cdots, 0,1,0, \cdots, 0)$, 对应于第 $k$ 个向量, 
    它仅在第 $k$ 个位置取 1 , 其他位置取 $0$). 
    并约定 $E=\mathbb{K}^{n}$ 赋予欧氏范数. 证明若 $u$ (或等价地 $[u]$) 可对角化, 
    则 $\|u\|=\max\{|\lambda_{1}|,\cdots,|\lambda_{n}|\}$, 这里 $\lambda_{1}, \cdots, \lambda_{n}$ 是 $u$ 的特征值.

    (c) $\{e_{1}, \cdots, e_{n}\}$ 如上, 试由 $[u]$ 中的元素分别确定在 $p=1$ 和 $p=\infty$ 时的
    范数 $\left\|u:\left(\mathbb{K}^{n},\|\cdot\|_{p}\right)\rightarrow\left(\mathbb{K}^{n},\|\cdot\|_{p}\right)\right\|$.
\end{exercise}

\begin{proof}
(a)记
\[u(e_k)=\sum_{m=1}^n u_{mk}e_m,\quad k=1,\cdots,n\]
并记矩阵 $[u]=(u_{mk})\in\mathbb{M}_n(\mathbb{K})$,
则 $(u(e_1),\cdots,u(e_n))=(e_1,\cdots,e_n)[u]$.
任取 $x\in E$, 存在唯一的一组数 $x_1,\cdots,x_n$ 使得 $x=\sum_{k=1}^n x_ke_k$. 则
\[u(x)=\sum_{k=1}^n x_ku(e_k)=(e_1,\cdots,e_n)[u]
\begin{pmatrix}
    x_1\\\vdots\\x_n
\end{pmatrix}\]
由此说明映射 $u\mapsto [u]$ 建立了从 $\mathcal{L}(E)$ 到所有 $n\times n$ 
矩阵构成的向量空间 $\mathbb{M}_n(\mathbb{K})$ 之间的同构映射.

(b) \textcolor{blue}{注: 这一问的题目条件稍微改一下, 将$u$可对角化改为$u$可酉对角化,
即存在酉矩阵 $P$ 使得 $P[u]P^{*}=\Lambda$, 其中 $\Lambda=\diag\{\lambda_1,\cdots,\lambda_n\}$.}

由题意知此时 $E$ 为有限维赋范空间, 故由定理 3.2.9 知 $\mathcal{L}(E)=\mathcal{B}(E)$,
即对于任意 $u\in\mathcal{L}(E)$, 都有 $u$ 为有界线性算子.
对于任意 $x\in E=\mathbb{K}^n$, 由 (a) 知 $u(x)$ 在 $e_1,\cdots,e_n$
下的坐标为 $[u]x$, 故
\begin{align*}
    \|u(x)\|^2
    & =x^{*}[u]^{*}[u]x\\
    & =x^{*}P^{*}\Lambda^{*}PP^{*}\Lambda Px\\
    & =(Px)^{*}\Lambda^{*}\Lambda(Px)\\
    & =(Px)^{*}\begin{pmatrix}|\lambda_1|^2& & \\ &\ddots& \\ & & |\lambda_n|^2\end{pmatrix}(Px)\\
    & \leq\max\{|\lambda_1|,\cdots,|\lambda_n|\}^2\|Px\|^2\\&=\max\{|\lambda_1|,\cdots,|\lambda_n|\}^2\|x\|^2,
\end{align*}
故 $\|u(x)\|\leq \max\{|\lambda_1|,\cdots,|\lambda_n|\}\|x\|$, 因此$||u||\leq \max\{|\lambda_1|,\cdots,|\lambda_n|\}$.

设$\max\{|\lambda_1|,\cdots,|\lambda_n|\}=|\lambda_k|$, 取$Px=(0,\cdots,0,1,0,\cdots,0)^T,\|x\|=1$,其中1位于第$k$个坐标位置,则
\[\|u(x)\|^2=|\lambda_k|^2\|x\|^2\Rightarrow \|u(x)\|=|\lambda_k|\|x\|\]
综上知$\|u\|=\max\{|\lambda_1|,\cdots,|\lambda_n|\}$.

(c)记$\|u\|_p=\|u:(\mathbb{K}^n,\|\cdot\|_p)\to(\mathbb{K}^n,\|\cdot\|_p)\|,p=1,\infty$.

(i) 当 $p=1$ 时, 任取 $x=(x_1,\cdots,x_n)^T\in\mathbb{K}^n$, 则
\[\begin{split}\|u(x)\|_1
&=\|[u]x\|_1=\sum_{j=1}^n\left|\sum_{i=1}^nu_{ji}x_i\right|\\
&\leq \sum_{j=1}^n\sum_{i=1}^n|u_{ji}|\cdot|x_i|=\sum_{i=1}^n\sum_{j=1}^n|u_{ji}|\cdot|x_i|\\
&=\sum_{i=1}^n\left(|x_i|\sum_{j=1}^n|u_{ji}|\right)\leq\left(\max\limits_{1\leq i\leq n}\sum_{j=1}^n|u_{ji}|\right)\|x\|_1,
\end{split}\]
故
\[\|u\|_1\leq \max\limits_{1\leq i\leq n}\sum_{j=1}^n|u_{ji}|.\]
设 $\max_{1\leq i\leq n}\sum_{j=1}^n|u_{ji}|=\sum_{j=1}^n|u_{jk}|$, 
取 $x=(0,\cdots,0,1,0,\cdots,0)^T,\|x\|_1=1$, 其中 1 位于第 $k$ 个坐标位置, 则
\[\|u(x)\|_1=\left(\sum_{j=1}^n|u_{jk}|\right)\|x\|_1.\]
综上得知
\[\|u\|_1=\max\limits_{1\leq i\leq n}\sum_{j=1}^n|u_{ji}|.\]

(ii) 当 $p=\infty$ 时, 同理可证明:
\[\|u\|_{\infty}=\max\limits_{1\leq j\leq n}\sum_{i=1}^n|u_{ji}|.\qedhere\]
\end{proof}




\begin{exercise}
    设 $E$ 是 Banach 空间.
    \begin{enumerate}[(a)]
    \item 设 $u\in\mathcal{B}(E)$ 且 $\|u\|<1$. 证明 $I_E-u$ 在 $\mathcal{B}(E)$ 中可逆.
    \item 设 $GL(E)$ 表示 $\mathcal{B}(E)$ 中可逆元构成的集合. 证明 $GL(E)$ 关于复合运算构成一个群且是 $\mathcal{B}(E)$ 中的开集.
    \item 证明 $u\to u^{-1}$ 是 $GL(E)$ 上的同胚映射.
    \end{enumerate}
\end{exercise}

\begin{proof}
因为 $\|u\|<1$, 且 $\|u^n\|\leq\|u\|^n$, 所以级数 $\sum_{n=0}^{\infty}\|u^n\|$ 收敛, 
又因为 $\mathcal{B}(E)$ 完备, 故 $\sum_{n=0}^{\infty}u^n$ 收敛, 记
\[v=\sum_{n=0}^{\infty}u^n\in\mathcal{B}(E).\]
则
\[(I_E-u)v=\lim_{k\to\infty}(I_E-u)\sum_{n=0}^ku^n=I_E.\]
同理可证 $v(I_E-u)=I_E$, 因此 $I_E-u$ 在 $\mathcal{B}(E)$ 中可逆.

(b)\begin{itemize}
\item $(uv)w=u(vw)$,即满足结合律
\item 恒等映射 $id$ 即为单位元
\item 任意元 $u$ 都存在 $u^{-1}\in GL(E),s.t.u\circ u^{-1}=id$
\end{itemize}
故 $GL(E)$ 关于复合运算构成一个群, 
下证 $GL(E)$ 是 $\mathcal{B}(E)$ 中的开集: 
任意 $u\in GL(E)$, 考虑 $u$ 的开球 $B(u,\|u^{-1}\|^{-1})$, 
则 $\forall v\in B(u,\|u^{-1}\|^{-1})$, 有 $\|v-u\|<\|u^{-1}\|^{-1}$, 
故 $\|u^{-1}(u-v)\|\leq \|u^{-1}\|\cdot\|u-v\|<1$, 从而
\[I-u^{-1}(u-v)=u^{-1}v\in GL(E).\]
由群中元素运算封闭性知
\[u\cdot u^{-1}v=v\in GL(E).\]
故
\[B(u,\|u^{-1}\|^{-1})\in GL(E).\]
由开集的定义知 $GL(E)$ 是 $\mathcal{B}(E)$ 中的开集.

(c)记$\Phi:GL(E)\to GL(E),u\mapsto u^{-1}$.
\begin{itemize}
\item 显然映射 $\Phi:u\mapsto u^{-1}$ 是 $GL(E)$ 上的双射;
\item $\Phi$ 连续: 由前面的证明过程知 $\forall v\in B(u,\|u^{-1}\|^{-1})$ 有
\[(I-u^{-1}(u-v))^{-1}=\sum_{n=0}^{\infty}(u^{-1}(u-v))^n.\]
故
\[v^{-1}=(u-(u-v))^{-1}=(u(I-u^{-1}(u-v)))^{-1}=\sum_{n=0}^{\infty}(u^{-1}(u-v))^nu^{-1}.\]
因此
\[\begin{split}
\|v^{-1}-u^{-1}\|
&=\|\sum_{n=1}^{\infty}(u^{-1}(u-v))^nu^{-1}\|\\
&\leq\|u^{-1}\|\cdot\sum_{n=1}^{\infty}(\|u-v\|\cdot\|u^{-1}\|)^n\\
&=\frac{\|u^{-1}\|^2\|u-v\|}{1-\|u^{-1}\|\cdot\|u-v\|}.
\end{split}\]
当 $\|u-v\|\to 0$ 时, $\|u^{-1}-v^{-1}\|\to 0$, 所以 $\Phi$ 连续;
\item $\Phi=\Phi^{-1}$
\end{itemize}
综上知 $\Phi$ 是 $GL(E)$ 上的同胚.
\end{proof}



% \setcounter{exercise}{9}
\begin{exercise}
    设 $f\in L_2(\FR)$, $g(x)=\frac{1}{x}\mathbbm{1}_{[1,\infty)}(x)$, 
    证明 $fg\in L_1(\FR)$. 给出例子说明 $f_1,f_2\in L_1(\FR)$, 但是 $f_1f_2\notin L_1(\FR)$.
\end{exercise}

\begin{proof}
(1)因为
\[g(x)=\frac{1}{x}\cdot\mathbbm{1}_{[1,+\infty)}(x),\]
所以
\[\int_{\FR} g^2(x)\diff x=\int_1^{\infty}\frac{1}{x^2}\diff x=1.\]
故 $g\in L_2(\FR)$, 又$f\in L_2(\FR)$, 所以 $fg\in L_1(\FR)$.

(2)取
\[f_1(x)=f_2(x)=\frac{1}{\sqrt{x}}\mathbbm{1}_{(0,1)}(x),\]
则
\[\int_{\FR} |f_1(x)|\diff x=\int_{\FR} |f_2(x)|\diff x=\int_0^1\frac{1}{\sqrt{x}}\diff x=2,\]
但是
\[\int_{\FR} |f_1(x)f_2(x)|\diff x=\int_0^1\frac{1}{x}\diff x=+\infty.\qedhere\]
\end{proof}




\begin{exercise}
    设 $(\varOmega,\mathcal{A},\mu)$ 为有限测度空间, 即有 $\mu(\varOmega)<\infty$.

    (a) 证明若 $0<p<q\leq\infty$, 则 $L_q(\varOmega)\subset L_p(\varOmega)$.
    用反例说明当 $\mu(\varOmega)=\infty$ 时, 结论不成立.

    (b) 证明若 $f\in L_{\infty}(\varOmega)$, 则 $f\in\bigcap\limits_{p<\infty}L_p(\varOmega)$
    且 $\|f\|_{\infty}=\lim_{p\to\infty}\|f\|_{p}$.

    (c) 设 $f\in\bigcap\limits_{p<\infty}L_p(\varOmega)$ 且满足 $\limsup_{p\to\infty}\|f\|_P<\infty$,
    证明 $f\in L_{\infty}(\varOmega)$.
\end{exercise}

\begin{proof}
    (a) 因 $0<p<q\leq\infty$, 故可设 $\frac{1}{p}=\frac{1}{q}+\frac{1}{r}$, 其中 $r>0$.
    因 $\mu(\varOmega)<\infty$, 故 $\int_{\varOmega}1^r\diff\mu=\mu(\varOmega)<\infty\Rightarrow 1\in L_r(\varOmega)$.
    任取 $f\in L_q(\varOmega)$, 由 H\"older 不等式可得 $f=f\cdot 1\in L_p(\varOmega)$, 且
    \[\|f\|_p\leq\|f\|_q\|1\|_r=\|f\|_q\cdot\big(\mu(\varOmega)\big)^{1/r},\]
    因此 $L_q(\varOmega)\subset L_p(\varOmega)$.

    当 $\mu(\varOmega)=\infty$ 时, $L_q(\varOmega)\subset L_p(\varOmega)$
    不一定成立, 例如取 $f(x)=\frac{1}{x}$, 则 $f\in L_2([1,\infty))$,
    但 $f\notin L_1([1,\infty))$.

    (b) 对于任意 $p<\infty$, 有 $\frac{1}{p}=\frac{1}{\infty}+\frac{1}{p}$.
    又因为 $f\in L_{\infty}(\varOmega)$, $1\in L_p(\varOmega)$, 所以由 H\"older 不等式
    知 $f=f\cdot 1\in L_p(\varOmega)$, 从而 $f\in\bigcap\limits_{p<\infty}L_p(\varOmega)$.
    又因
    \[\|f\|_p=\biggl(\int_{\varOmega}|f|^p\diff\mu\biggr)^{1/p}\leq\biggl(\int_{\varOmega}\|f\|_{\infty}^p\diff\mu\biggr)^{1/p}=\|f\|_{\infty}\bigl(\mu(\varOmega)\bigr)^{1/p},\]
    故
    \begin{equation}
        \limsup_{p\to\infty}\|f\|_p\leq\|f\|_{\infty}.\tag{$\star$}
    \end{equation}

    任意固定 $\delta>0$, 令 $\varOmega_{\delta}=\{x\mid |f(x)|>\|f\|_{\infty}-\delta\}$,
    则 $\mu(\varOmega_{\delta})>0$, 否则的话, 假设 $\mu(\varOmega_{\delta})=0$,
    则由本性上确界的定义知 $\|f\|_{\infty}\leq\|f\|_{\infty}-\delta$, 矛盾. 故
    \[\|f\|_p=\biggl(\int_{\varOmega}|f|^p\diff\mu\biggr)^{1/p}\geq\biggl(\int_{\varOmega_{\delta}}(\|f\|_{\infty}-\delta)^p\diff\mu\biggr)^{1/p}=(\|f\|_{\infty}-\delta)\bigl(\mu(\varOmega_{\delta})\bigr)^{1/p},\]
    两侧取下极限并结合 $\delta$ 的任意性, 得
    \begin{equation}
        \liminf_{p\to\infty}\|f\|_p\geq\|f\|_{\infty}.\tag{$\star\star$}
    \end{equation}
    由 $(\star)(\star\star)$ 得 $\lim_{p\to\infty}\|f\|_p=\|f\|_{\infty}$.

    (c) 假设 $f\notin L_{\infty}(\varOmega)$, 则对任意 $M>0$,
    存在 $A\in\mathcal{A}$, 使得 $\mu(A)>0$ 且在 $A$ 上 $|f|>M$, 则
    \[\|f\|_p=\biggl(\int_{\varOmega}|f|^p\diff\mu\biggr)^{1/p}\geq\biggl(\int_A M^p\diff\mu\biggr)^{1/p}=M\cdot(\mu(A))^{1/p},\]
    于是
    \[\limsup_{p\to\infty}\|f\|_p\geq M.\]
    由于 $M$ 是任意的, 故上式与 $\limsup_{p\to\infty}\|f\|_p<\infty$ 相矛盾.
\end{proof}




\begin{exercise}(插值不等式)
    设 $0<p<q\leq\infty$, $0\leq\theta\leq 1$. 并令
    \[\frac{1}{s}=\frac{\theta}{p}+\frac{1-\theta}{q}.\]
    证明若 $f\in L_p(\varOmega)\cap L_q(\varOmega)$, 则
    \[f\in L_s(\varOmega)\quad\text{且}\quad \|f\|_s\leq\|f\|_p^{\theta}\|f\|_q^{1-\theta}.\]
\end{exercise}

\begin{proof}
    $\theta=0$ 与 $\theta=1$ 的情形是平凡的, 故只需考虑 $0<\theta<1$.
    因 $f\in L_p(\varOmega)$, 故 $f^{\theta}\in L_{\frac{p}{\theta}}(\varOmega)$.
    又因 $f\in L_q(\varOmega)$, 故 $f^{1-\theta}\in L_{\frac{q}{1-\theta}}(\varOmega)$.
    而 $\frac{1}{s}=\frac{1}{p/\theta}+\frac{1}{q/(1-\theta)}$, 故由 H\"older 不等式知
    $f=f^{\theta}f^{1-\theta}\in L_s(\varOmega)$ 且
    \[\|f\|_s\leq\|f^{\theta}\|_{\frac{p}{\theta}} \|f^{1-\theta}\|_{\frac{q}{1-\theta}}=\|f\|_p^{\theta} \|f\|_q^{1-\theta}.\qedhere\]
\end{proof}




\begin{exercise}(广义 Minkowski 不等式) 
    设 $\left(\varOmega_{1}, \mathcal{A}_{1}, \mu_{1}\right)$ 
    和 $\left(\varOmega_{2}, \mathcal{A}_{2}, \mu_{2}\right)$ 是两个测度空间, $0<p<q<\infty$. 
    证明对任意可测函数 
    $f:\left(\varOmega_{1} \times \varOmega_{2}, \mathcal{A}_{1} \otimes \mathcal{A}_{2}\right) \rightarrow\FK$, 有
    \begin{align*}
        &\left(\int_{\varOmega_{2}}\left(\int_{\varOmega_{1}}\left|f\left(x_{1}, x_{2}\right)\right|^{p}\diff\mu_{1}(x_{1})\right)^{\frac{q}{p}}\diff \mu_{2}\left(x_{2}\right)\right)^{\frac{1}{q}} \\
   \leq &\left(\int_{\varOmega_{1}}\left(\int_{\varOmega_{2}}\left|f\left(x_{1}, x_{2}\right)\right|^{q}\diff\mu_{2}\left(x_{2}\right)\right)^{\frac{p}{q}}\diff\mu_{1}\left(x_{1}\right)\right)^{\frac{1}{p}}.
    \end{align*}
\end{exercise}

\begin{proof}
    首先由 Fubini 定理可得
    \begin{align*}
        & \int_{\varOmega_2}\biggl(\int_{\varOmega_1}|f(x_1,x_2)|^p\diff\mu_1(x_1)\biggr)^{\frac{q}{p}}\diff\mu_2(x_2) \\
    ={} & \int_{\varOmega_2}\biggl(\int_{\varOmega_1}|f(x_1,x_2)|^p\diff\mu_1(x_1)\biggr)^{\frac{q}{p}-1}\biggl(\int_{\varOmega_1}|f(x_1,x_2)|^p\diff\mu_1(x_1)\biggr)\diff\mu_2(x_2) \\
    ={} & \int_{\varOmega_1}\int_{\varOmega_2}\biggl(\int_{\varOmega_1}|f(x_1,x_2)|^p\diff\mu_1(x_1)\biggr)^{\frac{q}{p}-1}\cdot |f(x_1,x_2)|^p\diff\mu_2(x_2)\diff\mu_1(x_1).
    \end{align*}
    然后由 H\"older 不等式得
    \begin{align*}
        & \int_{\varOmega_2}\biggl(\int_{\varOmega_1}|f(x_1,x_2)|^p\diff\mu_1(x_1)\biggr)^{\frac{q}{p}-1}\cdot |f(x_1,x_2)|^p\diff\mu_2(x_2) \\
    \leq{} & \biggl[\int_{\varOmega_2}\biggl(\int_{\varOmega_1}|f(x_,x_2)|^p\diff\mu_1(x_1)\biggr)^{\frac{q}{p}}\diff\mu_2(x_2)\biggr]^{\frac{q-p}{q}}\biggl(\int_{\varOmega_2}|f(x_1,x_2)|^q\diff\mu_2(x_2)\biggr)^{\frac{p}{q}}.
    \end{align*}
    故
    \begin{align*}
        & \int_{\varOmega_2}\biggl(\int_{\varOmega_1}|f(x_1,x_2)|^p\diff\mu_1(x_1)\biggr)^{\frac{q}{p}}\diff\mu_2(x_2) \\
    \leq{} & \biggl[\int_{\varOmega_2}\biggl(\int_{\varOmega_1}|f(x_,x_2)|^p\diff\mu_1(x_1)\biggr)^{\frac{q}{p}}\diff\mu_2(x_2)\biggr]^{\frac{q-p}{q}}\cdot\int_{\varOmega_1}\biggl(\int_{\varOmega_2}|f(x_1,x_2)|^q\diff\mu_2(x_2)\biggr)^{\frac{p}{q}}\diff\mu_1(x_1).
    \end{align*}
    即
    \[\biggl[\int_{\varOmega_2}\biggl(\int_{\varOmega_1}|f(x_1,x_2)|^p\diff\mu_1(x_1)\biggr)^{\frac{q}{p}}\diff\mu_2(x_2)\biggr]^{\frac{p}{q}}\leq\int_{\varOmega_1}\biggl(\int_{\varOmega_2}|f(x_1,x_2)|^q\diff\mu_2(x_2)\biggr)^{\frac{p}{q}}\diff\mu_1(x_1).\]
    因此
    \begin{align*}
        & \left(\int_{\varOmega_{2}}\left(\int_{\varOmega_{1}}\left|f\left(x_{1}, x_{2}\right)\right|^{p}\diff\mu_{1}(x_{1})\right)^{\frac{q}{p}}\diff \mu_{2}\left(x_{2}\right)\right)^{\frac{1}{q}} \\
    \leq{} & \left(\int_{\varOmega_{1}}\left(\int_{\varOmega_{2}}\left|f\left(x_{1}, x_{2}\right)\right|^{q}\diff\mu_{2}\left(x_{2}\right)\right)^{\frac{p}{q}}\diff\mu_{1}\left(x_{1}\right)\right)^{\frac{1}{p}}.\qedhere
    \end{align*}
\end{proof}



% \setcounter{exercise}{13}
\begin{exercise}
    设 $0<p<\infty$.
    \begin{enumerate}[(a)]
        \item 对任意 $x=(x_n)\in\ell_p$ 定义 $(0,1)$ 上如下的函数
        \[T(x)(t)=\sum_{n\geq 1}[n(n+1)]^{\frac{1}{p}}x_n\mathbbm{1}_{(\frac{1}{n+1},\frac{1}{n})}(t).\]
        证明 $T$ 是 $\ell_p$ 到 $L_p(0,1)$ 的线性等距同构映射.
        \item 假设 $p\geq 1$ 且 $q$ 是 $p$ 的共轭数. 对任意 $f\in L_p(0,1)$, 定义
        \[S(f)_n=[n(n+1)]^{\frac{1}{q}}\int_{\frac{1}{n+1}}^{\frac{1}{n}}f(t)\diff t,\;\forall n\geq 1\]
        证明 $S$ 定义了从 $L_p(0,1)$ 到 $\ell_p$ 上的线性映射并且 $S\circ T$ 等于 $\ell_p$ 上的单位映射.
    \end{enumerate}
\end{exercise}

\begin{proof}
(a)\begin{itemize}
\item \[\begin{split}\int_0^1|T(x)(t)|^p\diff t
&=\int_0^1\left|\sum_{n\geq 1}[n(n+1)]^{\frac{1}{p}}x_n\mathbbm{1}_{(\frac{1}{n+1},\frac{1}{n})}(t)\right|^p\diff t\\
&=\sum_{n=1}^{\infty}\int_{\frac{1}{n+1}}^{\frac{1}{n}}\left|[n(n+1)]^{\frac{1}{p}}x_n\right|^p\diff t\\
&=\sum_{n=1}^{\infty}\left(\frac{1}{n}-\frac{1}{n+1}\right)n(n+1)|x_n|^p=\sum_{n=1}^{\infty}|x_n|^p<\infty
\end{split}\]
故$T(x)(t)\in L_p(0,1)$.
\item 线性:\[T(\lambda x+y)(t)=\sum_{n\geq 1}[n(n+1)]^{\frac{1}{p}}(\lambda x_n+y_n)\mathbbm{1}_{(\frac{1}{n+1},\frac{1}{n})}(t)=\lambda\cdot T(x)+T(y)\]
\item 等距:由第一条知$\forall x,y\in\ell_p$:
\[\|T(x-y)(t)\|_p=\left(\int_0^1(T(x-y)(t))^p\diff t\right)^{\frac{1}{p}}=\left(\sum_{n=1}^{\infty}(x_n-y_n)^p\right)^{\frac{1}{p}}=\|x-y\|_p\]故$T$是等距映射
\end{itemize}
注意:题目有一点小问题,$T$不是同构,因为不满足满射,例如$f(x)=x\in L_p(0,1)$不存在原像.

(b)\begin{itemize}
\item \[\sum_{i=1}^n|S(f)_n|^p=\sum_{n=1}^{\infty}[n(n+1)]^{\frac{p}{q}}\left|\int_{\frac{1}{n+1}}^{\frac{1}{n}}f(t)\diff t\right|^p\leq\sum_{n=1}^{\infty}[n(n+1)]^{p-1}\left(\int_{\frac{1}{n+1}}^{\frac{1}{n}}|f(t)|\diff t\right)^p\]
\begin{enumerate}[(i)]
\item 当$p=1$时:
\[\sum_{n=1}^{\infty}|S(f)_n|\leq\sum_{n=1}^{\infty}\int_{\frac{1}{n+1}}^{\frac{1}{n}}|f(t)|\diff t=\int_0^1|f(t)|\diff t<\infty\]
故$(S(f)_n)_{n\geq 1}\in\ell_1$
\item 当$p>1$时:
\[\begin{split}\sum_{n=1}^{\infty}|S(f)_n|^p&\leq\sum_{n=1}^{\infty}\frac{1}{n(n+1)}\cdot (n(n+1))^p\cdot\left(\int_{\frac{1}{n+1}}^{\frac{1}{n}}|f(t)|\diff t\right)^p\\&=\sum_{n=1}^{\infty}\frac{1}{n(n+1)}\left(\int_{\frac{1}{n+1}}^{\frac{1}{n}}n(n+1)|f(t)|\diff t\right)^p\end{split}\]
由 H\"older 不等式得:
\[\int_{\frac{1}{n+1}}^{\frac{1}{n}}n(n+1)|f(t)|\diff t\leq\left(\int_{\frac{1}{n+1}}^{\frac{1}{n}}|f(t)|^p\diff t\right)^{\frac{1}{p}}\cdot\left(\int_{\frac{1}{n+1}}^{\frac{1}{n}}(n(n+1))^q\right)^{\frac{1}{q}}\]
故\[\begin{split}
\left(\int_{\frac{1}{n+1}}^{\frac{1}{n}}n(n+1)|f(t)|\diff t\right)^p
&\leq\int_{\frac{1}{n+1}}^{\frac{1}{n}}|f(t)|^p\diff t\cdot\left((n(n+1))^q\frac{1}{n(n+1)}\right)^{\frac{p}{q}}\\
&=\int_{\frac{1}{n+1}}^{\frac{1}{n}}|f(t)|^p\diff t\cdot(n(n+1))^{(p-1)(q-1)}\\
&=n(n+1)\int_{\frac{1}{n+1}}^{\frac{1}{n}}|f(t)|^p\diff t
\end{split}\]
从而
\[\sum_{n=1}^{\infty}|S(f)_n|^p\leq\sum_{n=1}^{\infty}\int_{\frac{1}{n+1}}^{\frac{1}{n}}|f(t)|^p\diff t=\int_0^1|f(t)|^p\diff t<\infty\]
\end{enumerate}
综上知 $S$ 确实将 $L_p(0,1)$ 中得元素映到 $\ell_p$ 中
\item 线性:
\[S(\lambda f+g)_n=[n(n+1)]^{\frac{1}{q}}\int_{\frac{1}{n+1}}^{\frac{1}{n}}(\lambda f(t)+g(t))\diff t=\lambda S(f)_n+S(g)_n\]
\item 单位映射: $\forall x\in\ell_p$, 有
\[((S\circ T)(x))_n=(S(T(x)(t)))_n=[n(n+1)]^{\frac{1}{q}}\int_{\frac{1}{n+1}}^{\frac{1}{n}}[n(n+1)]^{\frac{1}{p}}x_n\diff t=x_n\]
故 $(S\circ T)(x)=x$, 即 $S\circ T$ 是 $\ell_p$ 上的单位映射.
\end{itemize}
证毕.
\end{proof}




\begin{exercise}
    (a) 证明: 若 $(E,d)$ 是可分的度量空间, $F\subset E$,
    则 $(F,d)$ 也是可分的度量空间.

    (b) 证明: $\FR^{n}, c_{0}, \ell_{p}, 1 \leq p<\infty, C([a, b], \FR), C_{0}(\FR, \FR)$ 
    和 $L_{p}(0,1), 1 \leq p<\infty$, 都是可分的.

    (c) 设 $C=\{-1,1\}^{\mathbb{N}}$ 是 $\ell_{\infty}$ 的子集, 
    它由所有的每项是 $1$ 或 $-1$ 的序列构成. 首先验证若 $x$ 和 $y$ 是 $C$ 中两个不同序列, 
    则 $\|x-y\|_{\infty}=2$. 再证明 $C$ 不可数, 由此导出 $\ell_{\infty}$ 不可分.
    类似证明 $L_{\infty}(0,1)$ 不可分.
\end{exercise}

\begin{proof}
    (a) 因 $(E,d)$ 可分, 故有可数稠密子集 $A$, 令 $B=A\cap F$,
    则 $B$ 为 $(F,d)$ 的可数稠密子集, 从而 $(F,d)$ 可分.

    (b) 令 $\FQ^n=\{(q_1,\cdots,q_n)\mid q_i\in\FQ, 1\leq i\leq n\}$, 
    则 $\FQ^n$ 为 $\FR^n$ 的可数稠密子集.

    $c_0$ 可分: 令 $S_k=\{(a_0,\cdots,a_k,0,\cdots)\mid a_i\in\FQ,1\leq i\leq k\}$,
    $S=\bigcup_{k\geq 1}S_k$, 则 $S$ 为可数集.
    任取 $y=(y_n)_{n\geq 1}\in c_0$, 由于 $\lim_{n\to\infty}|y_n|=0$,
    故对任意 $\epsilon>0$, 存在 $N$, 使得当 $n>N$ 时, $|y_n|<\epsilon$.
    由 $\FQ$ 在 $\FR$ 中稠密可知存在 $x=(x_1,\cdots,x_N,0,\cdots)\in S_N$,
    使得 $|x_i-y_i|<\epsilon(1\leq i\leq N)$, 从而 $\|x-y\|_{\infty}<\epsilon$,
    从而 $S$ 在 $c_0$ 中稠密.

    (c) 我们先说明一个引理:
    设 $(E,d)$ 为度量空间, $U\subset E$ 为不可数子集且存在 $r>0$,
    使得对任意 $x,y\in U$, $x\neq y$, 有 $d(x,y)\geq r$,
    则 $E$ 不可分.
    \begin{proof}[引理证明]
        假设 $E$ 存在可数稠密子集 $C$, 即 $C=(x_n)_{n\geq 1}$ 且 $\closure{C}=E$.
        首先我们有 $E=\bigcup_{n=1}^{\infty}B(x_n,\frac{r}{2})$,
        事实上, 任取 $x\in E=\closure{C}$, 由闭包的性质知 $B(x,\frac{r}{2})\cap C\neq\emptyset$,
        即存在 $C$ 中某 $x_n$ 使得 $x_n\in B(x,\frac{r}{2})$,
        而 $x_n\in B(x,\frac{r}{2})\Leftrightarrow x\in B(x_n,\frac{r}{2})$, 因此
        $x\in\bigcup_{n=1}^{\infty}B(x_n,\frac{r}{2})$, 从而 $E=\bigcup_{n=1}^{\infty}B(x_n,\frac{r}{2})$. 
        (从过程可看出, 这里的 $\frac{r}{2}$ 可替换为任意的正常数, 只不过为下面导出矛盾,
        故选择 $\frac{r}{2}$).

        于是 $U\subset\bigcup_{n=1}^{\infty}B(x_n,\frac{r}{2})$,
        而 $U$ 为不可数子集, 故必存在不同两点 $x,y\in U$ 包含于同一个球 $B(x_n,\frac{r}{2})$ 中,
        那么 $d(x,y)<r$, 矛盾. 这就说明 $E$ 不可分.
    \end{proof}
    
    $\ell_{\infty}$ 不可分: 考虑 $\ell_{\infty}$ 的子集 $C=\{1,-1\}^{\FN}$, 
    若 $x=(x_n)_{n\geq 1}$ 与 $y=(y_n)_{n\geq 1}$ 是 $C$ 中两个不同序列, 
    则必存在某 $n_0\geq 1$, 使得 $x_{n_0}\neq y_{n_0}\Rightarrow |x_{n_0}-y_{n_0}|=2$,
    从而 $\|x-y\|_{\infty}=2$. 根据实变函数中的技巧, 将 $C$ 中元素与无限小数表示对应可证
    $C$ 的基数为 $c$, 即 $C$ 不可数. 因此 $\ell_{\infty}$ 不可分.

    $L_{\infty}(0,1)$ 不可分: 考虑 $L_{\infty}(0,1)$ 的子集
    $A=(\mathbbm{1}_{(0,r)})_{0<r<1}$, 显然 $A$ 为不可数子集,
    且对于任意的 $r_1\neq r_2$, 有 $\|\mathbbm{1}_{(0,r_1)}-\mathbbm{1}_{(0,r_2)}\|_{\infty}=1$,
    故 $L_{\infty}(0,1)$ 不可分. 
\end{proof}




\begin{exercise}(卷积) 
     在实数集 $\FR$ 上取 Lebesgue $\sigma$-代数及 Lebesgue 测度, 并设 $f,g\in L_{1}(\FR)$.

     (a) 证明
    \begin{align*}
        \int_{\FR\times\FR} f(u)g(v)\diff u\diff v 
        & =\left[\int_{\FR} f(u) \diff u\right]\left[\int_{\FR} g(v) \diff v\right] \\
        & =\int_{\FR}\left[\int_{\FR} f(x-y)g(y) \diff y\right] \diff x.
    \end{align*}
    由此导出函数 $x\mapsto\int_{\FR} f(x-y)g(y)\diff y$ 在 $\FR$ 上几乎处处有定义.

    (b) 我们定义 $f$ 和 $g$ 的卷积 $f*g$ 为
    \[
    f*g(x)= \begin{cases}
        \int_{\FR} f(x-y)g(y) \diff y, & \text{当积分存在, } \\ 
        0, & \text{其他.}\end{cases}
    \]
    证明 $f*g\in L_{1}(\FR)$ 且 $\|f*g\|_1\leq\|f\|_1\|g\|_1$.

    (c) 取 $f=\mathbbm{1}_{[0,1]}$, 计算 $f*f$.
\end{exercise}

\begin{proof}
    首先容易验证 $f(x-y)g(y)$ 为可测函数, 故由 Tonelli 定理得
    \begin{align*}
        \int_{\FR^2} |f(x-y)g(y)|\diff(x,y)
        & =\int_{\FR}\int_{\FR} |f(x-y)g(y)|\diff x\diff y \\
        & =\int_{\FR}|g(y)|\int_{\FR} |f(x-y)|\diff x\diff y \\
        & =\|f\|_{L_1}\int_{\FR}|g(y)|\diff y=\|f\|_{L_1}\|g\|_{L_1}<\infty.
    \end{align*}
    故 $f(x-y)g(y)$ 在 $\FR^2$ 上可积, 由 Fubini 定理立即可得对于几乎处处的 $x\in\FR$, 有
    \[\int_{\FR}f(x-y)g(y)\diff y<\infty,\]
    也即函数 $x\mapsto\int_{\FR}f(x-y)g(y)\diff y$ 在 $\FR$ 上几乎处处有定义.

    (b) 由 (a) 中结论知
    \begin{align*}
        \int_{\FR}|f*g(x)|\diff x
        & =\int_{\FR}\left|\int_{\FR}f(x-y)g(y)\diff y\right|\diff x \\
        & \leq\int_{\FR}\int_{\FR}|f(x-y)g(y)|\diff y\diff x \\
        & =\|f\|_{L_1}\|g\|_{L_1}<\infty.
    \end{align*}
    故 $f*g\in L_1(\FR)$ 且 $\|f*g\|_{L_1}\leq\|f\|_{L_1}\|g\|_{L_1}$.

    (c) 由定义
    \begin{align*}
        f*f(x)
        & =\int_{\FR} f(x-y)f(y)\diff y=\int_{\FR} \mathbbm{1}_{[0,1]}(x-y)\mathbbm{1}_{[0,1]}(y)\diff y \\
        & =\int_0^1 \mathbbm{1}_{[0,1]}(x-y)\diff y.
    \end{align*}
    分类讨论可得, 当 $x<0$ 时, $f*f(x)=0$;
    当 $0\leq x\leq 1$ 时, $f*f(x)=x$;
    当 $1<x\leq 2$ 时, $f*f(x)=2-x$;
    当 $x>2$ 时, $f*f(x)=0$.
\end{proof}



\begin{exercise}
    在 $\FR$ 上考虑 Borel $\sigma$-代数和 Lebesgue 测度. 设 $1<p<\infty$ 且 $f \in L_{p}(0,+\infty)$. 定义
    \[
    F(x)=\frac{1}{x} \int_{0}^{x} f(t) \diff t, \quad \forall x>0.
    \]
    本题的目标是证明 Hardy 不等式:
    \begin{equation}
    \|F\|_{p} \leq \frac{p}{p-1}\|f\|_{p}, \forall f \in L_{p}(0,+\infty). \tag{$\star$}
    \end{equation}

    (a) 首先说明 $F$ 在 $(0,+\infty)$ 上的定义是合理的, 并且
    \[
    |x_1 F(x_1)-x_2 F(x_2)| \leq|x_1-x_2|^{\frac{1}{q}}\|f\|_{p}, \quad \forall x_{1}, x_{2}>0.
    \]
    这里 $q$ 是 $p$ 的共轭数. 并由此证明 $F$ 在 $(0,+\infty)$ 上连续, 故可测.

    (b) 假设 $f$ 是有紧支撑的连续函数且 $f \geq 0$. 证明 $F$ 在 $(0, \infty)$ 上连续可导且 有
    \[
    (p-1)\int_{0}^{+\infty} F(x)^{p}\diff x = p\int_{0}^{+\infty} F(x)^{p-1}f(x)\diff x.
    \]
    并由此导出公式 $(\star)$.

    (c) 证明公式 $(\star)$ 对所有的 $f \in L_{p}(0,+\infty)$ 成立.

    (d) 用反例说明当 $p=1$ 时, $(\star)$ 不成立, 即不存在任何常数 $C>0$, 使得
    \[
    \|F\|_{p} \leq C\|f\|_{p}, \quad \forall f \in L_{p}(0,+\infty).
    \]

    (e) 证明 $\frac{p}{p-1}$ 是使得 $(\star)$ 式成立的最优常数. 也就是说, 若有 $C>0$ 使得
    \[
    \|F\|_{p} \leq C\|f\|_{p}, \quad \forall f \in L_{p}(0,+\infty),
    \]
    则 $C\geq\frac{p}{p-1}$.

    提示: 考虑函数 $f(x)=x^{-\frac{1}{p}} \mathbbm{1}_{[1, n]}(x)$ 和极限
    \[
    \left\|F \mathbbm{1}_{[1, n]}(x)\right\|_{p} /\|f\|_{p}, n \rightarrow \infty.
    \]
\end{exercise}

\begin{proof}
    (a) 不妨设 $x_1\leq x_2$, 则
    \begin{align*}
        |x_1f(x_1)-x_2f(x_2)|
        & =\left|\int_0^{x_1}f(t)\diff t-\int_0^{x_2}f(t)\diff t\right| \\
        & =\left|\int_{x_1}^{x_2} f(t)\diff t\right| \\
        & \leq \int_{x_1}^{x_2}|f(t)|\diff t \\
        & \leq\biggl(\int_{x_1}^{x_2}|f(t)|^p\diff t\biggr)^{\frac{1}{p}}\biggl(\int_{x_1}^{x_2} 1 \diff t\biggr)^{\frac{1}{q}} \\
        & \leq |x_1-x_2|^{\frac{1}{q}}\|f\|_p.
    \end{align*}

    (b) 注意到 $(xF(x))'=f(x)$, 故由分部积分得
    \begin{align*}
        p\int_0^{+\infty} F(x)^{p-1}f(x)\diff x
        & =p\int_0^{+\infty} F(x)^{p-1}\diff (xF(x)) \\
        & =pxF(x)^p|_0^{+\infty}-p\int_0^{+\infty} xF(x)(p-1)F(x)^{p-2}\diff F(x) \\
        & =-p(p-1)\int_0^{+\infty}xF(x)^{p-1}\diff F(x) \\
        & =-(p-1)\int_0^{+\infty}x\diff F(x)^p \\
        & =-(p-1)xF(x)^p|_0^{+\infty}+(p-1)\int_0^{+\infty} F(x)^p\diff x \\
        & =(p-1)\int_0^{+\infty}F(x)^p\diff x.
    \end{align*}
\end{proof}



\begin{exercise}
    令 $2 \leq p<\infty$, 在本题中 $L_{p}(\mathbb{R})$ 简单记作 $L_{p}$.

    (a) 我们的第一个目标是证明 \textbf{Clarkson 不等式}:
    \[
    \left\|\frac{f+g}{2}\right\|_{p}^{p}+\left\|\frac{f-g}{2}\right\|_{p}^{p} \leq \frac{1}{2}\left(\|f\|_{p}^{p}+\|g\|_{p}^{p}\right), \quad \forall f, g \in L_{p}.
    \]
    \begin{enumerate}[(i)]
        \item 任取 $s, t \in[0,+\infty)$, 证明 $s^{p}+t^{p} \leq\left(s^{2}+t^{2}\right)^{\frac{p}{2}}$.
        \item 任取 $a, b \in \FR$, 证明
        \[
        \left|\frac{a+b}{2}\right|^{p}+\left|\frac{a-b}{2}\right|^{p} \leq \frac{1}{2}\left(|a|^{p}+|b|^{p}\right).
        \]
        \item 导出 Clarkson 不等式.
    \end{enumerate}

    (b) 设 $C$ 是 $L_{p}$ 空间中的非空闭凸集, 
    且 $f \in L_{p}$, 并记 $d=d(f, C)$. 我们的第二个目标是证明: 
    存在唯一的函数 $g_{0} \in C$, 使得 $d=\left\|f-g_{0}\right\|_{p}$.
    \begin{enumerate}[(i)]
        \item 解释为什么存在 $C$ 中的序列 $\left(g_{n}\right)_{n \geq 1}$, 使得
        \[
        \left\|f-g_{n}\right\|_{p}^{p} \leq d^{p}+\frac{1}{n}, \quad \forall n \in \mathbb{N}^{*}.
        \]
        \item 运用 Clarkson 不等式证明
        \[
        \left\|\frac{g_{n}-g_{m}}{2}\right\|_{p}^{p} \leq \frac{1}{2n}+\frac{1}{2m}, \quad \forall n, m \in \mathbb{N}^{*}.
        \]
        \item 导出: 存在函数 $g_{0} \in C$, 使得 $d(f, C)=\left\|f-g_{0}\right\|_{p}$.
        \item 证明这样的函数 $g_{0} \in C$ 是唯一的. 当证明了该命题后, 将 $g_{0}$ 记为 $P_{C}(f)$.
    \end{enumerate}

    (c) 最后我们的目标是证明映射 $P_{C}: L_{p} \rightarrow C$ 的连续性.
    \begin{enumerate}[(i)]
        \item 证明
        \[
        \left\|g-P_{C}(g)\right\|_{p} \leq\|f-g\|_{p}+\left\|f-P_{C}(f)\right\|_{p}, \quad \forall f, g \in L_{p}.
        \]
        \item 运用 Clarkson 不等式, 证明
        \[
        \left\|\frac{P_{C}(f)-P_{C}(g)}{2}\right\|_{p}^{p} \leq \frac{1}{2}\left\|f-P_{C}(g)\right\|_{p}^{p}-\frac{1}{2}\left\|f-P_{C}(f)\right\|_{p}^{p}, \quad \forall f, g \in L_{p}.
        \]
        \item 最后导出 $P_{C}$ 的连续性.
    \end{enumerate}
\end{exercise}

\begin{proof}
     (a)(i) 当 $t=0$ 时不等式显然成立, 当 $t\neq 0$ 时, 原不等式等价于
     \[\left(\frac{s}{t}\right)^p+1\leq\left(\left(\frac{s}{t}\right)^2+1\right)^{\frac{p}{2}}.\]
     令 $f(x)=(x^2+1)^{\frac{p}{2}}-x^p$, $x\geq 0$, 则
     $f'(x)=px(x^2+1)^{\frac{p}{2}-1}-px^{p-1}\geq 0$,
     故 $f(x)\geq f(0)=1$, 此蕴含所证不等式.

     (ii) 由 (i) 中结论和 $x\mapsto x^{\frac{p}{2}}$ 的凸性得
     \begin{align*}
         \left|\frac{a+b}{2}\right|^p+\left|\frac{a-b}{2}\right|^p
         & \leq\biggl(\left|\frac{a+b}{2}\right|^2+\left|\frac{a-b}{2}\right|^2\biggr)^{\frac{p}{2}}  =\biggl(\frac{a^2+b^2}{2}\biggr)^{\frac{p}{2}} \\
         & \leq\frac{1}{2}\left[(a^2)^{\frac{p}{2}}+(b^2)^{\frac{p}{2}}\right] =\frac{1}{2}(|a|^p+|b|^p).
     \end{align*}

     (iii) 由 (ii) 中结论可得
     \begin{align*}
        \left\|\frac{f+g}{2}\right\|_{p}^{p}+\left\|\frac{f-g}{2}\right\|_{p}^{p}
        & =\int_{\FR}\left|\frac{f(x)+g(x)}{2}\right|^p+\left|\frac{f(x)-g(x)}{2}\right|^p\diff x \\
        & \leq\frac{1}{2}\int_{\FR} |f(x)|^p+|g(x)|^p \diff x \\
        & =\frac{1}{2}(\|f\|_p^p+\|g\|_p^p).
     \end{align*} 

     (b)(i) 因 $d^p=d^p(f,C)=\inf\{\|f-g\|_p^p\mid g\in C\}$, 故由下确界的定义知
     对于任意 $n\in\FN^*$, 存在 $g_n\in C$, 使得
     \[\|f-g_n\|_p^p\leq d^p+\frac{1}{n}.\]

     (ii) 由于
     \[\|f-g_m\|_p^p\leq d^p+\frac{1}{m},\quad \|f-g_n\|_p^p\leq d^p+\frac{1}{n}.\]
     故结合 Clarkson 不等式得
     \[\left\| f-\frac{g_m+g_n}{2}\right\|_p^p+\left\|\frac{g_m-g_n}{2}\right\|_p^p\leq\frac{1}{2}\left(\|f-g_m\|_p^p+\|f-g_n\|_p^p\right)\leq d^p+\frac{1}{2m}+\frac{1}{2n}.\]
     因 $C$ 为凸集, 故 $\frac{g_m+g_n}{2}\in C$, 从而 $\left\|f-\frac{g_m+g_n}{2}\right\|_p^p\geq d^p$,
     代入上述不等式得
     \[\left\|\frac{g_m-g_n}{2}\right\|_p^p\leq\frac{1}{2m}+\frac{1}{2n}.\]

     (iii) 由 (ii) 知 $\|\frac{g_n-g_m}{2}\|_p^p\to 0$ ($m,n\to\infty$),
     故 $(g_n)_{n\geq 1}$ 为 $C$ 中 Cauchy 序列. 注意到 $C$ 为 Banach 空间 $L_p$
     的闭子集, 故 $C$ 完备, 从而 $(g_n)_{n\geq 1}$ 在 $C$ 中收敛, 记收敛值为 $g_0$.
     在 $\|f-g_n\|_p^p\leq d^p+\frac{1}{n}$ 两侧取极限, 即得 $\|f-g_0\|_p=d$.

     (iv) 假设存在 $g_1\in C$, 使得 $d(f,C)=\|f-g_1\|_p$, 则由 Clarkson 不等式得
     \begin{align*}
         d^p
         & =\frac{1}{2}\left(\|f-g_0\|_p^p+\|f-g_1\|_p^p\right) \\
         & \geq\left\|f-\frac{g_0+g_1}{2}\right\|_p^p+\left\|\frac{g_0-g_1}{2}\right\|_p^p\geq d^p+\left\|\frac{g_0-g_1}{2}\right\|_p^p.
     \end{align*}
     故 $g_0=g_1$, 唯一性得证.

     (c)(i) 由 Minkowski 不等式得
     \[\|g-P_C(g)\|_p\leq\|g-P_C(f)\|_p\leq\|f-g\|_p+\|f-P_C(f)\|_p.\]

     (ii) 由 Clarkson 不等式知
     \[\left\|f-\frac{P_C(f)+P_C(g)}{2}\right\|_p^p+\left\|\frac{P_C(f)-P_C(g)}{2}\right\|_p^p\leq\frac{1}{2}\left(\|f-P_C(f)\|_p^p+\|f-P_C(g)\|_p^p\right).\]
     结合 $\left\|f-\frac{P_C(f)+P_C(g)}{2}\right\|_p^p\geq \|f-P_C(f)\|_p^p$, 即得
     \[
         \left\|\frac{P_{C}(f)-P_{C}(g)}{2}\right\|_{p}^{p} \leq \frac{1}{2}\left\|f-P_{C}(g)\right\|_{p}^{p}-\frac{1}{2}\left\|f-P_{C}(f)\right\|_{p}^{p}.
     \]

     (iii) 
\end{proof}
% !TeX root = main.tex
% !TeX program = xelatex
\setcounter{chapter}{3}
\chapter{Hilbert空间}



\begin{exercise}
    设 $u:H\to K$ 是两个实内积空间之间的映射, 且有
    \[\|u(x)-u(y)\|=\|x-y\|,\quad\forall x,y\in H\;(\text{也就是说}, u\text{\ 是一个等距映射}).\]
    证明 $u-u(0)$ 是线性的.
\end{exercise}

\begin{proof}
记 $v=u-u(0)$, 则 $v(0)=0$, $\|v(x)-v(y)\|=\|x-y\|$ $(\forall x,y\in H)$,
即 $v$ 保距离(特别地, $v$ 还保范数), 将上式平方得 
\[\|v(x)\|^2+\|v(y)\|^2-2\langle v(x),v(y)\rangle=\|x\|^2+\|y\|^2-2\langle x,y\rangle.\]
故
\[\langle v(x),v(y)\rangle=\langle x,y\rangle.\]
因此 $v$ 保内积,下面证明 $v$ 是线性的:
\begin{itemize}
\item $v(x+y)=v(x)+v(y)$ $(\forall x,y\in H)$:
\begin{align*}
    &\langle v(x+y)-v(x)-v(y),v(x+y)-v(x)-v(y)\rangle\\
    &=\|v(x+y)\|^2+\|v(x)\|^2+\|v(y)\|^2-2\langle v(x+y),v(x)\rangle-2\langle v(x+y),v(y)\rangle+2\langle v(x),v(y)\rangle\\
    &=\|x+y\|^2+\|x\|^2+\|y\|^2-2\langle x+y,x\rangle-2\langle x+y,y\rangle+2\langle x,y\rangle\\
    &=\|x+y\|^2+\|x\|^2+\|y\|^2-2\|x+y\|^2+2\langle x,y\rangle\\
    &=0\\
    &\Rightarrow v(x+y)=v(x)+v(y).
\end{align*}
\item $v(\lambda x)=\lambda v(x)$ $(\forall\lambda\in\mathbb{R},x\in H)$:
\begin{align*}
    &\langle v(\lambda x)-\lambda v(x),v(\lambda x)-\lambda v(x)\rangle\\
    &=\|v(\lambda x)\|^2+\lambda^2\|v(x)\|^2-2\lambda\langle v(\lambda x),v(x)\rangle\\
    &=\|\lambda x\|^2+\lambda^2\|x\|^2-2\lambda\langle\lambda x,x\rangle\\
    &=0\\&\Rightarrow v(\lambda x)=\lambda v(x).
\end{align*}    
\end{itemize}
根据上面两点知 $v$ 是线性的.
\end{proof}

\begin{remark}
保距离+保原点 $\rightarrow$ 保内积.
\end{remark}




\begin{exercise}
    设 $A$ 是 $\ell_2$ 的子集, 其元素 $x=(x_n)_{n\geq 1}$ 满足 $|x_n|\leq\frac{1}{n}$, $n\geq 1$. 证明 $A$ 是紧集.
\end{exercise}

\begin{proof}
我们证明 $A$ 序列紧, 即证$A$中任意序列有收敛子列, 任取$A$中的序列$(x^{(m)})_{m\geq 1}$, 记
\[x^{(m)}=\left(x^{(m)}_1,x^{(m)}_2,\cdots,x^{(m)}_n,\cdots\right)\quad m=1,2,\cdots.\]

$\left(x_1^{(m)}\right)_{m\geq 1}$为有界序列,有收敛子列$\left(x_1^{(m_k^1)}\right)_{k\geq 1}$;

$\left(x_2^{(m_k^1)}\right)_{k\geq 1}$为有界序列,有收敛子列$\left(x_2^{(m_k^2)}\right)_{k\geq 1}$;

$\cdots\cdots$

$\left(x_n^{(m_k^{n-1})}\right)_{k\geq 1}$为有界序列,有收敛子列$\left(x_n^{(m_k^n)}\right)_{k\geq 1}$;

$\cdots\cdots$

根据对角线法选取指标列 $\left(m_k^k\right)_{k\geq 1}$, 
由此得到 $(x^{(m)})_{m\geq 1}$ 的子列 $(x^{(m_k^k)})_{k\geq 1}$, 
不妨将其简记为 $(x^{(m_k)})_{k\geq 1}$, 其每一个坐标分量都是收敛的, 
记 $(x^{(m_k)})_{k\geq 1}$ 依坐标收敛于 $x=(x_n)$, 且有 $|x_n|\leq\frac{1}{n}$, 
故 $x=(x_n)\in\ell_2$, 
下面证明 $(x^{(m_k)})_{k\geq 1}$ 依 $\ell_2$ 范数收敛到$x$: 
事实上, 对于每个$k\geq 1$, 都有 $|x_n^{(m_k)}|\leq\frac{1}{n}$, 故
\[|x_n^{(m_k)}-x_n|\leq\frac{2}{n}.\]
那么, 对任意 $\varepsilon>0$, 存在 $N\geq 1$, 使得
\[\sum_{n=N+1}^{\infty}|x_n^{(m_k)}-x_n|^2\leq\sum_{n=N+1}^{\infty}\frac{4}{n^2}<\varepsilon.\]
另一方面, 因为$(x^{(m_k)})_{k\geq 1}$ 依坐标收敛于$x=(x_n)$, 故存在$k_0\geq 1$, 使得当 $k\geq k_0$时, 有
\[\sum_{n=1}^N|x_n^{(m_k)}-x_n|^2\leq\sum_{n=1}^N\frac{\varepsilon}{N}=\varepsilon.\]
因此
\[\|x^{(m_k)}-x\|_{\ell_2}=\left(\sum_{n=1}^{\infty}|x_n^{(m_k)}-x_n|^2\right)^{\frac{1}{2}}<(2\varepsilon)^{\frac{1}{2}}.\]
即说明$(x^{(m_k)})_{k\geq 1}$依$\ell_2$范数收敛到$x$, 故$A$是紧集.
\end{proof}

\begin{remark}
从证明过程可以看出题目条件中的控制项 $\frac{1}{n}$ 可以换成任意 $(a_n)$, 只要其满足 $\sum_{n=1}^{\infty}a_n^2$ 收敛即可.
\end{remark}




\begin{exercise}
设 $E$ 和 $F$ 是内积空间 $H$ 的两个向量子空间. 证明存在常数 $\alpha\geq 0$ 使得
\[|\innerp{x}{y}|=\alpha\|x\|\|y\|,\quad\forall x\in E,\forall y\in F\]
的充分必要条件是或者 $\dim E=\dim F=1$, 或者 $\alpha=0$ (即 $E$ 与 $F$ 正交).
\end{exercise}

\begin{proof}
($\Leftarrow$)
充分性显然.

\necessary
已知存在 $\alpha\geq 0$, 使得 
$|\innerp{x}{y}|=\alpha \|x\|\cdot\|y\|$ $(\forall x\in E,y\in F)$.
当 $\alpha=0$时, 显然 $E$ 与 $F$ 正交;
当 $\alpha>0$时, 需证  $\dim E=\dim F=1$, 
反证法, 当 $\FK=\FR$ 时, 假设 $\dim F\geq 2$, 取 $F$ 中
两个不共线的单位向量 $e_1,e_2$, 取 $E$ 中一个单位向量 $e$,
则 $\innerp{e}{e_1}=\innerp{e}{e_2}=\alpha$, 且
\[\innerp{e}{e_1+e_2}=2\alpha=\alpha\|e_1+e_2\|,\]
因此 $\|e_1+e_2\|=\|e_1\|+\|e_2\|$, 于是 $e_1$ 与 $e_2$ 共线, 矛盾.
\end{proof}




\begin{exercise}
    设 $E$ 和 $F$ 是内积空间 $H$ 的两个向量子空间.
    假设 $E$ 和 $F$ 都不等于集合 $\{0\}$.
    定义 $E$ 和 $F$ 之间的夹角 $\theta$ 为
    \[\cos\theta=\sup\left\{\frac{|\innerp{x}{y}|}{\|x\|\|y\|}:x\in E, y\in F\right\},\quad\theta\in\left[0,\frac{\pi}{2}\right].\]
    证明: $\theta>0$ 当且仅当存在一个常数 $c>0$, 使得
    \[\|x+y\|^2\geq c(\|x\|^2+\|y\|^2),\quad\forall x\in E, y\in F.\]
\end{exercise}

\begin{proof}
    \necessary
    记 $m=\cos\theta$, 则 $\theta>0\iff 0\leq m<1$. 由夹角的定义知
    \[|\innerp{x}{y}|\leq m\|x\|\|y\|\leq\frac{m}{2}(\|x\|^2+\|y\|^2),\]
    故
    \begin{align*}
        \|x+y\|^2
        &=\|x\|^2+\|y\|^2+2\Re\innerp{x}{y}\geq \|x\|^2+\|y\|^2-2|\innerp{x}{y}| \\
        &\geq \|x\|^2+\|y\|^2-m\left(\|x\|^2+\|y\|^2\right) \\
        &=(1-m)\left(\|x\|^2+\|y\|^2\right).
    \end{align*}
    在上式中取 $c=1-m>0$ 即得所证.

    \sufficient
    注意到
    \begin{align*}
        \cos\theta
        & =\sup\biggl\{\frac{|\innerp{x}{y}|}{\|x\|\|y\|}\colon x\in E,y\in F\biggr\} \\
        & =\sup\{|\innerp{x}{y}|\colon \|x\|=\|y\|=1,x\in E,y\in F\}.
    \end{align*}
    故只需证 $\sup\{|\innerp{x}{y}|\colon \|x\|=\|y\|=1,x\in E,y\in F\}<1$.

    对任意 $x\in E$, $y\in F$, 且 $\|x\|=\|y\|=1$, 由条件 $\|x+y\|^2\geq c(\|x\|^2+\|y\|^2)$
    (不妨设 $0<c<1$) 得 $\Re\innerp{x}{y}\geq c-1$.
    当 $\|x\|=1$ 时, 亦有 $\|-x\|=1$, 故
    \[\Re\innerp{x}{y}=-\Re\innerp{-x}{y}\leq 1-c.\]
    因此
    \begin{align*}
        |\innerp{x}{y}|
        & =\sgn\innerp{x}{y}\cdot\innerp{x}{y} \\
        & =\innerp{\sgn\innerp{x}{y}\cdot x}{y}\quad(\text{real number})\\
        & =\Re\innerp{\sgn\innerp{x}{y}\cdot x}{y} \\
        & \leq 1-c.
    \end{align*}
    从而 $\sup\{|\innerp{x}{y}|\colon \|x\|=\|y\|=1,x\in E,y\in F\}\leq 1-c<1$, 由此即得 $\theta>0$.
\end{proof}



\begin{exercise}
    设 $H$ 是 Hilbert 空间, $(A_n)$ 是 $H$ 中递减的闭凸非空子集列. 
    任取 $x\in H$, 令 $d_n(x)=d(x,A_n)$ 且 $d(x)=\lim_{n\to\infty}d_n(x)$.
    \begin{enumerate}[(a)]
    \item 证明: 若对某一个 $x\in H$, 有 $d(x)<\infty$, 
    则对所有的 $x\in H$, $d(x)<\infty$. 我们在下面假设该命题成立, 
    并用 $A(x,\varepsilon,n)$ 表示中心在 $x$、半径为 $d(x)+\varepsilon$ 的闭球与 $A_n$ 的交集, 
    即 $A(x,\varepsilon,n)=A_n\cap\overline{B}(x,d(x)+\varepsilon)$.
    \item 证明
    \[\lim_{\varepsilon\to 0,n\to\infty}\diam(A(x,\varepsilon,n))=0.\]
    \item 证明所有 $A_n$ 的交集 $A$ 非空并且 $d(x)=d(x,A)$.
    \end{enumerate}
\end{exercise}

\begin{proof}
(a)假设存在 $x_0\in H,s.t.d(x_0)<\infty$, 记$y_n=P_{A_n}(x_0)$, 则
\[d_n(x_0)=d(x_0,A_n)=d(x_0,y_n)<\infty\]
故 $\forall x\in H$
\[d_n(x)=d(x,A_n)\leq d(x,y_n)\leq d(x,x_0)+d(x_0,y_n)<\infty\]
因此 $d(x)<\infty(\forall x\in H)$.

(b)仍记 $y_n=P_{A_n}(x)$, 则 $d_n(x)=d(x,y_n)\leq d(x)$, 因此$y_n\in A_n\cap\overline{B}(x,d(x)+\varepsilon)$. 由于$d_n(x)=d(x,y_n)\leq d(x)<\infty$, 故对于$\forall\varepsilon>0,\exists N(\varepsilon),s.t.\forall n\geq N(\varepsilon),d(x)<d_n(x)+\varepsilon$. 因此对$\forall z,w\in A_n\cap\overline{B}(x,d(x)+\varepsilon),\forall n\geq N(\varepsilon)$, 有:
\[d_n(x)\leq d(x,z)\leq d(x)+\varepsilon\leq d_n(x)+2\varepsilon\]
\[d_n(x)\leq d(x,w)\leq d(x)+\varepsilon\leq d_n(x)+2\varepsilon\]
由 $A_n$ 及 $\overline{B}(x,d(x)+\varepsilon)$ 均为凸集知 $\frac{z+w}{2}\in A_n\cap\overline{B}(x,d(x)+\varepsilon)$, 从而:
\[d_n(x)\leq d\left(x,\frac{z+w}{2}\right)\]
结合以上三式并根据平行四边形公式得:
\[\begin{split}
d_n(x)^2+\frac{1}{4}\|z-w\|^2
&\leq\left\|x-\frac{z+w}{2}\right\|^2+\left\|\frac{z-w}{2}\right\|^2\\
&=2\left(\left\|\frac{x-z}{2}\right\|^2+\left\|\frac{x-w}{2}\right\|^2\right)\\
&\leq (d_n(x)+2\varepsilon)^2\end{split}\]
即
\[\|z-w\|^2\leq 16\varepsilon(d_n(x)+\varepsilon)\]
对所有的 $z,w\in A(x,\varepsilon,n)$ 取上确界得:
\[\diam A(x,\varepsilon,n)\leq 16\varepsilon(d_n(x)+\varepsilon)(n\geq N(\varepsilon))\]
从而
\[\lim\limits_{\varepsilon\to 0,n\to\infty}\diam A(x,\varepsilon,n)=0\]

(c)取 $H$ 中集列 $\left(A(x,\frac{1}{n},n)\right)_{n\geq 1}$, 则
\begin{itemize}
\item $\left(A(x,\frac{1}{n},n)\right)_{n\geq 1}$是单调递减的闭集列
\item $\left(A(x,\frac{1}{n},n)\right)_{n\geq 1}$非空
\item $\lim_{n\to\infty}\diam\left(A(x,\frac{1}{n},n)\right)_{n\geq 1}=0$(在(b)中取$\varepsilon=1/n$知此式成立)
\end{itemize}
因为 $H$ 是完备的, 所以
\[\bigcap_{n\geq 1}A\left(x,\frac{1}{n},n\right)\text{是单点集}\]
又 $\left(A(x,\frac{1}{n},n)\right)\subset A_n$, 故$A=\bigcap_{n\geq 1}A_n$非空.

下证$d(x)=d(x,A)$:

因为 $d_n(x)=d(x,A_n)\leq d(x,A)$, 所以$d(x)\leq d(x,A)$, 假设$d(x)<d(x,A)$, 则存在$\delta,s.t.d(x,A_n)\leq d(x)<\delta<d(x,A)$, 同理可知:
\[\bigcap_{n\geq 1}A\left(x,\frac{\delta-d(x)}{n},n\right)=\bigcap_{n\geq 1}\left(A_n\cap\overline{B}\left(x,d(x)+\frac{\delta-d(x)}{n}\right)\right)\text{是单点集}\]
故
\[\bigcap_{n\geq 1}\left(A_n\cap\overline{B}\left(x,\delta\right)\right)=A\cap\overline{B}(x,\delta)\text{非空}\]
但事实是
\[A\cap\overline{B}(x,\delta)=\emptyset\]
故假设不成立, 即得$d(x)=d(x,A)$.
\end{proof}




\begin{exercise}
    设 $H$ 是内积空间, $x_n,x\in H$. 并假设
    \[\lim_{n\to\infty}\|x_n\|=\|x\|\quad\text{且}\quad \lim_{n\to\infty}\innerp{y}{x_n}=\innerp{y}{x}, \forall y\in H.\]
    证明 $\lim_{n\to\infty} \|x_n-x\|=0$.
\end{exercise}

\begin{proof}
    因 $\lim_{n\to\infty}\|x_n\|=\|x\|$,
    所以 $\lim_{n\to\infty}\langle x_n,x_n\rangle=\langle x,x\rangle\cdots(1)$

    因为 $\lim_{n\to\infty}\langle y,x_n\rangle=\langle y,x\rangle$,
    所以 $\lim_{n\to\infty}\langle x,x_n\rangle=\langle x,x\rangle\cdots(2)$

    两式相减得 $\lim_{n\to\infty}\langle x_n-x,x_n\rangle=0$,
    另外由第二式可得 $\lim_{n\to\infty}\langle x_n-x,x\rangle=0$.

    故
    \begin{align*}
        \lim_{n\to\infty}\|x_n-x\|^2
        & =\lim_{n\to\infty}\langle x_n-x,x_n-x\rangle\\
        & =\lim_{n\to\infty}\langle x_n-x,x_n\rangle-\lim_{n\to\infty}\langle x_n-x,x\rangle\\
        & =0-0=0.
    \end{align*}
    从而 $\lim_{n\to\infty}\|x_n-x\|=0$.
\end{proof}




\begin{exercise}
    设 $(x_n)$ 是 Hilbert 空间 $H$ 中的有界序列. 证明存在
    $(x_n)$ 的子序列 $(x_{n_k})$ 及 $x\in H$, 使得对任意 $y\in H$, 有
    $\lim_k\innerp{y}{x_{n_k}}=\innerp{y}{x}$.
\end{exercise}

\begin{proof}
(本题考查对角线选择法和Riesz表示定理)设$\|x_n\|\leq M(\forall n\in\mathbb{N^{*}})$,故对任意$m,n\in \mathbb{N^{*}},|\langle x_m,x_n\rangle|\leq\|x_m\|\cdot\|x_n\|\leq M^2$,考虑下面的一族有界内积序列
\[\begin{array}{ccccc}
\langle x_1,x_1\rangle&\langle x_1,x_2\rangle&\cdots&\langle x_1,x_n\rangle&\cdots\\
\langle x_2,x_1\rangle&\langle x_2,x_2\rangle&\cdots&\langle x_2,x_n\rangle&\cdots\\
\vdots&\vdots&\vdots&\vdots\\
\langle x_m,x_1\rangle&\langle x_m,x_2\rangle&\cdots&\langle x_m,x_n\rangle&\cdots\\
\vdots&\vdots&\vdots&\vdots\\
\end{array}\]

第一行序列$(\langle x_1,x_n\rangle)_{n\geq 1}$存在收敛子列$(\langle x_1,x_{n_k^1}\rangle)_{k\geq 1}$

第二行子列$(\langle x_2,x_{n_k^1}\rangle)_{k\geq 1}$存在收敛子列$(\langle x_2,x_{n_k^2}\rangle)_{k\geq 1}$

$\cdots\cdots$

第$m$行子列$(\langle x_m,x_{n_k^{m-1}}\rangle)_{k\geq 1}$
存在收敛子列$(\langle x_m,x_{n_k^m}\rangle)_{k\geq 1}$
 
依此下来, 并运用对角线选择法取出 $(x_n)$ 的子列 $(x_{n_k^k})_{k\geq 1}$,
不妨将其简记为 $(x_{n_k})_{k\geq 1}$.

设 $E=span((x_n)_{n\geq 1})$, 
则任意 $y\in E,(\langle y,x_{n_k}\rangle)_{k\geq 1}$ 收敛,
进一步容易验证对 $\forall y\in\overline{E},(\langle y,x_{n_k}\rangle)_{k\geq 1}$ 收敛,
最后任意 $y\in H$, 由正交分解定理得 $y=y_1+y_2,y_1\in\overline{E},y_2\in E^{\perp}$, 故
\[\langle y,x_{n_k}\rangle=\langle y_1,x_{n_k}\rangle+\langle y_2,x_{n_k}\rangle=\langle y_1,x_{n_k}\rangle.\]
因此对任意 $y\in H,\lim_{k\to\infty}\langle y,x_{n_k}\rangle$ 存在, 而且
\[\left|\lim_{k\to\infty}\langle y,x_{n_k}\rangle\right|\leq\limsup_{k\to\infty}|\langle y,x_{n_k}\rangle|\leq M\|y\|.\]
因此 $y\mapsto\lim_{k\to\infty}\langle y,x_{n_k}\rangle$ 是连续线性泛函, 
由 Riesz 表示定理知存在 $x\in H$ 使得 $\lim_{k\to\infty}\langle y,x_{n_k}\rangle=\langle y,x\rangle(\forall y\in H)$.
\end{proof}




\begin{exercise}
    设 $A$ 和 $B$ 都是 Hilbert 空间 $H$ 的非空闭凸子集,
    并设它们其中一个有界. 证明存在 $a\in A$ 和 $b\in B$,
    使得 $d(a,b)=d(A,B)$, 这里
    \[d(A,B)=\inf\{d(x,y)\mid x\in A, y\in B\}.\]
\end{exercise}

\begin{proof}
    不妨设$A$有界, 由距离的定义知对任意 $n\geq1,\exists x_n\in A$ 使得
    \[d(x_n,B)<d(A,B)+\frac{1}{n}.\]
    因为 $(x_n)_{n\geq1}$ 是有界序列, 故由上一题结论知存在
    $a\in H$ 及$(x_n)_{n\geq 1}$ 的子列 (不妨仍记为 $(x_n)_{n\geq 1}$) 使得
    \[\lim_{n\to\infty}\innerp{x_n}{y}=\innerp{a}{y},\quad\forall y\in H.\]
    下面证明 $a\in A$, 由投影的性质知
    \begin{align*}
        \|a-P_A(a)\|^2
        &=\langle a-P_A(a),a-P_A(a)\rangle\\
        &=\lim_{n\to\infty}\langle a-P_A(a),x_n-P_A(a)\rangle\quad(\text{注意到虚部的极限为}\ 0)\\
        &=\lim_{n\to\infty}\Re\innerp{a-P_A(a)}{x_n-P_A(a)}\leq 0.
    \end{align*}
    故 $a=P_A(a)$, 从而 $a\in A$, 又
    \[\begin{split}
    \|x_n-P_B(x_n)\|^2=
    &\|x_n-a+a-P_B(a)+P_B(a)-P_B(x_n)\|^2\\
    =&\|a-P_B(a)\|^2+\|x_n-a+P_B(a)-P_B(x_n)\|^2\\
    &+2\langle a-P_B(a),x_n-a\rangle+2\langle a-P_B(a),P_B(a)-P_B(x_n)\rangle\\
    \geq
    &\|a-P_B(a)\|^2+2\langle a-P_B(a),x_n-a\rangle\end{split}\]
    故
    \[\begin{split}
    (d(A,B))^2
    &\leq \|a-P_B(a)\|^2\leq\|x_n-P_B(x_n)\|^2-2\langle a-P_B(a),x_n-a\rangle\\
    &<\left(d(A,B)+\frac{1}{n}\right)^2-2\langle a-P_B(a),x_n-a\rangle\to(d(A,B))^2(n\to\infty)\end{split}\]
    因此 $\|a-P_B(a)\|=d(A,B)$, 记 $b=P_B(a)\in B$, 即得 $d(a,b)=d(A,B)$.
\end{proof}




\begin{exercise}
    将上一习题中的条件换成 $A$ 和 $B$ 无界, 但假设 $\|x\|$
    和 $\|y\|$ 都趋向 $\infty$ 时, 必有 $d(x,y)$ 趋向 $\infty$.
    证明结论仍然成立. 在 $\FR^2$ 中用反例说明若条件不符合假设时, 结论不成立.
\end{exercise}

\begin{proof}
反例: 取 $\mathbb{R}^2$ 中的区域 $A=\{(x,y)\mid xy\leq-1,x<0\}$,
$B=\{(x,y)|xy\geq1,x>0\}$, 
则 $d(A,B)=0$, 但是不存在 $a\in A,b\in B$ 使得 $d(a,b)=0$.
\end{proof}




\begin{exercise}
    (a) 设 $H$ 是 Hilbert 空间, $D_{n}=\{-1,1\}^{n}$. 证明
    \[
    \frac{1}{2^{n}} \sum_{(\varepsilon_{k}) \in D_{n}}\left\|\varepsilon_{1} x_{1}+\cdots+\varepsilon_{n} x_{n}\right\|^{2}=\left\|x_{1}\right\|^{2}+\cdots+\left\|x_{n}\right\|^{2}, \quad \forall x_1,\cdots,x_n \in H.
    \]

    (b) 设 $(X,\|\cdot\|)$ 是 Banach 空间, 并假设有一个 $X$ 上的内积范数 $|\cdot|$ 
    等价于 $\|\cdot\|$. 证明存在正常数 $a$ 和 $b$, 使得
    \[
    a\sum_{k=1}^{n}\left\|x_{k}\right\|^{2}\leq\frac{1}{2^{n}} \sum_{\left(\varepsilon_{k}\right) \in D_{n}}\left\|\sum_{k=1}^{n} \varepsilon_{k} x_{k}\right\|^{2} \leq b \sum_{k=1}^{n}\left\|x_{k}\right\|^{2}, \quad \forall x_{1}, \cdots, x_{n} \in X.
    \]

    (c) 设 $1 \leq p \neq 2 \leq \infty$, 证明空间 $c_{0}, \ell_{p}$ 和 $L_{p}(0,1)$ 没有等价的内积范数.
\end{exercise}

\begin{proof}
    (a) 原等式等价于
    \[\sum_{(\varepsilon_k)\in D_n}\|\varepsilon_1x_1+\cdots+\varepsilon_nx_n\|^2=2^n\left(\|x_1\|^2+\cdots+\|x_n\|^2\right).\]
    上式左边
    \begin{align*}
        \mathrm{RHS}
        &=\sum_{(\varepsilon_k)\in D_n}\innerp{\sum_{i=1}^n\varepsilon_ix_i}{\sum_{i=1}^n\varepsilon_ix_i} \\
        &=\sum_{(\varepsilon_k)\in D_n}\sum_{i=1}^n\sum_{j=1}^n \varepsilon_i\varepsilon_j\innerp{x_i}{x_j} \\
        &=\sum_{(\varepsilon_k)\in D_n}\sum_{i=1}^n\varepsilon_i^2\innerp{x_i}{x_i}+\sum_{(\varepsilon_k)\in D_n}\sum_{\substack{1\leq i,j\leq n \\ i\neq j}}\varepsilon_i\varepsilon_j\innerp{x_i}{x_j} \\
        &=\sum_{(\varepsilon_k)\in D_n}\sum_{i=1}^n\innerp{x_i}{x_i}+\sum_{(\varepsilon_k)\in D_n}\sum_{\substack{1\leq i,j\leq n \\ i\neq j}}\varepsilon_i\varepsilon_j\innerp{x_i}{x_j} \\
        &=2^n\sum_{i=1}^n \|x_i\|^2+\sum_{(\varepsilon_k)\in D_n}\sum_{\substack{1\leq i,j\leq n \\ i\neq j}}\varepsilon_i\varepsilon_j\innerp{x_i}{x_j}.
    \end{align*}
    观察最后一项中 $\innerp{x_i}{x_j}$ 的系数:
    $\varepsilon_i=\varepsilon_j=1$ 有 $2^{n-2}$ 项,
    $\varepsilon_i=\varepsilon_j=-1$ 有 $2^{n-2}$ 项,
    $\varepsilon_i=1$, $\varepsilon_j=-1$ 有 $2^{n-2}$ 项,
    $\varepsilon_i=-1$, $\varepsilon_j=1$ 有 $2^{n-2}$ 项,
    因此 
    \[\innerp{x_i}{x_j}\text{\ 的系数\ }=2\cdot 2^{n-2}-2\cdot 2^{n-2}=0.\]
    于是即证所需.

    (b) 由于 $|\cdot|$ 等价于 $\|\cdot\|$, 故存在正常数 $C_1$ 和 $C_2$
    使得 $C_1\|\cdot\|\leq |\cdot|\leq C_2\|\cdot\|$. 由 (a) 知
    \[\frac{1}{2^n}\sum_{(\varepsilon_k)\in D_n}\left|\sum_{k=1}^n \varepsilon_kx_k\right|^2=\sum_{k=1}^n |x_k|^2.\]
    故
    \[\sum_{k=1}^n C_2^2\|x_k\|^2\geq\sum_{k=1}^n |x_k|^2=\frac{1}{2^n}\sum_{(\varepsilon_k)\in D_n}\left|\sum_{k=1}^n \varepsilon_kx_k\right|^2\geq\frac{1}{2^n}\sum_{(\varepsilon_k)\in D_n}C_1^2\left\|\sum_{k=1}^n \varepsilon_kx_k\right\|^2,\]
    因此
    \[\frac{1}{2^n}\sum_{(\varepsilon_k)\in D_n}\left\|\sum_{k=1}^n \varepsilon_kx_k\right\|^2\leq\left(\frac{C_2}{C_1}\right)^2\sum_{k=1}^n \|x_k\|^2.\]
    故取 $b=\left(\frac{c_2}{c_1}\right)^2$ 即得
    \[\frac{1}{2^n}\sum_{(\varepsilon_k)\in D_n}\left\|\sum_{k=1}^n \varepsilon_kx_k\right\|^2\leq b\sum_{k=1}^n \|x_k\|^2.\]
    同理取 $a=\left(\frac{C_1}{C_2}\right)^2$ 可得左半边不等式.
\end{proof}




\begin{exercise}
    设 $\left(C_{n}\right)$ 是 Hilbert 空间 $H$ 中的一个递增的非空闭凸子集列, $C$ 是所有 $C_n$ 的并集的闭包. 证明
    \[
    P_{C}(x)=\lim_{n\to\infty} P_{C_n}(x), \quad \forall x \in H.
    \]
\end{exercise}

\begin{proof}
    首先容易验证 $C$ 是闭凸集, 从而 $P_C(x)$ 是有定义的, 
    接下来证明 $P_C(x)=\lim_{n\to\infty}P_{C_n}(x)$ $(\forall x\in H)$, 分几步进行:

    $d(x,C)=\lim_{n\to\infty}d(x,C_n)$:
    因为对于每个 $n$, $d(x,C)\leq d(x,C_n)$,
    故 $d(x,C)\leq \lim_{n\to\infty}d(x,C_n)$,
    假设 $d(x,C)<\lim_{n\to\infty}d(x,C_n)$,
    则存在 $y\in C$, 使得 $d(x,y)<\lim_{n\to\infty}d(x,C_n)$,
    不妨设 $y\in\bigcup_{n=1}^{\infty}C_n$, 也就是说存在 $n_0$ 使得 $y\in C_{n_0}$,
    从而 $d(x,y)\geq d(x,C_{n_0})\geq\lim_{n\to\infty}d(x,C_n)$, 矛盾, 故 $d(x,C)=\lim_{n\to\infty}d(x,C_n)$.
    
    $(P_{C_n}(x))_{n\geq 1}$收敛:
    因为 $(d(x,C_n))_{n\geq 1}$ 单调递减趋于 $d(x,C)$,
    故对 $\forall\varepsilon>0$, 存在 $N\geq 1$,
    使得当 $n>N$ 时, $d(x,C_n)<d(x,C)+\varepsilon$,
    故 $\forall m,n>N$, 有
    \begin{align*}
        4(d(x,C)+\varepsilon)^2
        &\geq 2(\|x-P_{C_n}(x)\|^2+\|x-P_{C_m}(x)\|^2) \\
        &=4\left\|x-\frac{P_{C_n}(x)+P_{C_m}(x)}{2}\right\|^2+\|P_{C_n}(x)-P_{C_m}(x)\|^2 \\
        &\geq 4d(x,C)^2+\|P_{C_n}(x)-P_{C_m}(x)\|^2.
    \end{align*}
    由上式知 $(P_{C_n}(x))_{n\geq 1}$ 是 Cauchy 序列, 
    由 $C$ 的完备性知其在 $C$ 中收敛, 记为 $\lim_{n\to\infty}P_{C_n}(x)=y\in C$.
    
    $y=P_C(x)$: 
    对于任意 $\forall\varepsilon>0$,
    存在 $N\geq 1$, 使得对 $\forall n>N$,
    有 $d(y,P_{C_n}(x))<\varepsilon$, $d(x,C_n)<d(x,C)+\varepsilon$, 故
    \[d(x,y)\leq d(y,P_{C_n}(x))+d(x,C_n)<d(x,C)+2\varepsilon,\] 
    由 $\varepsilon$ 的任意性知 $d(x,y)\leq d(x,C)$, 又因为 $y\in C$, 
    故 $d(x,y)=d(x,C)$, 由投影的唯一性知 $y=P_C(x)$, 证毕.
\end{proof}




\begin{exercise}
    设 $H$ 是内积空间. $(x_{1}, \cdots, x_{n})$ 是 $H$ 中的任一向量组,
    称矩阵 $(\innerp{x_i}{x_j})_{1\leq i,j\leq n}$ 
    的行列式为向量组 $\left(x_{1},\cdots, x_{n}\right)$ 的 Gram 行列式, 记作 $G(x_{1},\cdots,x_{n})$.
    
    (a) 证明 $G(x_{1},\cdots,x_{n})\geq 0$; 
    且 $G\left(x_{1},\cdots,x_{n}\right)>0$ 当且仅当向量组 $\left(x_{1}, \cdots, x_{n}\right)$ 线性独立.
    
    (b) 假设向量组 $\left(x_{1},\cdots,x_{n}\right)$ 线性独立. 
    令 $E=\operatorname{span}\left(x_{1},\cdots, x_{n}\right)$. 证明
    \[
        d(x, E)^{2}=\frac{G\left(x, x_{1}, x_{2}, \cdots, x_{n}\right)}{G\left(x_{1}, x_{2}, \cdots, x_{n}\right)}, \quad \forall x\in H.
    \]
\end{exercise}

\begin{proof}
    (参考《高等代数与解析几何》 陈志杰习题 6.3.13 及 6.4.6)

    (a) 设 $W=\operatorname{span}\{x_1,\cdots,x_n\}$ 且 $\dim W=k$,
    取 $W$ 的规范正交基 $(e_i)_{1\leq i\leq k}$. 由于
    \[x_i=\sum_{m=1}^k \innerp{x_i}{e_m}e_m,\quad x_j=\sum_{k=1}^m \innerp{x_j}{e_m}e_m,\]
    故
    \begin{align*}
        \innerp{x_i}{x_j}
        &=\innerp{\sum_{m=1}^k \innerp{x_i}{e_m}e_m}{\sum_{m=1}^k \innerp{x_j}{e_m}e_m} \\
        &=\sum_{m=1}^k \innerp{x_i}{e_m}\innerp{x_j}{e_m}=\sum_{k=1}^m \innerp{x_i}{e_m}\overline{\innerp{e_m}{x_j}}.
    \end{align*}
    记
    \[M=\begin{pmatrix}
        \innerp{x_1}{e_1} & \cdots & \innerp{x_1}{e_k} \\
        \vdots            &        & \vdots            \\
        \innerp{x_n}{e_1} & \cdots & \innerp{x_n}{e_k}
    \end{pmatrix}_{n\times k}.\]
    则
    \[M^\T
    =\begin{pmatrix}
        \overline{\innerp{e_1}{x_1}} & \cdots & \overline{\innerp{e_1}{x_n}} \\
        \vdots & & \vdots \\
        \overline{\innerp{e_k}{x_1}} & \cdots & \overline{\innerp{e_k}{x_n}} 
    \end{pmatrix}_{k\times n},\]
    且 $(\innerp{x_i}{x_j})_{1\leq i,j\leq n}=MM^\T$, 从而 $G(x_1,\cdots,x_n)=\det(MM^\T)$.

    若 $k<n$, 则 $\rank(M)\leq k<n$, 故 $\rank(\innerp{x_i}{x_j})<n$, 故 $|G(x_1,\cdots,x_n)|=0$.

    若 $k=n$, 则 $x_1,\cdots, x_n$ 线性无关,
    即关于 $\lambda_1,\cdots,\lambda_n$ 的方程
    \[\lambda_1 x_1+\cdots+\lambda_nx_n=0\]
    只有零解. 考虑关于 $\lambda_1,\cdots,\lambda_n$ 的齐次线性方程组
    \[\begin{cases}
        \lambda_1\innerp{x_1}{e_1}+\cdots+\lambda_n\innerp{x_n}{e_1}=0 \\
        \cdots \\
        \lambda_1\innerp{x_1}{e_n}+\cdots+\lambda_n\innerp{x_n}{e_n}=0.
    \end{cases}\]
    上述方程组的系数矩阵即为 $M^\T$, 将上述方程组的第 $i$ ($1\leq i\leq n$) 个方程乘以 $e_i$ 并求和即得
    \[\lambda_1x_1+\cdots+\lambda_nx_n=0,\]
    于是 $\lambda_1=\cdots=\lambda_n=0$, 因此 $\det(M)\neq 0$,
    从而 
    \[G(x_1,\cdots,x_n)=\det(MM^\T)=\det(M)\det(M^\T)=(\det(M))^2>0.\]

    (b) 略.
\end{proof}




\begin{exercise}
    设 $E=C([0,1])$ 上装备有如下的内积
    \[
    \innerp{f}{g}=\int_{0}^{1} f(t) \overline{g(t)}\diff t.
    \]
    并设 $E_{0}$ 表示在 $[0,1]$ 上积分为 $0$ 的函数组成的 $E$ 的向量子空间. 考虑 $E$ 的 向量子空间:
    \[
    H=\{f \in E: f(1)=0\} \text {\ 且\ } H_{0}=E_{0} \cap H.
    \]

    (a) 验证 $H_{0}$ 是 $H$ 的闭的真向量子空间.

    (b) 设 $h(t)=t-\frac{1}{2}, t \in[0,1]$. 证明
    \begin{enumerate}[(i)]
    \item $E=\operatorname{span}(H, h)$ 且有 $E_{0}=\operatorname{span}\left(H_{0}, h\right)$;
    \item $h$ 属于 $H_{0}$ 在 $E$ 中的闭包.
    \end{enumerate}

    (c) 证明 $H_{0}^{\perp}=\{0\}$. 解释所得结果蕴含的意义.
\end{exercise}

\begin{proof}
    (a) 任取一列 $(f_n)\subset E_0$ 且 $f_n\rightarrow f$,
    则 $\int_0^1 f_n(t)\diff t=0$, $\int_0^1|f_n(t)-f(t)|^2\diff t\rightarrow0$,
    故 $f_n(t)-f(t)=0\almosteverywhere$ $(n\to\infty)$,
    因此 $\int_0^1 f(t)\diff t=0$, 也即 $f\in E_0$,
    从而说明 $E_0$ 是闭子空间, 故 $H_0=E_0\cap H$ 是 $H$ 的闭子空间.
    取 $f(t)=1-t$, 显然 $f(t)\in H$, 但是 $f(t)\notin H_0$, 故 $H_0$ 是 $H$ 的真子空间.

    (b)(i) 对于 $\forall f\in E$,
    令 $g(t)=f(t)-2f(1)h(t)$,
    则 $g(1)=0$, 故 $g\in H$, 所以 $E=\operatorname{span}(H,h)$.

    对于 $\forall f\in E_0$,
    令 $g(t)=f(t)-2f(1)h(t)$, 则 $g(1)=0$,
    $\int_0^1 g(t)\diff t=0$, 即 $g\in H_0$, 所以 $E_0=\operatorname{span}(H_0,h)$.

    (ii)取 $g_n(t)=\sin2\pi nt$, 则 $g_n\in H_0$,由于$h$关于点$(1/2,0)$对称,
    故 $h$ 的 Fourier 展开式中只含有形如 $g_n$ 的项, 因此 $h\in\bar{H}_0$
\end{proof}



\begin{exercise}
    仍设 $E$ 为上一习题中的内积空间, 并令 $0<c<1$. 记
\[
F=\left\{f \in E:f|_{[0, c]}=0\right\}.
\]

(a) 验证 $F$ 是 $E$ 的闭的真向量子空间.

(b) 证明 $F \oplus F^{\perp} \neq E$. 解释所得结果蕴含的意义.
\end{exercise}
% 14.
% \begin{proof}
% (a)任取一列$(f_n)\subset F$且$f_n\rightarrow f$,则$f_n|_{[0,c]}=0,\int_0^1|f_n(t)-f(t)|^2\diff t\rightarrow 0$\\
% 故$\int_0^c|f_n(t)-f(t)|^2\diff t=\int_0^c|f(t)|^2\rightarrow 0\Rightarrow f|_{[0,c]}=0$, 因此$f\in F$,从而$F$是$E$的闭子空间, 而$F$是$E$的真子空间是显然的.

% (b)要证明$F\oplus F^{\perp}\not= E$, 即需要证明存在$E$中元素使其没有基于$F$的正交分解.

% 取$f\equiv 1\in E$, 对任意$g\in F,\exists\delta>0,s.t.\forall t\in [c,c+\delta],(f-g)(t)>0$(不妨设$c+\delta<1$), 取实值函数$h(t)$满足:
% \[h(t)
% \begin{cases}
% =0&t\in [0,c]\cup [c+\delta,1]\\
% >0&t\in (c,c+\delta)
% \end{cases}\]
% 则
% \[\int_0^1(f-g)(t)\cdot h(t)\diff t=\int_c^{c+\delta}(f-g)(t)\cdot h(t)\diff t>0\]
% 故 $f-g\notin F^{\perp}$, 从而证明了$F\oplus F^{\bot}\not= E$, 此结论说明$E$不是Hilbert空间.
% \end{proof}



\begin{exercise}
    (a) 设 $E$ 和 $F$ 是 Hilbert 空间 $H$ 的两个正交向量子空间. 证明 $E+F$ 是闭的当且仅当 $E$ 和 $F$ 都是闭的.
    
    (b) $(e_{n})$ 表示 $\ell_{2}$ 中的标准正交基. 
    设 $E$ 是 $\left\{e_{2 n}: n \geq 1\right\}$ 的线性扩张的闭包, 
    而 $F$ 是 $\left\{e_{2 n}+\frac{1}{n} e_{2 n+1}: n \geq 1\right\}$ 的线性扩张的闭包. 
    证明 $E\cap F=\{0\}$ 并 且 $E+F$ 在 $\ell_{2}$ 中不是闭的.
\end{exercise}

\begin{proof}
    (a) 由 $E$ 与 $F$ 正交知 $E+F=E\oplus F$, 即任取 $z\in E+F$,
    存在唯一的 $x\in E$ 和 $y\in F$, 使得 $z=x+y$.

    \necessary
    任取 $E$ 中收敛列 $(x_n)_{n\geq 1}$, 设 $x_n\to x\in E+F$,
    即 $\lim_{n\to\infty}\|x_n-x\|=0$. 由于 $x\in E+F$, 故存在 $x'\in E$, $x''\in F$,
    使得 $x=x'+x''$, 那么
    \begin{align*}
        \lim_{n\to\infty}\|x_n-x\|^2
        & =\lim_{n\to\infty}\|x_n-x'-x''\|^2 \\
        & =\lim_{n\to\infty}\|x_n-x'\|^2+\|x''\|^2-2\Re\innerp{x_n-x'}{x''} \\
        & =\lim_{n\to\infty}\|x_n-x'\|^2+\|x''\|^2=0.
    \end{align*}
    故必有 $x''=0$, 从而 $x=x'\in E$, 因此 $E$ 为闭集. 同理可证 $F$ 为闭集.

    \sufficient
    任取 $E+F$ 中 Cauchy 序列 $(z_n)_{n\geq 1}$, 设 $z_n=x_n+y_n$, 其中 $x_n\in E$, $y_n\in F$, 则
    \begin{align*}
        \|z_m-z_n\|^2
        & =\|x_m+y_m-x_n-y_n\|^2 \\
        & =\|x_m-x_n\|^2+\|y_m-y_n\|^2+2\Re\innerp{x_m-x_n}{y_m-y_n} \\
        & =\|x_m-x_n\|^2+\|y_m-y_n\|^2\to 0\quad (m,n\to\infty).
    \end{align*}
    故 $(x_n)_{n\geq 1}$ 和 $(y_n)_{n\geq 1}$ 分别为 $E$ 和 $F$ 中的 Cauchy 序列,
    而 $E,F$ 皆完备, 故设 $x_n\to x\in E$, $y_n\to y\in F$.
    令 $z=x+y\in E+F$, 则当 $n\to\infty$ 时
    \[\|z_n-z\|^2=\|x_n+y_n-x-y\|^2=\|x_n-x\|^2+\|y_n-y\|^2\to 0.\]
    即 $z_n\to z\in E+F$, 故 $E+F$ 完备, 从而为闭集.

    (b) 任取 $x\in E\cap F$, 由于 $E$ 是 Hilbert 空间, 且
    $\{e_{2n}:n\geq 1\}$ 是 $E$ 的一组规范正交基, 故存在唯一的系数列 $(x_n)_{n\geq 1}$,
    使得 $x=\sum_{n=1}^{\infty}x_n e_{2n}$. 类似地, $F$ 为 Hilbert 空间, 且规范化后的
    $\{\frac{n}{\sqrt{n^2+1}}(e_{2n}+\frac{1}{n}e_{2n+1}):n\geq 1\}$ 是 $F$ 的一组规范正交基,
    故存在唯一的系数列 $(y_n)_{n\geq 1}$, 
    使得 $x=\sum_{n=1}^{\infty}y_n\frac{n}{\sqrt{n^2+1}}(e_{2n}+\frac{1}{n}e_{2n+1})$.
    于是对任意 $n\geq 1$, 有
    \[x_n=y_n\cdot\frac{n}{\sqrt{n^2+1}},\quad\frac{y_n}{\sqrt{n^2+1}}=0\Longrightarrow x_n=y_n=0.\]
    故 $x=0$, 因此 $E\cap F=\{0\}$. 下证 $E+F$ 不是闭集,
    取 $x^{(m)}=\sum_{n=1}^m -e_{2n}\in E$, $y^{(m)}=\sum_{n=1}^m (e_{2n}+\frac{1}{n}e_{2n+1})\in F$,
    则
    \[x^{(m)}+y^{(m)}=\sum_{n=1}^m \frac{1}{n}e_{2n+1}\in E+F\]
    且
    \[x^{(m)}+y^{(m)}\xrightarrow{\ell_2}\sum_{n=1}^{\infty}\frac{1}{n}e_{2n+1}.\]
    但 $\sum_{n=1}^{\infty}\frac{1}{n}e_{2n+1}\notin E+F$, 事实上, 若存在
    $x=\sum_{n=1}^{\infty}x_n e_{2n}\in E$ 和 $y=\sum_{n=1}^{\infty}y_n\frac{n}{\sqrt{n^2+1}}(e_{2n}+\frac{1}{n}e_{2n+1})\in F$,
    使得 $x+y=\sum_{n=1}^{\infty}\frac{1}{n}e_{2n+1}$, 则
    \[x_n+\frac{ny_n}{\sqrt{n^2+1}}=0,\quad\frac{y_n}{\sqrt{n^2+1}}=\frac{1}{n}\Longrightarrow x_n=-1,y_n=\frac{\sqrt{n^2+1}}{n}.\]
    但此时 $x=\sum_{n=1}^{\infty}-e_{2n}\notin\ell_2$, 矛盾.
\end{proof}



\begin{exercise}
    设 $H$ 是 Hilbert 空间, $E$ 是 $H$ 的非零的闭向量子空间. 
    设 $P$ 是 $H$ 到 $E$ 的投 影 (投影意味着 $P$ 是 $H$ 上的线性算子且满足 $P^{2}=P$ ). 证明以下命题等价:
    
    (a) $P=P_{E}$.
    
    (b) $\|P\|=1$.
    
    (c) $|\langle x, P(x)\rangle| \leq\|x\|^{2}, \forall x \in H$.
\end{exercise}

\begin{proof}
    (a) $\Rightarrow$ (b) 显然.

    (b) $\Rightarrow$ (c)
    由 $\|P\|=1$ 知 $\|P(x)\|\leq\|x\|$, 故 $|\langle x,P(x)\rangle|\leq\|x\|\cdot\|P(x)\|\leq\|x\|^2$.

    (c) $\Rightarrow$ (a)
    分三步进行
    \begin{itemize}
    \item $\forall y\in E,P(y)=y$: 任意 $y\in E,\exists x\in H,s.t.P(x)=y$,故$P(y)=P(P(x))=P(x)=y$.
    \item $\forall y\in E^{\perp},P(y)=0$: 根据$P$的线性性知:
    \[P(y+nP(y))=(n+1)P(y).\]
    故结合条件得
    \[|\langle y+nP(y),(n+1)P(y)\rangle|\leq\|y+nP(y)\|^2=\|y\|^2+n^2\|P(y)\|^2.\]
    又
    \[|\langle y+nP(y),(n+1)P(y)\rangle|=|\langle nP(y),(n+1)P(y)\rangle|=(n^2+n)\|P(y)\|^2.\]
    结合两式得
    \[\|P(y)\|^2\leq\frac{1}{n}\|y\|^2.\]
    上式对于任意正整数 $n$ 成立, 故只能有 $P(y)=0$.
    \item $\forall x\in H,P(x)=P_E(x)$: 根据前两步的结果可知
    \[P(x)=P(P_E(x))+P(x-P_E(x))=P_E(x).\qedhere\]
    \end{itemize}
\end{proof}




\begin{exercise}
    设 $H$ 是 Hilbert 空间, $E$ 是 $H$ 的向量子空间. 
    设 $F$ 为赋范空间, $u: E \rightarrow F$ 是连续线性映射. 
    证明 $u$ 有连续的线性延拓 $\widehat{u}: H \rightarrow F$, 且 $\|\widehat{u}\|=\|u\|$.
\end{exercise}

\begin{proof}
    假设 $F$ 为 Banach 空间, 由定理 3.2.13 知连续线性映射 $u:E\to F$
    可以唯一地扩展为连续线性映射 $\tilde{u}:\closure{E}\to F$ 且 $\|\tilde{u}\|=\|u\|$.
    对任意 $x\in H$, 定义
    \[\hat{u}(x):=\tilde{u}\left(P_{\closure{E}}(x)\right).\]
    若 $x\in E$, 则 $\hat{u}(x)=\tilde{u}(x)=u(x)$, 故 $\hat{u}$
    为 $u$ 的扩展映射.

    $\hat{u}$ 为连续映射: 对任意 $x,y\in H$ 和 $\lambda\in\FK$,
    由 $\tilde{u}$ 和 $P_{\closure{E}}$ 的线性性得
    \begin{align*}
        \hat{u}(\lambda x+y)
        & =\tilde{u}(P_{\closure{E}}(\lambda x+y)) \\
        & =\tilde{u}(\lambda P_{\closure{E}}(x)+P_{\closure{E}}(y)) \\
        & =\lambda\tilde{u}(P_{\closure{E}}(x))+\tilde{u}(P_{\closure{E}}(y)) \\
        & =\lambda\hat{u}(x)+\hat{u}(y).
    \end{align*}

    $\hat{u}$ 为有界映射: 对任意 $x\in H$, 有
    \[\|\hat{u}(x)\|=\|\tilde{u}(P_{\closure{E}}(x))\|\leq\|\tilde{u}\|\|P_{\closure{E}}\|\|x\|=\|u\|\|x\|.\]
    故 $\|\tilde{u}\|\leq\|u\|$, 又
    \[\|u\|=\sup_{x\in E,x\neq 0}\frac{\|u(x)\|}{\|x\|}\leq\sup_{x\in H,x\neq 0}\frac{\|\hat{u}(x)\|}{\|x\|}=\|\hat{u}\|,\]
    所以 $\|\hat{u}\|=\|u\|$.
\end{proof}



\begin{exercise}
    设 $[0,1]$ 上赋予 Lebesgue 测度, $H=L_{2}(0,1)$. 并假设 $K \in L_{2}([0,1] \times[0,1])$. 我们定义
    \[
    T_{K}(f)(x)=\int_{0}^{1} K(x, y) f(y)\diff y, \quad f \in H, x \in[0,1].
    \]

    (a) 证明 $T_{K}(f)$ 在 $[0,1]$ 上几乎处处有定义.

    (b) 证明 $T_{K} \in\mathcal{B}(H)$ 且
    \[
    \left\|T_{K}\right\|\leq\|K\|_{L_{2}([0,1] \times[0,1])}.
    \]

    (c) 设 $\widetilde{K}(x, y)=\overline{K(y, x)}$, $x,y\in[0,1]$. 证明 $T_{K}^{*}=T_{\tilde{K}}$.

    (d) 定义
    \[
    T(f)(x)=\int_{0}^{x} f(1-y)\diff y, \quad f \in H, x \in[0,1].
    \]
    证明 $T\in\mathcal{B}(H)$ 且有 $T^{*}=T$.
    最后给出 $T$ 的非零特征值并证明相应的特征子空间两两正交.
\end{exercise}

\begin{proof}
    (a)任意固定 $x$, 将 $K(x,y)$ 看作关于 $y$ 的一元函数, 由 Cauchy-Schwarz 不等式得:
    \begin{align*}
        |\innerp{K}{\bar{f}}|^2
        & =\left|\int_0^1K(x,y)f(y)\diff y\right|^2 \\
        & \leq\int_0^1|K(x,y)|^2\diff y\cdot\int_0^1|f(y)|^2\diff y.
    \end{align*}
    因为 $f\in L_2(0,1)$, 所以
    \[\int_0^1|f(y)|^2\diff y<\infty.\]
    因为 $K\in L_2([0,1]\times [0,1])$, 所以
    \[\int_0^1\int_0^1|K(x,y)|^2\diff y\diff x<\infty\Rightarrow\int_0^1|K(x,y)|^2\diff y<\infty, \almosteverywhere\]
    结合以上三式得
    \[\left|\int_0^1K(x,y)f(y)\diff y\right|^2<\infty, \almosteverywhere\]
    也就证明了 $T_K(f)$ 在 $[0,1]$ 上几乎处处有定义.

    (b) 由 (a) 中结论知: $\forall f\in H$, $T_K(f)\in H$.

    首先, $T_K$ 为线性算子. 对于任意 $f,g\in H$ 和 $\lambda\in\mathbb{K}$, 有
    \[T_K(\lambda f+g)=\int_0^1K(x,y)(\lambda f(y)+g(y))\diff y=\lambda T_K(f)+T_K(g).\] 

    其次, $T_K$ 为有界算子. 对于任意 $f\in H$, 有
    \begin{align*}
        \|T_K(f)\|^2
        &=\int_0^1|T_K(f)(x)|^2\diff x\\
        &=\int_0^1\left|\int_0^1K(x,y)f(y)\diff y\right|^2\diff x\\
        &\leq \int_0^1\left(\int_0^1|K(x,y)|^2\diff y\cdot\int_0^1|f(y)|^2\diff y\right)\diff x\\
        &=\int_0^1|f(y)|^2\diff y\cdot\int_0^1\int_0^1|K(x,y)|^2\diff x\diff y\\
        &=\|f\|^2\cdot\|K\|_{L_2([0,1]\times[0,1])}^2.
    \end{align*}
    故 $T_K$ 为有界算子且 $\|T_K\|\leq \|K\|_{L_2([0,1]\times [0,1])}$.


    (c) 对于 $\forall f,g\in H$, 有
    \begin{align*}
        \langle T_{\widetilde{K}}(f),g\rangle&=\int_0^1\left(\int_0^1\widetilde{K}(x,y)f(y)\diff y\right)\conjugate{g(x)}\diff x\\
        &=\int_0^1\left(\int_0^1\overline{K(y,x)}f(y)\diff y\right)\conjugate{g(x)}\diff x \\
        &=\int_0^1\left(\int_0^1\overline{K(x,y)}f(x)\diff x\right)\conjugate{g(y)}\diff y \\
        &=\int_0^1\left(\int_0^1\overline{K(x,y)g(y)}f(x)\diff x\right)\diff y \\
        &=\int_0^1\left(\int_0^1\overline{K(x,y)g(y)}f(x)\diff y\right)\diff x \\
        &=\int_0^1f(x)\overline{\left(\int_0^1K(x,y)g(y)\diff y\right)}\diff x \\
        &=\langle f,T_K(g)\rangle.
    \end{align*}
    因此由伴随算子的定义知 $T_K^{*}=T_{\widetilde{K}}$.

    (d) $T(f)(x)=\int_0^x f(1-y)\diff y=\int_{1-x}^1 f(y)\diff y$, 取:
    \[K(x,y)=
    \begin{cases}
    0, & 0\leq y\leq 1-x,\\
    1, & 1-x<y\leq 1.
    \end{cases}\]
    显然 $K(x,y)\in L_2([0,1]\times [0,1])$, 且
    \[T_K(f)(x)=\int_0^1 K(x,y)f(y)\diff y=\int_{1-x}^1 f(y)\diff y=T(f)(x).\]
    即在此情形下 $T_K$ 和 $T$ 是同一个算子, 利用(b)中结论知 $T\in\mathcal{B}(H)$.

    由 $K(x,y)$ 的定义知 $\widetilde{K}(x,y)=\overline{K(y,x)}=K(y,x)=K(x,y)$, 所以由 (c) 中结论知:
    \[T^{*}=T_K^{*}=T_{\widetilde{K}}=T_K=T.\]
    因为 $T^{*}=T$, 所以 $T$ 的特征值全部都为实数, 
    任取两个特征值 $\lambda,\mu\in\mathbb{R}$, 任取两个相应的特征向量 $f,g\in H$, 
    即$T(f)=\lambda f,T(g)=\mu g$,则:
    \[\mu\langle f,g\rangle=\langle f,\mu g\rangle=\langle f,T(g)\rangle=\langle T(f),g\rangle=\lambda\langle f,g\rangle.\]
    从而
    \[(\mu-\lambda)\langle f,g\rangle=0.\]
    故 $\langle f,g\rangle=0$, 因此相应的特征子空间两两正交.

    下面具体求特征值. 任取非零特征值 $\lambda$ 及其相应的特征向量 $f$, 则
    \[T(f)(x)=\int_{1-x}^1 f(y)\diff y=\lambda f(x),\quad\forall x\in [0,1].\]
    故 $f(0)=0$ 且 $\lambda f(1)=\int_0^1 f(y)\diff y$. 将上式求导一次得
    \begin{equation}
        f(1-x)=\lambda f'(x)\Longrightarrow f(x)=\lambda f'(1-x).\tag{$\star$}
    \end{equation}
    再将上式求导一次得
    \begin{equation}
        -f'(1-x)=\lambda f''(x).\tag{$\star\star$}
    \end{equation}
    结合 $(\star)(\star\star)$ 两式即得 ODE
    \[f''(x)+\frac{1}{\lambda^2}f(x)=0.\]
    上述常微分方程的解为 $f(x)=C_1\cos\frac{x}{\lambda}+C_2\sin\frac{x}{\lambda}$.
    由 $f(0)=0$, 得 $f(x)=C_2\sin\frac{x}{\lambda}$, 再由 $\lambda f(1)=\int_0^1 f(x)\diff x$ 得
    \[\lambda C_2\sin\frac{1}{\lambda}=\int_0^1 C_2\sin\frac{x}{\lambda}\diff x.\]
    由上式直接解得 $\sin\frac{1}{\lambda}+\cos\frac{1}{\lambda}=1$,
    故 $\lambda=\frac{1}{2k\pi}$ ($k\in\FZ,k\neq 0$) 或 $\frac{1}{\frac{\pi}{2}+2k\pi}$ ($k\in\FZ$).
\end{proof}

% \begin{remark}
% 如果 $u\in\mathcal{B}(H)$ 是正规的且 $u(f)=\lambda f$, 则 $u^{*}(f)=\bar{\lambda}f$ 利用 $\lambda\mathbbm{1}-u$ 的正规性可以证明此结果, 即 $(\lambda\mathbbm{1}-u)^{*}(\lambda\mathbbm{1}-u)=(\lambda\mathbbm{1}-u)(\lambda\mathbbm{1}-u)^{*}$
% \end{remark}


\begin{exercise}
    和上一习题一样, 令 $H=L_{2}(0,1)$; 并设 $\left(e_{n}\right)_{n \geq 1}$ 是 $H$ 中的规范正交集. 
    证明: $\left(e_{n}\right)_{n \geq 1}$ 是 $H$ 上的规范正交基的充分必要条件是
    \[
    \sum_{n\geq 1}\left|\int_{0}^{x} e_{n}(t)\diff t\right|^{2}=x, \quad \forall x \in[0,1].
    \]
\end{exercise}

\begin{proof}
    \necessary
    取 $f_x=\mathbbm{1}_{(0,x)}$, 由 Parseval 恒等式得 
    \begin{equation}
        \|f_x\|_{L_2}^2=\sum_{n\geq 1}|\innerp{f_x}{e_n}|^2,\tag{$\star$}
    \end{equation}
    换个马甲即为
    \[x=\sum_{n\geq 1}\left|\int_{0}^{x} e_{n}(t)\diff t\right|^{2}.\]

    \sufficient
    将 $(e_n)_{n\geq 1}$ 扩展成为 $L_2(0,1)$ 的规范正交基 $(e_n)_{n\geq 1}\cup(\tilde{e}_n)_{n\geq 1}$,
    则由 Parseval 恒等式得
    \[\|f_x\|_{L_2}^2=\sum_{n\geq 1}|\innerp{f_x}{e_n}|^2+\sum_{n\geq 1}|\innerp{f_x}{\tilde{e}_n}|^2.\]
    而由 $(\star)$ 式得 $\sum_{n\geq 1}|\innerp{f_x}{\tilde{e}_n}|^2=0$, 即对任意的 $n\geq 1$, 有
    \[\innerp{f_x}{\tilde{e}_n}=\int_0^x \tilde{e}_n(t)\diff t=0,\quad\forall x\in [0,1].\]
    故 $\tilde{e}_n=0$, $\forall n\geq 1$, 从而 $(e_n)_{n\geq 1}$ 为 $L_2(0,1)$ 的规范正交基.
\end{proof}



\begin{exercise}
    设 $\Omega$ 是复数域 $\mathbb{C}$ 中的开集, 约定 $\mathbb{C}$ 
    上的测度为 $\mathbb{R}^{2}$ 上的 Lebesgue 测度, 记为 $\diff \lambda(z)$. 令
    \[H_{\Omega}=\{f\in L_{2}(\Omega): f \text{\ 是\ }\Omega\text{\ 上的全纯函数}\}.\]
    对任一点 $z\in\Omega$, $\delta_{z}$ 表示 $z$ 处在 $H_{\Omega}$ 上的演化, 即有 $\delta_{z}(f)=f(z), f \in H_{\Omega}$.

    (a) 若 $\closure{B}(z, r)=\{w\in\mathbb{C}:|w-z|\leq r\}\subset\Omega$, 证明
    \[
    f(z)=\frac{1}{\pi r^2} \int_{\closure{B}(z, r)} f(w)\diff\lambda(w), \quad\forall f\in H_{\Omega}.
    \]

    (b) 证明
    \[
    f \in H_{\Omega}, z \in\Omega, d(z,\Omega^{c})>r\Longrightarrow|f(z)| \leq \frac{1}{\sqrt{\pi} r}\|f\|_{2}.
    \]
    
    (c) 证明: 当在 $L_{2}(\Omega)$ 上赋予内积运算时, $H_{\Omega}$ 是一个可分的 Hilbert 空间.
\end{exercise}



\begin{exercise}
    设 $H$ 是一个 Hilbert 空间, 并设 $T\in\mathcal{B}(H)$ 且 $\|T\|\leq 1$. 证明:

    (a) $T(x)=x$ 当且仅当 $T^*(x)=x$, $x\in H$.

    (b) $\ker(I-T)=\ker(I-T^*)$.

    (c) $H=\ker(I-T)\oplus\closure{(I-T)(H)}$.
\end{exercise}

\begin{proof}
    (a)由伴随算子的性质知 $\|T^*\|=\|T\|\leq 1$. 当 $T(x)=x$ 时,
    \begin{align*}
        \|T^*(x)-x\|^2
        & =\|T^*(x)\|^2+\|x\|^2-2\Re\innerp{T^*}{x} \\
        & =\|T^*(x)\|^2+\|x\|^2-2\Re\innerp{x}{T(x)} \\
        & =\|T^*(x)\|^2+\|x\|^2-2\Re\innerp{x}{x} \\
        & =\|T^*(x)\|^2-\|x\|^2 \\
        & \leq\|T^*\|^2\|x\|^2-\|x\|^2\leq 0,
    \end{align*}
    故 $T^*(x)=x$. 同理可证当 $T^*(x)=x$ 时有 $T(x)=x$.

    (b) $x\in\ker(I-T)\Leftrightarrow x-T(x)=0\Leftrightarrow x-T^{*}(x)=0\Leftrightarrow x\in\ker(I-T^{*})$, 
    故 $\ker(I-T)=\ker(I-T^{*})$.

    (c) 由正交分解定理, 只需证明:
    \[\ker(I-T)=[(I-T)(H)]^{\perp}\]

    先证: $(I-T)^*=I-T^*$. 对于任意 $x,y\in H$, 有
    \[\innerp{(I-T)(x)}{y}=\innerp{x}{y}-\innerp{T(x)}{y}=\innerp{x}{y}-\innerp{x}{T^*(y)}=\innerp{x}{(I-T^*)(y)},\]
    故 $(I-T)^*=I-T^*$.

    再证: $\ker(I-T)=(I-T)(H)^{\perp}$.
    一方面, 对任意 $x\in (I-T)(H)^{\perp}$ 和 $y\in H$, 有
    \[\innerp{(I-T^*)(x)}{y}=\innerp{(I-T)^*(x)}{y}=\innerp{x}{(I-T)(y)}=0.\]
    由 $y$ 的任意性知 $x\in\ker(I-T^*)=\ker(I-T)$, 故 $(I-T)(H)^{\perp}\subset\ker(I-T)$.
    另一方面, 对任意 $x\in\ker(I-T)=\ker(I-T^*)$ 和 $(I-T)(y)\in (I-T)(H)$, 有
    \[\innerp{x}{(I-T)(y)}=\innerp{(I-T)^*(x)}{y}=\innerp{(I-T^*)(x)}{y}=0,\]
    故 $\ker(I-T)\subset (I-T)(H)^{\perp}$.
\end{proof}



\begin{exercise}
    设 $H$ 是 Hilbert 空间. 称映射 $A\in\mathcal{B}(H)$ 为压缩算子, 若 $\|A\|\leq 1$;
    称 $A$ 是正的, 若对任一 $x\in H$, 有 $\innerp{A(x)}{x}\geq 0$.

    (a) 证明 $H$ 上任意压缩正算子 $A$ 满足
    \[\|x-A(x)\|^2\leq\|x\|^2-\|A(x)\|^2,\quad x\in H.\]
\end{exercise}

\begin{proof}
    (a) 先证明 $A$ 是自伴算子, 即 $\innerp{A(x)}{y}=\innerp{x}{A(y)}$, $\forall x,y\in H$. 因
    \[\innerp{A(x+y)}{x+y}=\innerp{A(x)}{x}+\innerp{A(y)}{y}+\innerp{A(y)}{x}+\innerp{A(x)}{y}\in\FR,\]
    故 $\innerp{A(y)}{x}+\innerp{A(x)}{y}\in\FR$, 故
    \begin{equation}
            \innerp{A(y)}{x}+\innerp{A(x)}{y}=\conjugate{\innerp{A(y)}{x}}+\conjugate{\innerp{A(x)}{y}}=\innerp{x}{A(y)}+\innerp{y}{A(x)}.\tag{$\star$}
    \end{equation}
    又因
    \[\innerp{A(x-\ii y)}{x-\ii y}=\innerp{A(x)}{x}+\innerp{A(y)}{y}+\ii\innerp{A(x)}{y}-\ii\innerp{A(y)}{x}\in\FR,\]
    故 $\ii\innerp{A(x)}{y}-\ii\innerp{A(y)}{x}\in\FR$, 故
    \[\ii(\innerp{A(x)}{y}-\innerp{A(y)}{x})=(-\ii)(\innerp{y}{A(x)}-\innerp{x}{A(y)}),\]
    从而
    \begin{equation}
        \innerp{A(x)}{y}-\innerp{A(y)}{x}=\innerp{x}{A(y)}-\innerp{y}{A(x)}.\tag{$\star\star$}
    \end{equation}
    结合 $(\star)(\star\star)$ 即得 $\innerp{A(x)}{y}=\innerp{x}{A(y)}$.

    再证明 $I-A$ 是正算子, 从而是自伴算子. 对任意 $x\in H$, 由 Cauchy-Schwarz 不等式有
    \[\innerp{A(x)}{x}\leq\|A(x)\|\|x\|\leq\|A\|\|x\|^2\leq\|x\|^2=\innerp{x}{x},\]
    即 $\innerp{(I-A)(x)}{x}\geq 0$, 故 $I-A$ 是正算子.

    原不等式 $\|x-A(x)\|^2\leq\|x\|^2-\|A(x)\|^2$ 等价于 $\innerp{A(x)}{(I-A)(x)}\geq 0$,
    故只需证后者即可, 由 $I-A$ 为自伴算子得
    \begin{align*}
        \innerp{A(x)}{(I-A)(x)}
        & =\innerp{(I-A)A(x)}{x} \\
        & =\innerp{(I-A)A(x)}{(I-A)(x)+A(x)} \\
        & =\innerp{(I-A)A(x)}{(I-A)(x)}+\innerp{(I-A)A(x)}{A(x)} \\
        & =\innerp{A(I-A)(x)}{(I-A)(x)}+\innerp{(I-A)A(x)}{A(x)} \\
        & \geq 0.
    \end{align*}
    最后用到了 $A$ 和 $I-A$ 皆为正算子, 证毕.
\end{proof}

\begin{remark}
    从 (a) 可以得到一个结论: 正算子必为自伴算子. 事实上, 我们也可以用极化恒等式证明此结论,
    但这里所说的极化恒等式是广义的极化恒等式, 设 $H$ 为 $\FC$ 上的向量空间,
    映射 $S:H\times H\to\FC$, $(x,y)\to S(x,y)$ 关于第一个变量是线性的,
    关于第二个变量是共轭线性的, 则我们有极化恒等式
    \[4S(x,y)=\sum_{k=0}^3 \ii^k S(x+\ii^k y,x+\ii^k y).\]
    证明很容易, 将上式右侧展开验证即可. 特别地, 取 $S$ 为内积, 则得到经典的极化恒等式
    \[4\innerp{x}{y}=\sum_{k=0}^3 \ii^k\|x+\ii^k y\|^2.\]
    如果假设 $S$ 关于第一个变量是共轭线性的而
    关于第二个变量是线性的, 则也有相应的极化恒等式
    \[4S(x,y)=\sum_{k=0}^3(-\ii)^k S(x+\ii^k y,x+\ii^k y).\]

    在本问题中, 定义 $\mathcal{S}(x,y)=\innerp{A(x)}{y}$ 和 $\mathcal{T}(x,y)=\innerp{x}{A(y)}$,
    则 $\mathcal{S}$ 和 $\mathcal{T}$ 都关于第一个变量为线性且关于第二个变量为共轭线性,
    由极化恒等式得
    \[4\mathcal{S}(x,y)=\sum_{k=0}^3 \ii^k \mathcal{S}(x+\ii^k y,x+\ii^k y),\]
    \[4\mathcal{T}(x,y)=\sum_{k=0}^3 \ii^k \mathcal{T}(x+\ii^k y,x+\ii^k y).\]
    即
    \[4\innerp{A(x)}{y}=\sum_{k=0}^3 \ii^k \innerp{A(x+\ii^k y)}{x+\ii^k y},\]
    \[4\innerp{x}{A(y)}=\sum_{k=0}^3 \ii^k \innerp{x+\ii^k y}{A(x+\ii^k y)}.\]
    结合 $A$ 为正算子即得 $\innerp{A(x)}{y}=\innerp{x}{A(y)}$.
\end{remark}
\chapter{连续函数空间}
\thispagestyle{empty}



\begin{exercise}
    对任意 $x\in [0,1]$, 设 $f_n(x)=x^n$. 在 $[0,1]$ 上的哪些点处, $(f_n)_{n\geq 1}$ 等度连续?
\end{exercise}

\begin{solve}
    因为$(f_n)_{n\geq 1}$在区间$[0,1)$上一致收敛到常值函数$f\equiv0$,
    所以$(f_n)_{n\geq 1}$在$[0,1)$上等度连续,并且容易看出$(f_n)_{n\geq 1}$在$x=1$处不是等度连续的.
\end{solve}



\begin{exercise}
    设 $K$ 是度量空间, $E$ 是赋范空间, $\left(f_{n}\right)_{n\geq 1}$ 是一列从 $K$ 到 $E$ 的连续函数. 
    证明若 $(f_{n})_{n\geq 1}$ 在一个点 $x$ 处等度连续, 则对任一收敛到 $x$ 
    的点列 $(x_{n})_{n\geq 1}$, 都有 $(f_{n}(x)-f_{n}(x_n))_{n\geq 1}$ 收敛到 $0$. 
    进而证明如果 $(f_{n}(x))_{n\geq 1}$ 在 $E$ 中收敛到 $y$, 
    那么对任一收敛到 $x$ 的点列 $(x_{n})_{n\geq 1}$, $(f_{n}(x_{n}))_{n\geq 1}$ 也收敛到 $y$.

    取 $f_{n}(x)=\sin(nx)$. 证明 $(f_{n})_{n\geq 1}$ 在 $\FR$ 上每一点都不等度连续.
\end{exercise}

\begin{proof}
    (1) 若 $(f_n)_{n\geq 1}$在点$x$处等度连续, 则对 $\forall \varepsilon>0$, $\exists \delta>0$, 
    使得当 $d(x,y)<\delta$ 时, 对 $\forall n\in\mathbb{N}^*$, $\|f_n(x)-f_n(y)\|<\varepsilon$. 
    且对任意收敛到 $x$ 的点列 $(x_n)_{n\geq 1}$, 存在正整数$N$, 当$n>N$时, $d(x_n,x)<\delta$, 
    故此时对任意正整数 $k$, $\|f_k(x)-f_k(x_n)\|<\varepsilon$. 取$k=n$, 
    得$\|f_n(x)-f_n(x_n)\|<\varepsilon$.
    
    综上, 对$\forall \varepsilon>0$, 存在正整数 $N$, 当 $n>N$ 时, $\|f_n(x)-f_n(x_n)\|<\varepsilon$, 
    这就是说$(f_n(x)-f_n(x_n))_{n\geq 1}$收敛到$0$.

	(2) 由于 $(f_n(x))_{n\geq 1}$ 在 $E$ 中收敛到 $y$, 故对 $\varepsilon>0$, 存在 $N_1\in\mathbb{N}^*$, 
    当$n>N_1$时, $\|f_n(x)-y\|<\frac{\varepsilon}{2}$, 又根据 (1), 
    设 $(x_n)_{n\geq 1}$ 收敛于 $x$, 则存在 $N_2\in\mathbb{N}^*$, 使得 $n>N_2$ 时, 
    $\|f_n(x)-f_n(x_n)\|<\frac{\varepsilon}{2}$. 取 $N=\max\left\{N_1,N_2\right\}$, 当$n>N$时, 
	\[\|f_n(x_n)-y\|\leq \|f_n(x_n)-f_n(x)\|+\|f_n(x)-y\|\leq\frac{\varepsilon}{2}+\frac{\varepsilon}{2}=\varepsilon.\]
	故 $f_n(x_n)_{n\geq 1}$ 也收敛到 $y$.

	(3) 若 $x=k\pi$, $k\in\FZ$, 则取 $x_n=k\pi+\frac{1}{n}$, 注意到$f_n(k\pi)=\sin(nk\pi)=0$, 
    故 $\lim\limits_{n\rightarrow\infty}f_n(k\pi)=0$, 而
    \[\begin{aligned}
		f_n(x_n)&=\sin(n(k\pi+\frac{1}{n}))\\&=\sin(nk\pi+1)\\
		&=\sin(nk\pi)\cos1+\cos(nk\pi)\sin1\\&=\cos(nk\pi)\sin1.
	\end{aligned}\]
    从而 $|f_n(x_n)|=\sin 1$ 对任意正整数 $n$ 都成立, 
    因此 $f_n(x_n)$ 在 $n$ 趋于 $\infty$ 时极限不可能为 $0$. 由 (2) 知 $(f_n)_{n\geq 1}$ 在 $x=k\pi$ 处不等度连续. 

    若 $x\not= k\pi$, $k\in\FZ$, 取 $x_n=x+\frac{\pi}{n}$, 从而
    \[\begin{aligned}
    \|f_n(x)-f_n(x_n)\|&=|\sin (nx)-\sin(nx+\pi)|\\
    &=|\sin (nx)-\sin(nx)\cos\pi-\cos(nx)\sin\pi|\\
                    &=2|\sin nx|.
    \end{aligned}\]

    下面我们说明当 $x\not=k\pi$ 时, $\lim\limits_{n\rightarrow \infty}\sin nx$不存在. 
    事实上, 设 $x\not=k\pi$, $k\in\FZ$, 若 $\lim\limits_{n\rightarrow \infty} \sin nx$ 存在, 那么
    \[\lim\limits_{n\rightarrow \infty} (\sin((n+1)x)-\sin((n-1)x))=0.\]
    由和差化积, 我们知道 $\sin((n+1)x)-\sin((n-1)x)=2\sin x\cos nx$, 从而
    \[\lim\limits_{n\rightarrow \infty} \cos nx=0.\]
    接着注意到 $\cos((n+1)x)=\cos nx\cos x-\sin nx\sin x$, 故
    \[\lim\limits_{n\rightarrow \infty} \sin nx =0,\]
    而这与 $\sin^2 nx+\cos^2 nx=1$矛盾! 从而 $\lim\limits_{n\rightarrow \infty} \sin nx$不存在.

    因此当 $n$ 趋近于 $\infty$ 时, $\|f_n(x)-f_n(x_n)\|$ 极限不存在, 
    由 (1) 知 $(f_n)_{n\geq 1}$ 在 $x\not= k\pi$ 处不等度连续.
    
    综上, $(f_n)_{n\geq 1}$ 在 $\mathbb{R}$上每一点都不等度连续. 
\end{proof}





\begin{exercise}
    设 $K$ 是拓扑空间, $(E,d)$ 是度量空间. 证明: 
    若 $(f_{n})$ 在 $C(K, E)$ 中依一致范数收敛, 则 $(f_{n})$ 等度连续.
\end{exercise}

\begin{proof}
    设$(f_n)_{n\geq 1}$一致收敛到$f$,容易验证$f\in C(K,E)$,则$\forall\varepsilon>0,\exists N>0,s.t.\forall n>N,sup_{x\in K}d(f_n(x),f(x))<\varepsilon/3$,任取$x\in K$,因为$f\in C(K,E)$,所以存在$V\in\mathcal{N}(x)$,使得当$y\in V$时,$d(f(y),f(x))<\varepsilon/3$\\
    因此对于上述的$\varepsilon,N$,当$n>N$且$y\in V$时,有
    \[d(f_n(y),f_n(x))\leq d(f_n(y),f(y))+d(f(y),f(x))+d(f(x),f_n(x))<\varepsilon\]
    从而集合$(f_n)_{n>N}$是等度连续的,而增加有限个元素不改变等度连续性,因此$(f_n)_{n\geq 1}$等度连续.
\end{proof}



\begin{exercise}
    设 $K$ 是拓扑空间, $(E, d)$ 是度量空间, $(f_{n})$ 是 $C(K, E)$ 上等度连续序列. 
    证明所有使得 $(f_{n}(x))$ 是 Cauchy 序列的点 $x$ 构成的集合是 $K$ 中的闭子集.
\end{exercise}

\begin{proof}
记所有使得$(f_n(x))$是Cauchy序列的点$x$构成的集合为$B$,要证明$B$为闭集,只需证明其任意收敛序列的极限点仍在其中
设$(x_k)_{k\geq 1}$是$B$中任意一个收敛的序列,且$x_k\to x$.

因为$f_n\in C(K,E)$,所以$\forall\varepsilon>0,\exists K,\forall k>K,d(f_n(x_k),f_n(x))<\varepsilon$\\
又$x_k\in B$,所以$(f_n(x_k))_{n\geq 1}$是Cauchy序列,故对于上述$\varepsilon>0,\exists N,\forall m,n>N$,有
\[d(f_n(x_k),f_m(x_k))<\varepsilon\]
从而\[d(f_n(x),f_m(x))\leq d(f_n(x),f_n(x_k))+d(f_n(x_k),f_m(x_k))+d(f_m(x_k),f_m(x))<3\varepsilon\]
这说明$(f_n(x))_{n\geq 1}$是Cauchy序列,故$x\in B$,所以$B$是闭集.
\end{proof}



\begin{exercise}
    考虑函数序列 $(f_{n})$, 这里 $f_{n}(t)=\sin\left(\sqrt{t+4(n\pi)^2}\right)$, $t\in[0,\infty)$.

    (a) 证明 $(f_{n})$ 等度连续并且逐点收敛到 $0$ 函数.

    (b) $C_{b}([0,\infty),\FR)$ 表示 $[0, \infty)$ 上所有有界连续实函数构成的空间, 并赋予范数
    \[\|f\|_{\infty}=\sup_{t\geq 0}|f(t)|.\]
    $(f_n)$ 在 $C_b([0,\infty),\FR)$ 中是否相对紧?
\end{exercise}

\begin{proof}
    (a) 任意取定 $t_0\geq 0$, 对于 $\forall\varepsilon>0$,
    取 $\delta=4\pi\varepsilon$, 则当 $t\in B(t_0,\delta)\cap [0,+\infty)$ 时, 对于任意的 $f_n$ 有
    \begin{align*}
        |f_n(t)-f_n(t_0)| & =\left|\sin(\sqrt{t+4(n\pi)^2})-\sin(\sqrt{t_0+4(n\pi)^2})\right| \\
                        & \leq |\sqrt{t+4(n\pi)^2}-\sqrt{t_0+4(n\pi)^2}|\\
                        & =\frac{|t-t_0|}{\sqrt{t+4(n\pi)^2}+\sqrt{t_0+4(n\pi)^2}}\\
                        & \leq\frac{|t-t_0|}{4\pi}<\varepsilon.
    \end{align*}
    因此 $(f_n)$ 等度连续.
    对任意的 $t\in [0,\infty)$,因为
    \[\begin{split}
    \lim_{n\to\infty}|\sin(\sqrt{t+4(n\pi)^2})-0|&=\lim_{n\to\infty}|\sin(\sqrt{t+4(n\pi)^2})-\sin(2n\pi)|\\
    &\leq \lim_{n\to\infty}\frac{t}{\sqrt{t+4(n\pi)^2}+2n\pi}\\
    &\leq \lim_{n\to\infty}\frac{t}{4n\pi}=0.
    \end{split}\]
    故 $(f_n)$ 逐点收敛到 $0$ 函数.

    (b) 注意到依范数 $\|\cdot\|_{\infty}$ 下的收敛即为在 $[0,\infty)$ 下的一致收敛,
    假设 $(f_n)$有依范数 $\|\cdot\|_{\infty}$ 收敛的子列, 则由 (a) 知该子列必一致收敛于 $0$ 函数,
    然而对于 $\forall n$,
    \[\|f_n\|_{\infty}=\sup_{t\geq 0}|\sin(\sqrt{t+4(n\pi)^2})|=1.\]
    故 $(f_n)$ 不存在依范数 $\|\cdot\|_{\infty}$ 收敛的子列, 因此 $(f_n)$ 不是相对紧的.
\end{proof}




% 7.Proof:(a)对于任意$m\geq 1$,有$K=\bigcup_{x\in K}B(x,1/m)$,因为$K$是紧的,所以存在有限子集$D_m$,使得
% \[K=\bigcup_{x\in D_m}B(x,1/m)\]
% 令\[D=\bigcup_{m\geq 1}D_m\]
% 显然$D$是$K$的可数稠密子集\\
% (b)因为$D$可数,故可记$D=\{x_1,x_2,\cdots\}$\\
% $\{f_n(x_1):n\geq 1\}$相对紧:有收敛子列$\{f_{n_{1k}}(x_1):k\geq 1\}$\\
% $\{f_n(x_2):n\geq 1\}$相对紧:$\{f_{n_{1k}}(x_2):k\geq 1\}$有收敛子列$\{f_{n_{2k}}(x_2):k\geq 1\}$\\
% $\cdots$\\
% $\{f_n(x_l):n\geq 1\}$相对紧:$\{f_{n_{(l-1)k}}(x_l):k\geq 1\}$有收敛子列$\{f_{n_{lk}}(x_l):k\geq 1\}$\\
% $\cdots$\\
% 如此进行下去,利用对角线法,可挑选出一列$(f_{n_{kk}})_{k\geq 1}$,使得对任意的$x\in D$,$(f_{n_{kk}}(x))_{k\geq 1}$收敛\\
% (c)对任意$x\in K$,存在$(x_m)\subset D$使得$x_m\to x$,故$\forall\varepsilon>0$,有:
% \[d(f_{n_{kk}}(x),f_{n_{qq}}(x))\leq d(f_{n_{kk}}(x),f_{n_{kk}}(x_m))+d(f_{n_{kk}}(x_m),f_{n_{qq}}(x_m))+d(f_{n_{qq}}(x_m),f_{n_{qq}}(x))\]由$(f_n)$等度连续及(b)中结论知当指标$m,k,q$都取得足够大时,有
% \[d(f_{n_{kk}}(x),f_{n_{qq}}(x))<\varepsilon\]因此$(f_{n_{kk}}(x))$是Cauchy序列\\
% 由$\Delta(f_{n_{kk}},f_{n_{qq}})=sup_{x\in K}d((f_{n_{kk}}(x)),(f_{n_{qq}}(x)))$知$(f_{n_{kk}})$是Cauchy序列,故其在$C(K,E)$中收敛\\\\


\setcounter{exer}{9}
\begin{exercise}
    设 $(K,d)$ 是紧度量空间. 证明所有从 $K$ 到 $\FR$ 的 Lipschitz 函数构成的集合在 $(C(K,\FR),\|\cdot\|_{\infty})$ 中稠密.
\end{exercise}

\begin{proof}
    记所有从 $K$ 到 $\FR$ 的 Lipschitz 函数构成的集合为 $\mathcal{A}$.
    \begin{itemize}
    \item $\mathcal{A}$是 $C(K,\FR)$的子代数:
    容易验证 $\mathcal{A}$ 中元素关于加法和数乘封闭,下面说明关于乘法封闭,
    任意 $f,g\in\mathcal{A}$, 存在 $\lambda_1>0,\lambda_2>0$ 使得对任意 $x,y\in K$, 有
    \[|f(x)-f(y)|\leq\lambda_1d(x,y).\]
    \[|g(x)-g(y)|\leq\lambda_2d(x,y).\]
    又因为 $K$ 为紧集, 故存在 $M_1,M_2$, 使得对于任意 $x\in K$, 
    有 $|f(x)|\leq M_1$, $|g(x)|\leq M_2$, 故
    \[\begin{split}
    |f(x)g(x)-f(y)g(y)|
    & =|f(x)g(x)-f(x)g(y)+f(x)g(y)-f(y)g(y)|\\
    & \leq |f(x)g(x)-f(x)g(y)|+|f(x)g(y)-f(y)g(y)|\\
    & \leq M_1\lambda_2d(x,y)+M_2\lambda_1d(x,y)\\
    & =(M_1\lambda_2+M_2\lambda_1)d(x,y).
    \end{split}\]
    从而 $\mathcal{A}$ 中元素关于乘法封闭.
    \item 常值函数 $1\in\mathcal{A}$.
    \item 任意取定 $y\in K$, 取 $f(x)=d(x,y)\in\mathcal{A}$, 则当 $x\neq y$ 时, $f(x)\neq f(y)$.
    \end{itemize}
    由 Stone-Weierstrass 定理知 $\mathcal{A}$ 在 $C(K,\FR)$ 中稠密.
\end{proof}



\begin{exercise}
    设 $K_{1}$ 和 $K_{2}$ 都是紧 Hausdorff 空间. 
    对 $f\in C(K_{1},\FC), g\in C(K_{2},\FC)$ 定义
    \[
    f\otimes g(x_1,x_2)=f(x_1) g(x_2), \quad\forall(x_1,x_2)\in K_{1}\times K_{2}.
    \]
    并定义集合
    \[
    \mathcal{A}=\biggl\{\sum_{\text{有限和}} a_{i}f_{i}\otimes g_{i}: a_{i}\in\FC, f_{i}\in C(K_1,\FC), g_{i}\in C(K_2, \FC)\biggr\}.
    \]
    证明 $\mathcal{A}$ 在 $C(K_1\times K_2,\FC)$ 中稠密.
\end{exercise}

\begin{proof}
    \begin{enumerate}[(i)]
    \item $K_1\times K_2$是紧的Hausdorff空间
    \item $\mathcal{A}$是$C(K_1\times K_2,\FC)$的子代数:容易验证$\mathcal{A}$是向量子空间,下证$\mathcal{A}$关于乘法封闭:\\
    任意$\ell_1,\ell_2\in\mathcal{A}$,记$\ell_1=\sum_{i\in I}a_if_{1i}\otimes g_{1i},\ell_2=\sum_{j\in J}b_jf_{2j}\otimes g_{2j}$,其中$I,J$都为有限指标集,对任意$(x_1,x_2)\in K_1\times K_2$,有\[\begin{split}\ell_1(x_1,x_2)\cdot\ell_2(x_1,x_2)&=\left(\sum_{i\in I}a_if_{1i}(x_1)g_{1i}(x_2)\right)\left(\sum_{j\in J}b_jf_{2j}(x_1)g_{2j}(x_2)\right)\\&=\sum_{i\in I}\sum_{j\in J}a_ib_j(f_{1i}f_{2j})(x_1)\cdot(g_{1i}g_{2j})(x_2)\\&=\sum_{i\in I}\sum_{j\in J}a_ib_j\left((f_{1i}f_{2j})\otimes(g_{1i}g_{2j})\right)(x_1,x_2)\end{split}\]
    因此\[\ell_1\cdot\ell_2=\sum_{i\in I}\sum_{j\in J}a_ib_j\left((f_{1i}f_{2j})\otimes(g_{1i}g_{2j})\right)\]
    上式仍为有限和,结合$C(K_1,\FC),C(K_2,\FC)$都为代数知$\mathcal{A}$关于乘法封闭
    \item 常值函数$1\in\mathcal{A}$
    \item 可分点:设$x=(x_1,x_2),y=(y_1,y_2)\in K_1\times K_2$且$x\neq y$,不妨设$x_1\neq y_1$,因为$K_1$是紧Hausdorff空间,所以由Urysohn引理知存在$f\in C(K_1,[0,1])\subset C(K_1,\FC)$使得$f(x_1)=0,f(y_1)=1$,令$g(x)\equiv1(\forall x\in K_2)$,则
    \[f\otimes g(x)=f\otimes g(x_1,x_2)=0\]
    \[f\otimes g(y)=f\otimes g(y_1,y_2)=1\]
    因此$\mathcal{A}$是可分点的
    \item $\mathcal{A}$是自伴的
    \end{enumerate}
    综上知$\mathcal{A}$在$C(K_1\times K_2,\FR)$中稠密.
\end{proof}



\begin{exercise}
    $[0,1]$ 上所有的偶多项式构成的集合 $\mathcal{Q}$ 是否在 $C([0,1], \FR)$ 上稠密?
    $[-1,1]$ 上所有的偶多项式构成的集合 $\mathcal{R}$ 是否在 $C([-1,1], \FR)$ 上稠密?
\end{exercise}

\begin{solve}
    $\mathcal{Q}$在$C([0,1],\FR)$中稠密,理由:
    \begin{enumerate}[(i)]
    \item $\mathcal{Q}$ 是子代数.
    \item $1\in\mathcal{Q}$.
    \item $\forall x,y\in [0,1]$, 取 $f(x)=x^2$, 则 $f(x)\neq f(y)$.
    \end{enumerate}
    但是 $\mathcal{R}$ 在 $C([-1,1],\FR)$ 中不稠密, 因为 $[-1,1]$ 中的任意一个非零点和其相反数不可分.
\end{solve}



\begin{exercise}
    本习题的目的是证明 Bernstein 定理: 令 $f\in C([0,1], \FK)$, 并设
    \[
    B_{n}(f)(x)=\sum_{k=0}^{n} \mathrm{C}_{n}^{k} f\left(\frac{k}{n}\right) x^{k}(1-x)^{n-k}.
    \]
    则 $B_{n}$ 在 $[0,1]$ 上一致收敛到 $f$.

    (a) 首先导出对任一正整数 $n$, 有公式
    \[
    \sum_{k=0}^{n} \mathrm{C}_{n}^{k} k x^{k}(1-x)^{n-k}=nx \quad\text{和}\quad\sum_{k=0}^{n} \mathrm{C}_{n}^{k} k^2 x^k(1-x)^{n-k}=nx+n(n-1)x.
    \]
    并由此证明
    \[
    \sum_{k=0}^{n} \mathrm{C}_{n}^{k}(k-nx)^2 x^k (1-x)^{n-k}=nx(1-x).
    \]

    (b) 对任意 $\varepsilon>0$, 选择适当的 $\delta>0$, 使得
    \[
    x, y \in[0,1],|x-y|<\delta \Rightarrow|f(x)-f(y)|<\varepsilon.
    \]
    对任意固定的 $x\in [0,1]$, 令 $I=\{k:|x-\frac{k}{n}|<\delta\}$
    及 $J=\{k:|x-\frac{k}{n}|>\delta\}$. 证明
    \begin{align*}
        |f(x)-B_n(f)(x)|<{}
        & \varepsilon\sum_{k\in I}\mathrm{C}_n^k x^k(1-x)^{n-k} \\
        & +\frac{2\|f\|_{\infty}}{\delta^2}\sum_{k\in J}\mathrm{C}_n^k \biggl(x-\frac{k}{n}\biggr)^2x^k(1-x)^{n-k}.
    \end{align*}
    从而导出
    \[|f(x)-B_n(f)(x)|<\varepsilon+\frac{2\|f\|_{\infty}}{\delta}\frac{x(1-x)}{n}.\]

    (c) 得出结论
    \[\lim_{n\to\infty}\|f-B_n(f)\|_{\infty}=0.\]
\end{exercise}

\begin{proof}
(a)引入二项分布$X\sim B(n,x)$,则:
\[\sum_{k=0}^nC_n^kkx^k(1-x)^{n-k}=E(X)=nx\]
\[\sum_{k=0}^nC_n^kk^2x^k(1-x)^{n-k}=E(X^2)=Var(X)+E^2(X)=nx+n(n-1)x^2\]
\[\mbox{故}\sum_{k=0}^nC_n^k(k-nx)^2x^k(1-x)^{n-k}=nx+n(n-1)x^2-2nx\cdot nx+n^2x^2=nx(1-x)\]

(b)\[\begin{split}
|f(x)-B_n(f)(x)|&=\left|f(x)-\sum_{k=0}^nC_n^kf\left(\frac{k}{n}\right)x^k(1-x)^{n-k}\right|\\
&=\left|\sum_{k=0}^nC_n^k\left(f(x)-f\left(\frac{k}{n}\right)\right)x^k(1-x)^{n-k}\right|\\
&\leq\sum_{k=0}^nC_n^k\left|f(x)-f\left(\frac{k}{n}\right)\right|x^k(1-x)^{n-k}\\
&<\varepsilon\sum_{k\in I}C_n^kx^k(1-x)^{n-k}+\sum_{k\in J}C_n^k\left|f(x)-f\left(\frac{k}{n}\right)\right|x^k(1-x)^{n-k}\\
&\leq\varepsilon\sum_{k\in I}C_n^kx^k(1-x)^{n-k}+\frac{2\|f\|_{\infty}}{\delta^2}\sum_{k\in J}C_n^k\left(x-\frac{k}{n}\right)^2x^k(1-x)^{n-k}
\end{split}\]
由(a)知$\sum_{k=0}^nC_n^k(x-k/n)^2x^k(1-x)^{n-k}=x(1-x)/n$,故
\[|f(x)-B_n(f)(x)|<\varepsilon+\frac{2\|f\|_{\infty}}{\delta^2}\frac{x(1-x)}{n}\]

(c)由(b)中所得不等式知:
\[\begin{split}\lim_{n\to\infty}\|f-B_n(f)\|_{\infty}
&=\lim_{n\to\infty}\sup\limits_{0\leq x\leq 1}|f(x)-B_n(f)(x)|\\
&\leq\lim_{n\to\infty}\sup\limits_{0\leq x\leq 1}\left(\varepsilon+\frac{2\|f\|_{\infty}}{\delta^2}\frac{x(1-x)}{n}\right)\\&=\varepsilon\end{split}\]
由$\varepsilon$的任意性知\[\lim_{n\to\infty}\|f-B_n(f)\|_{\infty}=0\]
\end{proof}


    % 15.\textit{Proof}:\\\\
    % 16.\textit{Proof}:(a)首先显然$F(x)$是连续的,又因为$\hat{f}(0)=\frac{1}{2\pi}\int_0^{2\pi}f(\theta)\diff\theta=0$,所以
    % \[F(x+2\pi)=\int_0^{x+2\pi}f(t)\diff t=\int_0^xf(t)\diff t+\int_x^{x+2\pi}f(t)\diff t=\int_0^xf(t)\diff t=F(x)\]
    % 故$F\in C_{2\pi}$,并且\\
    % \[\begin{split}\hat{F}(n)&=\frac{1}{2\pi}\int_0^{2\pi}F(\theta)e^{-in\theta}\diff\theta\\
    % &=\frac{1}{2\pi}\int_0^{2\pi}F(\theta)\frac{1}{-in}\diff(e^{-in\theta})\\
    % &=\frac{1}{2\pi}\left(\frac{i}{n}e^{-in\theta}F(\theta)\right)\bigg|_0^{2\pi}-\frac{1}{2\pi}\int_0^{2\pi}\frac{i}{n}e^{-in\theta}f(\theta)\diff\theta\\
    % &=-\frac{i}{2\pi n}\int_0^{2\pi}e^{-in\theta}f(\theta)\diff\theta\\
    % &=-\frac{i}{n}\hat{f}(n)=\begin{cases}-\frac{i}{|n|}\hat{f}(|n|)=-\frac{i}{|n|}\frac{a_{|n|}}{2i}=-\frac{a_{|n|}}{2|n|}&n>0\mbox{时}\\\frac{i}{|n|}\hat{f}(-|n|)=\frac{i}{|n|}\left(-\frac{a_{|n|}}{2i}\right)=-\frac{a_{|n|}}{2|n|}&n<0\mbox{时}\end{cases}\\
    % &=-\frac{a_{|n|}}{2|n|}\end{split}\]
    % 由上面结果知对任意$n\geq 1$,有\[\begin{split}\frac{a_n}{n}&=-(\hat{F}(n)+\hat{F}(-n))\\&=-\frac{1}{2\pi}\int_0^{2\pi}F(\theta)\left(e^{-in\theta}+e^{in\theta}\right)\diff\theta\\
    % &=-\frac{1}{\pi}\int_0^{2\pi}F(\theta)\cos(n\theta)\diff\theta\\&=-\frac{1}{\pi}\int_0^{2\pi}\frac{F(\theta)}{n}\diff\sin(n\theta)\\
    % &=-\frac{F(\theta)\sin(n\theta)}{n\pi}\bigg|_0^{2\pi}+\frac{1}{\pi}\int_0^{2\pi}\frac{\sin(n\theta)}{n}f(\theta)\diff\theta\\
    % &=\frac{1}{\pi}\int_0^{2\pi}\frac{\sin(n\theta)}{n}f(\theta)\diff\theta\end{split}\]
    % 故\[\sum_{n\geq 1}\frac{a_n}{n}=\frac{1}{\pi}\int_0^{2\pi}\sum_{n\geq 1}\frac{\sin(n\theta)}{n}f(\theta)\diff\theta\]
    % 由级数$\sum_{n\geq 1}\frac{\sin(n\theta)}{n}$收敛及$f\in L_{2\pi}^1$知$\sum_{n\geq 1}\frac{a_n}{n}$收敛\\
    % (b)假设级数$\sum_{n\geq 2}\frac{\sin(nx)}{\log n}$是Fourier级数,则由(a)中结论知级数
    % \[\sum_{n\geq 2}\frac{1}{n\log n}\mbox{收敛}\]
    % 显然矛盾,因此$\sum_{n\geq 2}\frac{\sin(nx)}{\log n}$不是Fourier级数\\\\
    % 17.\textit{Proof}:因为$C_{2\pi}$在$L_{2\pi}^p$中稠密,所以存在函数序列$(f_n)\subset C_{2\pi}$,使得
    % \[\|f-f_n\|_p\to0(n\to\infty)\]
    % 不妨设$p\geq 1(0<p<1\mbox{的情形同理可证})$,由Minkowski不等式得
    % \[\|\tau_a(f)-f\|_p\leq\|\tau_a(f)-\tau_a(f_n)\|_p+\|\tau_a(f_n)-f_n\|_p+\|f_n-f\|_p\]
    % 由$f_n$是一致连续函数且$\tau_a$是连续变换不难知:
    % \[\lim_{a\to0}\|\tau_a(f)-f\|_p=0\]
\setcounter{chapter}{5}
\chapter{Baire定理及其应用}
\thispagestyle{empty}


\begin{exercise}
    (a) 证明: $\FR\setminus\FQ$ 是 $\FR$ 上的 $\mathcal{G}_{\delta}$ 集.
    并导出 $\FQ$ 不是 $\mathcal{G}_{\delta}$ 集, 且不存在函数 $f:\FR\to\FR$ 使得 $\cont{f}=\FQ$.

    (b) 定义函数 $f:\FR\to\FR$: 当 $x\in\FR\setminus\FQ$ 时, 令 $f(x)=0$; $f(0)=1$;
    若 $x$ 是非零的有理数 $\frac{p}{q}$, 这里 $\frac{p}{q}$ 是 $x$ 的不可约形式, $p\in\FZ$,
    $q\in\FN$, 令 $f(x)=\frac{1}{q}$. 证明: $\cont{f}=\FR\setminus\FQ$.

    (c) 设 $f=\mathbbm{1}_{\FQ}$. 证明: $f$ 不是第一纲的 (即不是任一函数列的极限函数),
    但是存在一列第一纲的函数逐点收敛到 $f$.
\end{exercise}

\begin{proof}
  \begin{enumerate}[(a)]
    \item 记 $\FQ=\{r_k\}_{k=1}^{\infty}$, 对于每一个 $k$, 令
      \[U_k=\FR\setminus\{r_k\}=(-\infty,r_k)\cup(r_k,+\infty).\]
      显然 $U_k$ 是开集并且
      \[\bigcap_{k=1}^{\infty}U_k=\FR\setminus\FQ.\]
      因此 $\FR\setminus\FQ$ 是 $\FR$ 上的 $\mathcal{G}_{\delta}$集.
      假设 $\FQ$ 也是 $\mathcal{G}_{\delta}$ 集, 则存在一列开集 $\{V_k\}_{k=1}^{\infty}$,使得
      \[\FQ=\bigcap_{k=1}^{\infty}V_k.\]
      每个 $V_k$ 包含 $\FQ$, 故必在 $\FR$ 中稠密, 由 Baire 定理可知集合
      \[\biggl(\bigcap_{k=1}^{\infty}U_k\biggr)\bigcap\biggl(\bigcap_{k=1}^{\infty}V_k\biggr)\]
      在 $\FR$ 中稠密, 然而
      \[\biggl(\bigcap_{k=1}^{\infty}U_k\biggr)\bigcap\biggl(\bigcap_{k=1}^{\infty}V_k\biggr)=(\FR\setminus\FQ)\cap\FQ=\varnothing.\]
      矛盾, 从而假设不成立, 说明 $\FQ$ 不是 $\mathcal{G}_{\delta}$集,
      由于任一映射的连续点集都是 $\mathcal{G}_{\delta}$ 集,
      所以不存在函数 $f:\FR\to\FR$使得$\textrm{Cont}(f)=\FQ$.

    \item 容易验证该函数是周期为 $1$ 的函数, 因此只需要考虑该函数在区间 $[0,1]$ 上的情况:
   
      首先, $f$在无理数点处连续: 任意取定 $x_0\in\FR\setminus\FQ$,
      有 $f(x_0)=0$. 对于 $\forall\varepsilon>0$, 在 $[0,1]$ 上至多只有有限个 $\frac{p}{q}$
      使得 $f(\frac{p}{q})=\frac{1}{q}\geq\varepsilon$, 取 $\delta>0$, 
      使得 $U(x_0,\delta)$ 不包含上述有限个有理数(这样的 $\delta$ 显然是可以取到的), 则当 $x\in U(x_0,\delta)$ 时,
      \[|f(x)-f(x_0)|=|f(x)|<\varepsilon.\]

      其次, $f$ 在有理数点处不连续: 同理可证.

      因此 $\cont{f}=\FR\setminus\FQ$.

    \item 见 Hirsch-Lacombe \cite[Page 65]{hirsch2012elements}.

      假设 $f=\mathbbm{1}_{\FQ}$ 是第一纲的, 则由定理6.1.7知 $\cont{f}$ 是
      $\FR$ 中稠密的 $\mathcal{G}_{\delta}$ 集, 但是 $f$ 在任意点处都不连续, 
      从而矛盾, 故 $f=\mathbbm{1}_{\FQ}$ 不是第一纲的. 取
      \[f_m(x)=\lim_{n\to+\infty}\cos(m!\pi x)^{2n}.\]
      由定义知 $(f_m)_{m\geq 1}$ 是一列第一纲的函数, 下面我们证明 $(f_m)_{m\geq 1}$ 逐点收敛到 $f$:
      \begin{itemize}
      \item 若$x\in\FQ$, 则可表示为 $x=\frac{p}{q}$, 当 $m\geq q$时,
            $m!\pi x=k\pi\Rightarrow f_m(x)=\lim_{n\to\infty}1=1=\mathbbm{1}_{\FQ}(x)$.
      \item 若 $x\in\FR\setminus\FQ$, 必有 $\cos(m!\pi x)\in(-1,1)\Rightarrow f_m(x)=0=\mathbbm{1}_{\FQ}(x)$.
      \end{itemize}
      综上知 $\lim_{m\to\infty}f_m=\mathbbm{1}_{\FQ}$. \qedhere
    \end{enumerate}
\end{proof}



\begin{exercise}
    证明局部紧的 Hausdorff 空间是 Baire 空间.
\end{exercise}

\begin{proof}
  设$X$是局部紧的Hausdorff空间,我们来证明$X$是Baire空间:

  已有定理:局部紧的Hausdorff空间中每一点都有一个紧邻域基.

  设$\{U_n\}_{n\geq 1}$是一列在$X$中稠密的开子集,且$D=\bigcap_{n\geq 1}U_n$,
  设$U$是任一非空开集,需证$D\cap U\neq\varnothing$.
  因为$U\cap U_1$是非空开集,所以存在非空开集$V_1$,使得$\bar{V}_1$紧且
  \[\bar{V}_1\subset U\cap U_1.\]
  因为$V_1\cap U_2$是非空开集,所以存在非空开集$V_2$,使得$\bar{V}_2$紧且\[\bar{V}_2\subset V_1\cap U_2\subset U\cap U_1\cap U_2.\]

  因为$V_{n-1}\cap U_n$是非空开集, 所以存在非空开集$V_n$, 使得$\bar{V}_n$紧且\[\bar{V}_n\subset V_{n-1}\cap U_n\subset U\cap U_1\cap\cdots\cap U_n\]
  依此可以得到一列开集$\{V_n\}$使得$\{\bar{V}_n\}$是单调递减的紧集列,故
  \[\bigcap_{n\geq 1}\bar{V}_n\neq\varnothing\]
  所以
  \[D\cap U=\left(\bigcap_{n\geq 1}U_n\right)\cap U\neq\varnothing\]这就说明了$D$在$X$中稠密,因此$X$是Baire空间.
\end{proof}



\begin{exercise}
    (a) 设 $f:\FR\to\FR$ 是可微函数. 证明: $\cont{f'}$ 是 $\FR$ 上稠密的 $\mathcal{G}_{\delta}$ 集.

    (b) 设 $f:\FR^2\to\FR$ 连续且在 $\FR^2$ 上存在偏导数 $\frac{\partial f}{\partial x}$ 和 $\frac{\partial f}{\partial y}$.
    证明: $f$ 的可微点包含 $\FR^2$ 中一个稠密的 $\mathcal{G}_{\delta}$ 集.
\end{exercise}

\begin{proof}
    (a)记
    \[g_n(x)=n\left[f\left(x+\frac{1}{n}\right)-f(x)\right].\]
    则 $(g_n)_{n\geq 1}$ 是 $\FR$ 上的连续函数序列且对于任意 $x\in\FR$, 有
    \[\lim_{n\to\infty}g_n(x)=f'(x).\]
    由定理 6.1.7 知 $\cont{f'}$ 是 $\FR$ 中稠密的 $\mathcal{G}_{\delta}$ 集.

    (b) 因为 
    \[\frac{\partial f(x,y)}{\partial x}=\lim\limits_{n\to\infty}n\left[f\left(x+\frac{1}{n},y\right)-f(x,y)\right]\overset{\Delta}{=}\lim\limits_{n\to\infty}F_n(x,y).\]
    且 $F_n(x,y)\in C(\FR^2)$, 所以 $\frac{\partial f(x,y)}{\partial x}$ 
    的连续点集是稠密的 $\mathcal{G}_{\delta}$集, 记为 $G_x$;
    同理 $\frac{\partial f(x,y)}{\partial y}$ 的连续点集也是稠密的 $\mathcal{G}_{\delta}$集,
    记为 $G_y$. 令 $G=G_x\cap G_y$, 则 $G$ 也是稠密的 $\mathcal{G}_{\delta}$ 集, 并且 $f$ 在 $G$ 上可微.
\end{proof}



\begin{exercise}
    证明: 完备度量空间中的任何一个可数子集至少含有一个孤立点.
\end{exercise}

\begin{proof}
  Suppose that $(X,d)$ is a complete metric space.
  Without loss of generality, we may assume that $X$ is countable, i.e., $X = \{x_n\}_{n\geq 1}$.

  To prove that $X$ has at least one isolated point, we proceed by contradiction.
  Suppose that $X$ has no isolated point. Let $O_n = X\setminus \{x_n\}$.
  On the one hand, $O_n$ is open since $\{x_n\}$ is closed.
  On the other hand, $O_n$ is dense in $X$ since $\{x_n\}$ is not an isolated point.
  By Baire's theorem we have
  $\bigcap_{n=1}^\infty O_n$ is dense in $X$.
  However,
  \[\bigcap_{n=1}^\infty O_n = \bigcap_{n=1}^\infty X\setminus\{x_n\} = \emptyset,\]
  which is a contradiction.
\end{proof}


\begin{exercise}
  (...)
\end{exercise}

\begin{proof}
  \begin{enumerate}[(a)]
    \item For any $f\in B$, by the definition of $B$ we know that $f$ is differentiable
      at at least one point $x\in [0,1]$ and thus
      \[\lim_{y\to x} \biggl|\frac{f(y)-f(x)}{y-x}\biggr| = |f'(x)|.\]
      Hence there exists some $\delta>0$ such that
      \[\biggl|\frac{f(y)-f(x)}{y-x}\biggr| \leq |f'(x)| + 1\]
      when $|y - x| < \delta$.
      Let $M := \max_{x\in [0,1]} |f|$ and choose $n$ large enough such that
      \[n > \max\biggl(|f'(x)|+1, \frac{2M}{\delta}\biggr).\]
      Then $|f(y)-f(x)| \leq n |y-x|$ for all $y\in [0,1]$, so $f\in F_n$.
    \item Assume that $(f_k)_{k\geq 1}$ is a sequence in $F_n$ and $f_k\to f$.
      We need to prove that $f\in F_n$. For any $f_k$, there exists some $x_k\in [0,1]$
      such that $f_k$ satisfies the $n$-Lipschitz condition at $x_k$.
      By Bolzano-Weierstrass theorem, we may assume that $x_k\to x$
      (upon extracting a subsequence) and next we show that $f$ satisfies the
      $n$-Lipschitz condition at $x$. Indeed, for any $y\in [0,1]$,
      we obtain by the triangle inequality
      \begin{align*}
        |f(x)-f(y)|
        & \leq |f(x)-f_k(x)| + |f_k(x)-f_k(x_k)| + |f_k(x_k)-f_k(y)| + |f_k(y)-f(y)| \\
        & \leq 2\|f-f_k\|_\infty + n|x-x_k| + n|y-x_k|.
      \end{align*}
      Letting $k\to\infty$, we have
      \[|f(x)-f(y)| \leq n|x-y|\quad \forall y\in [0,1].\]
    \item 
      \begin{enumerate}[(i)]
        \item This can be guaranteed by the Bernstein theorem (see Exercise~14 in Chapter~5).
        \item Since $h = P + g$, we have
          \[\|f - h\|_\infty \leq \|f - P\|_\infty + \|g\|_\infty \leq \frac{r}{2} + \frac{r}{4}
            < r.\]
          Hence $h\in B(f,r)$. Now we prove that $h\notin F_n$.
          Suppose by contradiction that $h\in F_n$, then there exists some $x_0\in [0,1]$
          such that $|h(x_0)-h(y)| \leq n|x_0-y|$ for all $y\in [0,1]$, i.e.,
          \[|P(x_0) - P(y) + g(x_0) - g(y)| \leq n|x_0-y| \quad \forall y\in [0,1].\]
          So
          \[\biggl|\frac{g(x_0)-g(y)}{x_0-y}\biggr| \leq
            \biggl|\frac{P(x_0)-P(y)}{x_0-y}\biggr| + n\]
          for all $y\in [0,1]$. Let $y\to x_0+$, we obtain that
          \[\lim_{y\to x_0+} \biggl|\frac{g(x_0)-g(y)}{x_0-y}\biggr| \leq M+n\]
          since $M = \|P'\|_\infty$. However, by the definition of $g$, we have
          \[\lim_{y\to x_0+} \biggl|\frac{g(x_0)-g(y)}{x_0-y}\biggr|
            = \frac{r/4}{1/2N} = \frac{rN}{2} > M + n + 1,\]
          a contradiction.
        \item From (i)--(ii) we know that $F_n$ has empty interior.
      \end{enumerate}
    \item Since $B = E\setminus A \subset \bigcup_{n\geq 1} F_n$, we obtain
      \[A \supset \bigcap_{n\geq 1} F_n^c,\]
      the right-hand side of which is a dense $\mathcal{G}_\delta$ set according to
      Baire's theorem. \qedhere
  \end{enumerate}
\end{proof}


\begin{exercise}
  设 $E$ 和 $F$ 都是 Banach 空间, $(u_n)$ 是 $\mathcal{B}(E,F)$ 的序列.
  证明下列命题等价:
  \begin{enumerate}[(a)]
    \item $(u_n(x))$ 在每个 $x\in E$ 处收敛.
    \item $A\subset E$ 且 $\Span(A)$ 在 $E$ 中稠密, $(u_n(a))$ 在每个 $a\in A$ 处收敛,
    且 $(u_n)$ 有界.
  \end{enumerate}
\end{exercise}

\begin{proof}
    $(a)\Rightarrow(b)$ 是显然的, 下证 $(b)\Rightarrow(a)$,
    即证 $\forall x\in E\setminus A$, $(u_n(x))$ 收敛:
    因为 $(u_n(a))$在每个$a\in A$处收敛, 所以$(u_n(a))$在每个$a\in\Span(A)$ 处收敛,
    因为 $\Span(A)$ 在 $E$ 中稠密, 所以存在 $(x_m)_{m\geq1}\subset\Span(A)$,
    使得 $x_m\to x(m\to\infty)$, 显然 $(x_m)_{m\geq1}$ 是 $E$ 中的 Cauchy 序列. 
    \[\begin{split}
    \lim_{n\to\infty}u_n(x)
    &=\lim_{n\to\infty}u_n\left(\lim_{m\to\infty}x_m\right)\\
    &=\lim_{n\to\infty}\lim_{m\to\infty}u_n(x_m)\\
    &=\lim_{m\to\infty}\lim_{n\to\infty}u_n(x_m)\\
    &=\lim_{m\to\infty}y_m\end{split}\]
    下面我们只需要证明$\lim_{m\to\infty}y_m$存在,结合$F$是Banach空间可知只需要证明$(y_m)$是$F$中Cauchy序列即可,而这个结论通过下式即得:
    \[\begin{split}
    \|y_m-y_k\|
    &=|\lim_{n\to\infty}u_n(x_m)-\lim_{n\to\infty}u_n(x_k)|\\
    &=|\lim_{n\to\infty}u_n(x_m-x_k)|\\
    &\leq\mathop{\textrm{lim inf}}\limits_{n\to\infty}\|u_n\|\cdot\|x_m-x_k\|\left(\mbox{注意}(x_m)_{m\geq1}\mbox{是}\mathrm{Cauchy}\mbox{序列}\right)
    \end{split}\]
\end{proof}


\begin{exercise}
  (...)
\end{exercise}

\begin{proof}
  \begin{enumerate}[(a)]
    \item Since $f$ is Lipschitz function, there exists some positive constant $C>0$
      such that $|f(x)-f(y)|\leq C|x-y|$ for all $x,y\in [0,1]$ and thus
      \begin{align*}
        \left|\frac{u_n(f)}{n}\right|
        & = \left|\int_0^1 f(t) \diff t-\frac{1}{n}\sum_{k=1}^nf\left(\frac{k}{n}\right)\right| \\
        & = \left|\sum_{k=1}^n\int_{\frac{k-1}{n}}^{\frac{k}{n}} f(t) \diff t
            - \frac{1}{n}\sum_{k=1}^nf\left(\frac{k}{n}\right)\right| \\
        & \leq \frac{1}{n}\sum_{k=1}^n\left|f(\xi_k)-f\left(\frac{k}{n}\right)\right|
            \qquad \text{for some } \xi_k\in\left(\frac{k-1}{n},\frac{k}{n}\right) \\
        & \leq \frac{1}{n}\sum_{k=1}^nC\left|\xi_k-\frac{k}{n}\right| \\
        & \leq \frac{1}{n}\sum_{k=1}^nC\cdot\frac{1}{n}=\frac{C}{n},
      \end{align*}
      Therefore
      \[\frac{u_n(f)}{n}=O\left(\frac{1}{n}\right)\quad\text{as } n\to\infty.\]
    \item Note here we should regard $n$ as a fixed integer. One the one hand,
      \begin{align*}
        |u_n(f)|
        & = \left|n\int_0^1f(t)\diff t-\sum_{k=1}^nf\left(\frac{k}{n}\right)\right| \\
        & \leq n\int_0^1|f(t)|\diff t+\sum_{k=1}^n\left|f\left(\frac{k}{n}\right)\right| \\
        & \leq n\|f\|_{\infty}+n\|f\|_{\infty} = 2n\|f\|_{\infty}.
      \end{align*}
      On the other hand, construct a sequence of functions $(f_m)$ as follows:
      \[f_m = \begin{cases}
        -mx+\frac{km}{n}-1, & \text{for } \frac{k}{n}-\frac{2}{m}\leq x\leq\frac{k}{n},\, 1\leq k\leq n \\
        mx-\frac{km}{n}-1,  & \text{for } \frac{k}{n}\leq x\leq\frac{k}{n}+\frac{2}{m},\, 1\leq k\leq n-1.
      \end{cases}\]
      Then $f_m\in E$, $\|f_m\|_\infty = 1$ and $f_m(x) = -1$ at $x = \frac{k}{n}$
      for $1\leq k\leq n$. By the construction of $f_m$ we obtain that
      \[|u_n(f_{m})| = \left|n\int_0^1 f_m(t) \diff t
        - \sum_{k=1}^n f_{m}\left(\frac{k}{n}\right)\right|
        = \left|2n-\frac{2n(2n-1)}{m}\right|\to 2n,\]
      as $m\to\infty$.
    \item Since $E$ is a Banach space and $\sup_{n\geq 1} \|u_n\| = \infty$,
     we have by Theorem 6.2.4 that
     \[G=\Bigl\{f\in E:\sup_{n\geq 1}|u_n(f)|=+\infty\Bigr\}\]
     is a dense $\mathcal{G}_\delta$ set in $C([0,1])$. \qedhere
  \end{enumerate}
\end{proof}


\begin{exercise}[8]
    设 $E$ 是 $(C([0,1]),\|\cdot\|_{\infty})$ 的闭向量子空间, 并假设 $E$ 中的元素都是 Lipschitz 函数.

    (a) 设 $x,y\in [0,1]$ 且 $x\neq y$, 定义泛函 $\varPhi_{x,y}:E\to\FR$ 为
    \[\varPhi_{x,y}(f)=\frac{f(y)-f(x)}{y-x}.\]
    证明: $\{\varPhi_{x,y}\mid x,y\in [0,1],x\neq y\}$ 是 $E^*$ 中的有界集.

    (b) 导出 $E$ 中闭单位球在 $[0,1]$ 上等度连续, 且 $\dim E<\infty$.
\end{exercise}

\begin{proof}
    (a) 因为完备度量空间的闭子空间完备, 所以 $E$ 是 Banach 空间,
    容易验证 $\{\varPhi_{x,y}\}\subset E^*$, 又因为对任意 $f\in E$, 有
    \[\sup_{x,y\in[0,1],x\neq y}|\varPhi_{x,y}(f)|=\sup_{x,y\in[0,1],x\neq y}\left|\frac{f(y)-f(x)}{y-x}\right|\leq C.\]
    这里的 $C$ 是函数 $f$ 的 Lipschitz 常数, 故由 Banach-Steinhaus 定理知
    \[\sup_{x,y\in[0,1],x\neq y}\|\varPhi_{x,y}\|<\infty.\]
    也即 $\{\varPhi_{x,y}\mid x,y\in[0,1],x\neq y\}$ 是 $E^*$ 中的有界集.

    (b) 记 $E$ 中的闭单位球为 $\closure{B_E}$, 则由 (a) 中结论知
    \[\sup_{x,y\in[0,1],x\neq y}\sup_{f\in \closure{B_E}}\|\varPhi_{x,y}(f)\|<\infty.\]
    即
    \[\sup_{x,y\in[0,1],x\neq y}\sup_{f\in \closure{B_E}}\left|\frac{f(y)-f(x)}{y-x}\right|<\infty.\]
    这说明 $\closure{B_E}$ 在 $[0,1]$ 上一致等度连续, 故必然等度连续.
    又对任意 $x\in[0,1]$, $\closure{B_E}$ 的轨道
    \[\closure{B_E}(x)=\{f(x):f\in \closure{B_E}\}=\{f(x):\max_{0\leq x\leq 1}|f(x)|=1\}\]
    有界, 故由 Ascoli 定理知 $\closure{B_E}$ 在 $E$ 中相对紧, 从而紧, 根据 Riesz 引理知 $\textrm{dim}E<\infty$.
\end{proof}


\begin{exercise}
  Assume $E$ is Banach space and $u,v\in\mathcal{B}(E)$. Prove that
  if $u(E)\subset v(E)$, then there exists some constant $k\geq 0$
  such that for all $x\in E$, there exists $y\in E$ such that
  $\|y\|\leq k\|x\|$ and $u(x) = v(y)$.
\end{exercise}

\begin{proof}
  Since $u(E)\subset v(E)$, for any $x\in E$, $u(x)\in v(E)$. Thus
  there exists some $y\in E$ such that $u(x) = v(y)$.
\end{proof}


\begin{exercise}[10]
    设 $E,F$ 都是 Banach 空间, $u\in\mathcal{B}(E,F)$ 并满足 $u(B_E)$ 在 $B_F$ 中稠密.

    (a) 计算 $\|u\|$.

    (b) 证明: $u(B_E)=B_F$. 因此 $u$ 是满射.

    (c) 设 $v\in B(E/\ker u,F)$ 并满足 $v\circ q=u$, 这里 $q:E\to E/\ker u$
    是商映射. 证明: $v$ 是从 $E/\ker u$ 到 $F$ 上的等距映射.
\end{exercise}

\begin{proof}
    (a) 因为 $u(B_E)$ 在 $B_F$ 中稠密,
    所以 $B_F\subset\overline{u(B_E)}=\overline{B_F}$,
    又由 $u$ 连续知 $u(\overline{B_E})\subset\overline{u(B_E)}$, 故
    \[\|u\|=\sup_{x\in\overline{B_E}}\|u(x)\|=\sup_{u(x)\in u(\overline{B_E})}\|u(x)\|\leq\sup_{u(x)\in \overline{u(B_E)}}\|u(x)\|=\sup_{u(x)\in\overline{B_F}}\|u(x)\|=1.\]
    对任意 $\varepsilon>0$, 存在 $y\in B_F$,
    使得 $\|y\|\geq 1-\varepsilon$, 对于上述 $y\in B_F$, 存在 $x\in B_E$,
    使得 $\|u(x)-y\|\leq\varepsilon$, 故
    \[\|u(x)\|\geq\|y\|-\|u(x)-y\|\geq 1-2\varepsilon.\]
    由 $\varepsilon$ 的任意性知 $\|u\|=1$.

    (b) 由条件知, $u(B_{E})\subset B_{F}$ 且 $B_{F}\subset\closure{u\left(B_{E}\right)}$. 
    我们采用和教材中开映射定理 6.3.1 类似的证明过程, 首先任取常数 $0<\delta<1$, 对任意 $y\in B_{F}$, 取 $x_0\in B_{E}$, 使得
    \[
    \|y-u(x_0)\|<\delta.
    \] 
    并设 $y_{1}=\frac{1}{\delta}(y-u(x_{0}))$, 则 $y_1\in B_F$. 再取 $x_1\in B_E$, 使得
    \[
    \|y_{1}-u(x_{1})\|<\delta .
    \]
    再设 $y_2=\frac{1}{\delta}(y_1-u(x_1))$, 则 $y_2\in B_F$. 依次下来, 
    可得一列 $(y_n)_{n\geq 1}\subset B_F$ 及相应序列 $(x_n)_{n\geq 1}\subset B_E$, 满足
    \[
    y_{n+1}=\frac{1}{\delta}(y_n-u(x_n)),\|y_{n}-u(x_{n})\|<\delta, \quad n\geq 1.
    \]
    由以上构造过程, 可得
    \begin{equation}
        \begin{aligned}
            y &=\delta y_{1}+u(x_{0})=\delta^{2} y_{2}+u(x_{0})+\delta u(x_{1})=\cdots \cdots \\
            &=\delta^{n+1} y_{n+1}+u(x_{0})+\delta u(x_{1})+\delta^{2} u(x_{2})+\cdots+\delta^{n} u(x_{n}) \\
            &=\delta^{n+1} y_{n+1}+u\biggl(\sum_{k=0}^{n} \delta^{k} x_{k}\biggr).
        \end{aligned}\tag{$\star$}
    \end{equation}
    在上式中, $\sum_{k=0}^n \delta^k x_k$ 在 $n\to\infty$ 时收敛于某一点 $x\in E$, 且
    \[\|x\|\leq\sum_{n=1}^{\infty}\delta^n\|x_n\|<\frac{1}{1-\delta}.\]
    在 $(\star)$ 式中取 $n\to\infty$, 得 $y=u(x)$, 故 $B_F\subset u(\frac{1}{1-\delta}B_E)$.
    由 $u$ 的线性性, 有 $(1-\delta)B_F\subset u(B_E)$.
    任取 $y\in B_F$, 总可取到 $0<\delta<1$ 使得 $1-\delta>\|y\|$, 故
    $y\in u(B_E)$, 从而 $B_F\subset u(B_E)$. 这样就证明了 $u(B_E)=B_F$.

    (c) 对任意 $x\in E$, 用 $[x]$ 表示以 $x$ 为代表元的等价类. 由定义可知, 若 $[x]=[y]$, 则
    $u(x-y)=0$. 而且, 由于 $u$ 是连续的, 则 $\ker u$ 是 $E$ 的闭向量子空间, $E/\ker u$ 自然
    成为一个赋范空间, 其上的范数 $\|\cdot\|$ 约定为
    \[\|[x]\|=\inf_{y\in\ker u}\|x+y\|=\inf_{y\in[x]}\|y\|.\]
    由于 $v\in\mathcal{B}(E/\ker u, F)$ 满足 $v\circ q=u$, 则 $v([x])=u(x)$, $\forall x\in E$. 
    因 $u$ 是满射, 故 $v$ 也是满射; 而 $[x]\neq[y]$ 等价于 $u(x)\neq u(y)$, 故 $v$ 也是单射. 实际上, 由开
    映射定理立即得到, $v$ 是 $E/\ker u$ 到 $F$ 的线性同构映射.

    任取 $[x]\in E/\ker u$, 设 $y=v([x])$, 则也有 $y=u(x)$. 那么由 $u(B_{E})=B_{F}$, 可
    知对任意 $0<\varepsilon<1$, 存在 $x_{0}\in B_{E}$, 使得 
    $u(x_0)=\varepsilon \frac{y}{\|y\|}$, 则又有 $y=u(\varepsilon^{-1}\|y\| x_{0})$.
    于是得 $u(x-\varepsilon^{-1}\|y\| x_{0})=0$, 这表明 $\varepsilon^{-1}\|y\| x_{0}\in [x]$. 因此
    \[
    \|[x]\| \leq\bigl\|\varepsilon^{-1}\|y\|x_{0}\bigr\|\leq\varepsilon^{-1}\|y\|=\varepsilon^{-1}\|v([x])\| .
    \]
    由 $\varepsilon$ 的任意性, 即得 $\|[x]\|\leq\|v([x])\|$.

    另一方面, 对任意 $[x]\in E/\ker u$, 任取 $[x]$ 的代表元 $y$, 都有
    \[
    \|v([x])\|=\|u(y)\| \leq\|u\|\cdot\|y\|=\|y\|.
    \]
    对上式右边所有代表元的范数取下确界, 即得
    \[\|v([x])\|\leq\inf_{y\in [x]}\|y\|=\|[x]\|.\]
    综合以上讨论, 我们证明了 $v$ 是从 $E/\ker u$ 到 $F$ 上的等距同构映射.

    \textbf{另一种更直接的证明}:
    任取 $x\in B_E$, 则有 $\|[x]\|\leq\|x\|<1$, 即
    \[q(B_E)\subset B_{E/\ker u}.\]
    反过来, 任取 $[x]\in B_{E/\ker u}$, 则必定存在代表元 $y\in[x]$, 
    使得 $\|[x]\|\leq\|y\|<1$. 于是得 $y\in B_{E}$, 满足 $[x]=q(y)\in q(B_{E})$, 也就有
    \[
    B_{E/\ker u}\subset q(B_{E}).
    \]
    因此, 我们得到 $B_{E/\ker u}=q(B_{E})$. 再由 $u(B_{E})=B_{F}$, 以及 $u=v \circ q$, 立即得到
    \[
    v(B_{E/\ker u})=B_{F}.
    \]
    而且, 因 $v\in E/\ker u\to F$ 是同构映射, 故也有
    \[
    v^{-1}(B_{F})=B_{E/\ker u}.
    \]
    由以上两式立即得到 $\|v\|=\|v^{-1}\|=1$. 故 $v$ 是从 $E/\ker u$ 到 $F$ 上的等距同构映射.
\end{proof}


\begin{exercise}
  (...)
\end{exercise}

\begin{proof}
\begin{enumerate}[(a)]
  \item By definition
    \[\|q(g)-f\|_{\infty}=\sup_{y\in Y}|q(g)(y)-f(y)|
      = \sup_{y\in Y}\left|\frac{d(y,B)-d(y,A)}{3[d(y,B)+d(y,A)]}-f(y)\right|.\]
    We proceed in three cases as follows:
    \begin{enumerate}[(i)]
      \item For $y\in A$,
        \[\|q(g)-f\|_{\infty}=\sup_{y\in A}\left|\frac{d(y,B)}{3d(y,B)}-f(y)\right|=\sup_{y\in A}\left|\frac{1}{3}-f(y)\right|\leq\frac{2}{3}.\]
      \item For $y\in B$,
        \[\|q(g)-f\|_{\infty}=\sup_{y\in B}\left|\frac{-d(y,A)}{3d(y,A)}-f(y)\right|=\sup_{y\in B}\left|-\frac{1}{3}-f(y)\right|\leq\frac{2}{3}.\]
      \item For $y\in Y\setminus (A\cup B)$, since
        \[-\frac{1}{3}<f(y)<\frac{1}{3}
          \quad\text{and}\quad -\frac{1}{3}<\frac{d(y,B)-d(y,A)}{3[d(y,B)+d(y,A)]}<\frac{1}{3}.\]
        It follows that
        \[\|q(g)-f\|_{\infty} < 2/3.\]
    \end{enumerate}
    Note that the above discussions involve the special cases when $A$ or $B$
    is emptyset.
  \item For any $f\in F$, let
    \[C = \biggl\{y\in Y:f(y)\geq\frac{1}{3}\|f\|_{\infty}\biggr\},\quad 
      D = \left\{y\in Y:f(y)\leq-\frac{1}{3}\|f\|_{\infty}\right\}.\]
    Define
    \[g(x) := \frac{d(x,D)-d(x,C)}{3[d(x,D)+d(x,C)]}\cdot\|f\|_{\infty}.\]
    Then $\|g\|_{\infty}\leq\frac{1}{3}\|f\|_{\infty}$ and from (a) we have that
    $\|q(g)-f\|_{\infty}\leq\frac{2}{3}\|f\|_{\infty}$.
  \item For the $f$ and $g$ that satisfy the conditions in (b), we have
    \[\frac{2}{3}\|f\|_{\infty}\leq\|f\|_{\infty}-\|g\|_{\infty}
      \leq\|f\|_{\infty}-\|q(g)\|_{\infty}\leq\|f-q(g)\|_{\infty}\leq\frac{2}{3}\|f\|_{\infty}.\]
    Thus
    \[\|g\|_{\infty} = \frac{1}{3}\|f\|_{\infty}
      \text{ and }
      \|f-q(g)\|_{\infty } =\frac{2}{3}\|f\|_{\infty}.\]
    We use the induction method to generate a sequence of functions $(g_n)$ as follows:

    There exists some $g_1\in E$ such that
    \begin{align*}
      \|g_1\|_{\infty} & = \frac{1}{3}\|f\|_{\infty}, \\
      \|f-q(g_1)\|_{\infty} & = \frac{2}{3}\|f\|_{\infty}.
    \end{align*}

    There exists some $g_2\in E$ such that
    \begin{align*}
      \|g_2\|_{\infty}=\frac{1}{3}\|f-q(g_1)\|_{\infty}
        & = \frac{1}{3}\cdot\frac{2}{3}\|f\|_{\infty}, \\
      \|f-q(g_1)-q(g_2)\|_{\infty}=\frac{2}{3}\|f-q(g_1)\|_{\infty}
        & = \left(\frac{2}{3}\right)^2\|f\|_{\infty}.
    \end{align*}

    At the $n$-th step, there exists some $g_n\in E$ such that
    \begin{align*}
      \|g_n\|_{\infty}
        & = \frac{1}{3}\left\|f-\sum_{k=1}^{n-1}q(g_k)\right\|_{\infty}=\frac{1}{3}\left(\frac{2}{3}\right)^{n-1}\|f\|_{\infty}, \\
      \left\|f-\sum_{k=1}^nq(g_k)\right\|_{\infty}
        & = \frac{2}{3}\left\|f-\sum_{k=1}^{n-1}q(g_k)\right\|_{\infty}=\left(\frac{2}{3}\right)^n\|f\|_{\infty}.
    \end{align*}
    So we obtain a sequence $(g_n)_{n\geq 1}\subset E$ which satisfies two properties:
    \[\text{(i) } \|g_n\|_{\infty}=\frac{1}{3}\left(\frac{2}{3}\right)^{n-1}\|f\|_{\infty}
      \quad\text{and}\quad
      \text{(ii) } \left\|f-\sum_{k=1}^nq(g_k)\right\|_{\infty}=\left(\frac{2}{3}\right)^n\|f\|_{\infty}.\]
    From property (i) we know that $\sum_{n=1}^{\infty}\|g_n\|_{\infty}$ converges,
    so $\sum_{n=1}^{\infty}g_n$ converges to some $g\in E$ since $E$ is complete.

    From property (ii) we know that $\sum_{n=1}^{\infty}q(g_n)$ converges uniformly to $f$.

    Since $q$ is continuous, we have
    \[\sum_{n=1}^{\infty}q(g_n)=q\left(\sum_{n=1}^{\infty}g_n\right)\to q(g),\]
    and hence $f = q(g)$. Finally we prove that $\|f\|_\infty = \|g\|_\infty$.
    To this end, note that
    \[\|g\|_{\infty}=\left\|\sum_{n=1}^{\infty}g_n\right\|
      \leq \sum_{n=1}^{\infty}\|g_n\|_{\infty}
      = \sum_{n=1}^{\infty}\frac{1}{3}\left(\frac{2}{3}\right)^{n-1}\|f\|_{\infty}
      = \|f\|_{\infty},\]
    and $\|f\|_{\infty}=\|q(g)\|_{\infty}\leq\|g\|_{\infty}$.
  \item Obvious from (c).
  \item Choose arbitrarily $f\in C(Y,\FR)$,
    \begin{itemize}
      \item If $f$ is bounded, by (c) there exists some $g\in C(X,\FR)$ such that $q(g)=f$.
      \item If $f$ is unbounded, let $f_1 = \arctan f\in F$.
        There exists some $g_1\in E$ such that $q(g_1) = f_1$.
        Let $g = \tan g_1$, then
        \[q(g)=q(\tan g_1)=\tan(q(g_1))=\tan(f_1) = f.\qedhere\]
    \end{itemize}
\end{enumerate}
\end{proof}

\begin{exercise}[13]
  (...)
\end{exercise}

\begin{proof}
  \begin{enumerate}[(a)]
    \item 假设$F$在$E$中的内部不是空集,则存在$x\in F,r>0$使得$B(x,r)\subset F$,这里的$B(x,r)$是$E$中的开球,由$F$是向量子空间可得$B(0,r)\subset F\Rightarrow B(0,n)\subset F$ $(\forall n)$, 从而
    \[E=\bigcup_{n\geq 1}B(0,n)\subset F\Rightarrow E=F\]
    矛盾,故假设不成立,即证$F$在$E$中的内部为空集.
    \item 记所有多项式构成的空间为$\mathcal{P}$,所有次数不超过$n$的多项式构成的空间为$\mathcal{P}_n$,则
    \[\mathcal{P}=\bigcup_{n\geq 1}\mathcal{P}_n.\]
    假设$\mathcal{P}$上有完备范数,由(a)知$\mathcal{P}_n^{\circ}=\emptyset$,由Baire定理知$\mathcal{P}^{\circ}=\emptyset$,矛盾,故假设不成立,所以$\mathcal{P}$不能赋予完备范数. \qedhere
  \end{enumerate}
\end{proof}


\begin{exercise}[14]
    设 $E$ 是 Banach 空间, $F$ 和 $G$ 都是 $E$ 的闭向量子空间, 并且 $F+G$
    也是闭向量子空间. 证明: 存在一个常数 $C\geq 0$, 使得 $\forall x\in F+G$,
    存在 $(f,g)\in F\times G$, 满足
    \[x=f+g,\;\|f\|\leq C\|x\|,\;\|g\|\leq C\|x\|.\]
\end{exercise}

\begin{proof}
    考虑乘积 Banach 空间 $F\times G$ (赋予范数 $\|(f,g)\|=\|f\|+\|g\|$)
    和 Banach 空间 $F+G$ (范数即为 $E$ 中范数). 映射
    \[u:F\times G\to F+G,\;(f,g)\mapsto f+g\]
    为连续线性的满射, 由开映射定理, $u(B_{F\times G}(0,1))$ 为 $F+G$ 中含原点的开集,
    取常数 $c>0$, 使得 $B_{F+G}(0,c)\subset u(B_{F\times G}(0,1))$.
    则对于任意 $x\in F+G$ 且 $\|x\|<c$, 存在 $f\in F$, $g\in G$
    且 $\|f\|+\|g\|<1$, 使得 $x=f+g$.

    对于一般的 $x\in F+G$, 任取 $0<c'<c$, 由于 $x=\frac{\|x\|}{c'}\bigl(\frac{c'}{\|x\|}x\bigr)$,
    其中 $\left\|\frac{c'}{\|x\|}x\right\|=c'<c$, 故存在 $f'\in F$, $g'\in G$,
    使得 $\frac{c'}{\|x\|}x=f'+g'$ 且 $\|f'\|+\|g'\|<1$.
    令 $f=\frac{\|x\|}{c'}f'$, $g=\frac{\|x\|}{c'}g'$, 则
    $x=f+g$ 且
    \[\|f\|+\|g\|=\frac{\|x\|}{c'}\bigl(\|f'\|+\|g'\|\bigr)<\frac{1}{c'}\|x\|.\]
    由 $c'$ 的任意性即得 $\|f\|+\|g\|\leq\frac{1}{c}\|x\|$.
    再令 $C=\frac{1}{c}$ 即证所需.
\end{proof}



\begin{exercise}
    设 $H$ 是 Hilbert 空间, 且线性映射 $u:H\to H$ 满足
    \[\innerp{u(x)}{y}=\innerp{x}{u(y)},\quad\forall x,y\in H.\]
    证明: $u$ 连续.
\end{exercise}

\begin{proof}
    考虑线性泛函
    \[f_x\colon H\to\mathbb{K},\quad y\mapsto\langle u(y),u(x)\rangle.\]
    记 $H$ 中的闭单位球为 $\closure{B}_H$, 对于任意 $y\in H$,由 Cauchy-Schwarz 不等式有
    \[\sup_{x\in\closure{B}_H}|f_x(y)|
      = \sup_{x\in\closure{B}_H}|\innerp{u(y)}{u(x)}|
      = \sup_{x\in \closure{B}_H}|\innerp{u(u(y))}{x}|\leq\|u(u(y))\|<\infty.\]
    故由 Banach-Steinhaus 定理知
    \[\sup_{x\in \closure{B}_H}\|f_x\|<\infty.\]
    即
    \[\sup_{x\in \closure{B}_H}\sup_{y\in \closure{B_H}}|\langle u(y),u(x)\rangle|<\infty.\]
    因此
    \[\|u\|^2=\sup_{x\in\closure{B}_H}\|u(x)\|^2
      = \sup_{x\in \closure{B}_H}\langle u(x),u(x)\rangle < \infty.\]
    从而 $u$ 为有界算子, 亦即为连续算子.
\end{proof}
% 18.\textit{Proof}:(a)显然$u$是线性的,且当$\|(a_n)_{n\geq 1}\|\leq 1$时,有$\|u((a_n)_{n\geq 1})\|=\|\sum_{n\geq 1}a_nx_n\|\leq\sum_{n\geq 1}|a_n|\cdot\|x_n\|\leq\sum_{n\geq 1}|a_n|\leq 1$
% 因此$\|u\|\leq 1$,从而$u\in\mathcal{B}(\ell_1,E)$\\
% (b)由(a)知$u(B_{\ell_1})\subset \closure{B_E}$\\
% 任取$y\in \closure{B_E}$,则$d(y,\frac{1}{2}\closure{B_E})<\frac{1}{2}$,故存在$n_1\geq 2$使得$\|y-\frac{1}{2}x_{n_1}\|<\frac{1}{2}$\\
% 令$y_1=y-\frac{1}{2}x_{n_1}$,则存在$n_2>n_1$使得$\|2y_1-\frac{1}{2}x_{n_2}\|<\frac{1}{2}$\\
% 令$y_2=y_1-\frac{1}{4}x_{n_2}$,则存在$n_3>n_2$使得$\|4y_2-\frac{1}{2}x_{n_3}\|<\frac{1}{2}$\\
% $\cdots\cdots$\\
% 这样就得到一列$(x_{n_k})_{k\geq 1}$及$(y_k)_{k\geq 1}$使得
% \[y=\sum_{k=1}^n\frac{1}{2^k}x_{n_k}+y_k\]
% 因为$\|2^ky_k-\frac{1}{2}x_{n_{k+1}}\|<\frac{1}{2}\Rightarrow\|y_k-\frac{1}{2^{k+1}}x_{n_{k+1}}\|<\frac{1}{2^{k+1}}$,所以$y_k\to0(k\to\infty)$,故\[y=\sum_{k=1}^{\infty}\frac{1}{2^k}x_{n_k}\]
% 记$a_{n_k}=\frac{1}{2^k}$,对于其他$j\in\mathbb{N}^{*}\backslash\{n_k\}_{k=1}^{\infty}$,令$a_j=0$
% ,则$(a_n)_{n\geq 1}\in\ell_1$,并且$y=\sum_{n\geq 1}a_nx_n$,从而$u(B_{\ell_1})=\closure{B_E}$\\\\
% 下面证明$u(\overline{B_{\ell_1}})=\closure{B_E}$:\\
% 因为$u(B_{\ell_1})=\closure{B_E}$,所以$u(\overline{B_{\ell_1}})\supset \closure{B_E}$,故只需要说明$u(\overline{B_{\ell_1}})\subset \closure{B_E}$,任取$(a_n)_{n\geq 1}\in\overline{B_{\ell_1}}$,有:$\|u((a_n)_{n\geq 1})\|=\|\sum_{n\geq 1}a_nx_n\|\leq\sum_{n\geq 1}|a_n|\cdot\|x_n\|<\sum_{n\geq 1}|a_n|\leq 1$,故$u(\overline{B_{\ell_1}})\subset \closure{B_E}$\\
% (c)\begin{enumerate}[(i)]
% \item 由(a)(b)中结论及$\ell_p$可分知存在$u\in\mathcal{B}(\ell_1,\ell_p),s.t.u(B_{\ell_1})=B_{\ell_p}$,显然$u$是满射
% \item 假设存在连续线性映射$v:\ell_p\to\ell_1,s.t.u\circ v=id$,因为$v$是单射,所以$\ell_p$与$v(\ell_p)$同构,又因为$\ell_1$与$v(\ell_p)$同构,故$\ell_1$于$\ell_p$同构,矛盾
% \end{enumerate}
\chapter{拓扑向量空间}
\thispagestyle{empty}
% 设$(x,y)\in\FR^2$,令$p(x,y)=|x|$,则$p$是半范数;设$f\in C(\FR)$,令$p_N(f)=\sup\limits_{x\in [-N,N]}|f(x)|$,则$p_N$是半范数\\\\
% 怎样证明定理7.2.6中半范数族$(p_i)_{i\in I}$诱导的拓扑$\tau$与$E$的线性结构相容?


% 1.\textit{Proof}:
% (a)容易验证$d$满足正定性和对称性,故只需要说明$d$满足三角形不等式:即证明对于任意$f,g,h\in C(\FR,\FR)$,有
% \[\begin{split}\min\left\{1,\sup_{x\in\FR}|f(x)-g(x)|\right\}\leq\min\bigg\{&1,\sup_{x\in\FR}|f(x)-h(x)|\bigg\}+\min\left\{1,\sup_{x\in\FR}|g(x)-h(x)|\right\}\\
% =\min\bigg\{&2,\sup_{x\in\FR}|f(x)-h(x)|+1,\sup_{x\in\FR}|g(x)-h(x)|+1,\\
% &\sup_{x\in\FR}|g(x)-h(x)|+\sup_{x\in\FR}|g(x)-h(x)|\bigg\}\end{split}\]
% 将$\min\{1,\sup_{x\in\FR}|f(x)-g(x)|\}$与右侧括号里面四项逐一比较知此不等式显然成立\\
% 下面证明距离$d$是完备的:取$C(\FR,\FR)$中的柯西序列$(f_n)_{n\geq 1}$,即
% \[\forall\varepsilon>0(\mbox{不妨设}\varepsilon<1),\exists N>0,\forall m,n>N,d(f_m,f_n)<\varepsilon\]
% 从而\[\sup_{x\in\FR}|f_m(x)-f_n(x)|<\varepsilon(m,n>N)\]
% 那么对任意给定的$x\in\FR$,$(f_n(x))_{n\geq 1}$是$\FR$中的Cauchy序列,从而必然收敛,记为
% \[f_n(x)\to f(x)\in\FR(n\to\infty)\]
% 这样就得到一个映射$f:\FR\to\FR$,由$f(x)-f(y)=\lim_{n\to\infty}(f_n(x)-f_n(y))$知$f\in C(\FR,\FR)$,又$d(f_n,f)=\min\{1,\sup_{x\in\FR}|f_n(x)-f(x)|\}\to0(n\to\infty)$,故$f_n\to f(n\to\infty)$,所以$d$是完备的\\
% (b)\\\\

\begin{exercise}[2]
    设 $E$ 是拓扑向量空间, $A,B\subset E$.

    (a) 证明: 若 $A$ 是开集, 则 $A+B$ 也是开集.

    (b) 证明: 若 $A$ 和 $B$ 是紧的且 $E$ 是一个 Hausdorff 空间, 则 $A+B$ 也是紧的.

    (c) 构造 $\FR^2$ 上的例子, 说明 $A$ 和 $B$ 是闭集, 但 $A+B$ 不是闭集.
\end{exercise}

\begin{proof}
    (a) 因为
    \[A+B=\bigcup_{y\in B}(A+y).\]
    所以 $A+B$ 是开集.

    (b) $\varPhi:(x,y)\mapsto x+y$ 是连续映射, 因$A,B$紧, 故 $A\times B$紧, 故 $A+B=\varPhi(A\times B)$紧.

    (c) 取 $A=\{(x,0)\mid x\in\FR\}$, $B=\{xy=1\mid x>0\}$,
    则 $A$ 和 $B$ 都是 $\FR^2$ 中闭集, 但是 $A+B$ 不是闭集.
    事实上, $A+B$ 中序列
    \[(-n,0)+(n,\frac{1}{n})=(0,\frac{1}{n})\to (0,0),\]
    但是 $(0,0)\notin A+B$, 因此 $A+B$ 不是闭集.
\end{proof}



\begin{exercise}
    设 $E$ 是拓扑向量空间, $f$ 是 $E$ 到 $F$ 的线性泛函 ($f$ 不恒为 $0$). 
    并假设 $H=f^{-1}(0)$ 是闭集. 本题的目的是证明在该假设下 $f$ 是连续的.

    (a) 证明: 存在元素 $a\in E$, 使得 $f(a)=1$.

    (b) 证明: $E\setminus f^{-1}(1)$ 是含有原点的开集.

    (c) 设 $V$ 是包含于 $E\setminus f^{-1}(1)$ 的原点处的平衡邻域. 证明: $|f|$ 在 $V$ 上被 $1$ 严格控制, 进而导出 $f$ 连续.
\end{exercise}

\begin{proof}
    (a) 由于 $f$ 不恒为 $0$, 故存在 $x\in E$, 使得 $f(x)\neq 0$.
    取 $a=\frac{x}{f(x)}\in E$, 则 $f(a)=1$.

    (b)显然 $E\setminus f^{-1}(1)$包含原点,
    下证其为开集. 任取 $x\in E\setminus f^{-1}(1)$, 则 
    \[f(x)\neq 1\Rightarrow f(x-a)\neq 0\Rightarrow x-a\in E\setminus f^{-1}(0).\]
    而 $E\setminus f^{-1}(0)$ 为开集, 故存在开集 $U$ 使得 
    \[x-a\in U\subset E\setminus f^{-1}(0).\]
    那么 $a+U$ 也为开集且
    \[x\in a+U\subset E\setminus f^{-1}(1).\]
    因此 $E\setminus f^{-1}(1)$ 为开集.

    (c) (反证法) 假设存在 $x\in V$, 使得 $|f(x)|=\lambda>1$,
    则由 $V$ 是平衡的可得 $\frac{x}{\lambda}\in V$ 且 $|f(\frac{x}{\lambda})|=1$,
    这与 $V\subset E\setminus f^{-1}(1)$ 相矛盾.
    因此在 $V$ 上, $|f|<1$, 由推论 7.1.11 知 $f$ 连续.
\end{proof}




% 4.\textit{Proof}:(a)记$B=\{\sum_{k=1}^nt_ka_k:a_k\in A,t_k\geq 0,\sum_{k=1}^nt_k=1,n\geq1\},C=\bigcup_{\lambda\in\mathbb{K},|\lambda|\leq1}\lambda A$,先证明$\conv(A)=B$:
% \begin{enumerate}[(i)]
% \item $B$是凸集:$\forall x,y\in B$,设
% \[x=\sum_{k=1}^mt_{k1}a_{k1},y=\sum_{k=1}^nt_{k2}a_{k2}\left(\mbox{其中}\sum_{k=1}^mt_{k1}=\sum_{k=1}^nt_{k2}=1\right)\]
% 则对于任意$\lambda\in [0,1]$
% \[\lambda x+(1-\lambda)y=\sum_{k=1}^m\lambda t_{k1}a_{k1}+\sum_{k=1}^n(1-\lambda)t_{k2}a_{k2}\]
% 因为$\sum_{k=1}^m\lambda t_{k1}+\sum_{k=1}^n(1-\lambda)t_{k2}=1$,所以$\lambda x+(1-\lambda)y\in B$,故$B$是凸集
% \item $B$是包含$A$的最小凸集:设$B'$是包含$A$的任意一个凸集,下面对指标$n$用数学归纳法证明$B\subset B'$:\\
% 当$n=1$时,显然$\sum_{k=1}^nt_ka_k=a_1\in B'$\\
% 假设当$n=m$时结论成立,即$\sum_{k=1}^mt_ka_k\in B'$,其中$\sum_{k=1}^mt_k=1$,则当$n=m+1$时:\\
% 因为$\sum_{k=1}^{m+1}t_k=1$,所以$\sum_{k=1}^m\frac{t_k}{1-t_{m+1}}=1$,由假设条件知
% \[\sum_{k=1}^m\frac{t_ka_k}{1-t_{m+1}}\in B'\]
% 即存在$a_0\in B'$使得\[a_0=\sum_{k=1}^m\frac{t_ka_k}{1-t_{m+1}}\Rightarrow\sum_{k=1}^mt_ka_k=(1-t_{m+1})a_0\]
% 结合$B'$是凸集知\[\sum_{k=1}^{m+1}t_ka_k=\sum_{k=1}^mt_ka_k+t_{m+1}a_{m+1}=(1-t_{m+1})a_0+t_{m+1}a_{m+1}\in B'\]
% \end{enumerate}
% 再证明$ba(A)=C$:
% \begin{enumerate}[(i)]
% \item $C$是平衡集:对任意$|\mu|\leq1$\[\mu C=\bigcup_{|\lambda|\leq1}\lambda\mu A\subset\bigcup_{|\lambda|\leq1}\lambda A=C\]
% \item $C$是包含$A$的最小的平衡集:显然
% \end{enumerate}
% (b)设$A$是平衡集,则对于任意$x=\sum_{k=1}^nt_ka_k\in\conv(A),|\lambda|\leq1$,有
% \[\lambda x=\lambda\sum_{k=1}^nt_ka_k=\sum_{k=1}^nt_k(\lambda a_k)\in\conv(A)\]
% 故$\conv(A)$是平衡集\\
% (c)考虑凸集$\{(x,y)|x^2+y^2=1,x\geq0,y\geq0\}$,显然其平衡包是$\{(x,y)|x^2+y^2=1,xy\geq0\}$,其不是凸集\\\\


\begin{exercise}[5]
    设 $\varOmega$ 表示开圆盘 $\{z\in\FC\mid |z|<3\}$, $K$ 表示闭单位圆盘 $\{z\in\FC\mid |z\leq 1|\}$.
    对 $f\in H(\varOmega)$ 定义
    \[p(f)=\sup_{z\in K}|f(z)|.\]

    (a) 证明: $p$ 是 $H(\varOmega)$ 上的范数.

    (b) 证明: 由 $p$ 诱导的拓扑不同于在 $\varOmega$ 的紧子集上一致收敛的拓扑.
    (提示: 可以考虑函数 $f_n(z)=\e^{n(z-2)}$.)
\end{exercise}

\begin{proof}
    (a) 直接按定义验证.

    (b) 考虑 $f_n(z)=\e^{n(z-2)}$, 则在 $K$ 上有
    \[\sup_{z\in K}|f_n(z)|=\sup_{z\in K}\left|\e^{n(z-2)}\right|=\e^{-n}\to 0.\]
    故 $(f_n)_{n\geq 1}$ 依 $p$ 范数收敛于 $0$.
    但在紧子集 $\{z\in\FC: |z|\leq 2\}$ 上, $f_n(2)=1$, 故 $(f_n)_{n\geq 1}$ 不是一致收敛的.
\end{proof}



\begin{exercise}
    设 $A$ 是 $[0,1]$ 的可数子集, 且映射 $\alpha: A \rightarrow(0,+\infty)$ 
    满足 $\sum_{t\in A}\alpha(t)<+\infty$. 对 $f\in C([0,1],\FR)$ 定义
    \[\|f\|_{A,\alpha}=\sum_{t\in A}\alpha(t)|f(t)|.\]

    (a) 证明: $\|\cdot\|_{A,\alpha}$ 是 $C([0,1],\FR)$ 上的半范数. 什么时候它是一个范数? 
    什么时候它等价于一致范数 $\|\cdot\|_{\infty}$?

    (b) 证明: 两个半范数 $\|\cdot\|_{A, \alpha}$ 和 $\|\cdot\|_{A',\alpha'}$ 诱导相同的拓扑当且仅当 $A=A'$ 且
    \[
    0<\inf_{t\in A} \alpha'(t)/\alpha(t)\leqslant \sup_{t\in A} \alpha'(t)/\alpha(t)<\infty.
    \]
\end{exercise}

\begin{proof}
    (a) 首先由 $f \in C([0,1], \FR)$, 知 $\sup_{t \in A}|f(t)|<\infty$. 那么
    \[\|f\|_{A,\alpha}=\sum_{t\in A}\alpha(t)|f(t)|\leq\sup_{t\in A}|f(t)|\sum_{t \in A} \alpha(t)<+\infty.\]
    并且容易验证 $\|\cdot\|_{A, \alpha}$ 是满足半范数满足的其它公理. 
    而且可以看到, 若取 $A=\{1,\frac{1}{2},\ldots,\frac{1}{n},\ldots\}$, 
    令映射 $\alpha: \frac{1}{n} \rightarrow \frac{1}{2^{n}}$, 
    并设 $f$ 为在 $A$ 上取 $0$ 的 “锯齿” 函数, 则 $\|f\|_{A,\alpha}=0$, 但 $f\not\equiv 0$, 即 $\|f\|_{A,\alpha}$ 不是一个范数.

    由定义可知, $\|f\|_{A, \alpha}=0$ 等价于 $f(t)=0$, $t\in A$. 由此可以证明: $\|\cdot\|_{A, \alpha}$ 
    是一个范数当且仅当 $A$ 在 $[0,1]$ 中稠密.

    实际上, 若 $A$ 在 $[0,1]$ 中稠密, 则由 $f(t)=0$, $t\in A$, 及 $f$ 的连续性, 得 $f(t)=0$, $\forall t \in[0,1]$. 
    反过来, 假设 $A$ 在 $[0,1]$ 中不稠密, 那么存在点 $t\in[0,1]$
    及 $t$ 的开邻域 $I$, 使得 $A\cap I=\emptyset$, 即 $A \subset[0,1]\setminus I$.
    那么可取 $[0,1]$ 上的连续函数 $f$ 在 $[0,1]\setminus I$ 上为 $0$ , 而在 $I$ 上不为 $0$, 
    这与 $\|\cdot\|_{A,\alpha}$ 是一个范数相矛盾.

    最后, 我们来证明: 对任意一个半范数 $\|\cdot\|_{A,\alpha}$, 它都不等价于一致范数 $\|\cdot\|_{\infty}$.
    对可列集 $A$ 中的元素排序, 记 $A=(t_{i})_{i\geq 1}$. 
    由 $\sum_{t_i\in A} \alpha(t)<+\infty$, 存在 $N>0$, 使得当 $i\geq N$ 时, 
    有 $\sum_{i \geq N} \alpha(t_i)<\varepsilon$. 接下来, 选择一个连续函数 $f$ 满足 $\|f\|_{\infty}=1$, 且
    \[
    f(t)= \begin{cases}0, & t=t_{i}, i<N \\ 1, & t=t_{N+1}\end{cases}
    \]
    那么
    \[
    \|f\|_{A,\alpha}=\sum_{i>N} \alpha(t_i)|f(t_i)|\leq\sum_{i>N} \alpha(t_i)<\varepsilon,
    \]
    这意味着一致范数 $\|\cdot\|_{\infty}$ 关于半范数 $\|\cdot\|_{A,\alpha}$ 不是有界的.

    (b) 首先证明充分性. 设 $C_1=\inf_{t\in A}\alpha'(t)/\alpha(t)$, 
    $C_{2}=\sup_{t \in A} \alpha'(t)/\alpha(t)$, 则由充分性条件知 $0<C_1\leq C_2<\infty$. 故
    \[C_{1}\|f\|_{A,\alpha}\leq\|f\|_{A',\alpha'}=\sum_{t\in A'} \alpha'(t)|f(t)|=\sum_{t\in A} \frac{\alpha'(t)}{\alpha(t)} \alpha(t)|f(t)|\leq C_{2}\|f\|_{A,\alpha}.\]
    即证两个范数 $\|\cdot\|_{A, \alpha}$ 和 $\|\cdot\|_{A', \alpha'}$ 等价.

    下面证明必要性, 我们先假设 $A\neq A'$. 
    不妨设存在 $t_{i_0}\in A$, 但 $t_{i_{0}}\notin A'$. 任取 $\varepsilon>0$, 则存在 $N>0$, 
    当 $i\geq N$ 时, 有 $\sum_{i\geq N} \alpha'(t_i')<\varepsilon$. 取连续函数 $f$ 满足 $\|f\|_{\infty}=1$, 且
    \[f(t)=\begin{cases} 0, & t=t_{i}',\forall i<N; \\ 1, & t=t_{i_{0}} .\end{cases}\]
    则有
    \[\|f\|_{A,\alpha'}=\sum_{i>N} \alpha'(t_{i}')|f(t_{i}')|<\varepsilon,\]
    而
    \[\|f\|_{A, \alpha}\geq\alpha(t_{i_0}),\]
    这意味着两个半范数 $\|\cdot\|_{A, \alpha}$ 和 $\|\cdot\|_{A', \alpha'}$ 诱导的拓扑不同. 因此必有 $A=A'$.

    接下来, 我们假设 $A=A'=(t_i)_{i\geq 1}$. 两个半范数 $\|\cdot\|_{A,\alpha}$ 和 $\|\cdot\|_{A',\alpha'}$
    诱导的拓扑相同, 意味着存在常数 $C_1,C_2>0$, 使得
    \begin{equation}
        C_1\|f\|_{A,\alpha}\leq\|f\|_{A',\alpha'}\leq C_2\|f\|_{A,\alpha}.\tag{$\star$}
    \end{equation}
    任取一个 $t_{i_0}$, 则 $N>0$, 使得当 $i\geq N$ 时, 
    有 $\sum_{i\geq N} \alpha(t_i)\leq\alpha(t_{i_0})$. 取连续函数 $f$ 满足 $\|f\|_{\infty}=1$, 且
    \[
    f(t)=
    \begin{cases}
        0, & t=t_{i},\forall i<N \text {\ 且\ }i\neq i_{0}; \\
        1, & t=t_{i_{0}}.
    \end{cases}\]
    则有
    \[\|f\|_{A, \alpha}\leq 2\alpha(t_{i_0})\quad\text{ 且 }\quad\|f\|_{A,\alpha'}\geq\alpha'(t_{i_{0}}).\]
    综合上面的两个不等式以及 $(\star)$ 式, 立即可得
    \[\alpha'(t_{i_{0}}) \leq\|f\|_{A, \alpha'} \leq C_{2}\|f\|_{A,\alpha}\leq 2C_2\alpha(t_{i_{0}}).\]
    因此对任意 $t\in A$, 有 $\alpha'(t)/\alpha(t)\leq 2C_2$. 类似上面的讨论, 也有
    \[
    \alpha(t_{i_0})\leq\frac{2}{C_{1}}\alpha'(t_{i_{0}}).
    \]
    故结论成立.
\end{proof}

% 7.\textit{Proof}:(a)因为任取原点的邻域$V$,存在$\alpha>0,s.t.B\subset\alpha V$,即$\frac{1}{\alpha}B\subset V$,由定义知$(rB)_{r>0}$是原点的邻域基\\
% (b)因为$B$是原点的邻域,所以存在平衡邻域$V$使得$V\subset B$,由于平衡集的凸包仍然平衡,故$\conv(V)\subset B$且$\conv(V)$是平衡的凸集,取其内部,即得原点的凸平衡开邻域$(\conv(V))^{\circ}\subset B$\\\\
% 11.\textit{Proof}:(a)因为$E$是局部凸空间,所以$E$上必存在有界的凸平衡开邻域(记为$\Omega$),由第七题结论知其对应的Minkowski泛函$p_{\Omega}$是$E$上的范数,因此也是$F$上的范数\\\\
% 12.\textit{Proof}:(a)由定理7.1.7(2)知$\forall V\in\mathcal{N}_{\tau}(0)$,存在$W_1\in\mathcal{N}_{\tau}(0)$,使得$W_1+W_1\subset V$,又存在$W_2\in\mathcal{N}_{\tau}(0)$,使得$W_2+W_2\subset W_1$,依次进行下去,存在$W_{n-1}\in\mathcal{N}_{\tau}(0)$,使得$W_{n-1}+W_{n-1}\subset W_{n-2}$,这样就得到一列单调递减的集列$(W_i)_{i=1}^{n-1}$满足
% \[W_{n-1}+W_{n-1}+W_{n-2}+W_{n-3}+\cdots +W_1\subset V\]
% 取$W=W_{n-1}\in\mathcal{N}_{\tau}(0)$,即得$W+W+\cdots +W\subset V$($n$个$W$求和)\\
% (b)由$W$的吸收性知:\\
% 存在$\alpha_1>0$,当$|x_1|<\alpha_1$时,$x_1e_1\in W$\\
% 存在$\alpha_2>0$,当$|x_2|<\alpha_2$时,$x_2e_2\in W$\\
% $\cdots$\\
% 存在$\alpha_n>0$,当$|x_n|<\alpha_n$时,$x_ne_n\in W$\\
% 故取$r=\min\limits_{1\leq k\leq n}\alpha_k>0$,当$\max\limits_{1\leq k\leq n}|x_k|<r$时:
% \[x=\sum_{k=1}^nx_ke_k\in W+W+\cdots+W\subset V\]
% 也即$B_{\tau_0}(0,r)\subset V$,从而导出$\tau\subset\tau_0$\\
% (c)这里不是向量空间怎么谈同构呢?下面证明$id:(E,\tau_0)\to (E,\tau)$连续:\\
% 首先$f:\mathbb{K}^n\to(E,\tau_0),(x_1,\cdots,x_n)\mapsto\sum_{k=1}^nx_ke_k$为同胚,显然连续,又由$(E,\tau)$是拓扑向量空间知映射$id\circ f:\mathbb{K}^n\to (E,\tau_0)$连续,故$id$连续\\
% (d)令$S=\overline{B}\backslash B$,则$B$为$(E,\tau_0)$中的紧集,由$id:(E,\tau_0)\to(E,\tau)$连续知$S$是$(E,\tau)$中的紧集,而$\tau$是Hausdorff拓扑,故$S$是闭集,故$E\backslash S$为$(E,\tau)$中开集,令$V=E\backslash S\in\mathcal{N}_{\tau}(0)$,则$B=V\cap\overline{B}$\\
% (e)由$U$是平衡集知$U$是$(E,\tau_0)$中的单连通集且$0\in U$,又$U\cap S=\emptyset$,故必有$U\subset B$\\
% (f)显然\\\\




\begin{exercise}[16]
    令 $0<p<1$, 考虑空间 $L_p=L_p(0,1)$, 并在 $L_p$ 上赋予距离 $d_{p}(f, g)=\|f-g\|_p^p$. 
    本习题的目标是证明 $L_p$ 不是局部凸的且没有非零线性泛函.

    (a) 证明: $L_p$ 是拓扑向量空间.

    接下来, 我们先考虑 $p=\frac{1}{2}$ 的情形, 用 $B(r)$ 表示中心在原点、半径为 $r$ 的 $L_{\frac{1}{2}}$ 中的闭单位球: 
    $B(r)=\left\{f \in L_{\frac{1}{2}}\colon \|f\|_{\frac{1}{2}}^{\frac{1}{2}} \leqslant r\right\}$.

    (b) 取 $f\in B(\sqrt{2}r)$. 证明: 存在 $t_0 \in(0,1)$, 使得
    \[
    \int_{0}^{t_0}|f(t)|^{\frac{1}{2}}\diff t=\frac{\|f\|_{\frac{1}{2}}^{\frac{1}{2}}}{2}.
    \]

    (c) 定义
    \[g(t)=
    \begin{cases}
        2f(t), & 0\leqslant t\leqslant t_{0}, \\
        0, & t_{0}<t\leqslant 1,
    \end{cases}\quad\text{且}\quad 
    h(t)=\begin{cases}
        0, & 0 \leqslant t \leqslant t_{0}, \\
        2f(t), & t_{0}<t \leqslant 1.
    \end{cases}\]
    证明:
    \[g, h \in B(r) \quad \text{且} \quad f=\frac{g}{2}+\frac{h}{2}.\]

    (d) 由此导出 $B(\sqrt{2} r)\subset\conv(B(r))$, 并有 $\conv(B(r))=L_{\frac{1}{2}}$.

    (e) 得出 $L_{\frac{1}{2}}$ 是非局部凸的.

    (f) 把以上结论推广到 $0<p<1$ 的情形.

    (g) 证明: $L_{p}$ 上只有零线性泛函是连续的.
\end{exercise}

\begin{proof}
    (a) 只证明映射 $\varPhi:L_p\times L_p\to L_p,(f,g)\mapsto f+g$ 连续,
    对于加法的连续性同理可证, 对于以 $f+g$ 为中心的任意开球 $V=B(f+g,r)$,
    取 $U_1=B(f,r/2),U_2=B(g,r/2)$, 则对任意 $f_1\in B(f,r/2),g_1\in B(g,r/2)$, 有
    \[\|f_1-f\|_p^p<\frac{r}{2},\|g_1-g\|_p^p<\frac{r}{2}.\]
    由距离的三角不等式有
    \[\|(f_1+g_1)-(f+g)\|_p^p\leq\|f_1-f\|_p^p+\|g_1-g\|_p^p<\frac{r}{2}+\frac{r}{2}=r.\]
    所以 $f_1+g_1\in V$, 故 $U_1+U_2\subset V$, 由定义知 $\varPhi$连续.

    (b) 令
    \[\varPhi(t)=\int_0^t|f(x)|^{\frac{1}{2}}\diff x.\]
    则 $\varPhi(0)=0$, $\varPhi(1)=\|f\|_{\frac{1}{2}}^{\frac{1}{2}}$,
    且 $\varPhi(t)$ 是连续函数, 由介值性定理知存在 $t_0\in(0,1)$ 使得
    \[\varPhi(t_0)=\int_0^{t_0}|f(t)|^{\frac{1}{2}}\diff t=\frac{\|f\|_{\frac{1}{2}}^{\frac{1}{2}}}{2}.\]

    (c) 因为
    \[\|g\|_{\frac{1}{2}}^{\frac{1}{2}}=\int_0^1|g(t)|^{\frac{1}{2}}\diff t=\int_0^{t_0}|2f(t)|^{\frac{1}{2}}\diff t=\sqrt{2}\cdot\frac{\|f\|_{\frac{1}{2}}^{\frac{1}{2}}}{2}\leq\frac{\sqrt{2}}{2}\cdot\sqrt{2}r=r.\]
    所以 $g\in B(r)$. 同理 $h\in B(r)$, 而 $f=\frac{g}{2}+\frac{h}{2}$ 是显然的.

    (d) (注意, 在度量空间中, 球并不一定为凸集, 所以不要对本题中的 $B(r)$ 取凸包的操作感到惊讶! 但是在赋范空间中, 球一定为凸集.)
    对任意的 $f\in B(\sqrt{2}r)$, 利用 (c) 中的构造方式得到 $g,h\in B(r)$ 使得 $f=\frac{g}{2}+\frac{h}{2}$,
    故 $f\in \conv(B(r))$, 因此
    \[B(\sqrt{2}r)\subset \conv(B(r)).\]
    又
    \[B(2r)\subset \conv(B(\sqrt{2}r))\subset \conv(\conv(B(r)))=\conv(B(r)).\]
    故对任意的 $n$, 有
    \[B(2^nr)\subset \conv(B(r)).\]
    因此
    \[L_{\frac{1}{2}}=\bigcup_{n=1}^{\infty} B(2^nr)\subset \conv(B(r)).\]
    结合 $\conv(B(r))\subset L_{\frac{1}{2}}$ 知 $\conv(B(r))=L_{\frac{1}{2}}$.

    (e) 由(d) 知原点的凸开邻域只有 $L_{\frac{1}{2}}$, 因此 $L_{\frac{1}{2}}$ 是非局部凸的.

    (f)同 (c) 的构造亦可得 $L_p(0<p<1)$ 是非局部凸的.

    (g) 设 $f:L_p\to\FR$ 是连续的线性泛函,
    则 $\forall r>0$, $(-r,r)$ 为 $\FR$ 中开凸集,
    由 $f$ 的线性性知 $f^{-1}((-r,r))$ 为 $L_p$ 中凸集, 
    由 $f$ 的连续性知 $f^{-1}((-r,r))$ 为 $L_p$ 中的开集, 结合(f)中结论知
    \[f^{-1}(-r,r)=L_p.\]
    故
    \[f^{-1}(0)=\bigcap_{n=1}^{\infty}f^{-1}\left(\left(-\frac{1}{n},\frac{1}{n}\right)\right)=L_p.\]
    因此$f\equiv 0$.
\end{proof}
\setcounter{chapter}{7}
\chapter{Hahn-Banach 定理}
\thispagestyle{empty}


\begin{exercise}
    设 $1\leq p\leq\infty$, 考虑 $\FR^2$ 上的 $p$ 范数:
    \[\|(x_1,x_2)\|_p=\bigl(|x_1|^p+|x_2|^p\bigr)^{\frac{1}{p}},\;p<\infty;\quad \|(x_1,x_2)\|_{\infty}=\max\{|x_1|,|x_2|\}.\]
    设 $F=\FR\times\{0\}$, 即由 $e_1=(1,0)$ 生成的向量子空间, 并设 $f:F\to\FR$
    是线性泛函, 满足 $f(e_1)=1$.

    (a) 当 $\FR^2$ 上赋予 $\|\cdot\|_1$ 范数时, 确定 $f$ 从 $F$ 到 $\FR^2$ 的所有保范延拓.

    (b) 当 $\FR^2$ 上赋予 $\|\cdot\|_p$ 范数时, 考虑同样的问题.
\end{exercise}

\begin{proof}
    (a)首先 $\|f\|=\sup\limits_{x=te_1,t\neq 0}\frac{|f(x)|}{\|x\|_1}=\sup\limits_{x=te_1,t\neq0}\frac{|t|}{|t|}=1$,
    记 $e_2=(0,1)$, 则对任意 $x=x_1e_1+x_2e_2\in\FR^2$, 有
    \[\tilde{f}(x)=x_1f(e_1)+x_2\tilde{f}(e_2)=x_1+x_2\tilde{f}(e_2).\]
    则
    \[\|\tilde{f}\|=\sup_{x\in\FR^2,x\neq 0}\frac{|\tilde{f}(x)|}{\|x\|_1}=\sup_{x\in\FR^2,x\neq 0}\frac{|x_1+x_2\tilde{f}(e_2)|}{|x_1|+|x_2|}.\]
    要使得 $\|\tilde{f}\|=\|f\|=1$, 即
    \[\sup_{x\in\FR^2,x\neq 0}\frac{|x_1+x_2\tilde{f}(e_2)|}{|x_1|+|x_2|}=1.\]
    容易验证当且仅当 $|\tilde{f}(e_2)|\leq 1$ 时, 上式得以成立, 
    因此当 $\FR^2$ 上赋予 $\|\cdot\|_1$ 范数时, $f$ 从 $F$ 到 $\FR^2$ 的所有保范延拓为:
    \[\left\{\tilde{f}\colon\tilde{f}(x)=x_1+x_2\tilde{f}(e_2),|\tilde{f}(e_2)|\leq1\right\}.\]

    (b) 当 $1<p<\infty$ 时, 此时目标是:
    \[\|\tilde{f}\|=\sup_{x\in\FR^2,x\neq0}\frac{|x_1+x_2\tilde{f}(e_2)|}{\left(|x_1|^p+|x_2|^p\right)^{\frac{1}{p}}}=1.\]
    为叙述简便, 记 $t=|\tilde{f}(e_2)|$, 则$x_1\cdot x_2\tilde{f}(e_2)\geq 0$ 时
    \[\begin{split}\frac{|x_1+x_2\tilde{f}(e_2)|}{\left(|x_1|^p+|x_2|^p\right)^{\frac{1}{p}}}\leq 1
    &\Leftrightarrow \frac{|x_1|+|x_2|\cdot t}{\left(|x_1|^p+|x_2|^p\right)^{\frac{1}{p}}}\leq1\\
    &\Leftrightarrow (|x_1|+|x_2|\cdot t)^p\leq|x_1|^p+|x_2|^p(\mbox{不妨}|x_2|\neq0)\\
    &\Leftrightarrow\left(\frac{|x_1|}{|x_2|}+t\right)^p\leq\left(\frac{|x_1|}{|x_2|}\right)^p+1\\
    &\Leftrightarrow (\alpha+t)^p\leq\alpha^p+1(0\leq\alpha\leq\infty)\\
    &\Leftrightarrow t=0.
    \end{split}\]
    因此保范延拓为 $\left\{\tilde{f}:\tilde{f}(x)=x_1\right\}$.
    
    当 $p=\infty$ 时, 此时目标是:
    \[\sup_{x\in\FR^2,x\neq 0}\frac{|x_1+x_2\tilde{f}(e_2)|}{\max\{|x_1|,|x_2|\}}=1\Leftrightarrow\tilde{f}(e_2)=0.\]
    因此保范延拓为 $\left\{\tilde{f}:\tilde{f}(x)=x_1\right\}$.

    综上知当$\FR^2$上赋予$\|\cdot\|_p(1<p\leq\infty)$范数时,所有的保范延拓为$\left\{\tilde{f}:\tilde{f}(x)=x_1\right\}$.
\end{proof}



% \begin{exercise}
%     通过 $\FR^2$ 上的反例说明在几何形式的 Hahn-Banach 定理中, 一个凸子集是开集的条件是必要的.
% \end{exercise}

% \begin{proof}
% 将四个点$(\pm1,\pm1)$围成的正方形用$x$轴分成两部分,
% 上半部分去掉线段 $\{(x,y)|-1\leq x\leq0,y=0\}$, 得到区域 $A$,
% 下半部分去掉线段 $\{(x,y)|0\leq x\leq 1,y=0\}$, 得到区域$B$, 易知$A,B$无法被隔离.
% \end{proof}



% \begin{exercise}
%     设 $E$ 是数域 $\FK$ 上的赋范空间, $A\subset E$, 并设 $f:A\to\FK$
%     以及常数 $\lambda\geq 0$. 证明: 存在 $\widehat{f}\in E^*$, 使得
%     $\widehat{f}|_A=f\quad\text{且}\quad\|\widehat{f}\|\leq\lambda$
%     的充分必要条件是
%     \[\left|\sum_{k=1}^n \alpha_kf(a_k)\right|\leq\lambda\left\|\sum_{k=1}^n\alpha_ka_k\right\|,\forall n\in\FN^*,
%     \forall(a_1,\cdots,a_n)\in A^n,\forall (\alpha_1,\cdots,\alpha_n)\in\FK^n.\]
% \end{exercise}

% \begin{proof}
%     必要性显然,下面证明充分性. 由
%     \[\left|\sum_{k=1}^n\alpha_kf(a_k)\right|=\left|f\left(\sum_{k=1}^n\alpha_ka_k\right)\right|\leq\lambda\left\|\sum_{k=1}^n\alpha_ka_k\right\|.\]
%     知 $f$ 是连续线性泛函, 由 Hahn-Banach 定理知存在 $\hat{f}\in E^{*}$, 使得 $\hat{f}|_A=f,\|\hat{f}\|\leq\lambda$.
% \end{proof}



\begin{exercise}[4]
    设 $E$ 是 Hausdorff 拓扑向量空间, $A$ 是 $E$ 中包含原点的开凸集以及 $x_0\in E\setminus A$.

    (a) 证明: 存在 $f\in E^*$, 使得
    \[\Re f(x_0)=1,\;\text{且在\ }A\text{\ 上\ }\Re f<1.\]

    (b) 假设 $A$ 还是平衡的. 证明: 可以选择 $f\in E^*$, 使其满足
    \[f(x_0)=1,\;\text{且在\ }A\text{\ 上\ }|f|<1.\]
\end{exercise}

\begin{proof}
    (a) 由于 $\{x_0\}$ 为凸集, $A$ 为开凸集且二者不相交, 故由 Hahn-Banach 定理知存在
    $\widetilde{f}\in E^*$ 和 $\alpha>0$, 使得
    \[\Re\widetilde{f}(a)<\alpha\leq\Re\widetilde{f}(x_0),\quad\forall a\in A.\]
    令 $f=\frac{\widetilde{f}}{\Re\widetilde{f}(x_0)}$, 则 $\Re f(x_0)=1$ 且对任意 $a\in A$, 有
    \[\Re f(a)=\frac{\Re\widetilde{f}(a)}{\Re\widetilde{f}(x_0)}<\frac{\alpha}{\alpha}=1.\]

    (b) $\widetilde{f}$ 仍为 (a) 中所得有界线性泛函, 令 $f(x)=\frac{\widetilde{f}(x)}{\widetilde{f}(x_0)}$, 则
    $f(x_0)=1$, 且对任意 $a\in A$, 由 $A$ 平衡可得
    \begin{align*}
        |f(a)|
        & =\left|\frac{\widetilde{f}(a)}{\widetilde{f}(x_0)}\right|=\frac{|\widetilde{f}(a)|}{|\widetilde{f}(x_0)|}=\frac{\widetilde{f}(a)\sgn\widetilde{f}(a)}{|\widetilde{f}(x_0|} \\
        & =\frac{\widetilde{f}(a\sgn\widetilde{f}(a))}{|\widetilde{f}(x_0)|}\leq\frac{\Re\widetilde{f}(a\sgn\widetilde{f}(a))}{\alpha}<\frac{\alpha}{\alpha}=1.\qedhere
    \end{align*}
\end{proof}



\begin{exercise}
    设 $E$ 是 Hausdorff 局部凸空间, $A$ 是 $E$ 中包含原点的闭凸集以及 $x_0\in E\setminus A$.

    (a) 证明: 存在 $f\in E^*$, 使得
    \[\Re f(x_0)>1\quad\text{且}\quad\sup_{x\in A}\Re f(x)\leq 1.\]

    (b) 假设 $A$ 还是平衡的. 证明: 可以选择 $f$, 使其满足
    \[f(x_0)=1\quad\text{且}\quad\sup_{x\in A}|f(x)|\leq 1.\]
\end{exercise}

\begin{proof}
    (a) 由于 $A$是包含原点的闭凸集, $\{x_0\}$是紧集, 且二者不相交,
    故由 Hahn-Banach 定理知存在 $\widetilde{f}\in E^*$和常数 $\alpha>0$, 使得
    \[\sup_{x\in A}\Re\widetilde{f}(x)<\alpha<\Re\widetilde{f}(x_0).\]
    令 $f=\frac{\widetilde{f}}{\alpha}\in E^*$, 则
    \[\Re f(x_0)=\frac{\Re\widetilde{f}(x_0)}{\alpha}>1\quad\text{且}\quad\sup_{x\in A}\Re f(x)=\sup_{x\in A}\frac{\Re\widetilde{f}(x)}{\alpha}<1.\]
    (b) $\widetilde{f}$ 仍为 (a) 中所得有界线性泛函, 令 $f=\frac{\widetilde{f}}{\widetilde{f}(x_0)}$, 则
    $f(x_0)=1$, 且由 $A$ 平衡可得
    \begin{align*}
        \sup_{x\in A}|f(x)|
        & =\sup_{x\in A}\left|\frac{\widetilde{f}(x)}{\widetilde{f}(x_0)}\right|=\sup_{x\in A}\frac{|\widetilde{f}(x)|}{|\widetilde{f}(x_0)|} \\
        & <\frac{1}{\alpha}\sup_{x\in A}\Re\widetilde{f}(x\sgn\widetilde{f}(x))<\frac{\alpha}{\alpha}=1.\qedhere
    \end{align*}
\end{proof}
% 6.\textit{Proof}:(a)$\forall x\in\overline{\conv(A)},\exists(x_n)_{n\geq 1}\subset\conv(A),s.t.x_n\to x(n\to\infty)$,由凸包的定义知:\[x_n=\sum_{k=1}^{N_n}t_ka_k,a_k\in A,t_k\geq0,\sum_{k=1}^{N_n}t_k=1\]
% 取$f\in E^*$满足对任意$a\in A$,有$f(a)\leq1$,则
% \[f(x)=f\left(\lim_{n\to\infty}\sum_{k=1}^{N_n}t_ka_k\right)=\lim_{n\to\infty}\sum_{k=1}^{N_n}t_kf(a_k)\leq1\]
% 故$x\in\widehat{A}\Rightarrow\overline{\conv(A)}\subset\widehat{A}$\\
% (b)\\$(\Rightarrow)$当$\overline{\conv(A)}=\widehat{A}$时,因为显然$0\in\widehat{A}$,所以$0\in\overline{\conv(A)}$\\
% $(\Leftarrow)$因为$0\in\overline{\conv(A)}$,所以$\overline{\conv(A)}$是包含原点的闭凸集,任取$x_0\in E\setminus\overline{\conv(A)}$,由上一题结论知存在$f\in E^*$,使得
% \[f(x_0)>1\mbox{且}\sup_{x\in\overline{\conv(A)}}f(x)\leq 1\]
% 故$x_0\not\in\widehat{A}$,即$x_0\in\left(\widehat{A}\right)^c$,所以
% \[E\setminus\overline{\conv(A)}\subset\left(\widehat{A}\right)^c\Rightarrow\widehat{A}\subset\overline{\conv(A)}\]
% 结合$(a)$中结论得$\overline{\conv(A)}=\widehat{A}$\\\\
% 8.\textit{Proof}:(注:本题需要加上连续性的条件)\\
% (a)\begin{enumerate}[(i)]
% \item $G$是凸集:$\forall y^{(1)},y^{(2)}\in G,\exists x_1,x_2\in C$使得
% \[f_i(x_1)\leq y^{(1)}_i,f_i(x_2)\leq y^{(2)}_i,1\leq i\leq m\]
% 对于任意$0<\lambda<1$,由$C$是凸集知$\lambda x_1+(1-\lambda)x_2\in C$,又由$(f_i)$是凸函数得
% \[f_i(\lambda x_1+(1-\lambda)x_2)\leq\lambda f_i(x_1)+(1-\lambda)f_i(x_2)\leq\lambda y^{(1)}_i+(1-\lambda)y^{(2)}_i,1\leq i\leq m\]
% 故\[\lambda y^{(1)}+(1-\lambda)y^{(2)}\in G\]
% 因此$G$是凸集
% \item $G$是闭集:任取$G$中收敛序列$(y^{(n)})\to y$,需要证明$y\in G$,由$G$的定义知存在$(x_n)_{n\geq1}\subset C$,使得
% \[f_i(x_n)\leq y^{(n)}_i,1\leq i\leq m\]
% 因为$C$紧,故其有收敛子列$(x_{n_k})_{k\geq1}\to x\in C$,同时$f_i(x_{n_k})\leq y^{(n_k)}_i$,令$k\to\infty$,得
% \[\lim_{k\to\infty}f_i(x_{n_k})=f_i(x)\leq\lim_{k\to\infty}y^{(n_k)}_i=y_i\]
% 故$y\in G$,这就证明了$G$是闭集
% \end{enumerate}
% $S=\varnothing$意味着$\forall x\in C,\exists f_i,s.t.f_i(x)>0$,而
% \[G^c=\{y=(y_1,\cdots,y_m)\in\FR^m:\forall x\in C,\exists f_i,s.t.f_i(x)>y_i\}\]
% 故$S=\varnothing$可以表示为$0\in G^c$\\
% (b)显然\\\\


\begin{exercise}
  设 $E$ 是数域 $\FK$ 上的拓扑向量空间. 称 $E$ 的向量子空间 $H$ 是超平面,
  若有某个 $x_0\in E\setminus H$, 使得 $E = H + \FK x_0$.
  \begin{enumerate}[(a)]
    \item 证明: 若 $H$ 是超平面, 则对任意的 $x_0\in E\setminus H$, $E=H+\FK x_0$ 成立.
    \item 证明: 一个超平面或者是 $E$ 的稠密集, 或者是闭集.
    \item 证明: $H$ 是超平面当且仅当存在 $E$ 上的一个非零线性泛函 $f$, 使得 $H=\ker f$.
      因而 $H$ 是闭的等价于 $f$ 是连续的.
  \end{enumerate}
\end{exercise}

\begin{proof}
    (a) $\forall x_1\in E\setminus H,x_1=y_1+\lambda x_0,y_1\in H,\lambda\neq0$, 故对 $\forall x\in E$,
    \[x=y+kx_0=y+k\frac{x_1-y_1}{\lambda}=\left(y-\frac{k}{\lambda}y_1\right)+\frac{k}{\lambda}x_1\in H+\mathbb{K}x_1.\]
    因此
    \[E=H+\mathbb{K}x_1,\quad\forall x_1\in E\setminus H.\]

    (b) 由定理 7.1.6 知 $\closure{H}$ 是向量子空间, 又 $\dim (E\setminus H)=1$, 故只可能有两种情况:
    当 $\closure{H}=H$ 时, $H$ 为闭集; 当 $\closure{H}=E$ 时, $H$ 在 $E$ 中稠密.

    (c) \sufficient
    假设存在 $E$ 上的非零线性泛函 $f$, 使得 $H=\ker f$, 首先因 $f\not\equiv 0$,
    故存在 $x_0\in E\setminus\ker f$, 使得 $f(x_0)=1$, 则$\forall x\in E$, 有
    \[x=x-f(x)x_0+f(x)x_0.\]
    因为 $f(x-f(x)x_0)=f(x)-f(x)f(x_0)=0$, 所以 $x-f(x)x_0\in\ker f$,
    并且 $f(x)x_0\in\mathbb{K}x_0$, 又容易验证表示 $x=h+kx_0$, $h\in\ker f,k\in\mathbb{K}$ 是唯一的, 因此
    \[E=H+\mathbb{K}x_0.\]
    也即 $H$ 是超平面.

    \necessary
    因为 $H$ 是超平面, 所以存在 $x_0\in E\setminus H$, 使得 $E=H+\mathbb{K}x_0$, 
    故对于 $\forall x\in E$, $x=h+kx_0$, 定义泛函
    \[f:E\to\mathbb{K},x=h+kx_0\mapsto k.\]
    容易验证 $f$ 是合理定义的线性泛函且 $\ker f=H$.
    当 $f$ 连续时, 因为 $\{0\}\subset\mathbb{K}$ 是闭集, 所以 $H=\ker f=f^{-1}(0)$ 是闭集.
\end{proof}



\begin{exercise}
  设 $(X,\|\cdot\|_X)$ 是实赋范空间, $\closure{B}_X$ 表示该空间中的闭单位球.
  假设 $K\geq 1$, $C$ 是 $X$ 中闭凸对称子集 ($C$ 对称是指 $x\in C\Rightarrow -x\in C$), 且满足
  \[B_X\subset C\subset K\closure{B}_X.\]
  定义
  \[p(x) = \inf\Bigl\{\lambda>0\colon\frac{x}{\lambda}\in C\Bigr\},\;\forall x\in X.\]
  \begin{enumerate}[(a)]
    \item 证明: $p$ 是 $X$ 上和 $\|\cdot\|_X$ 等价的范数. 更确切地说, 证明:
      \[\frac{1}{K}\|x\|\leq p(x)\leq\|x\|,\;\forall x\in X.\]
    \item 设 $x\in X$. 证明:
      \begin{enumerate}[(i)]
          \item $x\in X\setminus C\Longleftrightarrow p(x)>1$.
          \item $x\in\mathring{C}\Longleftrightarrow p(x)<1$.
          \item $x\in\partial C\Longleftrightarrow p(x)=1$.
      \end{enumerate}
    \item 任取 $x\in\partial C$. 证明: 存在 $X$ 上的连续线性泛函 $f$, 使得 $f(x)=1$ 且在集合 $C$ 上, $|f|\leq 1$.
    \end{enumerate}
\end{exercise}

\begin{proof}
    (a) 任取 $\varepsilon>0$, 对于任意 $x,y\in X$, 有 $\frac{x}{p(x)+\varepsilon}\in C$, $\frac{y}{p(y)+\varepsilon}\in C$, 于是
    \[\frac{x + y}{p(x) + p(y) + 2\varepsilon} = \frac{p(x) + \varepsilon}{p(x) + p(y) + 2\varepsilon}\frac{x}{p(x) + \varepsilon} + \frac{p(y) + \varepsilon}{p(x) + p(y) + 2\varepsilon}\frac{y}{p(y) + \varepsilon}\in C,\]
    因此 $p(x+y)\leq p(x)+p(y)+2\varepsilon$, 由 $\varepsilon$ 的任意性得 $p(x+y)\leq p(x)+p(y)$.

    任取 $\lambda\in\FR$ 和 $x\in X$, 当 $\lambda=0$ 时, $p(\lambda x)=|\lambda|p(x)$
    显然成立, 当 $\lambda\neq 0$ 时, 有
    \[\frac{\lambda x}{p(\lambda x)+\varepsilon}\in C.\]
    由 $C$ 对称得
    \[\frac{x}{\frac{1}{|\lambda|}(p(\lambda x)+\varepsilon)}\in C.\]
    故 $p(x)\leq\frac{1}{|\lambda|}(p(\lambda x)+\varepsilon)\Rightarrow|\lambda|p(x)\leq p(\lambda x)$.

    又因 $\frac{x}{p(x)+\varepsilon}\in C$, 由 $C$ 对称可得 $\frac{\lambda x}{|\lambda|(p(x)+\varepsilon)}\in C$,
    故 $p(\lambda x)\leq |\lambda|(p(x)+\varepsilon)$, 从而$p(\lambda x)\leq|\lambda|p(x)$.
    因此 $p(\lambda x)=|\lambda|p(x)$.

    任取 $x\in X$, 有 $\frac{x}{\|x\|}\in\closure{B}_X\subset C$, 故 $p(x)\leq\|x\|$.
    又 $\frac{x}{p(x)+\varepsilon}\in C\subset K\closure{B}_X$, 故
    $\|\frac{x}{p(x)+\varepsilon}\|\leq K\Rightarrow \frac{1}{K}\|x\|\leq p(x)$.

    综上可知 $p$ 是在 $X$ 上和 $\|\cdot\|$ 等价的范数且满足
    \[\frac{1}{K}\|x\|\leq p(x)\leq\|x\|,\;\forall x\in X.\]

    (b) (i) \sufficient 因 $p(x)>1$, 故 $x=\frac{x}{1}\notin C$, 即 $x\in X\setminus C$.
    \necessary 因 $x\in X\setminus C$ 且 $X\setminus C$ 为开集,
    故存在 $\mu\in(0,1)$, 使得 $(1-\mu)x\in X\setminus C$, 即 $\frac{x}{1/(1-\mu)}\notin C$,
    因此 $p(x)\geq\frac{1}{1-\mu}>1$.

    (ii) \necessary 因 $x\in\mathring{C}$ 且 $\mathring{C}$ 为开集,
    故存在 $\mu>0$, 使得 $(1+\mu)x\in\mathring{C}\subset C$, 故 $p(x)\leq\frac{1}{1+\mu}<1$.
    \sufficient 因 $p(x)<1$, 故存在 $\lambda$ 使得 $p(x)<\lambda<1$,
    于是 $\frac{x}{\lambda}\in C\Rightarrow x\in\lambda C\subset\mathring{C}$.

    (iii) 由 (i)(ii) 即得 $x\in\partial C\Longleftrightarrow p(x)=1$.

    (c) 由教材推论 8.1.10 知存在 $X$ 上的连续线性泛函 $f$, 使得 $f(x)=1$ 且在集合 $C$ 上, $|f(x)|\leq p(x)\leq 1$.
\end{proof}



\begin{exercise}
    考虑空间 $\ell_{\infty}$ 和它的子空间 $F$:
    \[F=\bigl\{x\in\ell_{\infty}\colon \lim_{n\to\infty}m_n(x)\text{\ 存在}\bigr\},\quad\text{其中\ }m_n(x)=\frac{1}{n}\sum_{k=1}^n x_k.\]

    (a) 定义 $f:F\to\FR$ 为 $f(x)=\lim_{n\to\infty}m_n(x)$. 证明: $f\in F^*$.

    (b) 证明: 存在 $\ell_{\infty}$ 上的连续线性泛函 $m$ 满足下面的性质:
    \begin{enumerate}[(i)]
        \item $\liminf_{n\to\infty}x_n\leq m(x)\leq\limsup_{n\to\infty}x_n$, $\forall x\in\ell_{\infty}$.
        \item $m\circ\tau=m$, 这里 $\tau:\ell_{\infty}\to\ell_{\infty}$ 是右移算子, 即 $\tau(x)_n=x_{n+1}$.
    \end{enumerate}
\end{exercise}

\begin{proof}
    (a) 线性性 $f(\lambda x+y)=\lambda f(x)+f(y)$ 直接验证. 下证 $f$ 有界, 对任意 $n\geq 1$,
    \[|m_n(x)|\leq\frac{1}{n}\sum_{k=1}^n |x_k|\leq\|x\|_{\infty}.\]
    故对任意 $x\in F$, 有 $|f(x)|=\lim_{n\to\infty}|m_n(x)|\leq\|x\|_{\infty}$.

    (b)
\end{proof}

\begin{proof}
    (a) We can obtain it from definition easily that 
    \[ |m_n(x)| \leq \frac1n\sum_{k=1}^n |x_k|\leq \|x\|_\infty. \]
    Then $|f(x)| = |\lim m_n(x)|\leq \|x_\infty\|.$
    (b) Define $p: l^\infty\to \mathbb R, (x_n)\mapsto \overline\lim \frac1n|\sum_{k = 1}^nx_k|$, and we can obtain $p(x) \geq 0, p(x + y)\leq p(x) + p(y)$ and $|\lambda|p(x) = p(\lambda x)$. Since $p(x) \leq \|x\|_\infty$, $p$ is a continous seminorm. Therefore, $\exists m\in (l^\infty)^*$ s.t. $m = f$ on $F$ and $|m|\leq p$ on $l^\infty$. 
    
    Let $x^*$ denote $\overline\lim x_n$, and $x_*$ denote $\underline \lim x_n$. Set $e = (1, \cdots)\in F$, and then $\lrangle{m}{e} = 1$. We obtain that 
    \[ \langle m, x\rangle  = \langle m, x-x_*e\rangle + x_* \leq p(x-x_*e) + x_*.\]
    We claim that $p(x-x_*)\leq \overline\lim |x_n - x_*|$. Indeed, there exist $N>0$ such that $\forall n > N$, $|x_n - x_*| \leq \overline\lim |x_n - x_*| + \varepsilon$ for any $\varepsilon>0$. Thus 
    \[ \frac1n|\sum_{k=1}^n x_k - x_*|\leq \frac1n \sum_{k=1}^N |x_k - x_*| + \frac{n-N}{n}\overline\lim|x_n - x_*| + \varepsilon. \]
    Since $\exists n_k$ such that $\overline\lim |x_n - x_*|=\lim |x_{n_k} - x_*| = |\lim x_{n_k} - x_*| \leq x^*-x_*$,  we have 
    \[ \lrangle{m}{x}\leq p(x - x_*e) + x_*\leq x^* - x_* + x_* = x^*. \]
    
    Conversely, we have $\langle m, -x\rangle \leq -x_*$, and it follows that $x_*\leq \langle m, x\rangle \leq x^*$.

    Since
    \[ p(\tau x - x) = \overline\lim \frac1n |\sum_{k = 1}^nx_k - x_{k+1}| = \overline\lim\frac1n(|x_{n+1}-x_1|)\leq \overline\lim \frac{2\|x\|_\infty}{n} = 0, \]
    and then $|\langle m, \tau x - x\rangle|\leq p(\tau x - x) = 0$, $\langle m, \tau x\rangle = \langle m, x\rangle$. 
\end{proof}
\chapter{Banach空间的对偶理论}



\begin{exercise}
    设 $E$ 是赋范空间, 并设 $E^*$ 是可分的.
    \begin{enumerate}[(a)]
        \item 令 $(f_n)_{n\geq 1}$ 是 $E^*$ 中的稠密子集.
          选出 $E$ 中的序列 $(x_n)$ 使得 $f_n(x_n)\geq\frac{\|f_n\|}{2}$.
        \item 任取 $f\in E^*$. 证明: 若对每个 $x_n$ 有 $f(x_n)=0$, 则 $f=0$.
        \item 由此导出$\Span(x_1, x_2, \cdots)$在 $E$ 中稠密且 $E$ 是可分的. 
        \item 证明: 一个 Banach 空间是可分且自反的当且仅当它的对偶空间是可分且自反的. 
        \item 举一个可分赋范空间但其对偶空间不可分的例子.
    \end{enumerate}
\end{exercise}

\begin{proof}
    (a) 由定义 $\|f_n\|=\sup\limits_{\|x\|\leq 1}|f_n(x)|$, 
    故存在序列 $(\widetilde{x_n})_{n\geq 1}\subset\closure{B_E}$, 
    使得 $|f_n(\widetilde{x_n})|\geq\frac{\|f_n\|}{2}$.
    令 $x_n=\widetilde{x_n}\sgn f_n(\widetilde{x_n})$, 则 $f_n(x_n)\geq\frac{\|f_n\|}{2}$.

    (b) 任意取定 $\varepsilon>0$, 因 $(f_n)$ 在 $E^*$ 中稠密, 
    故存在 $f_n$, 使得 $\|f_n-f\|<\varepsilon$, 又
    \[\|f_n-f\|\geq |f_n(x_n)-f(x_n)|=|f_n(x_n)|\geq\frac{\|f_n\|}{2}.\]
    故 $\|f_n\|<2\varepsilon$, 从而 $\|f\|\leq\|f-f_n\|+\|f_n\|<3\varepsilon$,
    由 $\varepsilon$ 的任意性即得 $f=0$.

    (c)记$A=\closure{\Span(x_1,x_2,\cdots)}$, 显然 $A$ 是 $E$ 的闭向量子空间,
    假设 $A\neq E$, 则存在 $x_0\in E\setminus A$, 由推论 8.1.16 知
    存在 $f\in E^*$, 使得 $f|_A=0$ 且 $f(x_0)=d(x_0,A)>0$.
    这与 (b) 中结论矛盾, 故假设不成立, 所以 $\closure{\Span(x_1,x_2,\cdots)}=E$.

    另法: $F:=\Span(x_1,x_2,\cdots)$ 为 $E$ 的向量子空间. 任取 $f\in E^*$ 且 $f|_F=0$,
    由 (b) 知 $f=0$, 由推论 8.2.7 知 $\Span(x_1,x_2,\cdots)$ 在 $E$ 中稠密.

    记 $\FQ=\{q_i\}_{i=1}^{\infty}$, 
    则 $\{\sum_{i=1}^nq_ix_i,n\geq 1\}$ 是 $E$ 的可数稠密子集, 故 $E$ 可分.

    (d) \necessary
    因为 $E$是可分且自反的 Banach 空间, 所以 $E^{**}=E$ 可分,由 (c) 中结论知 $E^*$ 可分且自反.
    \sufficient
    当 $E^*$ 是可分且自反的, 直接由 (c) 中结论知 $E$ 是可分且自反的.

    (e) $\ell_1$ 可分, 但$\ell_1^*=\ell_{\infty}$ 不可分.
\end{proof}



\begin{exercise}
  设 $E$ 是 Banach 空间, $B\subset E^*$.
  \begin{enumerate}[(a)]
    \item 证明 $B$ 是相对 $w^*$-紧的当且仅当 $B$ 是有界的.
    \item 假设 $B$ 是有界的且 $E$ 是可分的. 证明 $(B, \sigma(E^*,E))$可度量化. 
  \end{enumerate}
\end{exercise}

\begin{proof}
    (a) \sufficient 若 $B$ 有界, 则存在 $r>0$, 使得 $B\subset r\closure{B}_{E^*}$,
    但 $\closure{B}_{E^*}$ 是 $w^*$-紧的, 故 $r\closure{B}_{E^*}$ 是 $w^*$-紧的,
    从而 $B$ 是相对 $w^*$-紧的.

    \necessary
    若 $B$ 是相对 $w^*$-紧的, 则 $\closure{B}$ 是 $w^*$-紧的,
    而对任意 $x\in E$, $\hat{x}\in E\hookrightarrow E^{**}$ 连续,
    从而 $\hat{x}(\closure{B})$ 是 $\FK$ 中紧集, 有界, 故
    \[\sup_{f\in\closure{B}}|f(x)|=\sup_{f\in\closure{B}}|\hat{x}(f)|<\infty.\]
    由共鸣定理得
    \[\sup_{f\in\closure{B}}\|f\|<\infty.\]
    故 $\closure{B}$ 有界, 从而 $B$ 有界.

    (b) (See H.~Brezis \cite[Theorem 3.28]{brezis_functional_2011})
    从下面的证明当中可以看出, 我们只需要证明 $B=B_{E^*}$ 的情形即可.

    取 $B_E$ 中的可数稠密子集 $(x_n)_{n\geq 1}$. 对于每个 $f\in E^*$, 令
    \[[f] = \sum_{n=1}^{\infty} \frac{1}{2^n} |f(x_n)|.\]
    显然 $[\quad]$ 是 $E^*$ 上的范数且 $[f]\leq\norm{f}$.
    记 $d(f,g) = [f-g]$ 为相应的度量, 我们下面证明 $d$ 在 $B_{E^*}$
    上诱导的度量和 $\sigma(E^*,E)$ 在 $B_{E^*}$ 上的限制是一样的.

    一方面, 任取 $f_0\in B_{E^*}$ 和 $f_0$ 在 $\sigma(E^*,E)$ 中的邻域 $V$.
    我们需要找到某个 $r>0$ 使得
    \[U = \{f\in B_{E^*}\colon d(f, f_0) < r\} \subset V.\]
    可以假设 $V$ 具有如下形式
    \[V = \{f\in B_{E^*}\colon |(f-f_0)(y_i)|<\epsilon
      \quad \forall i=1,2,\ldots,k\},\]
    其中 $\epsilon>0$, $y_1,y_2,\ldots,y_k\in E$.
    不失一般性, 可以假设 $\norm{y_i}\leq 1$ ($\forall i=1,2,\ldots,k$).
    对于每个 $i$, 存在某个整数 $n_i$ 使得
    \[\norm{y_i - x_{n_i}} < \epsilon/4.\]
    选取 $r>0$ 足够小使得
    \[2^{n_i}r < \epsilon/2\quad\forall i=1,2,\ldots,k.\]
    我们断言对于这样的 $r$, 有 $U\subset V$. 事实上, 因为 $d(f,f_0)<r$, 故
    \[\frac{1}{2^{n_i}}|(f-f_0)(x_{n_i})| < r\quad\forall i=1,2,\dots,k.\]
    因此
    \[|(f-f_0)(y_i)| = |(f-f_0)(y_i-x_{n_i}) + (f-f_0)(x_{n_i})|
      < \frac{\epsilon}{2} + \frac{\epsilon}{2}.\]
    从而 $f\in V$.

    另一方面, 任取 $f_0\in B_{E^*}$ 和 $r>0$, 我们需要找到 $f_0$ 在
    $\sigma(E^*,E)$ 中的某个邻域 $V$ 使得
    \[V\subset U=\{f\in B_{E^*}\colon d(f,f_0)<r\}.\]
    取 $V$ 为
    \[V = \{f\in B_{E^*}\colon |(f-f_0)(x_i)|<\epsilon
      \quad\forall i=1,2,\dots,k\},\]
    其中 $\epsilon$ 和 $k$ 待定. 对于 $f\in V$, 我们有
    \begin{align*}
      d(f,f_0)
      & = \sum_{n=1}^k \frac{1}{2^n}|(f-f_0)(x_n)|
          + \sum_{n=k+1}^{\infty} \frac{1}{2^n}|(f-f_0)(x_n)| \\
      & < \epsilon + 2\sum_{n=k+1}^{\infty} \frac{1}{2^n}
        = \epsilon + \frac{1}{2^{k-1}}.
    \end{align*}
    因此只需要选取 $\epsilon=\frac{r}{2}$ 以及 $k$ 充分大 (使得 $\frac{1}{2^{k-1}}<\frac{r}{2}$)
    就可以了.
\end{proof}


\begin{exercise}
    设 $E$ 是赋范空间, $A\subset E$.
  \begin{enumerate}[(a)]
    \item 假设 $A$ 是弱紧的. 证明: 若对任意 $x^*\in E^*$,
      $\{x^*(x)\colon x\in A\}$ 是有界的, 则 $A$ 是有界的.
    \item 假设 $A$ 有界且 $E^*$ 可分. 证明: $(A, \sigma(E, E^*))$ 可度量化.
  \end{enumerate}
\end{exercise}

\begin{proof}
    (a) Let $\varphi_x : E^*\to \mathbb R,f\mapsto \lrangle{f}{x}$, and we obtain $\|\varphi_x\| = \|x\| < +\infty$ which means $\varphi_x\in E^{**}$. Since $A$ is compact in weak topology, then $f(A)$ is compact in $\mathbb R$ and 
    \[ \sup_{x\in A}|\lrangle{\varphi_x}{f}| = \sup_{x\in A} |\lrangle{f}{x}| < +\infty, \forall f \in E^* \]
    By Uniform Boundedness Principle, we have $\sup_{x\in A}\|x\| = \sup_{x\in A} \|\varphi_x\| < +\infty$. 
        
    (b) (See H.~Brezis \cite[Theorem 3.29]{brezis_functional_2011}) 
\end{proof}


\begin{exercise}
  设 $E$ 是自反空间. 证明: $E$ 中的每个有界序列 $(x_n)$ 有弱收敛子列.
\end{exercise}

\begin{proof}
  See H.~Brezis \cite[Theorem 3.18]{brezis_functional_2011}.
\end{proof}

\begin{exercise}[7]
    设 $E$ 和 $F$ 是两个 Banach 空间, $u: E^{*} \rightarrow F^{*}$ 是线性映射. 证明: 映射
    \[
    u:(E^{*},\sigma(E^{*}, E)) \rightarrow(F^{*}, \sigma(F^{*}, F)).
    \]
    连续的充分必要条件是存在 $v\in\mathcal{B}(F, E)$, 使得 $u=v^{*}$. 
\end{exercise}

\begin{proof}
    Let $J_X$ denote the canonial injection from $X$ to $X^{**}$. 
    
    If $\exists v\in B(F, E)$ s.t. $u = v^*$, we immediately find $u = v^*\in B(E^*, F^*)$. It suffices to check that $\forall x\in F$, $J_F(x)\circ u$ is continous from $E$ weak$^*$ to $\mathbb R$. We obtain  
    \[ \begin{aligned}
        J_F(x)\circ u(f) = \lrangle{uf}{x}_{F^*, F}=\lrangle{f}{vx}_{E^*, E}
    \end{aligned} \]
    and $vx\in E$, hence $J_F(x)\circ u = J_E(vx)$ is continous on $\sigma(E^*, E)$. Therefore $u$ is continous from $(E, \sigma(E^*, E))$ to $(F, \sigma(F^*, F))$. 

    Conversely, $\forall x \in F$, denote $g_x: E^*\to \mathbb R, f\mapsto \lrangle{u(f)}{x}$. We obtain that $g_x\in E^{**}$ is w$^*$ continous on $E^*$. Then exists $x^*\in E$ s.t. 
    \[\lrangle{g_x}{f^*}_{E^{**}, E^*} = \lrangle{f^*}{x^*}, \forall f^*\in E^*. \]
    Let $v: x\in F\mapsto x^*\in E$. 
    It remains to prove that $v\in B(F, E)$. $\forall (x_n, y_n)\in G(v)\to (x_0, y_0)\in F\times E$, where $x_n\in F, y_n = v(x_n)\in E$, we obtain
    \[ \lrangle{f}{y_0} = \lim\lrangle{f}{y_n} = \lim\lrangle{uf}{x_n} = \lrangle{uf}{x_0} = \lrangle{f}{vx_0}, \forall f\in E^*. \]
    Then $y_0 = vx_0$, $G(v)$ is close which means $v\in B(F, E)$.
\end{proof}



\begin{exercise}[9]
    构造空间 $\ell_{\infty}^{*}$ 的单位球面上的一个序列, 使其没有 $w^{*}-$收敛的子序列. 
    这是否与 Banach-Alaoglu 定理矛盾? 如果在 $\ell_{\infty}$ 上有什么结论?
\end{exercise}

\begin{proof}
    It can be happened that $B$ is compact but $(f_n)\subset B$ has no converge subsequence 
    if $B$ is not metrizable. However, if $B\subset \ell_\infty = \ell_1^*$, $\ell_1$ is separable 
    which means every bounded subsets of $\ell_1^*$ are metrizable. Therefore for any bounded subset 
    $B$ of $\ell_\infty$, any sequence $(x_n)\subset B$ has a converge subsequence in $(\ell_1^*, \sigma(\ell_1^*, \ell_1))$.  

    Denote $e_n = (0, \cdots, 0, 1, 0, \cdots)$, and $\{e_n\}\subset \ell_1\subset \ell_\infty^*$. 
    $\|e_n\|_{\ell_1} = 1$, and then $e_n\in S(\ell_\infty^*)$. Assume that exists $n_k$ such that 
    $e_{n_k}\to f\in \ell_\infty^*$ in $\sigma(\ell_\infty^*, \ell_1)$. Then $\forall (x_n)\in \ell^\infty$ 
    such that $x_n$ is not converged, $\lrangle{e_{n_k}}{x} = x_{n_k}\to \lrangle{f}{x}$ but $x_n$ is not converged. 
    Therefore $(e_{n})$ has not any $w^*$-converge subsequence. 
\end{proof}



\begin{exercise}
    刻画 $\ell_p$ 的对偶空间比刻画 $L_p$ 的对偶要容易. 试不用课程中的结论, 直接导出
    \[\ell_{p}^{*} \cong \ell_{q},\;1\leq p<\infty,\quad\frac{1}{p}+\frac{1}{q}=1.\]
    并且证明
    \[c_{0}^{*} \cong \ell_{1}.\]
    由此导出 $c_{0}, \ell_{1}$ 和 $\ell_{\infty}$ 都不是自反的.
\end{exercise}

\begin{proof}
    任取$a=(a_i)_{i\geq1}\in\ell_q$,令
    \[f_a(x)=\sum_{i=1}^{\infty}a_ix_i,x=(x_i)\in\ell_p.\]
    则由 H\"older 不等式得
    \[|f_a(x)|\leq\|x\|_{\ell_p}\|a\|_{\ell_q},x=(x_i)\in\ell_p.\]
    即得 $f_a\in\ell_p^*$,下面任取 $f\in\ell_p^*$,证明反向的结论成立.
    对任意 $x=(x_1,x_2,\cdots)\in\ell_p$,
    记 $x^{(n)}=(x_1,\cdots,x_n,0,\cdots)$, 则$x^{(n)}\xrightarrow{\|\cdot\|_p}x$.
    用 $e_1,e_2,\cdots$ 表示 $\ell_p$ 中的标准基, 因为$f\in\ell_p^*$, 所以
    \[f(x)=\lim_{n\to\infty}f(x^{(n)})=\lim_{n\to\infty}\sum_{i=1}^nf(e_i)x_i=\sum_{i=1}^{\infty}a_ix_i.\]
    下证 $(a_i)_{i\geq1}\in\ell_q.\forall n\geq 1$, 令 $a^{(n)}=(a_1,\cdots,a_n,0,\cdots)$, 取
    \[x^{(n)}=\left(\frac{|a_1|^{q-1}\sgn a_1}{\|a^{(n)}\|^{\frac{q}{p}}_q},\cdots,\frac{|a_n|^{q-1}\sgn a_n}{\|a^{(n)}\|^{\frac{q}{p}}_q},0,\cdots\right)\]
    则由$p(q-1)=q$知
    \[\|x^{(n)}\|_{\ell_p}=\left(\frac{|a_1|^q}{\|a^{(n)}\|^q_q}+\cdots+\frac{|a_n|^q}{\|a^{(n)}\|^q_q}\right)^{1/p}=1\]
    并且由$\sgn a_i\cdot f(e_i)=|a_i|$得
    \[f(x^{(n)})=\frac{|a_1|^{q-1}\cdot|a_1|}{\|a^{(n)}\|^{\frac{q}{p}}_q}+\cdots+\frac{|a_n|^{q-1}\cdot|a_n|}{\|a^{(n)}\|^{\frac{q}{p}}_q}=\frac{\|a^{(n)}\|_q^q}{\|a^{(n)}\|_q^\frac{q}{p}}=\|a^{(n)}\|_q\]
    因此有$\|a^{(n)}\|_q\leq\|f\|\Rightarrow\|a\|_q\leq\|f\|$,于是建立了$\ell_p^*$到$\ell_q$的等距同构映射:
    \[J:\ell_p^*\to\ell_q,f\mapsto a=(a_1,a_2,\cdots)\]
    下面证明$c_0^*=\ell_1,c_0=\{x=(x_n)_{n\geq1}:\lim_{n\to\infty}x_n=0\}\subset\ell_{\infty}$,
    这里的证明方法与上面证明$\ell_p^*=\ell_q(1\leq p<\infty)$是完全类似的.
    $\forall a=(a_i)_{i\geq 1}\in\ell_1$,令
    \[f_a(x)=\sum_{i=1}^{\infty}a_ix_i,x=(x_i)\in c_0\]
    则
    \[|f_a(x)|\leq\|a\|_{\ell_1}\|x\|_{c_0},x=(x_i)\in c_0\]
    故 $f_a\in c_0^*$且$\|f_a\|\leq\|a\|_{\ell_1}$.

    再任取$f\in c_0^*,\forall x=(x_1,x_2,\cdots)\in c_0$,记$x^{(n)}=(x_1,\cdots,x_n,0,\cdots)$,用$e_1,e_2,\cdots$表示$c_0$中的标准基,则
    \[f(x)=\lim_{n\to\infty}f(x^{(n)})=\lim_{n\to\infty}f(e_i)x_i=\sum_{i=1}^{\infty}a_ix_i\]
    下证$a=(a_i)_{i\geq 1}\in\ell_1.\forall n\geq1$,令$a^{(n)}=(a_1,\cdots,a_n,0,\cdots)$,取
    \[x^{(n)}=(\sgn a_1,\cdots,\sgn a_n,0,\cdots)\]
    则\[\|x^{(n)}\|_{c_0}=1\]
    且\[f(x^{(n)})=\sum_{i=1}^nf(e_i)\sgn a_i=\sum_{i=1}^n|a_i|=\|a^{(n)}\|_{\ell_1}\]
    故$\|a^{(n)}\|_{\ell_1}\leq\|f\|\Rightarrow\|a\|_{\ell_1}\leq\|f\|$,这样就建立了等距同构映射:
    \[J:c_0^*\to\ell_1,f\mapsto a=(a_1,a_2,\cdots)\]
    因$c_0^{**}=\ell_1^*=\ell_{\infty}$,故$c_0,\ell_1,\ell_{\infty}$不自反.
\end{proof}



\begin{exercise}[12]
    (a) 设 $E$ 是自反 Banach 空间. 证明: 每个 $\varphi \in E^{*}$ 可以达到范数, 
    即存在 $x_{0} \in E$, 使得 $\left|\varphi\left(x_{0}\right)\right|=\|\varphi\|$.
    
    (b) 由此导出已知的事实: $\ell_{1}, L_{1}(0,1)$ 和 $C([0,1])$ 都不是自反的. 
    (提示: 对空间 $C([0,1])$ 考察泛函 $\varphi=\int_{0}^{\frac{1}{2}}-\int_{\frac{1}{2}}^{1}$.)

    (c) 证明: $C^{1}([0,1])$ 不是自反的, 
    这里 $C^{1}([0,1])$ 上赋予的范数是 $\|f\|=$ $\|f\|_{\infty}+\left\|f^{\prime}\right\|_{\infty}$ .
    (提示: 可以考虑 $C^{1}([0,1])$ 中由在原点处取零值的函数构成的子空间.)
\end{exercise}

\begin{proof}
    (a) 由 Hahn-Banach 延拓定理知
    \[\|\varphi\|=\sup_{x^{**}\in E^{**},\|x^{**}\|\leq 1}|\lrangle{x^{**}}{\varphi}|\]
    中上确界可以达到, 又 $E$ 自反, 故
    \[\|\varphi\|=\sup_{x\in E,\|x\|\leq 1}|\varphi(x)|\]
    中上确界可以达到, 即存在 $x_0\in E$, 使得 $|\varphi(x_0)|=\|\varphi\|$.

    (b) 若 $\ell_1$ 自反, 则每个 $y\in\ell_1^*\cong\ell_{\infty}$ 可以达到范数,
    即存在 $x\in\ell_1$, $\|x\|_{\ell_1}\leq 1$, 使得 $|\sum_{n=1}^{\infty}x_ny_n|=\|y\|_{\infty}$, 因此不等式链
    \[\left|\sum_{n=1}^{\infty}x_ny_n\right|\leq\sum_{n=1}^{\infty}|x_ny_n|\leq\|y\|_{\infty}\sum_{n=1}^{\infty}|x_n|\leq\|y\|_{\infty}\]
    处处取等, 故对任意 $n\geq 1$, 有 $|y_n|=\|y\|_{\infty}$, 这与 $y$ 的任意性相矛盾, 故 $\ell_1$ 非自反.

    若 $L_1(0,1)$ 自反, 则每个 $g\in L_1(0,1)^*\cong L_{\infty}(0,1)$ 可以达到范数,
    即存在 $f\in L_1(0,1)$, $\|f\|_1\leq 1$, 使得 $|\int_0^1 fg|=\|g\|_{\infty}$, 故不等式链
    \[\left|\int_0^1 fg\right|\leq\int_0^1 |fg|\leq\|g\|_{\infty}\int_0^1 |f|\leq\|g\|_{\infty}\]
    处处取等, 故有 $|g|=\|g\|_{\infty}$ a.e., 这与 $g$ 的任意性相矛盾, 故 $L_1(0,1)$ 非自反.

    若 $C([0,1])$ 自反, 则考察其上线性泛函 $\varphi=\int_0^{\frac{1}{2}}-\int_{\frac{1}{2}}^1$.
    对于任意 $f\in C([0,1])$, 有 $|\varphi(f)|\leq\|f\|$, 故 $\varphi\in C([0,1])^*$, 且 $\|\varphi\|\leq 1$.
    另一方面, 对任意 $\varepsilon>0$, 取 $f\in C([0,1])$ 满足
    \[f(x)=\begin{cases}
        1, & 0\leq x\leq\frac{1}{2}-\varepsilon, \\
        \frac{1}{\varepsilon}\left(\frac{1}{2}-x\right), & \frac{1}{2}-\varepsilon<x<\frac{1}{2}+\varepsilon, \\
        -1, & \frac{1}{2}+\varepsilon\leq x\leq 1.
    \end{cases}\]
    则有 $|\varphi(f)|>1-2\varepsilon$, 故 $\|\varphi\|=1$.

    由于 $C([0,1])$ 自反, 故 $\varphi$ 可以达到范数, 即存在 $f\in C([0,1])$, $\|f\|\leq 1$, 使得
    $|\int_0^{\frac{1}{2}}f-\int_{\frac{1}{2}}^1 f|=1$, 于是不等式链
    \[\left|\int_0^{\frac{1}{2}}f-\int_{\frac{1}{2}}^1f\right|\leq\int_0^{\frac{1}{2}}|f|+\int_{\frac{1}{2}}^1 |f|\leq 1\]
    处处取等, 故必须 $f|_{[0,\frac{1}{2})}=1$ a.e., $f|_{(\frac{1}{2},1]}=-1$ a.e.,
    矛盾, 故 $C([0,1])$ 非自反.

    (c) 若 $E=C^1([0,1])$ 自反, 考虑闭子空间 $F=\{f\in E\mid f(0)=0\}$, 则 $F$ 自反.
    考虑
    \[\varphi: C([0,1])\to F,\;\varphi(f)=\int_0^x f(t)\diff t.\]
    则 $\varphi$ 为线性双射, 且有
    \[\|\varphi(f)\|=\|\varphi(f)\|_{\infty}+\|f\|_{\infty}\leq 2\|f\|_{\infty}.\]
    故 $\varphi$ 为连续线性双射, 由开映射定理知 $\varphi$ 是同构, 但 $C([0,1])$ 不自反,
    矛盾, 故 $C^1([0,1])$ 不自反.
\end{proof}


\begin{exercise}[16]
  称 Banach 空间 $E$ 为一致凸的, 若对任意 $\varepsilon>0$, 存在 $\delta>0$ 使得
  \[x,y\in\closure{B}_E, \|x-y\|\geq\varepsilon
    \Longrightarrow\biggl\|\frac{x+y}{2}\biggr\|\leq 1-\delta.\]
  设 $(x_n)$ 是一致凸空间 $E$ 中弱收敛到 $x$ 的序列, 并有 $\lim_n\|x_n\| = \|x\|$.
  证明: $(x_n)$ 依范数收敛到 $x$. 举例说明条件 $\lim_n\|x_n\|=\|x\|$ 是必需的.
\end{exercise}

\begin{proof}
  若 $x = 0$, 则结论显然成立, 下面假设 $x\neq 0$. 令
  \[\lambda_n = \max(\|x_n\|, \|x\|),\quad y_n = \lambda_n^{-1}x_n,
    \quad y = \|x\|^{-1}x.\]
  由于 $\lim_{n\to\infty}\|x_n\| = \|x\|$,
  故 $\lambda_n\to \|x\|$, 且在弱拓扑 $\sigma(E,E^*)$ 中 $y_n\rightharpoonup y$. 因此
  \[\|y\| \leq \liminf_{n\to\infty} \|(y_n+y)/2\|.\]
  注意到 $\|y\|=1$ 且 $\|y_n\|\leq 1$, 结合上式可知 $\|(y_n+y)/2\|\to 1$.
  而 $E$ 是一致凸的, 从而必有 $\|y_n-y\|\to 0$. 因此 $x_n\to x$.
\end{proof}


\begin{exercise}
  \begin{enumerate}[(a)]
    \item 证明: Hilbert 空间是一致凸的.
    \item 用 Clarkson 不等式证明: 若 $2\leq p<\infty$, 则 $L_p$ 空间是一致凸的.
  \end{enumerate}
\end{exercise}

\begin{proof}
  (a) 对于任意的 $\varepsilon>0$, 由平行四边形公式
  \[\|x\|^2 + \|y\|^2 = 2\biggl(\biggl\|\frac{x+y}{2}\biggr\|^2
    + \biggl\|\frac{x-y}{2}\biggr\|^2\biggr),\]
  知当 $x,y\in\closure{B}_E$ 且 $\|x-y\|\geq\varepsilon$ 时, 有
  \[\biggl\|\frac{x-y}{2}\biggr\|^2 \leq 1-\varepsilon^2.\]
  故 $\bigl\|\frac{x-y}{2}\bigr\| < 1-\delta$, 其中 $\delta = 1-(1-\varepsilon^2)^{1/2}$.
\end{proof}


\begin{exercise}<Milman Pettis>
  证明一致凸空间是自反的.
\end{exercise}

\begin{proof}
  See H.~Brezis \cite[Theorem 3.31]{brezis_functional_2011}.
\end{proof}
\setcounter{chapter}{10}
\chapter{紧算子}


\begin{exercise}
    设 $E$ 是 Banach 空间, $T\in\mathcal{B}(E)$.
    并设 $(\lambda_n)$ 是 $\rho(T)$ 中收敛到 $\lambda\in\FK$ 的数列.
    证明: 若 $(R(\lambda_n,T))$ 在 $\mathcal{B}(E)$ 中有界, 则 $\lambda\in\rho(T)$.
\end{exercise}



\begin{exercise}
    设 $E$ 是 Banach 空间, $T\in\mathcal{B}(E)$.
    证明: 对任意 $\varepsilon>0$, 存在 $\delta>0$, 使得对任意 $S\in\mathcal{B}(E)$, 有
    \[\|T-S\|<\delta\Rightarrow\sigma(S)\subset\{\lambda\in\FK\mid d(\lambda,\sigma(T))<\varepsilon\}.\]
\end{exercise}



\begin{exercise}
    设 $1\leq p\leq\infty$, 定义 $\ell_p$ 上的算子 $S$ (前移算子) 为 $S(x)(n)=x(n+1)$,
    这里 $x=(x(n))_n\in\ell_p$.

    (a) 证明: 当 $p<\infty$ 时, $\sigma_p(S)=\{\lambda\in\FK\mid |\lambda|<1\}$;
    当 $p=\infty$ 时, $\sigma_p(S)=\{\lambda\in\FK\mid |\lambda|\leq 1\}$.

    (b) 由此导出 $\sigma(S)=\{\lambda\in\FK\mid |\lambda|\leq 1\}$.
\end{exercise}



\begin{exercise}
    设 $E$ 是 Banach 空间, $T$ 是 $E$ 上的线性等距映射. 并记
    \[D=\{\lambda\in\FK\mid |\lambda|<1\},\quad C=\{\lambda\in\FK\mid |\lambda|=1,\quad\closure{D}=D\cup C\}.\]

    (a) 证明: $\sigma_p(T)\subset C$, $\sigma(T)\subset\closure{D}$; 并且当 $\lambda\in D$ 时, 有
    \[\lambda\in\rho(T)\Leftrightarrow (\lambda-T)(E)=E.\]

    (b) 假设 $(\lambda_n)\subset D\cap\rho(T)$ 收敛到 $D$ 中元素 $\lambda$. 证明 $\lambda\in\rho(T)$.

    (c) 证明: $D\cap\rho(T)$ 在 $D$ 中既是开集又是闭集. 由此导出 $D\cap\rho(T)$ 是空集或者 $D\cap\rho(T)=D$.

    (d) 证明: $\sigma(T)$ 或者包含于 $C$ 中或者等于 $\closure{D}$, 并且前者成立的充分必要条件是 $T$ 为满射.

    (e) 假设 $E=\ell_p$, $1\leq p\leq\infty$, 且 $T$ 为 $E$ 的后移算子:
    \[T(x)(1)=0\quad]\text{且}\quad T(x)(n)=x(n-1),\; n>1.\]
    证明: $\sigma(T)=\closure{D}$ 并且 $\sigma_p(T)=\varnothing$.
\end{exercise}



\begin{exercise}
    设 $X$ 是紧 Hausdorff 空间, 并有 $\varphi\in C(X)$.
    设 $M_{\varphi}$ 表示 $C(X)$ 上由 $\varphi$ 确定的乘法算子: $M_{\varphi}(f)=\varphi f$.
    证明: $\sigma(M_{\varphi})=\varphi(X)$, 并且 $\sigma_p(M_{\varphi})$ 由满足如下性质的 $\lambda$ 构成:
    $\{\varphi=\lambda\}$ 内部是空集.

    此外, 当 $M_{\varphi}$ 定义在 $L_p(\mu)$ 上, $1\leq p\leq\infty$, 其中 $\mu$ 是 $X$ 上的正则测度,
    我们有什么结论?
\end{exercise}



\begin{exercise}
    设 $X$ 是度量空间, $E=C_b(X)$ 是 $X$ 上的有界连续复函数构成的 Banach 空间, 其上赋予范数:
    \[\|f\|_{\infty}=\sup_{x\in X}|f(x)|.\]
    并设 $T$ 是 $E$ 上的正算子, 即对每个 $f\geq 0$, 有 $T(f)\geq 0$.

    (a) 证明: 任取 $f\in E$, 有 $|T(f)|\leq T(|f|)$.

    (b) 设 $\lambda\in\FC$ 且 $|\lambda|>r(T)$. 证明
    \[|R(\lambda,T)(f)|\leq R(|\lambda|,T)(|f|).\]
    由此导出
    \[\|R(\lambda,T)\|\leq\|R(|\lambda|,T)\|.\]

    (c) 证明: $r(T)\in\sigma(T)$.
\end{exercise}



\begin{exercise}
    设 $H$ 是 Hilbert 空间, $E$ 和 $F$ 是 $H$ 的两个闭的正交补子空间.
    假设 $T\in\mathcal{B}(E)$ 且 $T(E)\subset E$, $T(F)\subset F$. 证明
    \[\sigma(T)=\sigma(T|_E)\cup\sigma(T|_F).\]
    作为应用, 确定 $\sigma(T)$, 其中 $T\in\mathcal{B}(\ell_2)$ 定义为
    \[T(x)(n)=x(n+2)+\frac{1+(-1)^n}{2}x(n),\quad\forall n\geq 1,\forall x=(x(n))_{n\geq 1}\in\ell_2.\]
\end{exercise}



\begin{exercise}
    设$E$和$F$是赋范空间.证明下面的命题成立:

    (a) 若$(x_n)$是$E$的弱收敛序列, 则$(x_n)$有界.

    (b) 若$T\in\mathcal{B}(E,F)$且$x_n$弱收敛到$x$, 则$T(x_n)$弱收敛到$T(x)$.

    (c) 若$T\in\mathcal{B}(E,F)$是紧算子且$x_n$弱收敛到$x$, 则$T(x_n)$依范数收敛到$T(x)$.

    (d) 若$E$自反, $T\in\mathcal{B}(E,F)$且当$x_n$弱收敛到$x$时, 有$T(x_n)$依范数收敛到 $T(x)$, 则$T$是紧算子.

    (e) 若$E$自反, 且$T\in\mathcal{B}(E,\ell_1)$或$T\in\mathcal{B}(c_0,E)$, 则$T$是紧算子. 
\end{exercise}

\begin{proof}
    (a) 若 $(x_n)$ 是 $E$ 中的弱收敛序列, 设其极限为$x$, 
    则对任意 $f\in E^*$, 有 $\lim\limits_{n\rightarrow \infty}f(x_n)=f(x)$.
    令 $\widehat{x_n}(f)=f(x_n)$, 则$\widehat{x_n}\in\mathcal{B}(E^*,\mathbb{R})$, 
    且对任意 $f\in E^*$, $(\widehat{x_n}(f))_{n\geq 1}$有界. 因此由Banach-Steinhaus定理, 
    $\sup\limits_{n\geq 1}\|x_n\|=\sup\limits_{n\geq 1}\|\widehat{x_n}\|<\infty$, 
    从而$(x_n)_{n\geq 1}$有界. 

    (b) 若$T\in\mathcal{B}(E,F)$, 则$T^*\in\mathcal{B}(F^*,E^*)$, 则对任意$f\in F^*$, 
    \[\lim_{n\rightarrow \infty}\langle f,T(x_n)\rangle=\lim_{n\rightarrow \infty}\langle T^*(f),x_n\rangle=\langle T^*(f),x\rangle=\langle f,T(x)\rangle.\] 

    (c) 由 (a) 知 $(x_n)_{n\geq 1}$有界, 而 $T$ 是紧算子, 故 $(T(x_n))_{n\geq 1}$ 相对紧, 
    从而 $(T(x_n))_{n\geq 1}$ 任意子列有收敛子列 $(T(x_{n_k}))_{k\geq 1 }$, 
    且该子列必依范数收敛于 $T(x)$, 这是因为若$(T(x_{n_k}))$ 依范数收敛到 $y$, 
    则对任意 $f\in F^*$, 我们有
    \[|f(T(x_{n_k}))-f(y)|\leq \|f\|\|T(x_{n_k})-y\|.\]
    因此 $T(x_{n_k})$ 弱收敛于 $y$, 但由 (b) 知, $T(x_n)$ 弱收敛于 $T(x)$, 
    因此 $y=f(x)$. 若$T(x_n)$不依范数收敛于$T(x)$, 则对任意正整数$N$,
    存在 $n_0>N$, 使得 $\|T(x_{n_0})-T(x)\|>1$, 
    从而可选取一子列 $(T(x_{n_k}))_{k\geq 1}$, 使得 $\|T(x_{n_k})-T(x)\|\geq1$, 
    但由上述讨论可知 $T((x_{n_k}))_{k\geq 1}$ 有收敛于 $T(x)$ 的子列, 矛盾!
    因此$T(x_n)$依范数收敛于$T(x)$.

    (d)由第九章第4题的结论, 若 $E$ 是自反的, 设$(x_n)_{n\geq 1}\subset B_E$, 则存在子列$(x_{n_k})_{k\geq 1}$,
    使得$(x_{n_k})_{k\geq1}$弱收敛于$x$.由题目条件知$T(x_{n_k})$依范数收敛到$T(x)$, 这说明$T(B_E)$相对紧, 从而$T$是紧算子.

    (e)由第八章第22题的结论知, $\ell_1$的依范数收敛与弱收敛等价.故若$\ell_1$中的序列$(x_n)$弱收敛到$x$, 
    则$(x_n)$依范数收敛到$x$, 从而$T(x_n)$依范数收敛到$T(x)$, 由(d)知, $T$是紧算子. 
    若$T\in \mathcal{B}(c_0,E)$, 则 $T^*\in \mathcal{B}(E^*,\ell_1)=\mathcal{B}(\ell_1,E)$, 
    故由上一段讨论知 $T^*$ 是紧算子, 从而$T$是紧算子.
\end{proof}



\begin{exercise}
    设 $(e_n)$ 是 $\ell_2$ 中的标准基. 定义算子 $T:\ell_2\rightarrow \ell_2$ 为
    \[T\biggl(\sum_{n\geq 1}x_n e_n\biggr)=\sum_{n\geq 1}\dfrac{x_n}{n}e_n,\quad (x_n)_{n\geq 1}\in \ell_2.\]
    证明: $T\in \mathcal{K}(\ell_2)$.
\end{exercise}

\begin{proof}
    定义
    \[T_N\biggl(\sum_{n\geq 1}x_n e_n\biggr)=\sum_{n=1}^{N}\dfrac{x_n}{n}e_n.\]
    因 $T_N$ 为连续线性算子且 $\dim T_N(\ell_2)<\infty$, 故 $T_N$ 是有限秩算子, 
    且对任意 $x=(x_n)_{n\geq 1}\in\ell_2$, 有
    \[\begin{aligned}
        \left\|T\biggl(\sum_{n\geq 1}x_n e_n\biggr)-T_N \biggl(\sum_{n\geq 1}x_n e_n\biggr)\right\|_{}^2
        & =\left\|\sum_{n=N+1}^{\infty}\dfrac{x_n}{n}e_n\right\|^2=\sum_{n=N+1}^\infty \dfrac{|x_n|^2}{n^2}\\
        & \leq\frac{1}{(N+1)^2}\sum_{n=N+1}^\infty |x_n|^2\\
        & \leq\frac{1}{(N+1)^2}\sum_{n=1}^\infty |x_n|^2=\frac{1}{(N+1)^2}\|x\|_{}^2.
    \end{aligned}\]
    故
    \[\|T-T_N\|=\sup_{x\in\ell_2,x\neq 0}\frac{\|T(x)-T_N(x)\|_{}}{\|x\|_{}}\leq\frac{1}{N+1}.\]
    因此
    \[\lim_{N\rightarrow \infty}\|T-T_N\|\leq\lim_{N\to\infty}\frac{1}{N+1}=0.\] 
    从而 $T$ 是紧算子.
\end{proof}



\begin{exercise}
    设$(\alpha_n)_{n\geq 1}\subset\FC$. 定义算子$T\in \mathcal{B}(c_0)$为
    \[T(x)=(\alpha_n x_n)_{n\geq1},\quad x=(x_n)_{n\geq 1}\in c_0.\]
    证明: $T\in \mathcal{K}(c_0)$ 当且仅当 $\lim\limits_{n\rightarrow \infty}\alpha_n=0$.
\end{exercise}

\begin{proof}
    \necessary
    若$\lim\limits_{n\rightarrow \infty} \alpha_n\not=0$, 
    则存在 $\varepsilon_0>0$ 和子列 $(\alpha_{n_k})_{k\geq 1}$, 
    使得对任意 $k\geq 1$, 都有 $|\alpha_{n_k}|\geq\varepsilon_0$. 
    设 $e_i=(0,\cdots,0,1,0,\cdots)$ (在第$i$个坐标处是 $1$, 其余坐标都是 $0$),  
    则 $A=\left\{e_i\colon i\geq 1\right\}$在$c_0$中有界, 且 $T(e_{n_i})=\alpha_{n_i}e_{n_i}$,
    但对任意 $i\neq j$, 
    \[\|T(e_{n_i})-T(e_{n_j})\|=\max\left\{|\alpha_{n_i}|,|\alpha_{n_j}|\right\}>\varepsilon_0,\]
    故 $(Te_{n_j})_{j\geq1}$ 的任一子列都不是 Cauchy 列, 从而不收敛, 
    这说明 $T(A)$ 在 $c_0$ 不是相对紧的. 这与 $T$ 是紧算子矛盾, 故 $\lim\limits_{n\rightarrow \infty}\alpha_n=0$.

    \sufficient
    设 $T_N(x)=\sum_{n=1}^N x_n T(e_n)$, 则 $T_N$ 是有限秩算子, 且
    \[\|T(x)-T_N(x)\|=\sup_{n\geq N+1}|\alpha_n x_n|\leq \sup_{n\geq N+1}|\alpha_n| \sup_{n\geq 1}|x_n|.\]
    由于 $\lim\limits_{N\rightarrow \infty} \sup\limits_{n\geq N+1}|\alpha_n|=\limsup\limits_{n\rightarrow \infty}|\alpha_n|=\lim\limits_{n\rightarrow \infty}|\alpha_n|=0$.故
    \[\lim\limits_{N\rightarrow \infty}\|T-T_N\|\leq  \lim\limits_{N\rightarrow \infty}\sup_{n\geq N+1}|\alpha_n|=0.\]
    从而 $T$ 是紧算子.
\end{proof}



\begin{exercise}
    设 $(\varOmega,\mathcal{A},\mu)$ 是测度空间, $p\in [1,\infty)$.
    设 $\mathcal{P}$ 是由所有有限个两两不相交的具有有限正测度的可测子集构成的集合.

    (a) 对任意 $\pi=\{A_k\}_{1\leq k\leq n}\in\mathcal{P}$, 定义 $L_p(\varOmega,\mathcal{A},\mu)$ 上的算子 $P_{\pi}$ 为
    \[P_{\pi}(f)=\sum_{k=1}^n \Bigl(\frac{1}{\mu(A_k)}\int_{A_k}f\diff\mu\Bigr)\mathbbm{1}_{A_k}.\]
    证明: $P_{\pi}\in\mathcal{B}(L_p(\varOmega,\mathcal{A},\mu))$ 且 $\|P_{\pi}\|=1$.

    (b) 证明: 对任意 $L_p(\varOmega,\mathcal{A},\mu)$ 中的紧子集 $K$ 及 $\varepsilon>0$, 存在 $\pi\in P$, 使得
    \[\|P_{\pi}(f)-f\|_p<\varepsilon,\quad\forall f\in K.\]

    (c) 导出 $\mathcal{F}_r(L_p(\varOmega,\mathcal{A},\mu))$ 在 $\mathcal{K}(L_p(\varOmega,\mathcal{A},\mu))$ 中稠密.

    (d) 假设 $X$ 是一个局部紧的 Hausdorff 空间. 证明在空间 $C_0(X)$ 上有和上面类似的结果.
\end{exercise}



\begin{exercise}
    设 $E=C([0,1])$ 上赋予一致范数 $\|\cdot\|_{\infty}$, 且 $\varPhi$ 是 $[0,1]\times [0,1]$
    上的连续函数. 定义算子 $T:E\to E$ 为
    \[T(f)(s)=\int_0^1 \varPhi(s,t)f(t)\diff t,\quad s\in [0,1].\]
    证明: $T$ 是紧的.
\end{exercise}



\begin{exercise}
    设 $E=L_2(0,1)$ 且 $\varPhi\in L_2([0,1]\times [0,1])$. 定义算子 $T:E\to E$ 为
    \[T(f)(s)=\int_0^1 \varPhi(s,t)f(t)\diff t,\quad s\in [0,1].\]
    证明: $T$ 是紧的.
\end{exercise}



\begin{exercise}
    定义 $C([0,1])$ 上的算子 $T$:
    \[Tf(x)=\int_0^{1-x} f(t)\diff t,\quad x\in [0,1].\]
    证明 $T$ 是紧的并确定 $\sigma(T)$.

    设 $1\leq p\leq\infty$, 在 $L_p([0,1])$ 上定义和上面一样的算子 $T$, 回答同样的问题.
\end{exercise}



\begin{exercise}
    设 $E$ 是空间 $C([0,1])$ 或 $L_p(0,1)$, 其中 $p\in [1,\infty]$.

    (a) 在 $E$ 上定义算子 $T$ 如下:
    \[Tf(x)=\int_0^1 \min\{x,y\}f(y)\diff y,\quad x\in [0,1].\]
    求 $T$ 的谱集.

    (b) 在 $L_2(0,1)$ 上定义如下的算子 $S$:
    \[Sf(x)=\int_0^x f(y)\diff y.\]
    确定 $S^*$, 证明 $SS^*$ 和上面的算子 $T$ 在空间 $E=L_2(0,1)$ 上相同, 求出 $\|S\|$.
\end{exercise}



\begin{exercise}
    设 $H$ 是 Hilbert 空间, 并有 $T\in\mathcal{B}(H)$. 令
    \[W(T)=\{\innerp{T(x)}{x}\colon x\in H,\|x\|=1\}.\]
    设 $\sigma_{pa}(T)$ 满足如下性质的所有 $\lambda\in\FC$ 构成的集合:
    $H$ 的单位球中存在一个序列 $(x_n)$, 使得 $\innerp{T(x_n)}{x_n}\to\lambda$.

    (a) 证明: $\sigma(T)=\sigma_{pa}(T)\cup\closure{\sigma_p(T^*)}$.

    (b) 证明: $\sigma(T)\subset\closure{W(T)}$.

    (c) 在 $\FK=\FC$ 的情形下, 导出
    \[r(T)\leq\sup_{\|x\|=1}|\innerp{T(x)}{x}|\leq\|T\|.\]
\end{exercise}



\begin{exercise}
    设 $H$ 是复 Hilbert 空间. 称 $T\in\mathcal{B}(H)$ 是正规的, 若 $T^*T=TT^*$.

    (a) 证明: $T$ 是正规的当且仅当对任意 $x\in H$, 有 $\|T(x)\|=\|T^*(x)\|$.

    (b) 假设 $T$ 是正规的. 证明: 对任意 $\lambda\in\FC$, 有
    \[\ker(\lambda-T)=\ker(\conjugate{\lambda}-T^*).\]
    由此导出: $\lambda\in\sigma_p(T)$ 等价于 $\conjugate{\lambda}\in\sigma_p(T^*)$.

    (c) 证明: $T$ 的相应于不同特征值的特征子空间相互正交.

    (d) 依然假设 $T$ 是正规的.
    (i) 证明: $r(T)=\|T\|$; (ii) 导出结论:
    \[\|T\|=\sup_{\|x\|=1}|\innerp{T(x)}{x}|.\]
\end{exercise}



\begin{exercise}
    试把谱分解定理扩展到复 Hilbert 空间的正规算子上.
\end{exercise}



\begin{exercise}(Fredholm 选择定理)
    设 $T$ 是 Hilbert 空间上的紧算子, $\lambda\in\FK$ 且 $\lambda\neq 0$. 证明: 方程
    \[\lambda x-T(x)=y\]
    或者对每一个 $y\in E$ 有唯一解, 或者对某些 $y$ 有无穷多个解但对其他的 $y$ 无解.
\end{exercise}

\begin{proof}
  由于 $T$ 为紧算子, 故
  \[\ker(I-T) = \{0\}\iff R(I-T) = E.\]

  当 $\ker(I-T) = \{0\}$ 时, $I-T\colon E\to E$ 为双射, 此时对于 $\forall y\in E$,
  方程 $x-T(x)=y$ 有唯一解.

  当 $\ker(I-T) \neq \{0\}$ 时, $R(I-T)\neq E$, 故当 $y\in E\setminus R(I-T)$ 时,
  方程 $x-T(x)=y$ 无解; 当 $y\in R(I-T)$ 时, 方程 $x-T(x)=y$ 有无穷多解并且可表示为
  \[x = \sum_{i=1}^n k_ie_i,\quad k_i\in\FK,\]
  其中 $\{e_i\}$ 为 $n$ 维空间 $\ker(I-T)$ 的一组基.
\end{proof}


\begin{exercise}
    设 $(\lambda_n)$ 是有界数列, 并定义算子 $T:\ell_2\to\ell_2$ 为 $(x_n)\mapsto (\lambda_n x_n)$.

    (a) 证明 $T\in\mathcal{B}(\ell_2)$, 确定 $\|T\|$ 和 $\sigma(T)$.

    (b) 什么情况下 $T$ 是紧的?
\end{exercise}



\begin{exercise}
    定义 Hilbert 算子 $u:\ell_2\to\ell_2$ 为
    \[u(x)=\biggl(\sum_{k=1}^{\infty}\frac{x_k}{j+k}\biggr)_{j\geq 1},\forall x=(x_k)_{k\geq 1}\in\ell_2.\]
    对每个 $n\geq 1$, 令 $a_n=\bigl(1,\frac{1}{\sqrt{2}},\cdots,\frac{1}{\sqrt{n}},0,0,\cdots\bigr)$, $b_n=\frac{a_n}{\|a_n\|}$.
    证明:
    \begin{enumerate}[(a)]
        \item $b_n$ 弱收敛到 $0$.
        \item $\innerp{u(a_n)}{a_n}\geq\pi\ln n+O(1)$.
        \item $\liminf_{n\to\infty}\|u(b_n)\|\geq\pi$.
        \item $\liminf_{n\to\infty}\|(u-v)(b_n)\|\geq\pi$, $\forall v\in\mathcal{K}(\ell_2)$.
        \item $\dist(u,\mathcal{K}(\ell_2))=\pi$, 结论意味着 $u$ 不是紧的.
    \end{enumerate}
\end{exercise}



\begin{exercise}
    设 $T$ 是 Hilbert 空间 $H$ 上的自伴紧算子, $\lambda$ 是 $T$ 的非零特征值. 证明:
    \[\ker(\lambda-T)\cap (\lambda-T)(H)=\{0\}.\]
    由此导出: $\ker(\lambda-T)=\ker(\lambda-T)^2$ 且
    \[H=\ker(\lambda-T)\oplus(\lambda-T)(H).\]
\end{exercise}



\begin{exercise}
    设 $T$ 是 Hilbert 空间 $H$ 上的自伴紧算子. 对 $T$ 的每个非零特征值 $\lambda$,
    令 $P_{\lambda}$ 表示相应的特征子空间上的正交投影. 取 $x\in H$, 证明:
    \begin{equation}
        T(y)=x\tag{$\star$}
    \end{equation}
    有解当且仅当 $x\in(\ker T)^{\perp}$ 且
    \[\sum_{\lambda\in\sigma_p(T)\setminus\{0\}} \frac{\|P_{\lambda}(x)\|^2}{\lambda^2}<\infty.\]
    并证明: 若以上充分条件成立, 则 $(\star)$ 式的通解为
    \[y=z+\sum_{\lambda\in\sigma_p(T)\setminus\{0\}} \frac{P_{\lambda}(x)}{\lambda^2},\text{\ 其中\ }z\in\ker T.\]
\end{exercise}



\begin{exercise}
    设 $H$ 是一个可分的 Hilbert 空间.
    
    (a) 设 $(e_n)_n$ 和 $(f_n)_n$ 是 $H$ 中的两个规范正交基. 证明: 任取 $T\in\mathcal{B}(H)$, 都有
    \[\sum_n \|Te_n\|^2=\sum_n \|Tf_n\|^2.\]
    后面我们固定 $H$ 中的规范正交基为 $(e_n)$.
    设 $\mathcal{S}_2(H)$ 为满足如下条件的算子 $T$ 构成的集合:
    \[\|T\|^2=\biggl(\sum_n \|Te_n\|^2\biggr)^{\frac{1}{2}}<\infty.\]
    我们称这样的算子为\emph{Hilbert-Schmidt 算子}.

    (b) 假设 $\dim H=N<\infty$. 利用算子 $T$ 在基 $(e_1,\cdots,e_N)$ 下的矩阵形式,
    给出范数 $\|T\|_2$ 的一个具体刻画.

    (c) 证明: $\mathcal{S}_2(H)$ 是 $\mathcal{B}(H)$ 的双边理想并包含所有的有限秩算子.

    (d) 证明: 任取 $T\in\mathcal{S}_2)(H)$, 有 $\|T\|\leq\|T\|_2$, 并且 $(\mathcal{S}_2(T),\|\cdot\|_2)$
    是一个 Hilbert 空间.

    (e) $P_n$ 表示到空间 $\Span(e_1,\cdots,e_n)$ 上的正交投影.
    证明: 在 $\dim H=\infty$ 时, 有
    \[\lim_n \|T-TP_n\|_2=0,\forall T\in\mathcal{S}_2(H).\]
    因而, 任意的 Hilbert-Schmidt 算子是紧的.

    (f) 假设 $T$ 是自伴紧算子. 对每个 $\lambda\in\sigma_p(T)\setminus\{0\}$,
    令 $d_{\lambda}=\dim\ker(\lambda-T)$. 证明: $T\in\mathcal{S}_2(H)$ 当且仅当
    \[\sum_{\lambda\in\sigma_p(T)\setminus\{0\}}d_{\lambda}\lambda^2<\infty.\]
\end{exercise}
\chapter*{补充题目}
\addcontentsline{toc}{chapter}{补充题目}


\begin{exercise}[1]
  设 $E$ 是赋范空间.
  \begin{enumerate}[(a)]
    \item 设 $A, B\subset E$, 且 $A$ 是紧的, $B$ 是闭的, 证明 $A+B$ 也是闭的.
    \item 设 $F$ 是 $E$ 的有限维子空间, $G$ 是 $E$ 的闭子空间, $F\cap G=\{0\}$.
      证明投影算子
      \[\mathcal{P}\colon F+G \to F,\quad f+g\mapsto f\]
      是连续线性算子.
    \item 在上一命题的条件下, 证明 $F+G$ 是 $E$ 中的闭子空间.
    \item 构造 $\mathbb{R}$ 上的例子, 说明 $A$ 和 $B$ 是闭集, 但 $A+B$ 不是闭集.
  \end{enumerate}
\end{exercise}

\begin{proof}
  (a) 略.

  (b) 考虑映射
  \[\alpha\colon F\oplus G \to E,\quad (f,g) \mapsto f+g.\]
  显然 $\alpha$ 是连续线性映射. 下面断言, 存在常数 $\delta>0$, 使得
  \[\norm{f+g} \geq \delta\norm{(f,g)}.\]
  否则的话, 存在序列 $(f_n)$ 和 $(g_n)$ 使得 $\norm{(f_n, g_n)} = \|f_n\|+\|g_n\| = 1$ 且
  $\norm{f_n+g_n}\to 0$.
  由于 $F$ 为有限维的, 故由 Riesz 定理知 $F$ 中的闭单位球是紧的, 从而序列
  $(f_n)$ 存在收敛子列 $f_{n_k}\to f_0\in F$.
  结合 $\norm{f_{n_k}+g_{n_k}}\to 0$ 知 $g_{n_k}\to -f_0\in G$.
  从而 $f_0\in F\cap G=\{0\}$. 然而, 我们由 $\|f_{n_k}\|+\|g_{n_k}\| = 1$
  知 $\|f_0\|=\frac{1}{2}$, 矛盾. 因此断言成立, 由此便有
  \[\|f\| \leq \|(f,g)\| \leq \frac{1}{\delta}\|f+g\|.\]
  因此 $\mathcal{P}$ 是有界线性算子.

  (c) 假设 $F+G$ 中的序列 $(f_n+g_n)$ 满足 $f_n+g_n\to x\in E$, 下证 $x\in F+G$.
  由于 $(f_n+g_n)$ 为 Cauchy 序列, 即
  \[\|(f_n+g_n) - (f_m+g_m)\| \to 0\quad\text{as } m,n\to\infty.\]
  而由 (b) 中结论知
  \[\|(f_n+g_n) - (f_m+g_m)\| \geq \delta \|(f_n-f_m, g_n-g_m)\|.\]
  从而 $(f_n)$ 和 $(g_n)$ 皆为 Cauchy 序列, 记 $f_n\to f\in F$, $g_n\to g\in G$.
  则 $x = f+g \in F+G$.

  (d) 令 $A=\{-1,-2,\ldots\}$, $B=\{n+\frac{1}{2^n}\mid n\geq 1\}$,
  则 $A$, $B$ 都为闭集, 但 $A+B$ 不是闭集.
\end{proof}


\begin{exercise}
  设 $E$ 是向量空间, $\norm{\cdot}_1$ 和 $\norm{\cdot}_2$ 是 $E$ 上两个范数. 证明:
  \begin{enumerate}[(a)]
    \item 若 $\norm{\cdot}_1$ 和 $\norm{\cdot}_2$ 等价, 即存在常数 $C_1$, $C_2>0$ 使得
      \[C_1\|x\|_1 \leq \|x\|_2 \leq C_2\|x\|_1\quad\forall x\in E,\]
      则 $\norm{\cdot}_1$ 和 $\norm{\cdot}_2$ 在 $E$ 上诱导相同的拓扑.
    \item 若 $\norm{\cdot}_1$ 和 $\norm{\cdot}_2$ 在 $E$ 上诱导相同的拓扑, 则 $\norm{\cdot}_1$
      和 $\norm{\cdot}_2$ 等价.
    \item 假设 $\norm{\cdot}_1$ 和 $\norm{\cdot}_2$ 都是完备的. 那么, 若 $\norm{\cdot}_1$
      和 $\norm{\cdot}_2$ 在 $E$ 上诱导的拓扑可以比较, 则 $\norm{\cdot}_1$ 和 $\norm{\cdot}_2$ 等价.
  \end{enumerate}
\end{exercise}

\begin{proof}
  (a) trivial.

  (b) 取 $(E,\norm{\cdot}_2)$ 中的单位球 $B_2(0,1)$, 存在 $r>0$
  使得 $B_1(0,r)\subset B_2(0,1)$.
  对于任意 $x\in E\setminus\{0\}$, 令 $y = \frac{rx}{2\|x\|_1}$, 有
  $\|y\|_1 = \frac{r}{2} < r$, 故 $\|y\|_2 < 1$, 由此得 $\|x\|_2\leq\frac{2}{r}\|x\|_1$.
  同理可得另一方向的不等式, 于是两范数等价.

  (c) 考虑恒等映射 $\id\colon (E,\norm{\cdot}_1)\to (E,\norm{\cdot}_2)$
  并运用 Banach 定理.
\end{proof}

\nocite{*}
\bibliography{FunctionalAnalysis}
\addcontentsline{toc}{chapter}{参考文献}

\end{document}