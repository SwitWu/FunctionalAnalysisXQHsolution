\setcounter{chapter}{5}
\chapter{Baire定理及其应用}
\thispagestyle{empty}


\begin{exercise}
    (a) 证明: $\FR\setminus\FQ$ 是 $\FR$ 上的 $\mathcal{G}_{\delta}$ 集.
    并导出 $\FQ$ 不是 $\mathcal{G}_{\delta}$ 集, 且不存在函数 $f:\FR\to\FR$ 使得 $\cont{f}=\FQ$.

    (b) 定义函数 $f:\FR\to\FR$: 当 $x\in\FR\setminus\FQ$ 时, 令 $f(x)=0$; $f(0)=1$;
    若 $x$ 是非零的有理数 $\frac{p}{q}$, 这里 $\frac{p}{q}$ 是 $x$ 的不可约形式, $p\in\FZ$,
    $q\in\FN$, 令 $f(x)=\frac{1}{q}$. 证明: $\cont{f}=\FR\setminus\FQ$.

    (c) 设 $f=\mathbbm{1}_{\FQ}$. 证明: $f$ 不是第一纲的 (即不是任一函数列的极限函数),
    但是存在一列第一纲的函数逐点收敛到 $f$.
\end{exercise}

\begin{proof}
  \begin{enumerate}[(a)]
    \item 记 $\FQ=\{r_k\}_{k=1}^{\infty}$, 对于每一个 $k$, 令
      \[U_k=\FR\setminus\{r_k\}=(-\infty,r_k)\cup(r_k,+\infty).\]
      显然 $U_k$ 是开集并且
      \[\bigcap_{k=1}^{\infty}U_k=\FR\setminus\FQ.\]
      因此 $\FR\setminus\FQ$ 是 $\FR$ 上的 $\mathcal{G}_{\delta}$集.
      假设 $\FQ$ 也是 $\mathcal{G}_{\delta}$ 集, 则存在一列开集 $\{V_k\}_{k=1}^{\infty}$,使得
      \[\FQ=\bigcap_{k=1}^{\infty}V_k.\]
      每个 $V_k$ 包含 $\FQ$, 故必在 $\FR$ 中稠密, 由 Baire 定理可知集合
      \[\biggl(\bigcap_{k=1}^{\infty}U_k\biggr)\bigcap\biggl(\bigcap_{k=1}^{\infty}V_k\biggr)\]
      在 $\FR$ 中稠密, 然而
      \[\biggl(\bigcap_{k=1}^{\infty}U_k\biggr)\bigcap\biggl(\bigcap_{k=1}^{\infty}V_k\biggr)=(\FR\setminus\FQ)\cap\FQ=\varnothing.\]
      矛盾, 从而假设不成立, 说明 $\FQ$ 不是 $\mathcal{G}_{\delta}$集,
      由于任一映射的连续点集都是 $\mathcal{G}_{\delta}$ 集,
      所以不存在函数 $f:\FR\to\FR$使得$\textrm{Cont}(f)=\FQ$.

    \item 容易验证该函数是周期为 $1$ 的函数, 因此只需要考虑该函数在区间 $[0,1]$ 上的情况:
   
      首先, $f$在无理数点处连续: 任意取定 $x_0\in\FR\setminus\FQ$,
      有 $f(x_0)=0$. 对于 $\forall\varepsilon>0$, 在 $[0,1]$ 上至多只有有限个 $\frac{p}{q}$
      使得 $f(\frac{p}{q})=\frac{1}{q}\geq\varepsilon$, 取 $\delta>0$, 
      使得 $U(x_0,\delta)$ 不包含上述有限个有理数(这样的 $\delta$ 显然是可以取到的), 则当 $x\in U(x_0,\delta)$ 时,
      \[|f(x)-f(x_0)|=|f(x)|<\varepsilon.\]

      其次, $f$ 在有理数点处不连续: 同理可证.

      因此 $\cont{f}=\FR\setminus\FQ$.

    \item 见 Hirsch-Lacombe \cite[Page 65]{hirsch2012elements}.

      假设 $f=\mathbbm{1}_{\FQ}$ 是第一纲的, 则由定理6.1.7知 $\cont{f}$ 是
      $\FR$ 中稠密的 $\mathcal{G}_{\delta}$ 集, 但是 $f$ 在任意点处都不连续, 
      从而矛盾, 故 $f=\mathbbm{1}_{\FQ}$ 不是第一纲的. 取
      \[f_m(x)=\lim_{n\to+\infty}\cos(m!\pi x)^{2n}.\]
      由定义知 $(f_m)_{m\geq 1}$ 是一列第一纲的函数, 下面我们证明 $(f_m)_{m\geq 1}$ 逐点收敛到 $f$:
      \begin{itemize}
      \item 若$x\in\FQ$, 则可表示为 $x=\frac{p}{q}$, 当 $m\geq q$时,
            $m!\pi x=k\pi\Rightarrow f_m(x)=\lim_{n\to\infty}1=1=\mathbbm{1}_{\FQ}(x)$.
      \item 若 $x\in\FR\setminus\FQ$, 必有 $\cos(m!\pi x)\in(-1,1)\Rightarrow f_m(x)=0=\mathbbm{1}_{\FQ}(x)$.
      \end{itemize}
      综上知 $\lim_{m\to\infty}f_m=\mathbbm{1}_{\FQ}$. \qedhere
    \end{enumerate}
\end{proof}



\begin{exercise}
    证明局部紧的 Hausdorff 空间是 Baire 空间.
\end{exercise}

\begin{proof}
  设$X$是局部紧的Hausdorff空间,我们来证明$X$是Baire空间:

  已有定理:局部紧的Hausdorff空间中每一点都有一个紧邻域基.

  设$\{U_n\}_{n\geq 1}$是一列在$X$中稠密的开子集,且$D=\bigcap_{n\geq 1}U_n$,
  设$U$是任一非空开集,需证$D\cap U\neq\varnothing$.
  因为$U\cap U_1$是非空开集,所以存在非空开集$V_1$,使得$\bar{V}_1$紧且
  \[\bar{V}_1\subset U\cap U_1.\]
  因为$V_1\cap U_2$是非空开集,所以存在非空开集$V_2$,使得$\bar{V}_2$紧且\[\bar{V}_2\subset V_1\cap U_2\subset U\cap U_1\cap U_2.\]

  因为$V_{n-1}\cap U_n$是非空开集, 所以存在非空开集$V_n$, 使得$\bar{V}_n$紧且\[\bar{V}_n\subset V_{n-1}\cap U_n\subset U\cap U_1\cap\cdots\cap U_n\]
  依此可以得到一列开集$\{V_n\}$使得$\{\bar{V}_n\}$是单调递减的紧集列,故
  \[\bigcap_{n\geq 1}\bar{V}_n\neq\varnothing\]
  所以
  \[D\cap U=\left(\bigcap_{n\geq 1}U_n\right)\cap U\neq\varnothing\]这就说明了$D$在$X$中稠密,因此$X$是Baire空间.
\end{proof}



\begin{exercise}
    (a) 设 $f:\FR\to\FR$ 是可微函数. 证明: $\cont{f'}$ 是 $\FR$ 上稠密的 $\mathcal{G}_{\delta}$ 集.

    (b) 设 $f:\FR^2\to\FR$ 连续且在 $\FR^2$ 上存在偏导数 $\frac{\partial f}{\partial x}$ 和 $\frac{\partial f}{\partial y}$.
    证明: $f$ 的可微点包含 $\FR^2$ 中一个稠密的 $\mathcal{G}_{\delta}$ 集.
\end{exercise}

\begin{proof}
    (a)记
    \[g_n(x)=n\left[f\left(x+\frac{1}{n}\right)-f(x)\right].\]
    则 $(g_n)_{n\geq 1}$ 是 $\FR$ 上的连续函数序列且对于任意 $x\in\FR$, 有
    \[\lim_{n\to\infty}g_n(x)=f'(x).\]
    由定理 6.1.7 知 $\cont{f'}$ 是 $\FR$ 中稠密的 $\mathcal{G}_{\delta}$ 集.

    (b) 因为 
    \[\frac{\partial f(x,y)}{\partial x}=\lim\limits_{n\to\infty}n\left[f\left(x+\frac{1}{n},y\right)-f(x,y)\right]\overset{\Delta}{=}\lim\limits_{n\to\infty}F_n(x,y).\]
    且 $F_n(x,y)\in C(\FR^2)$, 所以 $\frac{\partial f(x,y)}{\partial x}$ 
    的连续点集是稠密的 $\mathcal{G}_{\delta}$集, 记为 $G_x$;
    同理 $\frac{\partial f(x,y)}{\partial y}$ 的连续点集也是稠密的 $\mathcal{G}_{\delta}$集,
    记为 $G_y$. 令 $G=G_x\cap G_y$, 则 $G$ 也是稠密的 $\mathcal{G}_{\delta}$ 集, 并且 $f$ 在 $G$ 上可微.
\end{proof}



\begin{exercise}
    证明: 完备度量空间中的任何一个可数子集至少含有一个孤立点.
\end{exercise}

\begin{proof}
  Suppose that $(X,d)$ is a complete metric space.
  Without loss of generality, we may assume that $X$ is countable, i.e., $X = \{x_n\}_{n\geq 1}$.

  To prove that $X$ has at least one isolated point, we proceed by contradiction.
  Suppose that $X$ has no isolated point. Let $O_n = X\setminus \{x_n\}$.
  On the one hand, $O_n$ is open since $\{x_n\}$ is closed.
  On the other hand, $O_n$ is dense in $X$ since $\{x_n\}$ is not an isolated point.
  By Baire's theorem we have
  $\bigcap_{n=1}^\infty O_n$ is dense in $X$.
  However,
  \[\bigcap_{n=1}^\infty O_n = \bigcap_{n=1}^\infty X\setminus\{x_n\} = \emptyset,\]
  which is a contradiction.
\end{proof}


\begin{exercise}
  (...)
\end{exercise}

\begin{proof}
  \begin{enumerate}[(a)]
    \item For any $f\in B$, by the definition of $B$ we know that $f$ is differentiable
      at at least one point $x\in [0,1]$ and thus
      \[\lim_{y\to x} \biggl|\frac{f(y)-f(x)}{y-x}\biggr| = |f'(x)|.\]
      Hence there exists some $\delta>0$ such that
      \[\biggl|\frac{f(y)-f(x)}{y-x}\biggr| \leq |f'(x)| + 1\]
      when $|y - x| < \delta$.
      Let $M := \max_{x\in [0,1]} |f|$ and choose $n$ large enough such that
      \[n > \max\biggl(|f'(x)|+1, \frac{2M}{\delta}\biggr).\]
      Then $|f(y)-f(x)| \leq n |y-x|$ for all $y\in [0,1]$, so $f\in F_n$.
    \item Assume that $(f_k)_{k\geq 1}$ is a sequence in $F_n$ and $f_k\to f$.
      We need to prove that $f\in F_n$. For any $f_k$, there exists some $x_k\in [0,1]$
      such that $f_k$ satisfies the $n$-Lipschitz condition at $x_k$.
      By Bolzano-Weierstrass theorem, we may assume that $x_k\to x$
      (upon extracting a subsequence) and next we show that $f$ satisfies the
      $n$-Lipschitz condition at $x$. Indeed, for any $y\in [0,1]$,
      we obtain by the triangle inequality
      \begin{align*}
        |f(x)-f(y)|
        & \leq |f(x)-f_k(x)| + |f_k(x)-f_k(x_k)| + |f_k(x_k)-f_k(y)| + |f_k(y)-f(y)| \\
        & \leq 2\|f-f_k\|_\infty + n|x-x_k| + n|y-x_k|.
      \end{align*}
      Letting $k\to\infty$, we have
      \[|f(x)-f(y)| \leq n|x-y|\quad \forall y\in [0,1].\]
    \item 
      \begin{enumerate}[(i)]
        \item This can be guaranteed by the Bernstein theorem (see Exercise~14 in Chapter~5).
        \item Since $h = P + g$, we have
          \[\|f - h\|_\infty \leq \|f - P\|_\infty + \|g\|_\infty \leq \frac{r}{2} + \frac{r}{4}
            < r.\]
          Hence $h\in B(f,r)$. Now we prove that $h\notin F_n$.
          Suppose by contradiction that $h\in F_n$, then there exists some $x_0\in [0,1]$
          such that $|h(x_0)-h(y)| \leq n|x_0-y|$ for all $y\in [0,1]$, i.e.,
          \[|P(x_0) - P(y) + g(x_0) - g(y)| \leq n|x_0-y| \quad \forall y\in [0,1].\]
          So
          \[\biggl|\frac{g(x_0)-g(y)}{x_0-y}\biggr| \leq
            \biggl|\frac{P(x_0)-P(y)}{x_0-y}\biggr| + n\]
          for all $y\in [0,1]$. Let $y\to x_0+$, we obtain that
          \[\lim_{y\to x_0+} \biggl|\frac{g(x_0)-g(y)}{x_0-y}\biggr| \leq M+n\]
          since $M = \|P'\|_\infty$. However, by the definition of $g$, we have
          \[\lim_{y\to x_0+} \biggl|\frac{g(x_0)-g(y)}{x_0-y}\biggr|
            = \frac{r/4}{1/2N} = \frac{rN}{2} > M + n + 1,\]
          a contradiction.
        \item From (i)--(ii) we know that $F_n$ has empty interior.
      \end{enumerate}
    \item Since $B = E\setminus A \subset \bigcup_{n\geq 1} F_n$, we obtain
      \[A \supset \bigcap_{n\geq 1} F_n^c,\]
      the right-hand side of which is a dense $\mathcal{G}_\delta$ set according to
      Baire's theorem. \qedhere
  \end{enumerate}
\end{proof}


\begin{exercise}
  设 $E$ 和 $F$ 都是 Banach 空间, $(u_n)$ 是 $\mathcal{B}(E,F)$ 的序列.
  证明下列命题等价:
  \begin{enumerate}[(a)]
    \item $(u_n(x))$ 在每个 $x\in E$ 处收敛.
    \item $A\subset E$ 且 $\Span(A)$ 在 $E$ 中稠密, $(u_n(a))$ 在每个 $a\in A$ 处收敛,
    且 $(u_n)$ 有界.
  \end{enumerate}
\end{exercise}

\begin{proof}
    $(a)\Rightarrow(b)$ 是显然的, 下证 $(b)\Rightarrow(a)$,
    即证 $\forall x\in E\setminus A$, $(u_n(x))$ 收敛:
    因为 $(u_n(a))$在每个$a\in A$处收敛, 所以$(u_n(a))$在每个$a\in\Span(A)$ 处收敛,
    因为 $\Span(A)$ 在 $E$ 中稠密, 所以存在 $(x_m)_{m\geq1}\subset\Span(A)$,
    使得 $x_m\to x(m\to\infty)$, 显然 $(x_m)_{m\geq1}$ 是 $E$ 中的 Cauchy 序列. 
    \[\begin{split}
    \lim_{n\to\infty}u_n(x)
    &=\lim_{n\to\infty}u_n\left(\lim_{m\to\infty}x_m\right)\\
    &=\lim_{n\to\infty}\lim_{m\to\infty}u_n(x_m)\\
    &=\lim_{m\to\infty}\lim_{n\to\infty}u_n(x_m)\\
    &=\lim_{m\to\infty}y_m\end{split}\]
    下面我们只需要证明$\lim_{m\to\infty}y_m$存在,结合$F$是Banach空间可知只需要证明$(y_m)$是$F$中Cauchy序列即可,而这个结论通过下式即得:
    \[\begin{split}
    \|y_m-y_k\|
    &=|\lim_{n\to\infty}u_n(x_m)-\lim_{n\to\infty}u_n(x_k)|\\
    &=|\lim_{n\to\infty}u_n(x_m-x_k)|\\
    &\leq\mathop{\textrm{lim inf}}\limits_{n\to\infty}\|u_n\|\cdot\|x_m-x_k\|\left(\mbox{注意}(x_m)_{m\geq1}\mbox{是}\mathrm{Cauchy}\mbox{序列}\right)
    \end{split}\]
\end{proof}


\begin{exercise}
  (...)
\end{exercise}

\begin{proof}
  \begin{enumerate}[(a)]
    \item Since $f$ is Lipschitz function, there exists some positive constant $C>0$
      such that $|f(x)-f(y)|\leq C|x-y|$ for all $x,y\in [0,1]$ and thus
      \begin{align*}
        \left|\frac{u_n(f)}{n}\right|
        & = \left|\int_0^1 f(t) \diff t-\frac{1}{n}\sum_{k=1}^nf\left(\frac{k}{n}\right)\right| \\
        & = \left|\sum_{k=1}^n\int_{\frac{k-1}{n}}^{\frac{k}{n}} f(t) \diff t
            - \frac{1}{n}\sum_{k=1}^nf\left(\frac{k}{n}\right)\right| \\
        & \leq \frac{1}{n}\sum_{k=1}^n\left|f(\xi_k)-f\left(\frac{k}{n}\right)\right|
            \qquad \text{for some } \xi_k\in\left(\frac{k-1}{n},\frac{k}{n}\right) \\
        & \leq \frac{1}{n}\sum_{k=1}^nC\left|\xi_k-\frac{k}{n}\right| \\
        & \leq \frac{1}{n}\sum_{k=1}^nC\cdot\frac{1}{n}=\frac{C}{n},
      \end{align*}
      Therefore
      \[\frac{u_n(f)}{n}=O\left(\frac{1}{n}\right)\quad\text{as } n\to\infty.\]
    \item Note here we should regard $n$ as a fixed integer. One the one hand,
      \begin{align*}
        |u_n(f)|
        & = \left|n\int_0^1f(t)\diff t-\sum_{k=1}^nf\left(\frac{k}{n}\right)\right| \\
        & \leq n\int_0^1|f(t)|\diff t+\sum_{k=1}^n\left|f\left(\frac{k}{n}\right)\right| \\
        & \leq n\|f\|_{\infty}+n\|f\|_{\infty} = 2n\|f\|_{\infty}.
      \end{align*}
      On the other hand, construct a sequence of functions $(f_m)$ as follows:
      \[f_m = \begin{cases}
        -mx+\frac{km}{n}-1, & \text{for } \frac{k}{n}-\frac{2}{m}\leq x\leq\frac{k}{n},\, 1\leq k\leq n \\
        mx-\frac{km}{n}-1,  & \text{for } \frac{k}{n}\leq x\leq\frac{k}{n}+\frac{2}{m},\, 1\leq k\leq n-1.
      \end{cases}\]
      Then $f_m\in E$, $\|f_m\|_\infty = 1$ and $f_m(x) = -1$ at $x = \frac{k}{n}$
      for $1\leq k\leq n$. By the construction of $f_m$ we obtain that
      \[|u_n(f_{m})| = \left|n\int_0^1 f_m(t) \diff t
        - \sum_{k=1}^n f_{m}\left(\frac{k}{n}\right)\right|
        = \left|2n-\frac{2n(2n-1)}{m}\right|\to 2n,\]
      as $m\to\infty$.
    \item Since $E$ is a Banach space and $\sup_{n\geq 1} \|u_n\| = \infty$,
     we have by Theorem 6.2.4 that
     \[G=\Bigl\{f\in E:\sup_{n\geq 1}|u_n(f)|=+\infty\Bigr\}\]
     is a dense $\mathcal{G}_\delta$ set in $C([0,1])$. \qedhere
  \end{enumerate}
\end{proof}


\begin{exercise}[8]
    设 $E$ 是 $(C([0,1]),\|\cdot\|_{\infty})$ 的闭向量子空间, 并假设 $E$ 中的元素都是 Lipschitz 函数.

    (a) 设 $x,y\in [0,1]$ 且 $x\neq y$, 定义泛函 $\varPhi_{x,y}:E\to\FR$ 为
    \[\varPhi_{x,y}(f)=\frac{f(y)-f(x)}{y-x}.\]
    证明: $\{\varPhi_{x,y}\mid x,y\in [0,1],x\neq y\}$ 是 $E^*$ 中的有界集.

    (b) 导出 $E$ 中闭单位球在 $[0,1]$ 上等度连续, 且 $\dim E<\infty$.
\end{exercise}

\begin{proof}
    (a) 因为完备度量空间的闭子空间完备, 所以 $E$ 是 Banach 空间,
    容易验证 $\{\varPhi_{x,y}\}\subset E^*$, 又因为对任意 $f\in E$, 有
    \[\sup_{x,y\in[0,1],x\neq y}|\varPhi_{x,y}(f)|=\sup_{x,y\in[0,1],x\neq y}\left|\frac{f(y)-f(x)}{y-x}\right|\leq C.\]
    这里的 $C$ 是函数 $f$ 的 Lipschitz 常数, 故由 Banach-Steinhaus 定理知
    \[\sup_{x,y\in[0,1],x\neq y}\|\varPhi_{x,y}\|<\infty.\]
    也即 $\{\varPhi_{x,y}\mid x,y\in[0,1],x\neq y\}$ 是 $E^*$ 中的有界集.

    (b) 记 $E$ 中的闭单位球为 $\closure{B_E}$, 则由 (a) 中结论知
    \[\sup_{x,y\in[0,1],x\neq y}\sup_{f\in \closure{B_E}}\|\varPhi_{x,y}(f)\|<\infty.\]
    即
    \[\sup_{x,y\in[0,1],x\neq y}\sup_{f\in \closure{B_E}}\left|\frac{f(y)-f(x)}{y-x}\right|<\infty.\]
    这说明 $\closure{B_E}$ 在 $[0,1]$ 上一致等度连续, 故必然等度连续.
    又对任意 $x\in[0,1]$, $\closure{B_E}$ 的轨道
    \[\closure{B_E}(x)=\{f(x):f\in \closure{B_E}\}=\{f(x):\max_{0\leq x\leq 1}|f(x)|=1\}\]
    有界, 故由 Ascoli 定理知 $\closure{B_E}$ 在 $E$ 中相对紧, 从而紧, 根据 Riesz 引理知 $\textrm{dim}E<\infty$.
\end{proof}


\begin{exercise}
  Assume $E$ is Banach space and $u,v\in\mathcal{B}(E)$. Prove that
  if $u(E)\subset v(E)$, then there exists some constant $k\geq 0$
  such that for all $x\in E$, there exists $y\in E$ such that
  $\|y\|\leq k\|x\|$ and $u(x) = v(y)$.
\end{exercise}

\begin{proof}
  The space
  \[X=\{(y,z)\in E\times E: u(y)=v(z)\}\]
  is again a Banach space as a closed (by the continuity of $u$ and $v$) 
  subspace of $E\times E$ endowed, e.g., with the norm $\|(y,z)\|=\max\{\|y\|,\|z\|\}$
  (in category theory, $X$ is called a \emph{pullback} of $u$ and $v$).
  The assumption implies that the first projection
  \[\pi:X\to E,\, (y,z)\mapsto y\]
  is surjective and hence open. By the open mapping theorem,
  there is a constant $k\geq 0$ such that,
  for every $x\in E$, there is $(y,z)\in X$ with $\|(y,z)\|\le k\|x\|$ and $\pi(y,z)=x$.
  This means $y=x$ and $z\in E$ satisfies $\|z\|\le\|(y,z)\|\le k\|x\|$
  as well as $u(x)=u(y)=v(z)$.
\end{proof}


\begin{exercise}[10]
    设 $E,F$ 都是 Banach 空间, $u\in\mathcal{B}(E,F)$ 并满足 $u(B_E)$ 在 $B_F$ 中稠密.

    (a) 计算 $\|u\|$.

    (b) 证明: $u(B_E)=B_F$. 因此 $u$ 是满射.

    (c) 设 $v\in B(E/\ker u,F)$ 并满足 $v\circ q=u$, 这里 $q:E\to E/\ker u$
    是商映射. 证明: $v$ 是从 $E/\ker u$ 到 $F$ 上的等距映射.
\end{exercise}

\begin{proof}
    (a) 因为 $u(B_E)$ 在 $B_F$ 中稠密,
    所以 $B_F\subset\overline{u(B_E)}=\overline{B_F}$,
    又由 $u$ 连续知 $u(\overline{B_E})\subset\overline{u(B_E)}$, 故
    \[\|u\|=\sup_{x\in\overline{B_E}}\|u(x)\|=\sup_{u(x)\in u(\overline{B_E})}\|u(x)\|\leq\sup_{u(x)\in \overline{u(B_E)}}\|u(x)\|=\sup_{u(x)\in\overline{B_F}}\|u(x)\|=1.\]
    对任意 $\varepsilon>0$, 存在 $y\in B_F$,
    使得 $\|y\|\geq 1-\varepsilon$, 对于上述 $y\in B_F$, 存在 $x\in B_E$,
    使得 $\|u(x)-y\|\leq\varepsilon$, 故
    \[\|u(x)\|\geq\|y\|-\|u(x)-y\|\geq 1-2\varepsilon.\]
    由 $\varepsilon$ 的任意性知 $\|u\|=1$.

    (b) 由条件知, $u(B_{E})\subset B_{F}$ 且 $B_{F}\subset\closure{u\left(B_{E}\right)}$. 
    我们采用和教材中开映射定理 6.3.1 类似的证明过程, 首先任取常数 $0<\delta<1$, 对任意 $y\in B_{F}$, 取 $x_0\in B_{E}$, 使得
    \[
    \|y-u(x_0)\|<\delta.
    \] 
    并设 $y_{1}=\frac{1}{\delta}(y-u(x_{0}))$, 则 $y_1\in B_F$. 再取 $x_1\in B_E$, 使得
    \[
    \|y_{1}-u(x_{1})\|<\delta .
    \]
    再设 $y_2=\frac{1}{\delta}(y_1-u(x_1))$, 则 $y_2\in B_F$. 依次下来, 
    可得一列 $(y_n)_{n\geq 1}\subset B_F$ 及相应序列 $(x_n)_{n\geq 1}\subset B_E$, 满足
    \[
    y_{n+1}=\frac{1}{\delta}(y_n-u(x_n)),\|y_{n}-u(x_{n})\|<\delta, \quad n\geq 1.
    \]
    由以上构造过程, 可得
    \begin{equation}
        \begin{aligned}
            y &=\delta y_{1}+u(x_{0})=\delta^{2} y_{2}+u(x_{0})+\delta u(x_{1})=\cdots \cdots \\
            &=\delta^{n+1} y_{n+1}+u(x_{0})+\delta u(x_{1})+\delta^{2} u(x_{2})+\cdots+\delta^{n} u(x_{n}) \\
            &=\delta^{n+1} y_{n+1}+u\biggl(\sum_{k=0}^{n} \delta^{k} x_{k}\biggr).
        \end{aligned}\tag{$\star$}
    \end{equation}
    在上式中, $\sum_{k=0}^n \delta^k x_k$ 在 $n\to\infty$ 时收敛于某一点 $x\in E$, 且
    \[\|x\|\leq\sum_{n=1}^{\infty}\delta^n\|x_n\|<\frac{1}{1-\delta}.\]
    在 $(\star)$ 式中取 $n\to\infty$, 得 $y=u(x)$, 故 $B_F\subset u(\frac{1}{1-\delta}B_E)$.
    由 $u$ 的线性性, 有 $(1-\delta)B_F\subset u(B_E)$.
    任取 $y\in B_F$, 总可取到 $0<\delta<1$ 使得 $1-\delta>\|y\|$, 故
    $y\in u(B_E)$, 从而 $B_F\subset u(B_E)$. 这样就证明了 $u(B_E)=B_F$.

    (c) 对任意 $x\in E$, 用 $[x]$ 表示以 $x$ 为代表元的等价类. 由定义可知, 若 $[x]=[y]$, 则
    $u(x-y)=0$. 而且, 由于 $u$ 是连续的, 则 $\ker u$ 是 $E$ 的闭向量子空间, $E/\ker u$ 自然
    成为一个赋范空间, 其上的范数 $\|\cdot\|$ 约定为
    \[\|[x]\|=\inf_{y\in\ker u}\|x+y\|=\inf_{y\in[x]}\|y\|.\]
    由于 $v\in\mathcal{B}(E/\ker u, F)$ 满足 $v\circ q=u$, 则 $v([x])=u(x)$, $\forall x\in E$. 
    因 $u$ 是满射, 故 $v$ 也是满射; 而 $[x]\neq[y]$ 等价于 $u(x)\neq u(y)$, 故 $v$ 也是单射. 实际上, 由开
    映射定理立即得到, $v$ 是 $E/\ker u$ 到 $F$ 的线性同构映射.

    任取 $[x]\in E/\ker u$, 设 $y=v([x])$, 则也有 $y=u(x)$. 那么由 $u(B_{E})=B_{F}$, 可
    知对任意 $0<\varepsilon<1$, 存在 $x_{0}\in B_{E}$, 使得 
    $u(x_0)=\varepsilon \frac{y}{\|y\|}$, 则又有 $y=u(\varepsilon^{-1}\|y\| x_{0})$.
    于是得 $u(x-\varepsilon^{-1}\|y\| x_{0})=0$, 这表明 $\varepsilon^{-1}\|y\| x_{0}\in [x]$. 因此
    \[
    \|[x]\| \leq\bigl\|\varepsilon^{-1}\|y\|x_{0}\bigr\|\leq\varepsilon^{-1}\|y\|=\varepsilon^{-1}\|v([x])\| .
    \]
    由 $\varepsilon$ 的任意性, 即得 $\|[x]\|\leq\|v([x])\|$.

    另一方面, 对任意 $[x]\in E/\ker u$, 任取 $[x]$ 的代表元 $y$, 都有
    \[
    \|v([x])\|=\|u(y)\| \leq\|u\|\cdot\|y\|=\|y\|.
    \]
    对上式右边所有代表元的范数取下确界, 即得
    \[\|v([x])\|\leq\inf_{y\in [x]}\|y\|=\|[x]\|.\]
    综合以上讨论, 我们证明了 $v$ 是从 $E/\ker u$ 到 $F$ 上的等距同构映射.

    \textbf{另一种更直接的证明}:
    任取 $x\in B_E$, 则有 $\|[x]\|\leq\|x\|<1$, 即
    \[q(B_E)\subset B_{E/\ker u}.\]
    反过来, 任取 $[x]\in B_{E/\ker u}$, 则必定存在代表元 $y\in[x]$, 
    使得 $\|[x]\|\leq\|y\|<1$. 于是得 $y\in B_{E}$, 满足 $[x]=q(y)\in q(B_{E})$, 也就有
    \[
    B_{E/\ker u}\subset q(B_{E}).
    \]
    因此, 我们得到 $B_{E/\ker u}=q(B_{E})$. 再由 $u(B_{E})=B_{F}$, 以及 $u=v \circ q$, 立即得到
    \[
    v(B_{E/\ker u})=B_{F}.
    \]
    而且, 因 $v\in E/\ker u\to F$ 是同构映射, 故也有
    \[
    v^{-1}(B_{F})=B_{E/\ker u}.
    \]
    由以上两式立即得到 $\|v\|=\|v^{-1}\|=1$. 故 $v$ 是从 $E/\ker u$ 到 $F$ 上的等距同构映射.
\end{proof}


\begin{exercise}
  (...)
\end{exercise}

\begin{proof}
\begin{enumerate}[(a)]
  \item By definition
    \[\|q(g)-f\|_{\infty}=\sup_{y\in Y}|q(g)(y)-f(y)|
      = \sup_{y\in Y}\left|\frac{d(y,B)-d(y,A)}{3[d(y,B)+d(y,A)]}-f(y)\right|.\]
    We proceed in three cases as follows:
    \begin{enumerate}[(i)]
      \item For $y\in A$,
        \[\|q(g)-f\|_{\infty}=\sup_{y\in A}\left|\frac{d(y,B)}{3d(y,B)}-f(y)\right|=\sup_{y\in A}\left|\frac{1}{3}-f(y)\right|\leq\frac{2}{3}.\]
      \item For $y\in B$,
        \[\|q(g)-f\|_{\infty}=\sup_{y\in B}\left|\frac{-d(y,A)}{3d(y,A)}-f(y)\right|=\sup_{y\in B}\left|-\frac{1}{3}-f(y)\right|\leq\frac{2}{3}.\]
      \item For $y\in Y\setminus (A\cup B)$, since
        \[-\frac{1}{3}<f(y)<\frac{1}{3}
          \quad\text{and}\quad -\frac{1}{3}<\frac{d(y,B)-d(y,A)}{3[d(y,B)+d(y,A)]}<\frac{1}{3}.\]
        It follows that
        \[\|q(g)-f\|_{\infty} < 2/3.\]
    \end{enumerate}
    Note that the above discussions involve the special cases when $A$ or $B$
    is emptyset.
  \item For any $f\in F$, let
    \[C = \biggl\{y\in Y:f(y)\geq\frac{1}{3}\|f\|_{\infty}\biggr\},\quad 
      D = \left\{y\in Y:f(y)\leq-\frac{1}{3}\|f\|_{\infty}\right\}.\]
    Define
    \[g(x) := \frac{d(x,D)-d(x,C)}{3[d(x,D)+d(x,C)]}\cdot\|f\|_{\infty}.\]
    Then $\|g\|_{\infty}\leq\frac{1}{3}\|f\|_{\infty}$ and from (a) we have that
    $\|q(g)-f\|_{\infty}\leq\frac{2}{3}\|f\|_{\infty}$.
  \item For the $f$ and $g$ that satisfy the conditions in (b), we have
    \[\frac{2}{3}\|f\|_{\infty}\leq\|f\|_{\infty}-\|g\|_{\infty}
      \leq\|f\|_{\infty}-\|q(g)\|_{\infty}\leq\|f-q(g)\|_{\infty}\leq\frac{2}{3}\|f\|_{\infty}.\]
    Thus
    \[\|g\|_{\infty} = \frac{1}{3}\|f\|_{\infty}
      \text{ and }
      \|f-q(g)\|_{\infty } =\frac{2}{3}\|f\|_{\infty}.\]
    We use the induction method to generate a sequence of functions $(g_n)$ as follows:

    There exists some $g_1\in E$ such that
    \begin{align*}
      \|g_1\|_{\infty} & = \frac{1}{3}\|f\|_{\infty}, \\
      \|f-q(g_1)\|_{\infty} & = \frac{2}{3}\|f\|_{\infty}.
    \end{align*}

    There exists some $g_2\in E$ such that
    \begin{align*}
      \|g_2\|_{\infty}=\frac{1}{3}\|f-q(g_1)\|_{\infty}
        & = \frac{1}{3}\cdot\frac{2}{3}\|f\|_{\infty}, \\
      \|f-q(g_1)-q(g_2)\|_{\infty}=\frac{2}{3}\|f-q(g_1)\|_{\infty}
        & = \left(\frac{2}{3}\right)^2\|f\|_{\infty}.
    \end{align*}

    At the $n$-th step, there exists some $g_n\in E$ such that
    \begin{align*}
      \|g_n\|_{\infty}
        & = \frac{1}{3}\left\|f-\sum_{k=1}^{n-1}q(g_k)\right\|_{\infty}=\frac{1}{3}\left(\frac{2}{3}\right)^{n-1}\|f\|_{\infty}, \\
      \left\|f-\sum_{k=1}^nq(g_k)\right\|_{\infty}
        & = \frac{2}{3}\left\|f-\sum_{k=1}^{n-1}q(g_k)\right\|_{\infty}=\left(\frac{2}{3}\right)^n\|f\|_{\infty}.
    \end{align*}
    So we obtain a sequence $(g_n)_{n\geq 1}\subset E$ which satisfies two properties:
    \[\text{(i) } \|g_n\|_{\infty}=\frac{1}{3}\left(\frac{2}{3}\right)^{n-1}\|f\|_{\infty}
      \quad\text{and}\quad
      \text{(ii) } \left\|f-\sum_{k=1}^nq(g_k)\right\|_{\infty}=\left(\frac{2}{3}\right)^n\|f\|_{\infty}.\]
    From property (i) we know that $\sum_{n=1}^{\infty}\|g_n\|_{\infty}$ converges,
    so $\sum_{n=1}^{\infty}g_n$ converges to some $g\in E$ since $E$ is complete.

    From property (ii) we know that $\sum_{n=1}^{\infty}q(g_n)$ converges uniformly to $f$.

    Since $q$ is continuous, we have
    \[\sum_{n=1}^{\infty}q(g_n)=q\left(\sum_{n=1}^{\infty}g_n\right)\to q(g),\]
    and hence $f = q(g)$. Finally we prove that $\|f\|_\infty = \|g\|_\infty$.
    To this end, note that
    \[\|g\|_{\infty}=\left\|\sum_{n=1}^{\infty}g_n\right\|
      \leq \sum_{n=1}^{\infty}\|g_n\|_{\infty}
      = \sum_{n=1}^{\infty}\frac{1}{3}\left(\frac{2}{3}\right)^{n-1}\|f\|_{\infty}
      = \|f\|_{\infty},\]
    and $\|f\|_{\infty}=\|q(g)\|_{\infty}\leq\|g\|_{\infty}$.
  \item Obvious from (c).
  \item Choose arbitrarily $f\in C(Y,\FR)$,
    \begin{itemize}
      \item If $f$ is bounded, by (c) there exists some $g\in C(X,\FR)$ such that $q(g)=f$.
      \item If $f$ is unbounded, let $f_1 = \arctan f\in F$.
        There exists some $g_1\in E$ such that $q(g_1) = f_1$.
        Let $g = \tan g_1$, then
        \[q(g)=q(\tan g_1)=\tan(q(g_1))=\tan(f_1) = f.\qedhere\]
    \end{itemize}
\end{enumerate}
\end{proof}

\begin{exercise}[13]
  (...)
\end{exercise}

\begin{proof}
  \begin{enumerate}[(a)]
    \item 假设$F$在$E$中的内部不是空集,则存在$x\in F,r>0$使得$B(x,r)\subset F$,这里的$B(x,r)$是$E$中的开球,由$F$是向量子空间可得$B(0,r)\subset F\Rightarrow B(0,n)\subset F$ $(\forall n)$, 从而
    \[E=\bigcup_{n\geq 1}B(0,n)\subset F\Rightarrow E=F\]
    矛盾,故假设不成立,即证$F$在$E$中的内部为空集.
    \item 记所有多项式构成的空间为$\mathcal{P}$,所有次数不超过$n$的多项式构成的空间为$\mathcal{P}_n$,则
    \[\mathcal{P}=\bigcup_{n\geq 1}\mathcal{P}_n.\]
    假设$\mathcal{P}$上有完备范数,由(a)知$\mathcal{P}_n^{\circ}=\emptyset$,由Baire定理知$\mathcal{P}^{\circ}=\emptyset$,矛盾,故假设不成立,所以$\mathcal{P}$不能赋予完备范数. \qedhere
  \end{enumerate}
\end{proof}


\begin{exercise}[14]
    设 $E$ 是 Banach 空间, $F$ 和 $G$ 都是 $E$ 的闭向量子空间, 并且 $F+G$
    也是闭向量子空间. 证明: 存在一个常数 $C\geq 0$, 使得 $\forall x\in F+G$,
    存在 $(f,g)\in F\times G$, 满足
    \[x=f+g,\;\|f\|\leq C\|x\|,\;\|g\|\leq C\|x\|.\]
\end{exercise}

\begin{proof}
    考虑乘积 Banach 空间 $F\times G$ (赋予范数 $\|(f,g)\|=\|f\|+\|g\|$)
    和 Banach 空间 $F+G$ (范数即为 $E$ 中范数). 映射
    \[u:F\times G\to F+G,\;(f,g)\mapsto f+g\]
    为连续线性的满射, 由开映射定理, $u(B_{F\times G}(0,1))$ 为 $F+G$ 中含原点的开集,
    取常数 $c>0$, 使得 $B_{F+G}(0,c)\subset u(B_{F\times G}(0,1))$.
    则对于任意 $x\in F+G$ 且 $\|x\|<c$, 存在 $f\in F$, $g\in G$
    且 $\|f\|+\|g\|<1$, 使得 $x=f+g$.

    对于一般的 $x\in F+G$, 任取 $0<c'<c$, 由于 $x=\frac{\|x\|}{c'}\bigl(\frac{c'}{\|x\|}x\bigr)$,
    其中 $\left\|\frac{c'}{\|x\|}x\right\|=c'<c$, 故存在 $f'\in F$, $g'\in G$,
    使得 $\frac{c'}{\|x\|}x=f'+g'$ 且 $\|f'\|+\|g'\|<1$.
    令 $f=\frac{\|x\|}{c'}f'$, $g=\frac{\|x\|}{c'}g'$, 则
    $x=f+g$ 且
    \[\|f\|+\|g\|=\frac{\|x\|}{c'}\bigl(\|f'\|+\|g'\|\bigr)<\frac{1}{c'}\|x\|.\]
    由 $c'$ 的任意性即得 $\|f\|+\|g\|\leq\frac{1}{c}\|x\|$.
    再令 $C=\frac{1}{c}$ 即证所需.
\end{proof}



\begin{exercise}
    设 $H$ 是 Hilbert 空间, 且线性映射 $u:H\to H$ 满足
    \[\innerp{u(x)}{y}=\innerp{x}{u(y)},\quad\forall x,y\in H.\]
    证明: $u$ 连续.
\end{exercise}

\begin{proof}
    考虑线性泛函
    \[f_x\colon H\to\mathbb{K},\quad y\mapsto\langle u(y),u(x)\rangle.\]
    记 $H$ 中的闭单位球为 $\closure{B}_H$, 对于任意 $y\in H$,由 Cauchy-Schwarz 不等式有
    \[\sup_{x\in\closure{B}_H}|f_x(y)|
      = \sup_{x\in\closure{B}_H}|\innerp{u(y)}{u(x)}|
      = \sup_{x\in \closure{B}_H}|\innerp{u(u(y))}{x}|\leq\|u(u(y))\|<\infty.\]
    故由 Banach-Steinhaus 定理知
    \[\sup_{x\in \closure{B}_H}\|f_x\|<\infty.\]
    即
    \[\sup_{x\in \closure{B}_H}\sup_{y\in \closure{B_H}}|\langle u(y),u(x)\rangle|<\infty.\]
    因此
    \[\|u\|^2=\sup_{x\in\closure{B}_H}\|u(x)\|^2
      = \sup_{x\in \closure{B}_H}\langle u(x),u(x)\rangle < \infty.\]
    从而 $u$ 为有界算子, 亦即为连续算子.
\end{proof}


\begin{exercise}
  Let $X = C^1([0,1],\FR)$ and $Y = C([0,1],\FR)$ both endowed with the uniform
  norm $\|\cdot\|_\infty$. Define $u:X\to Y$, $u(f) = f'$. Prove that the graph $G(u)$
  of $u$ is closed but $u$ is not continuous.
\end{exercise}

\begin{proof}
  \begin{itemize}
    \item The graph $G(u)$ of $u$ is closed. Just prove by definition,
      take a sequence $(f_n,f_n')\in X\times Y$ and let $(f_n,f_n')\to (f,g)$.
      Show that $(f,g)\in X\times Y$ and $f' = g$.
    \item $u$ is not continuous. Take $f_n = x^n$, then $\|f_n\|_\infty = 1$
      and $\|f_n'\|_\infty = n$.
  \end{itemize}
  The $C^1[0,1]$ functions are a dense subset of $C[0,1]$ in the sup norm.
  However the inclusion is proper so the $C^1[0,1]$ functions are not a complete
  subspace of $C[0,1]$.
\end{proof}


\begin{exercise}[18]
  Let $E$ be a separable Banach and $E\neq \{0\}$.
  Let $(x_n)_{n\geq 1}$ be a dense sequence in $B_E$.
  \begin{enumerate}[(a)]
    \item Prove that there exists $u\in \mathcal{B}(\ell_1, E)$ such that
      \[u((a_n)_{n\geq 1}) = \sum_{n\geq 1} a_n x_n.\]
    \item Prove that the image under $u$ of the open unit ball of $\ell_1$ is $B_E$.
      What is the image under $u$ of the closed unit ball of $\ell_1$?
    \item Let $1<p<\infty$.
      \begin{enumerate}[(i)]
        \item Prove there exists a continuous linear surjection $u: \ell_1\to \ell_p$.
        \item Prove there does not exist a continuous linear map $v:\ell_p\to \ell_1$
          such that $u\circ v$ is the identity map on $\ell_p$.
        \item Prove there does not exist any closed subspace $F$ in $\ell_1$ such that
          $F\oplus \ker u = \ell_1$.
      \end{enumerate}
  \end{enumerate}
\end{exercise}

\begin{proof}
  \begin{enumerate}[(a)]
    \item This question seems a little strange, we just need to verify that
      $u$ is linear and bounded, which is straightforward.
    \item 
  \end{enumerate}
\end{proof}
% 18.\textit{Proof}:(a)显然$u$是线性的,且当$\|(a_n)_{n\geq 1}\|\leq 1$时,有$\|u((a_n)_{n\geq 1})\|=\|\sum_{n\geq 1}a_nx_n\|\leq\sum_{n\geq 1}|a_n|\cdot\|x_n\|\leq\sum_{n\geq 1}|a_n|\leq 1$
% 因此$\|u\|\leq 1$,从而$u\in\mathcal{B}(\ell_1,E)$\\
% (b)由(a)知$u(B_{\ell_1})\subset \closure{B_E}$\\
% 任取$y\in \closure{B_E}$,则$d(y,\frac{1}{2}\closure{B_E})<\frac{1}{2}$,故存在$n_1\geq 2$使得$\|y-\frac{1}{2}x_{n_1}\|<\frac{1}{2}$\\
% 令$y_1=y-\frac{1}{2}x_{n_1}$,则存在$n_2>n_1$使得$\|2y_1-\frac{1}{2}x_{n_2}\|<\frac{1}{2}$\\
% 令$y_2=y_1-\frac{1}{4}x_{n_2}$,则存在$n_3>n_2$使得$\|4y_2-\frac{1}{2}x_{n_3}\|<\frac{1}{2}$\\
% $\cdots\cdots$\\
% 这样就得到一列$(x_{n_k})_{k\geq 1}$及$(y_k)_{k\geq 1}$使得
% \[y=\sum_{k=1}^n\frac{1}{2^k}x_{n_k}+y_k\]
% 因为$\|2^ky_k-\frac{1}{2}x_{n_{k+1}}\|<\frac{1}{2}\Rightarrow\|y_k-\frac{1}{2^{k+1}}x_{n_{k+1}}\|<\frac{1}{2^{k+1}}$,所以$y_k\to0(k\to\infty)$,故\[y=\sum_{k=1}^{\infty}\frac{1}{2^k}x_{n_k}\]
% 记$a_{n_k}=\frac{1}{2^k}$,对于其他$j\in\mathbb{N}^{*}\backslash\{n_k\}_{k=1}^{\infty}$,令$a_j=0$
% ,则$(a_n)_{n\geq 1}\in\ell_1$,并且$y=\sum_{n\geq 1}a_nx_n$,从而$u(B_{\ell_1})=\closure{B_E}$\\\\
% 下面证明$u(\overline{B_{\ell_1}})=\closure{B_E}$:\\
% 因为$u(B_{\ell_1})=\closure{B_E}$,所以$u(\overline{B_{\ell_1}})\supset \closure{B_E}$,故只需要说明$u(\overline{B_{\ell_1}})\subset \closure{B_E}$,任取$(a_n)_{n\geq 1}\in\overline{B_{\ell_1}}$,有:$\|u((a_n)_{n\geq 1})\|=\|\sum_{n\geq 1}a_nx_n\|\leq\sum_{n\geq 1}|a_n|\cdot\|x_n\|<\sum_{n\geq 1}|a_n|\leq 1$,故$u(\overline{B_{\ell_1}})\subset \closure{B_E}$\\
% (c)\begin{enumerate}[(i)]
% \item 由(a)(b)中结论及$\ell_p$可分知存在$u\in\mathcal{B}(\ell_1,\ell_p),s.t.u(B_{\ell_1})=B_{\ell_p}$,显然$u$是满射
% \item 假设存在连续线性映射$v:\ell_p\to\ell_1,s.t.u\circ v=id$,因为$v$是单射,所以$\ell_p$与$v(\ell_p)$同构,又因为$\ell_1$与$v(\ell_p)$同构,故$\ell_1$于$\ell_p$同构,矛盾
% \end{enumerate}