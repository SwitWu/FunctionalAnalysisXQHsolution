% !TeX root = main.tex
% !TeX program = xelatex
\setcounter{chapter}{3}
\chapter{Hilbert空间}



\begin{exercise}
    设 $u:H\to K$ 是两个实内积空间之间的映射, 且有
    \[\|u(x)-u(y)\|=\|x-y\|,\quad\forall x,y\in H\;(\text{也就是说}, u\text{\ 是一个等距映射}).\]
    证明 $u-u(0)$ 是线性的.
\end{exercise}

\begin{proof}
记 $v=u-u(0)$, 则 $v(0)=0$, $\|v(x)-v(y)\|=\|x-y\|$ $(\forall x,y\in H)$,
即 $v$ 保距离(特别地, $v$ 还保范数), 将上式平方得 
\[\|v(x)\|^2+\|v(y)\|^2-2\langle v(x),v(y)\rangle=\|x\|^2+\|y\|^2-2\langle x,y\rangle.\]
故
\[\langle v(x),v(y)\rangle=\langle x,y\rangle.\]
因此 $v$ 保内积,下面证明 $v$ 是线性的:
\begin{itemize}
\item $v(x+y)=v(x)+v(y)$ $(\forall x,y\in H)$:
\begin{align*}
    & \langle v(x+y)-v(x)-v(y),v(x+y)-v(x)-v(y)\rangle\\
    & = \|v(x+y)\|^2+\|v(x)\|^2+\|v(y)\|^2-2\langle v(x+y),v(x)\rangle \\
    & \quad -2\langle v(x+y),v(y)\rangle+2\langle v(x),v(y)\rangle\\
    & = \|x+y\|^2+\|x\|^2+\|y\|^2-2\langle x+y,x\rangle
      -2\langle x+y,y\rangle+2\langle x,y\rangle\\
    & = \|x+y\|^2+\|x\|^2+\|y\|^2-2\|x+y\|^2+2\langle x,y\rangle\\
    & = 0 \\
    & \Rightarrow v(x+y)=v(x)+v(y).
\end{align*}
\item $v(\lambda x)=\lambda v(x)$ $(\forall\lambda\in\mathbb{R},x\in H)$:
\begin{align*}
    &\langle v(\lambda x)-\lambda v(x),v(\lambda x)-\lambda v(x)\rangle\\
    &=\|v(\lambda x)\|^2+\lambda^2\|v(x)\|^2-2\lambda\langle v(\lambda x),v(x)\rangle\\
    &=\|\lambda x\|^2+\lambda^2\|x\|^2-2\lambda\langle\lambda x,x\rangle\\
    &=0\\&\Rightarrow v(\lambda x)=\lambda v(x).
\end{align*}    
\end{itemize}
根据上面两点知 $v$ 是线性的.
\end{proof}

\begin{remark}
保距离+保原点 $\rightarrow$ 保内积.
\end{remark}




\begin{exercise}
    设 $A$ 是 $\ell_2$ 的子集, 其元素 $x=(x_n)_{n\geq 1}$ 满足 $|x_n|\leq\frac{1}{n}$, $n\geq 1$. 证明 $A$ 是紧集.
\end{exercise}

\begin{proof}
我们证明 $A$ 序列紧, 即证$A$中任意序列有收敛子列, 任取$A$中的序列$(x^{(m)})_{m\geq 1}$, 记
\[x^{(m)}=\left(x^{(m)}_1,x^{(m)}_2,\cdots,x^{(m)}_n,\cdots\right)\quad m=1,2,\cdots.\]

$\left(x_1^{(m)}\right)_{m\geq 1}$为有界序列,有收敛子列$\left(x_1^{(m_k^1)}\right)_{k\geq 1}$;

$\left(x_2^{(m_k^1)}\right)_{k\geq 1}$为有界序列,有收敛子列$\left(x_2^{(m_k^2)}\right)_{k\geq 1}$;

$\cdots\cdots$

$\left(x_n^{(m_k^{n-1})}\right)_{k\geq 1}$为有界序列,有收敛子列$\left(x_n^{(m_k^n)}\right)_{k\geq 1}$;

$\cdots\cdots$

根据对角线法选取指标列 $\left(m_k^k\right)_{k\geq 1}$, 
由此得到 $(x^{(m)})_{m\geq 1}$ 的子列 $(x^{(m_k^k)})_{k\geq 1}$, 
不妨将其简记为 $(x^{(m_k)})_{k\geq 1}$, 其每一个坐标分量都是收敛的, 
记 $(x^{(m_k)})_{k\geq 1}$ 依坐标收敛于 $x=(x_n)$, 且有 $|x_n|\leq\frac{1}{n}$, 
故 $x=(x_n)\in\ell_2$, 
下面证明 $(x^{(m_k)})_{k\geq 1}$ 依 $\ell_2$ 范数收敛到$x$: 
事实上, 对于每个$k\geq 1$, 都有 $|x_n^{(m_k)}|\leq\frac{1}{n}$, 故
\[|x_n^{(m_k)}-x_n|\leq\frac{2}{n}.\]
那么, 对任意 $\varepsilon>0$, 存在 $N\geq 1$, 使得
\[\sum_{n=N+1}^{\infty}|x_n^{(m_k)}-x_n|^2\leq\sum_{n=N+1}^{\infty}\frac{4}{n^2}<\varepsilon.\]
另一方面, 因为$(x^{(m_k)})_{k\geq 1}$ 依坐标收敛于$x=(x_n)$, 故存在$k_0\geq 1$, 使得当 $k\geq k_0$时, 有
\[\sum_{n=1}^N|x_n^{(m_k)}-x_n|^2\leq\sum_{n=1}^N\frac{\varepsilon}{N}=\varepsilon.\]
因此
\[\|x^{(m_k)}-x\|_{\ell_2}=\left(\sum_{n=1}^{\infty}|x_n^{(m_k)}-x_n|^2\right)^{\frac{1}{2}}<(2\varepsilon)^{\frac{1}{2}}.\]
即说明$(x^{(m_k)})_{k\geq 1}$依$\ell_2$范数收敛到$x$, 故$A$是紧集.
\end{proof}

\begin{remark}
从证明过程可以看出题目条件中的控制项 $\frac{1}{n}$ 可以换成任意 $(a_n)$, 只要其满足 $\sum_{n=1}^{\infty}a_n^2$ 收敛即可.
\end{remark}




\begin{exercise}
设 $E$ 和 $F$ 是内积空间 $H$ 的两个向量子空间. 证明存在常数 $\alpha\geq 0$ 使得
\[|\innerp{x}{y}|=\alpha\|x\|\|y\|,\quad\forall x\in E,\forall y\in F\]
的充分必要条件是或者 $\dim E=\dim F=1$, 或者 $\alpha=0$ (即 $E$ 与 $F$ 正交).
\end{exercise}

\begin{proof}
($\Leftarrow$)
充分性显然.

\necessary
已知存在 $\alpha\geq 0$, 使得 
$|\innerp{x}{y}|=\alpha \|x\|\cdot\|y\|$ $(\forall x\in E,y\in F)$.
当 $\alpha=0$时, 显然 $E$ 与 $F$ 正交;
当 $\alpha>0$时, 需证  $\dim E=\dim F=1$, 
反证法, 当 $\FK=\mathbb{R}$ 时, 假设 $\dim F\geq 2$, 取 $F$ 中
两个不共线的单位向量 $e_1,e_2$, 取 $E$ 中一个单位向量 $e$,
则 $\innerp{e}{e_1}=\innerp{e}{e_2}=\alpha$, 且
\[\innerp{e}{e_1+e_2}=2\alpha=\alpha\|e_1+e_2\|,\]
因此 $\|e_1+e_2\|=\|e_1\|+\|e_2\|$, 于是 $e_1$ 与 $e_2$ 共线, 矛盾.
\end{proof}




\begin{exercise}
    设 $E$ 和 $F$ 是内积空间 $H$ 的两个向量子空间.
    假设 $E$ 和 $F$ 都不等于集合 $\{0\}$.
    定义 $E$ 和 $F$ 之间的夹角 $\theta$ 为
    \[\cos\theta=\sup\left\{\frac{|\innerp{x}{y}|}{\|x\|\|y\|}:x\in E, y\in F\right\},\quad\theta\in\left[0,\frac{\pi}{2}\right].\]
    证明: $\theta>0$ 当且仅当存在一个常数 $c>0$, 使得
    \[\|x+y\|^2\geq c(\|x\|^2+\|y\|^2),\quad\forall x\in E, y\in F.\]
\end{exercise}

\begin{proof}
    \necessary
    记 $m=\cos\theta$, 则 $\theta>0\iff 0\leq m<1$. 由夹角的定义知
    \[|\innerp{x}{y}|\leq m\|x\|\|y\|\leq\frac{m}{2}(\|x\|^2+\|y\|^2),\]
    故
    \begin{align*}
        \|x+y\|^2
        & = \|x\|^2+\|y\|^2+2\Re\innerp{x}{y}\geq \|x\|^2+\|y\|^2-2|\innerp{x}{y}| \\
        & \geq \|x\|^2+\|y\|^2-m\left(\|x\|^2+\|y\|^2\right) \\
        & = (1-m)\left(\|x\|^2+\|y\|^2\right).
    \end{align*}
    在上式中取 $c=1-m>0$ 即得所证.

    \sufficient
    注意到
    \begin{align*}
        \cos\theta
        & =\sup\biggl\{\frac{|\innerp{x}{y}|}{\|x\|\|y\|}\colon x\in E,y\in F\biggr\} \\
        & =\sup\{|\innerp{x}{y}|\colon \|x\|=\|y\|=1,x\in E,y\in F\}.
    \end{align*}
    故只需证 $\sup\{|\innerp{x}{y}|\colon \|x\|=\|y\|=1,x\in E,y\in F\}<1$.

    对任意 $x\in E$, $y\in F$, 且 $\|x\|=\|y\|=1$, 由条件 $\|x+y\|^2\geq c(\|x\|^2+\|y\|^2)$
    (不妨设 $0<c<1$) 得 $\Re\innerp{x}{y}\geq c-1$.
    当 $\|x\|=1$ 时, 亦有 $\|-x\|=1$, 故
    \[\Re\innerp{x}{y}=-\Re\innerp{-x}{y}\leq 1-c.\]
    因此
    \begin{align*}
        |\innerp{x}{y}|
        & =\sgn\innerp{x}{y}\cdot\innerp{x}{y} \\
        & =\innerp{\sgn\innerp{x}{y}\cdot x}{y}\quad(\text{real number})\\
        & =\Re\innerp{\sgn\innerp{x}{y}\cdot x}{y} \\
        & \leq 1-c.
    \end{align*}
    从而 $\sup\{|\innerp{x}{y}|\colon \|x\|=\|y\|=1,x\in E,y\in F\}\leq 1-c<1$, 由此即得 $\theta>0$.
\end{proof}



\begin{exercise}
    设 $H$ 是 Hilbert 空间, $(A_n)$ 是 $H$ 中递减的闭凸非空子集列. 
    任取 $x\in H$, 令 $d_n(x)=d(x,A_n)$ 且 $d(x)=\lim_{n\to\infty}d_n(x)$.
    \begin{enumerate}[(a)]
    \item 证明: 若对某一个 $x\in H$, 有 $d(x)<\infty$, 
    则对所有的 $x\in H$, $d(x)<\infty$. 我们在下面假设该命题成立, 
    并用 $A(x,\varepsilon,n)$ 表示中心在 $x$、半径为 $d(x)+\varepsilon$ 的闭球与 $A_n$ 的交集, 
    即 $A(x,\varepsilon,n)=A_n\cap\overline{B}(x,d(x)+\varepsilon)$.
    \item 证明
    \[\lim_{\varepsilon\to 0,n\to\infty}\diam(A(x,\varepsilon,n))=0.\]
    \item 证明所有 $A_n$ 的交集 $A$ 非空并且 $d(x)=d(x,A)$.
    \end{enumerate}
\end{exercise}

\begin{proof}
  (a) 假设存在 $x_0\in H$, 使得 $d(x_0)<\infty$, 记 $y_n=P_{A_n}(x_0)$, 则
  \[d_n(x_0) = d(x_0,A_n) = d(x_0,y_n) < \infty.\]
  故 $\forall x\in H$
  \[d_n(x) = d(x,A_n) \leq d(x,y_n)\leq d(x,x_0)+d(x_0,y_n) < \infty.\]
  因此 $d(x)<\infty$ $(\forall x\in H)$.

  (b)仍记 $y_n=P_{A_n}(x)$, 则 $d_n(x)=d(x,y_n)\leq d(x)$,
  因此 $y_n\in A_n\cap\overline{B}(x,d(x)+\varepsilon)$.
  由于 $d_n(x)=d(x,y_n)\leq d(x)<\infty$, 故对于 $\forall\varepsilon>0$,
  存在 $N(\varepsilon)$, 使得 $\forall n\geq N(\varepsilon)$, $d(x)<d_n(x)+\varepsilon$.
  因此对 $\forall z,w\in A_n\cap\overline{B}(x,d(x)+\varepsilon),\forall n\geq N(\varepsilon)$, 有
  \[d_n(x)\leq d(x,z)\leq d(x)+\varepsilon\leq d_n(x)+2\varepsilon,\]
  \[d_n(x)\leq d(x,w)\leq d(x)+\varepsilon\leq d_n(x)+2\varepsilon.\]
  由 $A_n$ 及 $\overline{B}(x,d(x)+\varepsilon)$ 均为凸集知 $\frac{z+w}{2}\in A_n\cap\overline{B}(x,d(x)+\varepsilon)$, 从而
  \[d_n(x)\leq d\left(x,\frac{z+w}{2}\right).\]
  结合以上三式并根据平行四边形公式得
  \[\begin{split}
    d_n(x)^2+\frac{1}{4}\|z-w\|^2
    & \leq \left\|x-\frac{z+w}{2}\right\|^2+\left\|\frac{z-w}{2}\right\|^2 \\
    & = 2\left(\left\|\frac{x-z}{2}\right\|^2+\left\|\frac{x-w}{2}\right\|^2\right) \\
    & \leq (d_n(x)+2\varepsilon)^2.
  \end{split}\]
  即
  \[\|z-w\|^2\leq 16\varepsilon(d_n(x)+\varepsilon).\]
  对所有的 $z,w\in A(x,\varepsilon,n)$ 取上确界得
  \[\diam A(x,\varepsilon,n)\leq 16\varepsilon(d_n(x)+\varepsilon)
    \quad (n\geq N(\varepsilon))\]
  从而
  \[\lim\limits_{\varepsilon\to 0,n\to\infty}\diam A(x,\varepsilon,n) = 0.\]

  (c)取 $H$ 中集列 $\left(A(x,\frac{1}{n},n)\right)_{n\geq 1}$, 则
  \begin{itemize}
  \item $\left(A(x,\frac{1}{n},n)\right)_{n\geq 1}$是单调递减的闭集列
  \item $\left(A(x,\frac{1}{n},n)\right)_{n\geq 1}$非空
  \item $\lim_{n\to\infty}\diam\left(A(x,\frac{1}{n},n)\right)_{n\geq 1}=0$(在(b)中取$\varepsilon=1/n$知此式成立)
  \end{itemize}
  因为 $H$ 是完备的, 所以
  \[\bigcap_{n\geq 1}A\left(x,\frac{1}{n},n\right)\text{是单点集}.\]
  又 $\left(A(x,\frac{1}{n},n)\right)\subset A_n$, 故$A=\bigcap_{n\geq 1}A_n$非空.

  下证 $d(x)=d(x,A)$:

  因为 $d_n(x)=d(x,A_n)\leq d(x,A)$, 所以$d(x)\leq d(x,A)$, 假设$d(x)<d(x,A)$,
  则存在 $\delta$, 使得 $d(x,A_n)\leq d(x)<\delta<d(x,A)$, 同理可知:
  \[\bigcap_{n\geq 1}A\left(x,\frac{\delta-d(x)}{n},n\right)
    = \bigcap_{n\geq 1}\left(A_n\cap\overline{B}\left(x,d(x)+\frac{\delta-d(x)}{n}\right)\right)\text{ 是单点集}.\]
  故
  \[\bigcap_{n\geq 1}\left(A_n\cap\overline{B}\left(x,\delta\right)\right)=A\cap\overline{B}(x,\delta)\text{ 非空},\]
  但事实是
  \[A\cap\overline{B}(x,\delta) = \emptyset.\]
  故假设不成立, 即得 $d(x) = d(x,A)$.
\end{proof}


\begin{exercise}
    设 $H$ 是内积空间, $x_n,x\in H$. 并假设
    \[\lim_{n\to\infty}\|x_n\|=\|x\|\quad\text{且}\quad \lim_{n\to\infty}\innerp{y}{x_n}
      = \innerp{y}{x}, \forall y\in H.\]
    证明 $\lim_{n\to\infty} \|x_n-x\| = 0$.
\end{exercise}

\begin{proof}
    因 $\lim_{n\to\infty}\|x_n\|=\|x\|$,
    所以 $\lim_{n\to\infty}\langle x_n,x_n\rangle=\langle x,x\rangle\cdots(1)$

    因为 $\lim_{n\to\infty}\langle y,x_n\rangle=\langle y,x\rangle$,
    所以 $\lim_{n\to\infty}\langle x,x_n\rangle=\langle x,x\rangle\cdots(2)$

    两式相减得 $\lim_{n\to\infty}\langle x_n-x,x_n\rangle=0$,
    另外由第二式可得 $\lim_{n\to\infty}\langle x_n-x,x\rangle=0$.

    故
    \begin{align*}
        \lim_{n\to\infty}\|x_n-x\|^2
        & =\lim_{n\to\infty}\langle x_n-x,x_n-x\rangle\\
        & =\lim_{n\to\infty}\langle x_n-x,x_n\rangle-\lim_{n\to\infty}\langle x_n-x,x\rangle\\
        & =0-0=0.
    \end{align*}
    从而 $\lim_{n\to\infty}\|x_n-x\|=0$.
\end{proof}


\begin{exercise}
  设 $(x_n)$ 是 Hilbert 空间 $H$ 中的有界序列. 证明存在
  $(x_n)$ 的子序列 $(x_{n_k})$ 及 $x\in H$, 使得对任意 $y\in H$, 有
  $\lim_k\innerp{y}{x_{n_k}}=\innerp{y}{x}$.%
  \footnote{其实这个是 Eberlein-\v Smulian 定理的推论:
    在自反 Banach 空间中, 任何有界序列都存在弱收敛的子列.}
\end{exercise}

\begin{proof}
  (本题考查对角线选择法和Riesz表示定理)设$\|x_n\|\leq M(\forall n\in\mathbb{N^{*}})$,故对任意$m,n\in \mathbb{N^{*}},|\langle x_m,x_n\rangle|\leq\|x_m\|\cdot\|x_n\|\leq M^2$,考虑下面的一族有界内积序列
  \[\begin{array}{ccccc}
  \langle x_1,x_1\rangle&\langle x_1,x_2\rangle&\cdots&\langle x_1,x_n\rangle&\cdots\\
  \langle x_2,x_1\rangle&\langle x_2,x_2\rangle&\cdots&\langle x_2,x_n\rangle&\cdots\\
  \vdots&\vdots&\vdots&\vdots\\
  \langle x_m,x_1\rangle&\langle x_m,x_2\rangle&\cdots&\langle x_m,x_n\rangle&\cdots\\
  \vdots&\vdots&\vdots&\vdots\\
  \end{array}\]

  第一行序列$(\langle x_1,x_n\rangle)_{n\geq 1}$存在收敛子列$(\langle x_1,x_{n_k^1}\rangle)_{k\geq 1}$

  第二行子列$(\langle x_2,x_{n_k^1}\rangle)_{k\geq 1}$存在收敛子列$(\langle x_2,x_{n_k^2}\rangle)_{k\geq 1}$

  $\cdots\cdots$

  第$m$行子列$(\langle x_m,x_{n_k^{m-1}}\rangle)_{k\geq 1}$
  存在收敛子列$(\langle x_m,x_{n_k^m}\rangle)_{k\geq 1}$
  
  依此下来, 并运用对角线选择法取出 $(x_n)$ 的子列 $(x_{n_k^k})_{k\geq 1}$,
  不妨将其简记为 $(x_{n_k})$.

  设 $E = \Span((x_n)_{n\geq 1})$, 
  则任意 $y\in E$, $(\langle y,x_{n_k}\rangle)_{k\geq 1}$ 收敛,
  进一步容易验证对 $\forall y\in\overline{E}$, $(\langle y,x_{n_k}\rangle)_{k\geq 1}$ 收敛,
  最后任意 $y\in H$, 由正交分解定理得 $y=y_1+y_2$, $y_1\in\overline{E}$, $y_2\in E^{\perp}$, 故
  \[\langle y,x_{n_k}\rangle
    = \langle y_1,x_{n_k}\rangle+\langle y_2,x_{n_k}\rangle=\langle y_1,x_{n_k}\rangle.\]
  因此对任意 $y\in H,\lim_{k\to\infty}\langle y,x_{n_k}\rangle$ 存在, 而且
  \[\left|\lim_{k\to\infty}\langle y,x_{n_k}\rangle\right|\leq\limsup_{k\to\infty}|\langle y,x_{n_k}\rangle|\leq M\|y\|.\]
  因此 $y\mapsto\lim_{k\to\infty}\langle y,x_{n_k}\rangle$ 是连续线性泛函, 
  由 Riesz 表示定理知存在 $x\in H$ 使得 $\lim_{k\to\infty}\langle y,x_{n_k}\rangle=\langle y,x\rangle(\forall y\in H)$.
\end{proof}


\begin{exercise}
  设 $A$ 和 $B$ 都是 Hilbert 空间 $H$ 的非空闭凸子集,
  并设它们其中一个有界. 证明存在 $a\in A$ 和 $b\in B$,
  使得 $d(a,b)=d(A,B)$, 这里
  \[d(A,B)=\inf\{d(x,y)\mid x\in A, y\in B\}.\]
\end{exercise}

\begin{proof}
  不妨设 $A$ 有界, 由距离的定义知对任意 $n\geq 1$, 存在 $x_n\in A$ 使得
  \[d(x_n,B) < d(A,B) + \frac{1}{n}.\]
  因为 $(x_n)_{n\geq1}$ 是有界序列, 故由上一题结论知存在
  $a\in H$ 及$(x_n)_{n\geq 1}$ 的子列 (不妨仍记为 $(x_n)_{n\geq 1}$) 使得
  \[\lim_{n\to\infty}\innerp{x_n}{y}=\innerp{a}{y},\quad\forall y\in H.\]
  下面证明 $a\in A$, 由投影的性质知
  \begin{align*}
    \|a-P_A(a)\|^2
    & = \langle a-P_A(a),a-P_A(a)\rangle\\
    & = \lim_{n\to\infty}\langle a-P_A(a),x_n-P_A(a)\rangle\quad(\text{注意到虚部的极限为}\ 0)\\
    & = \lim_{n\to\infty}\Re\innerp{a-P_A(a)}{x_n-P_A(a)}\leq 0.
  \end{align*}
  故 $a=P_A(a)$, 从而 $a\in A$. 又
  \begin{align*}
    \|x_n-P_B(x_n)\|^2
    & = \|x_n-a+a-P_B(a)+P_B(a)-P_B(x_n)\|^2 \\
    & = \|a-P_B(a)\|^2+\|x_n-a+P_B(a)-P_B(x_n)\|^2 \\
    & \quad + 2\Re\langle a-P_B(a),x_n-a\rangle
      + 2\Re\langle a-P_B(a),P_B(a)-P_B(x_n)\rangle \\
    & \geq \|a-P_B(a)\|^2 + 2\Re\langle a-P_B(a),x_n-a\rangle.
  \end{align*}
  故
  \begin{align*}
    (d(A,B))^2
    & \leq \|a-P_B(a)\|^2\leq\|x_n-P_B(x_n)\|^2 - 2\Re\langle a-P_B(a),x_n-a\rangle\\
    & < \biggl(d(A,B)+\frac{1}{n}\biggr)^2 - 2\Re\langle a-P_B(a),x_n-a\rangle \\
    & \quad \to (d(A,B))^2 \qquad \text{as } n\to\infty.
  \end{align*}
  因此 $\|a-P_B(a)\|=d(A,B)$, 记 $b=P_B(a)\in B$, 即得 $d(a,b)=d(A,B)$.
\end{proof}


\begin{exercise}
  将上一习题中的条件换成 $A$ 和 $B$ 无界, 但假设 $\|x\|$
  和 $\|y\|$ 都趋向 $\infty$ 时, 必有 $d(x,y)$ 趋向 $\infty$.
  证明结论仍然成立. 在 $\mathbb{R}^2$ 中用反例说明若条件不符合假设时, 结论不成立.
\end{exercise}

\begin{proof}
  令 $M = d(A, B) + 1$, 由假设条件知存在 $r_1>0$, 使得
  \[\forall x\in A, \forall y\in B :
    (\|x\|\geq r_1 \text{ and } \|y\|\geq r_2 \Longrightarrow d(x,y)\geq M).\]
  令 $r_2 := r_1 + M$, $A_0 := A\cap B(0, r_2)$, $B_0 := B\cap B(0, r_2)$.

  取 $x\in A$, $y\in B$ 满足 $x\in A\setminus A_0$ 或 $y\in B\setminus B_0$,
  \begin{enumerate}[(1)]
    \item 若 $\|x\|, \|y\|\geq r_1$, 则 $d(x,y) \geq M$.
    \item 若 $\|x\| < r_1$, 则 $x\in A_1$, 从而必有 $y\notin B_0$, 故
      \[\|x - y\| \geq \|y\| - \|x\| > r_2 - r_1 = M.\]
    \item 若 $\|y\| < r_1$, 同理可得 $\|x-y\| > M$.
  \end{enumerate}
  综上可知
  \[d(A,B) = \inf\{d(x,y) : x\in A_0, y\in B_0\}.\]
  注意到 $A_0$, $B_0$ 都是非空有界凸集, 由上一题结论知存在 $a\in A_0$, $b\in B_0$ 使得
  \[d(a,b) = d(A,B).\]
  
  若条件不符合假设时, 取 $\mathbb{R}^2$ 中的区域 $A=\{(x,y)\mid xy\leq -1, x < 0\}$,
  $B=\{(x,y) \mid xy\geq 1, x > 0\}$, 
  则 $d(A,B)=0$, 但是不存在 $a\in A,b\in B$ 使得 $d(a,b)=0$.
\end{proof}


\begin{exercise}
    (a) 设 $H$ 是 Hilbert 空间, $D_{n}=\{-1,1\}^{n}$. 证明
    \[
    \frac{1}{2^{n}} \sum_{(\varepsilon_{k}) \in D_{n}}\left\|\varepsilon_{1} x_{1}+\cdots+\varepsilon_{n} x_{n}\right\|^{2}=\left\|x_{1}\right\|^{2}+\cdots+\left\|x_{n}\right\|^{2}, \quad \forall x_1,\cdots,x_n \in H.
    \]

    (b) 设 $(X,\|\cdot\|)$ 是 Banach 空间, 并假设有一个 $X$ 上的内积范数 $|\cdot|$ 
    等价于 $\|\cdot\|$. 证明存在正常数 $a$ 和 $b$, 使得
    \[
      a\sum_{k=1}^{n}\left\|x_{k}\right\|^{2}
      \leq\frac{1}{2^{n}} \sum_{\left(\varepsilon_{k}\right) \in D_{n}}
        \left\|\sum_{k=1}^{n} \varepsilon_{k} x_{k}\right\|^{2}
      \leq b \sum_{k=1}^{n}\left\|x_{k}\right\|^{2}, \quad \forall x_{1}, \cdots, x_{n} \in X.
    \]

    (c) 设 $1 \leq p \neq 2 \leq \infty$, 证明空间 $c_{0}$, $\ell_{p}$ 和 $L_{p}(0,1)$
    没有等价的内积范数.
\end{exercise}

\begin{proof}
    (a) 原等式等价于
    \[\sum_{(\varepsilon_k)\in D_n}\|\varepsilon_1x_1+\cdots+\varepsilon_nx_n\|^2=2^n\left(\|x_1\|^2+\cdots+\|x_n\|^2\right).\]
    上式左边
    \begin{align*}
        \mathrm{RHS}
        & = \sum_{(\varepsilon_k)\in D_n} \innerp*{\sum_{i=1}^n\varepsilon_ix_i}{\sum_{i=1}^n\varepsilon_ix_i} \\
        & = \sum_{(\varepsilon_k)\in D_n}\sum_{i=1}^n\sum_{j=1}^n \varepsilon_i\varepsilon_j\innerp{x_i}{x_j} \\
        & = \sum_{(\varepsilon_k)\in D_n}\sum_{i=1}^n\varepsilon_i^2\innerp{x_i}{x_i}+\sum_{(\varepsilon_k)\in D_n}\sum_{\substack{1\leq i,j\leq n \\ i\neq j}}\varepsilon_i\varepsilon_j\innerp{x_i}{x_j} \\
        & = \sum_{(\varepsilon_k)\in D_n}\sum_{i=1}^n\innerp{x_i}{x_i}+\sum_{(\varepsilon_k)\in D_n}\sum_{\substack{1\leq i,j\leq n \\ i\neq j}}\varepsilon_i\varepsilon_j\innerp{x_i}{x_j} \\
        & = 2^n\sum_{i=1}^n \|x_i\|^2+\sum_{(\varepsilon_k)\in D_n}\sum_{\substack{1\leq i,j\leq n \\ i\neq j}}\varepsilon_i\varepsilon_j\innerp{x_i}{x_j}.
    \end{align*}
    观察最后一项中 $\innerp{x_i}{x_j}$ 的系数:
    $\varepsilon_i=\varepsilon_j=1$ 有 $2^{n-2}$ 项,
    $\varepsilon_i=\varepsilon_j=-1$ 有 $2^{n-2}$ 项,
    $\varepsilon_i=1$, $\varepsilon_j=-1$ 有 $2^{n-2}$ 项,
    $\varepsilon_i=-1$, $\varepsilon_j=1$ 有 $2^{n-2}$ 项,
    因此 
    \[\innerp{x_i}{x_j}\text{\ 的系数\ }=2\cdot 2^{n-2}-2\cdot 2^{n-2}=0.\]
    于是即证所需.

    (b) 由于 $|\cdot|$ 等价于 $\|\cdot\|$, 故存在正常数 $C_1$ 和 $C_2$
    使得 $C_1\|\cdot\|\leq |\cdot|\leq C_2\|\cdot\|$. 由 (a) 知
    \[\frac{1}{2^n}\sum_{(\varepsilon_k)\in D_n}\left|\sum_{k=1}^n \varepsilon_kx_k\right|^2=\sum_{k=1}^n |x_k|^2.\]
    故
    \[\sum_{k=1}^n C_2^2\|x_k\|^2\geq\sum_{k=1}^n |x_k|^2=\frac{1}{2^n}\sum_{(\varepsilon_k)\in D_n}\left|\sum_{k=1}^n \varepsilon_kx_k\right|^2\geq\frac{1}{2^n}\sum_{(\varepsilon_k)\in D_n}C_1^2\left\|\sum_{k=1}^n \varepsilon_kx_k\right\|^2,\]
    因此
    \[\frac{1}{2^n}\sum_{(\varepsilon_k)\in D_n}\left\|\sum_{k=1}^n \varepsilon_kx_k\right\|^2\leq\left(\frac{C_2}{C_1}\right)^2\sum_{k=1}^n \|x_k\|^2.\]
    故取 $b=\left(\frac{c_2}{c_1}\right)^2$ 即得
    \[\frac{1}{2^n}\sum_{(\varepsilon_k)\in D_n}\left\|\sum_{k=1}^n \varepsilon_kx_k\right\|^2\leq b\sum_{k=1}^n \|x_k\|^2.\]
    同理取 $a=\left(\frac{C_1}{C_2}\right)^2$ 可得左半边不等式.
\end{proof}




\begin{exercise}
    设 $\left(C_{n}\right)$ 是 Hilbert 空间 $H$ 中的一个递增的非空闭凸子集列, $C$ 是所有 $C_n$ 的并集的闭包. 证明
    \[
    P_{C}(x)=\lim_{n\to\infty} P_{C_n}(x), \quad \forall x \in H.
    \]
\end{exercise}

\begin{proof}
    首先容易验证 $C$ 是闭凸集, 从而 $P_C(x)$ 是有定义的, 
    接下来证明 $P_C(x)=\lim_{n\to\infty}P_{C_n}(x)$ $(\forall x\in H)$, 分几步进行:

    $d(x,C)=\lim_{n\to\infty}d(x,C_n)$:
    因为对于每个 $n$, $d(x,C)\leq d(x,C_n)$,
    故 $d(x,C)\leq \lim_{n\to\infty}d(x,C_n)$,
    假设 $d(x,C)<\lim_{n\to\infty}d(x,C_n)$,
    则存在 $y\in C$, 使得 $d(x,y)<\lim_{n\to\infty}d(x,C_n)$,
    不妨设 $y\in\bigcup_{n=1}^{\infty}C_n$, 也就是说存在 $n_0$ 使得 $y\in C_{n_0}$,
    从而 $d(x,y)\geq d(x,C_{n_0})\geq\lim_{n\to\infty}d(x,C_n)$, 矛盾, 故 $d(x,C)=\lim_{n\to\infty}d(x,C_n)$.
    
    $(P_{C_n}(x))_{n\geq 1}$收敛:
    因为 $(d(x,C_n))_{n\geq 1}$ 单调递减趋于 $d(x,C)$,
    故对 $\forall\varepsilon>0$, 存在 $N\geq 1$,
    使得当 $n>N$ 时, $d(x,C_n)<d(x,C)+\varepsilon$,
    故 $\forall m,n>N$, 有
    \begin{align*}
        4(d(x,C)+\varepsilon)^2
        &\geq 2(\|x-P_{C_n}(x)\|^2+\|x-P_{C_m}(x)\|^2) \\
        &=4\left\|x-\frac{P_{C_n}(x)+P_{C_m}(x)}{2}\right\|^2+\|P_{C_n}(x)-P_{C_m}(x)\|^2 \\
        &\geq 4d(x,C)^2+\|P_{C_n}(x)-P_{C_m}(x)\|^2.
    \end{align*}
    由上式知 $(P_{C_n}(x))_{n\geq 1}$ 是 Cauchy 序列, 
    由 $C$ 的完备性知其在 $C$ 中收敛, 记为
    \[\lim_{n\to\infty}P_{C_n}(x)=y\in C.\]
    
    $y=P_C(x)$: 
    对于任意 $\forall\varepsilon>0$,
    存在 $N\geq 1$, 使得对 $\forall n>N$,
    有 $d(y,P_{C_n}(x))<\varepsilon$, $d(x,C_n)<d(x,C)+\varepsilon$, 故
    \[d(x,y)\leq d(y,P_{C_n}(x))+d(x,C_n)<d(x,C)+2\varepsilon,\] 
    由 $\varepsilon$ 的任意性知 $d(x,y)\leq d(x,C)$, 又因为 $y\in C$, 
    故 $d(x,y)=d(x,C)$, 由投影的唯一性知 $y=P_C(x)$, 证毕.
\end{proof}




\begin{exercise}
    设 $H$ 是内积空间. $(x_{1}, \cdots, x_{n})$ 是 $H$ 中的任一向量组,
    称矩阵 $(\innerp{x_i}{x_j})_{1\leq i,j\leq n}$ 
    的行列式为向量组 $\left(x_{1},\cdots, x_{n}\right)$ 的 Gram 行列式, 记作 $G(x_{1},\cdots,x_{n})$.
    
    (a) 证明 $G(x_{1},\cdots,x_{n})\geq 0$; 
    且 $G\left(x_{1},\cdots,x_{n}\right)>0$ 当且仅当向量组 $\left(x_{1}, \cdots, x_{n}\right)$ 线性独立.
    
    (b) 假设向量组 $\left(x_{1},\cdots,x_{n}\right)$ 线性独立. 
    令 $E=\operatorname{span}\left(x_{1},\cdots, x_{n}\right)$. 证明
    \[
        d(x, E)^{2}=\frac{G\left(x, x_{1}, x_{2}, \cdots, x_{n}\right)}{G\left(x_{1}, x_{2}, \cdots, x_{n}\right)}, \quad \forall x\in H.
    \]
\end{exercise}

\begin{proof}
    (参考《高等代数与解析几何》 陈志杰习题 6.3.13 及 6.4.6)

    (a) 设 $W=\operatorname{span}\{x_1,\cdots,x_n\}$ 且 $\dim W=k$,
    取 $W$ 的规范正交基 $(e_i)_{1\leq i\leq k}$. 由于
    \[x_i=\sum_{m=1}^k \innerp{x_i}{e_m}e_m,\quad x_j=\sum_{k=1}^m \innerp{x_j}{e_m}e_m,\]
    故
    \begin{align*}
        \innerp{x_i}{x_j}
        &=\innerp{\sum_{m=1}^k \innerp{x_i}{e_m}e_m}{\sum_{m=1}^k \innerp{x_j}{e_m}e_m} \\
        &=\sum_{m=1}^k \innerp{x_i}{e_m}\innerp{x_j}{e_m}=\sum_{k=1}^m \innerp{x_i}{e_m}\overline{\innerp{e_m}{x_j}}.
    \end{align*}
    记
    \[M=\begin{pmatrix}
        \innerp{x_1}{e_1} & \cdots & \innerp{x_1}{e_k} \\
        \vdots            &        & \vdots            \\
        \innerp{x_n}{e_1} & \cdots & \innerp{x_n}{e_k}
    \end{pmatrix}_{n\times k}.\]
    则
    \[M^\T
    =\begin{pmatrix}
        \overline{\innerp{e_1}{x_1}} & \cdots & \overline{\innerp{e_1}{x_n}} \\
        \vdots & & \vdots \\
        \overline{\innerp{e_k}{x_1}} & \cdots & \overline{\innerp{e_k}{x_n}} 
    \end{pmatrix}_{k\times n},\]
    且 $(\innerp{x_i}{x_j})_{1\leq i,j\leq n}=MM^\T$, 从而 $G(x_1,\cdots,x_n)=\det(MM^\T)$.

    若 $k<n$, 则 $\rank(M)\leq k<n$, 故 $\rank(\innerp{x_i}{x_j})<n$, 故 $|G(x_1,\cdots,x_n)|=0$.

    若 $k=n$, 则 $x_1,\cdots, x_n$ 线性无关,
    即关于 $\lambda_1,\cdots,\lambda_n$ 的方程
    \[\lambda_1 x_1+\cdots+\lambda_nx_n=0\]
    只有零解. 考虑关于 $\lambda_1,\cdots,\lambda_n$ 的齐次线性方程组
    \[\begin{cases}
        \lambda_1\innerp{x_1}{e_1}+\cdots+\lambda_n\innerp{x_n}{e_1}=0 \\
        \cdots \\
        \lambda_1\innerp{x_1}{e_n}+\cdots+\lambda_n\innerp{x_n}{e_n}=0.
    \end{cases}\]
    上述方程组的系数矩阵即为 $M^\T$, 将上述方程组的第 $i$ ($1\leq i\leq n$) 个方程乘以 $e_i$ 并求和即得
    \[\lambda_1x_1+\cdots+\lambda_nx_n=0,\]
    于是 $\lambda_1=\cdots=\lambda_n=0$, 因此 $\det(M)\neq 0$,
    从而 
    \[G(x_1,\cdots,x_n)=\det(MM^\T)=\det(M)\det(M^\T)=(\det(M))^2>0.\]

    (b) 略.
\end{proof}




\begin{exercise}
    设 $E=C([0,1])$ 上装备有如下的内积
    \[
    \innerp{f}{g}=\int_{0}^{1} f(t) \overline{g(t)}\diff t.
    \]
    并设 $E_{0}$ 表示在 $[0,1]$ 上积分为 $0$ 的函数组成的 $E$ 的向量子空间. 考虑 $E$ 的 向量子空间:
    \[
    H=\{f \in E: f(1)=0\} \text {\ 且\ } H_{0}=E_{0} \cap H.
    \]

    (a) 验证 $H_{0}$ 是 $H$ 的闭的真向量子空间.

    (b) 设 $h(t)=t-\frac{1}{2}, t \in[0,1]$. 证明
    \begin{enumerate}[(i)]
    \item $E=\operatorname{span}(H, h)$ 且有 $E_{0}=\operatorname{span}\left(H_{0}, h\right)$;
    \item $h$ 属于 $H_{0}$ 在 $E$ 中的闭包.
    \end{enumerate}

    (c) 证明 $H_{0}^{\perp}=\{0\}$. 解释所得结果蕴含的意义.
\end{exercise}

\begin{proof}
    (a) 任取一列 $(f_n)\subset E_0$ 且 $f_n\rightarrow f$,
    则 $\int_0^1 f_n(t)\diff t=0$, $\int_0^1|f_n(t)-f(t)|^2\diff t\rightarrow0$,
    故 $f_n(t)-f(t)=0$ a.e. $(n\to\infty)$,
    因此 $\int_0^1 f(t)\diff t=0$, 也即 $f\in E_0$,
    从而说明 $E_0$ 是闭子空间, 故 $H_0=E_0\cap H$ 是 $H$ 的闭子空间.
    取 $f(t)=1-t$, 显然 $f(t)\in H$, 但是 $f(t)\notin H_0$, 故 $H_0$ 是 $H$ 的真子空间.

    (b)(i) 对于 $\forall f\in E$,
    令 $g(t)=f(t)-2f(1)h(t)$,
    则 $g(1)=0$, 故 $g\in H$, 所以 $E=\operatorname{span}(H,h)$.

    对于 $\forall f\in E_0$,
    令 $g(t)=f(t)-2f(1)h(t)$, 则 $g(1)=0$,
    $\int_0^1 g(t)\diff t=0$, 即 $g\in H_0$, 所以 $E_0=\operatorname{span}(H_0,h)$.

    (ii)取 $g_n(t)=\sin2\pi nt$, 则 $g_n\in H_0$,由于$h$关于点$(1/2,0)$对称,
    故 $h$ 的 Fourier 展开式中只含有形如 $g_n$ 的项, 因此 $h\in\bar{H}_0$
\end{proof}



\begin{exercise}
    仍设 $E$ 为上一习题中的内积空间, 并令 $0<c<1$. 记
\[
F=\left\{f \in E:f|_{[0, c]}=0\right\}.
\]

(a) 验证 $F$ 是 $E$ 的闭的真向量子空间.

(b) 证明 $F \oplus F^{\perp} \neq E$. 解释所得结果蕴含的意义.
\end{exercise}
% 14.
% \begin{proof}
% (a)任取一列$(f_n)\subset F$且$f_n\rightarrow f$,则$f_n|_{[0,c]}=0,\int_0^1|f_n(t)-f(t)|^2\diff t\rightarrow 0$\\
% 故$\int_0^c|f_n(t)-f(t)|^2\diff t=\int_0^c|f(t)|^2\rightarrow 0\Rightarrow f|_{[0,c]}=0$, 因此$f\in F$,从而$F$是$E$的闭子空间, 而$F$是$E$的真子空间是显然的.

% (b)要证明$F\oplus F^{\perp}\not= E$, 即需要证明存在$E$中元素使其没有基于$F$的正交分解.

% 取$f\equiv 1\in E$, 对任意$g\in F,\exists\delta>0,s.t.\forall t\in [c,c+\delta],(f-g)(t)>0$(不妨设$c+\delta<1$), 取实值函数$h(t)$满足:
% \[h(t)
% \begin{cases}
% =0&t\in [0,c]\cup [c+\delta,1]\\
% >0&t\in (c,c+\delta)
% \end{cases}\]
% 则
% \[\int_0^1(f-g)(t)\cdot h(t)\diff t=\int_c^{c+\delta}(f-g)(t)\cdot h(t)\diff t>0\]
% 故 $f-g\notin F^{\perp}$, 从而证明了$F\oplus F^{\bot}\not= E$, 此结论说明$E$不是Hilbert空间.
% \end{proof}



\begin{exercise}
    (a) 设 $E$ 和 $F$ 是 Hilbert 空间 $H$ 的两个正交向量子空间. 证明 $E+F$ 是闭的当且仅当 $E$ 和 $F$ 都是闭的.
    
    (b) $(e_{n})$ 表示 $\ell_{2}$ 中的标准正交基. 
    设 $E$ 是 $\left\{e_{2 n}: n \geq 1\right\}$ 的线性扩张的闭包, 
    而 $F$ 是 $\left\{e_{2 n}+\frac{1}{n} e_{2 n+1}: n \geq 1\right\}$ 的线性扩张的闭包. 
    证明 $E\cap F=\{0\}$ 并 且 $E+F$ 在 $\ell_{2}$ 中不是闭的.
\end{exercise}

\begin{proof}
    (a) 由 $E$ 与 $F$ 正交知 $E+F=E\oplus F$, 即任取 $z\in E+F$,
    存在唯一的 $x\in E$ 和 $y\in F$, 使得 $z=x+y$.

    \necessary
    任取 $E$ 中收敛列 $(x_n)_{n\geq 1}$, 设 $x_n\to x\in E+F$,
    即 $\lim_{n\to\infty}\|x_n-x\|=0$. 由于 $x\in E+F$, 故存在 $x'\in E$, $x''\in F$,
    使得 $x=x'+x''$, 那么
    \begin{align*}
        \lim_{n\to\infty}\|x_n-x\|^2
        & =\lim_{n\to\infty}\|x_n-x'-x''\|^2 \\
        & =\lim_{n\to\infty}\|x_n-x'\|^2+\|x''\|^2-2\Re\innerp{x_n-x'}{x''} \\
        & =\lim_{n\to\infty}\|x_n-x'\|^2+\|x''\|^2=0.
    \end{align*}
    故必有 $x''=0$, 从而 $x=x'\in E$, 因此 $E$ 为闭集. 同理可证 $F$ 为闭集.

    \sufficient
    任取 $E+F$ 中 Cauchy 序列 $(z_n)_{n\geq 1}$, 设 $z_n=x_n+y_n$, 其中 $x_n\in E$, $y_n\in F$, 则
    \begin{align*}
        \|z_m-z_n\|^2
        & =\|x_m+y_m-x_n-y_n\|^2 \\
        & =\|x_m-x_n\|^2+\|y_m-y_n\|^2+2\Re\innerp{x_m-x_n}{y_m-y_n} \\
        & =\|x_m-x_n\|^2+\|y_m-y_n\|^2\to 0\quad (m,n\to\infty).
    \end{align*}
    故 $(x_n)_{n\geq 1}$ 和 $(y_n)_{n\geq 1}$ 分别为 $E$ 和 $F$ 中的 Cauchy 序列,
    而 $E,F$ 皆完备, 故设 $x_n\to x\in E$, $y_n\to y\in F$.
    令 $z=x+y\in E+F$, 则当 $n\to\infty$ 时
    \[\|z_n-z\|^2=\|x_n+y_n-x-y\|^2=\|x_n-x\|^2+\|y_n-y\|^2\to 0.\]
    即 $z_n\to z\in E+F$, 故 $E+F$ 完备, 从而为闭集.

    (b) 任取 $x\in E\cap F$, 由于 $E$ 是 Hilbert 空间, 且
    $\{e_{2n}:n\geq 1\}$ 是 $E$ 的一组规范正交基, 故存在唯一的系数列 $(x_n)_{n\geq 1}$,
    使得 $x=\sum_{n=1}^{\infty}x_n e_{2n}$. 类似地, $F$ 为 Hilbert 空间, 且规范化后的
    $\{\frac{n}{\sqrt{n^2+1}}(e_{2n}+\frac{1}{n}e_{2n+1}):n\geq 1\}$ 是 $F$ 的一组规范正交基,
    故存在唯一的系数列 $(y_n)_{n\geq 1}$, 
    使得 $x=\sum_{n=1}^{\infty}y_n\frac{n}{\sqrt{n^2+1}}(e_{2n}+\frac{1}{n}e_{2n+1})$.
    于是对任意 $n\geq 1$, 有
    \[x_n=y_n\cdot\frac{n}{\sqrt{n^2+1}},\quad\frac{y_n}{\sqrt{n^2+1}}=0\Longrightarrow x_n=y_n=0.\]
    故 $x=0$, 因此 $E\cap F=\{0\}$. 下证 $E+F$ 不是闭集,
    取 $x^{(m)}=\sum_{n=1}^m -e_{2n}\in E$, $y^{(m)}=\sum_{n=1}^m (e_{2n}+\frac{1}{n}e_{2n+1})\in F$,
    则
    \[x^{(m)}+y^{(m)}=\sum_{n=1}^m \frac{1}{n}e_{2n+1}\in E+F\]
    且
    \[x^{(m)}+y^{(m)}\xrightarrow{\ell_2}\sum_{n=1}^{\infty}\frac{1}{n}e_{2n+1}.\]
    但 $\sum_{n=1}^{\infty}\frac{1}{n}e_{2n+1}\notin E+F$, 事实上, 若存在
    \[x=\sum_{n=1}^{\infty}x_n e_{2n}\in E\]
    和
    \[y=\sum_{n=1}^{\infty}y_n\frac{n}{\sqrt{n^2+1}}
      \biggl(e_{2n}+\frac{1}{n}e_{2n+1}\biggr)\in F\]
    使得 $x+y=\sum_{n=1}^{\infty}\frac{1}{n}e_{2n+1}$, 则
    \[x_n+\frac{ny_n}{\sqrt{n^2+1}}=0,\quad\frac{y_n}{\sqrt{n^2+1}}=\frac{1}{n}\Longrightarrow x_n=-1,y_n=\frac{\sqrt{n^2+1}}{n}.\]
    但此时 $x=\sum_{n=1}^{\infty}-e_{2n}\notin\ell_2$, 矛盾.
\end{proof}



\begin{exercise}
    设 $H$ 是 Hilbert 空间, $E$ 是 $H$ 的非零的闭向量子空间. 
    设 $P$ 是 $H$ 到 $E$ 的投 影 (投影意味着 $P$ 是 $H$ 上的线性算子且满足 $P^{2}=P$ ). 证明以下命题等价:
    
    (a) $P=P_{E}$.
    
    (b) $\|P\|=1$.
    
    (c) $|\langle x, P(x)\rangle| \leq\|x\|^{2}, \forall x \in H$.
\end{exercise}

\begin{proof}
    (a) $\Rightarrow$ (b) 显然.

    (b) $\Rightarrow$ (c)
    由 $\|P\|=1$ 知 $\|P(x)\|\leq\|x\|$, 故 $|\langle x,P(x)\rangle|\leq\|x\|\cdot\|P(x)\|\leq\|x\|^2$.

    (c) $\Rightarrow$ (a)
    分三步进行
    \begin{itemize}
    \item $\forall y\in E,P(y)=y$: 任意 $y\in E,\exists x\in H,s.t.P(x)=y$,故$P(y)=P(P(x))=P(x)=y$.
    \item $\forall y\in E^{\perp},P(y)=0$: 根据$P$的线性性知:
    \[P(y+nP(y))=(n+1)P(y).\]
    故结合条件得
    \[|\langle y+nP(y),(n+1)P(y)\rangle|\leq\|y+nP(y)\|^2=\|y\|^2+n^2\|P(y)\|^2.\]
    又
    \[|\langle y+nP(y),(n+1)P(y)\rangle|=|\langle nP(y),(n+1)P(y)\rangle|=(n^2+n)\|P(y)\|^2.\]
    结合两式得
    \[\|P(y)\|^2\leq\frac{1}{n}\|y\|^2.\]
    上式对于任意正整数 $n$ 成立, 故只能有 $P(y)=0$.
    \item $\forall x\in H,P(x)=P_E(x)$: 根据前两步的结果可知
    \[P(x)=P(P_E(x))+P(x-P_E(x))=P_E(x).\qedhere\]
    \end{itemize}
\end{proof}




\begin{exercise}
    设 $H$ 是 Hilbert 空间, $E$ 是 $H$ 的向量子空间. 
    设 $F$ 为赋范空间, $u: E \rightarrow F$ 是连续线性映射. 
    证明 $u$ 有连续的线性延拓 $\widehat{u}: H \rightarrow F$, 且 $\|\widehat{u}\|=\|u\|$.
\end{exercise}

\begin{proof}
    假设 $F$ 为 Banach 空间, 由定理 3.2.13 知连续线性映射 $u:E\to F$
    可以唯一地扩展为连续线性映射 $\tilde{u}:\closure{E}\to F$ 且 $\|\tilde{u}\|=\|u\|$.
    对任意 $x\in H$, 定义
    \[\hat{u}(x):=\tilde{u}\left(P_{\closure{E}}(x)\right).\]
    若 $x\in E$, 则 $\hat{u}(x)=\tilde{u}(x)=u(x)$, 故 $\hat{u}$
    为 $u$ 的扩展映射.

    $\hat{u}$ 为连续映射: 对任意 $x,y\in H$ 和 $\lambda\in\FK$,
    由 $\tilde{u}$ 和 $P_{\closure{E}}$ 的线性性得
    \begin{align*}
        \hat{u}(\lambda x+y)
        & =\tilde{u}(P_{\closure{E}}(\lambda x+y)) \\
        & =\tilde{u}(\lambda P_{\closure{E}}(x)+P_{\closure{E}}(y)) \\
        & =\lambda\tilde{u}(P_{\closure{E}}(x))+\tilde{u}(P_{\closure{E}}(y)) \\
        & =\lambda\hat{u}(x)+\hat{u}(y).
    \end{align*}

    $\hat{u}$ 为有界映射: 对任意 $x\in H$, 有
    \[\|\hat{u}(x)\|=\|\tilde{u}(P_{\closure{E}}(x))\|\leq\|\tilde{u}\|\|P_{\closure{E}}\|\|x\|=\|u\|\|x\|.\]
    故 $\|\tilde{u}\|\leq\|u\|$, 又
    \[\|u\|=\sup_{x\in E,x\neq 0}\frac{\|u(x)\|}{\|x\|}\leq\sup_{x\in H,x\neq 0}\frac{\|\hat{u}(x)\|}{\|x\|}=\|\hat{u}\|,\]
    所以 $\|\hat{u}\|=\|u\|$.
\end{proof}



\begin{exercise}
    设 $[0,1]$ 上赋予 Lebesgue 测度, $H=L_{2}(0,1)$. 并假设 $K \in L_{2}([0,1] \times[0,1])$. 我们定义
    \[
    T_{K}(f)(x)=\int_{0}^{1} K(x, y) f(y)\diff y, \quad f \in H, x \in[0,1].
    \]

    (a) 证明 $T_{K}(f)$ 在 $[0,1]$ 上几乎处处有定义.

    (b) 证明 $T_{K} \in\mathcal{B}(H)$ 且
    \[
    \left\|T_{K}\right\|\leq\|K\|_{L_{2}([0,1] \times[0,1])}.
    \]

    (c) 设 $\widetilde{K}(x, y)=\overline{K(y, x)}$, $x,y\in[0,1]$. 证明 $T_{K}^{*}=T_{\tilde{K}}$.

    (d) 定义
    \[
    T(f)(x)=\int_{0}^{x} f(1-y)\diff y, \quad f \in H, x \in[0,1].
    \]
    证明 $T\in\mathcal{B}(H)$ 且有 $T^{*}=T$.
    最后给出 $T$ 的非零特征值并证明相应的特征子空间两两正交.
\end{exercise}

\begin{proof}
    (a)任意固定 $x$, 将 $K(x,y)$ 看作关于 $y$ 的一元函数, 由 Cauchy-Schwarz 不等式得:
    \begin{align*}
        |\innerp{K}{\bar{f}}|^2
        & =\left|\int_0^1K(x,y)f(y)\diff y\right|^2 \\
        & \leq\int_0^1|K(x,y)|^2\diff y\cdot\int_0^1|f(y)|^2\diff y.
    \end{align*}
    因为 $f\in L_2(0,1)$, 所以
    \[\int_0^1|f(y)|^2\diff y<\infty.\]
    因为 $K\in L_2([0,1]\times [0,1])$, 所以
    \[\int_0^1\int_0^1|K(x,y)|^2\diff y\diff x<\infty\Rightarrow\int_0^1|K(x,y)|^2\diff y<\infty, \text{ a.e.}\]
    结合以上三式得
    \[\left|\int_0^1K(x,y)f(y)\diff y\right|^2<\infty, \text{ a.e.}\]
    也就证明了 $T_K(f)$ 在 $[0,1]$ 上几乎处处有定义.

    (b) 由 (a) 中结论知: $\forall f\in H$, $T_K(f)\in H$.

    首先, $T_K$ 为线性算子. 对于任意 $f,g\in H$ 和 $\lambda\in\mathbb{K}$, 有
    \[T_K(\lambda f+g)=\int_0^1K(x,y)(\lambda f(y)+g(y))\diff y=\lambda T_K(f)+T_K(g).\] 

    其次, $T_K$ 为有界算子. 对于任意 $f\in H$, 有
    \begin{align*}
        \|T_K(f)\|^2
        &=\int_0^1|T_K(f)(x)|^2\diff x\\
        &=\int_0^1\left|\int_0^1K(x,y)f(y)\diff y\right|^2\diff x\\
        &\leq \int_0^1\left(\int_0^1|K(x,y)|^2\diff y\cdot\int_0^1|f(y)|^2\diff y\right)\diff x\\
        &=\int_0^1|f(y)|^2\diff y\cdot\int_0^1\int_0^1|K(x,y)|^2\diff x\diff y\\
        &=\|f\|^2\cdot\|K\|_{L_2([0,1]\times[0,1])}^2.
    \end{align*}
    故 $T_K$ 为有界算子且 $\|T_K\|\leq \|K\|_{L_2([0,1]\times [0,1])}$.


    (c) 对于 $\forall f,g\in H$, 有
    \begin{align*}
        \langle T_{\widetilde{K}}(f),g\rangle&=\int_0^1\left(\int_0^1\widetilde{K}(x,y)f(y)\diff y\right)\conjugate{g(x)}\diff x\\
        &=\int_0^1\left(\int_0^1\overline{K(y,x)}f(y)\diff y\right)\conjugate{g(x)}\diff x \\
        &=\int_0^1\left(\int_0^1\overline{K(x,y)}f(x)\diff x\right)\conjugate{g(y)}\diff y \\
        &=\int_0^1\left(\int_0^1\overline{K(x,y)g(y)}f(x)\diff x\right)\diff y \\
        &=\int_0^1\left(\int_0^1\overline{K(x,y)g(y)}f(x)\diff y\right)\diff x \\
        &=\int_0^1f(x)\overline{\left(\int_0^1K(x,y)g(y)\diff y\right)}\diff x \\
        &=\langle f,T_K(g)\rangle.
    \end{align*}
    因此由伴随算子的定义知 $T_K^{*}=T_{\widetilde{K}}$.

    (d) $T(f)(x)=\int_0^x f(1-y)\diff y=\int_{1-x}^1 f(y)\diff y$, 取:
    \[K(x,y)=
    \begin{cases}
    0, & 0\leq y\leq 1-x,\\
    1, & 1-x<y\leq 1.
    \end{cases}\]
    显然 $K(x,y)\in L_2([0,1]\times [0,1])$, 且
    \[T_K(f)(x)=\int_0^1 K(x,y)f(y)\diff y=\int_{1-x}^1 f(y)\diff y=T(f)(x).\]
    即在此情形下 $T_K$ 和 $T$ 是同一个算子, 利用(b)中结论知 $T\in\mathcal{B}(H)$.

    由 $K(x,y)$ 的定义知 $\widetilde{K}(x,y)=\overline{K(y,x)}=K(y,x)=K(x,y)$, 所以由 (c) 中结论知:
    \[T^{*}=T_K^{*}=T_{\widetilde{K}}=T_K=T.\]
    因为 $T^{*}=T$, 所以 $T$ 的特征值全部都为实数, 
    任取两个特征值 $\lambda,\mu\in\mathbb{R}$, 任取两个相应的特征向量 $f,g\in H$, 
    即$T(f)=\lambda f,T(g)=\mu g$,则:
    \[\mu\langle f,g\rangle=\langle f,\mu g\rangle=\langle f,T(g)\rangle=\langle T(f),g\rangle=\lambda\langle f,g\rangle.\]
    从而
    \[(\mu-\lambda)\langle f,g\rangle=0.\]
    故 $\langle f,g\rangle=0$, 因此相应的特征子空间两两正交.

    下面具体求特征值. 任取非零特征值 $\lambda$ 及其相应的特征向量 $f$, 则
    \[T(f)(x)=\int_{1-x}^1 f(y)\diff y=\lambda f(x),\quad\forall x\in [0,1].\]
    故 $f(0)=0$ 且 $\lambda f(1)=\int_0^1 f(y)\diff y$. 将上式求导一次得
    \begin{equation}
        f(1-x)=\lambda f'(x)\Longrightarrow f(x)=\lambda f'(1-x).\tag{$\star$}
    \end{equation}
    再将上式求导一次得
    \begin{equation}
        -f'(1-x)=\lambda f''(x).\tag{$\star\star$}
    \end{equation}
    结合 $(\star)(\star\star)$ 两式即得 ODE
    \[f''(x)+\frac{1}{\lambda^2}f(x)=0.\]
    上述常微分方程的解为 $f(x)=C_1\cos\frac{x}{\lambda}+C_2\sin\frac{x}{\lambda}$.
    由 $f(0)=0$, 得 $f(x)=C_2\sin\frac{x}{\lambda}$, 再由 $\lambda f(1)=\int_0^1 f(x)\diff x$ 得
    \[\lambda C_2\sin\frac{1}{\lambda}=\int_0^1 C_2\sin\frac{x}{\lambda}\diff x.\]
    由上式直接解得 $\sin\frac{1}{\lambda}+\cos\frac{1}{\lambda}=1$,
    故 $\lambda=\frac{1}{2k\pi}$ ($k\in\FZ,k\neq 0$) 或 $\frac{1}{\frac{\pi}{2}+2k\pi}$ ($k\in\FZ$).
\end{proof}

% \begin{remark}
% 如果 $u\in\mathcal{B}(H)$ 是正规的且 $u(f)=\lambda f$, 则 $u^{*}(f)=\bar{\lambda}f$ 利用 $\lambda\mathbbm{1}-u$ 的正规性可以证明此结果, 即 $(\lambda\mathbbm{1}-u)^{*}(\lambda\mathbbm{1}-u)=(\lambda\mathbbm{1}-u)(\lambda\mathbbm{1}-u)^{*}$
% \end{remark}


\begin{exercise}
    和上一习题一样, 令 $H=L_{2}(0,1)$; 并设 $\left(e_{n}\right)_{n \geq 1}$ 是 $H$ 中的规范正交集. 
    证明: $\left(e_{n}\right)_{n \geq 1}$ 是 $H$ 上的规范正交基的充分必要条件是
    \[
    \sum_{n\geq 1}\left|\int_{0}^{x} e_{n}(t)\diff t\right|^{2}=x, \quad \forall x \in[0,1].
    \]
\end{exercise}

\begin{proof}
    \necessary
    取 $f_x=\mathbbm{1}_{(0,x)}$, 由 Parseval 恒等式得 
    \begin{equation}
        \|f_x\|_{L_2}^2=\sum_{n\geq 1}|\innerp{f_x}{e_n}|^2,\tag{$\star$}
    \end{equation}
    换个马甲即为
    \[x=\sum_{n\geq 1}\left|\int_{0}^{x} e_{n}(t)\diff t\right|^{2}.\]

    \sufficient
    将 $(e_n)_{n\geq 1}$ 扩展成为 $L_2(0,1)$ 的规范正交基 $(e_n)_{n\geq 1}\cup(\tilde{e}_n)_{n\geq 1}$,
    则由 Parseval 恒等式得
    \[\|f_x\|_{L_2}^2=\sum_{n\geq 1}|\innerp{f_x}{e_n}|^2+\sum_{n\geq 1}|\innerp{f_x}{\tilde{e}_n}|^2.\]
    而由 $(\star)$ 式得 $\sum_{n\geq 1}|\innerp{f_x}{\tilde{e}_n}|^2=0$, 即对任意的 $n\geq 1$, 有
    \[\innerp{f_x}{\tilde{e}_n}=\int_0^x \tilde{e}_n(t)\diff t=0,\quad\forall x\in [0,1].\]
    故 $\tilde{e}_n=0$, $\forall n\geq 1$, 从而 $(e_n)_{n\geq 1}$ 为 $L_2(0,1)$ 的规范正交基.
\end{proof}



\begin{exercise}
    设 $\Omega$ 是复数域 $\mathbb{C}$ 中的开集, 约定 $\mathbb{C}$ 
    上的测度为 $\mathbb{R}^{2}$ 上的 Lebesgue 测度, 记为 $\diff \lambda(z)$. 令
    \[H_{\Omega}=\{f\in L_{2}(\Omega): f \text{\ 是\ }\Omega\text{\ 上的全纯函数}\}.\]
    对任一点 $z\in\Omega$, $\delta_{z}$ 表示 $z$ 处在 $H_{\Omega}$ 上的演化, 即有 $\delta_{z}(f)=f(z), f \in H_{\Omega}$.

    (a) 若 $\closure{B}(z, r)=\{w\in\mathbb{C}:|w-z|\leq r\}\subset\Omega$, 证明
    \[
    f(z)=\frac{1}{\pi r^2} \int_{\closure{B}(z, r)} f(w)\diff\lambda(w), \quad\forall f\in H_{\Omega}.
    \]

    (b) 证明
    \[
    f \in H_{\Omega}, z \in\Omega, d(z,\Omega^{c})>r\Longrightarrow|f(z)| \leq \frac{1}{\sqrt{\pi} r}\|f\|_{2}.
    \]
    
    (c) 证明: 当在 $L_{2}(\Omega)$ 上赋予内积运算时, $H_{\Omega}$ 是一个可分的 Hilbert 空间.
\end{exercise}



\begin{exercise}
    设 $H$ 是一个 Hilbert 空间, 并设 $T\in\mathcal{B}(H)$ 且 $\|T\|\leq 1$. 证明:

    (a) $T(x)=x$ 当且仅当 $T^*(x)=x$, $x\in H$.

    (b) $\ker(I-T)=\ker(I-T^*)$.

    (c) $H=\ker(I-T)\oplus\closure{(I-T)(H)}$.
\end{exercise}

\begin{proof}
    (a)由伴随算子的性质知 $\|T^*\|=\|T\|\leq 1$. 当 $T(x)=x$ 时,
    \begin{align*}
        \|T^*(x)-x\|^2
        & =\|T^*(x)\|^2+\|x\|^2-2\Re\innerp{T^*}{x} \\
        & =\|T^*(x)\|^2+\|x\|^2-2\Re\innerp{x}{T(x)} \\
        & =\|T^*(x)\|^2+\|x\|^2-2\Re\innerp{x}{x} \\
        & =\|T^*(x)\|^2-\|x\|^2 \\
        & \leq\|T^*\|^2\|x\|^2-\|x\|^2\leq 0,
    \end{align*}
    故 $T^*(x)=x$. 同理可证当 $T^*(x)=x$ 时有 $T(x)=x$.

    (b) $x\in\ker(I-T)\Leftrightarrow x-T(x)=0\Leftrightarrow x-T^{*}(x)=0\Leftrightarrow x\in\ker(I-T^{*})$, 
    故 $\ker(I-T)=\ker(I-T^{*})$.

    (c) 由正交分解定理, 只需证明:
    \[\ker(I-T)=[(I-T)(H)]^{\perp}\]

    先证: $(I-T)^*=I-T^*$. 对于任意 $x,y\in H$, 有
    \[\innerp{(I-T)(x)}{y}=\innerp{x}{y}-\innerp{T(x)}{y}=\innerp{x}{y}-\innerp{x}{T^*(y)}=\innerp{x}{(I-T^*)(y)},\]
    故 $(I-T)^*=I-T^*$.

    再证: $\ker(I-T)=(I-T)(H)^{\perp}$.
    一方面, 对任意 $x\in (I-T)(H)^{\perp}$ 和 $y\in H$, 有
    \[\innerp{(I-T^*)(x)}{y}=\innerp{(I-T)^*(x)}{y}=\innerp{x}{(I-T)(y)}=0.\]
    由 $y$ 的任意性知 $x\in\ker(I-T^*)=\ker(I-T)$, 故 $(I-T)(H)^{\perp}\subset\ker(I-T)$.
    另一方面, 对任意 $x\in\ker(I-T)=\ker(I-T^*)$ 和 $(I-T)(y)\in (I-T)(H)$, 有
    \[\innerp{x}{(I-T)(y)}=\innerp{(I-T)^*(x)}{y}=\innerp{(I-T^*)(x)}{y}=0,\]
    故 $\ker(I-T)\subset (I-T)(H)^{\perp}$.
\end{proof}



\begin{exercise}
    设 $H$ 是 Hilbert 空间. 称映射 $A\in\mathcal{B}(H)$ 为压缩算子, 若 $\|A\|\leq 1$;
    称 $A$ 是正的, 若对任一 $x\in H$, 有 $\innerp{A(x)}{x}\geq 0$.

    (a) 证明 $H$ 上任意压缩正算子 $A$ 满足
    \[\|x-A(x)\|^2\leq\|x\|^2-\|A(x)\|^2,\quad\forall x\in H.\]

    (b) 设 $T=A_1\cdots A_r$ 是 $H$ 上 $r$ 个压缩正算子的乘积. 记 $N=\ker(I-T)$,
    并且 $P$ 是 $N$ 上的正交投影. 我们的目的是证明: 在强算子拓扑下, $T^n\to P$, 即有
    \begin{equation}
        \lim_{n\to\infty} \|T^n(x)-P(x)\|=0,\quad\forall x\in H. \tag{$*$}
    \end{equation}
    \begin{enumerate}[(i)]
        \item 证明 \[\|x-T(x)\|^2\leq r\bigl(\|x\|^2-\|T(x)\|^2\bigr),\quad\forall x\in H.\]
              由此导出: 对任意有界序列 $(x_n)\subset H$, 有
              \[\lim_{n\to\infty}(\|x_n\|-\|T(x_n)\|)=0\Rightarrow\lim_{n\to\infty} \|x_n-T(x_n)\|=0.\]
        \item 证明 ($*$) 式: 先对 $x\in N$ 证明, 再对 $x\in\closure{(I-T)(H)}$ 证明.
        \item 证明 \[N=\bigcap_{j=1}^r \ker(I-A_j).\]
    \end{enumerate}

    (c) 设 $P_1,\cdots,P_r$ 分别是 $H$ 的 $r$ 个闭向量子空间 $E_1,\cdots,E_r$ 上的投影算子,
    并令 $T=P_1\cdots P_r$. 证明 $(T^n)_{n\geq 1}$ 在强算子拓扑下收敛到一个投影算子.
\end{exercise}

\begin{proof}
    (a) 先证明 $A$ 是自伴算子, 即 $\innerp{A(x)}{y}=\innerp{x}{A(y)}$, $\forall x,y\in H$. 因
    \[\innerp{A(x+y)}{x+y}=\innerp{A(x)}{x}+\innerp{A(y)}{y}+\innerp{A(y)}{x}+\innerp{A(x)}{y}\in\mathbb{R},\]
    故 $\innerp{A(y)}{x}+\innerp{A(x)}{y}\in\mathbb{R}$, 故
    \begin{equation}
            \innerp{A(y)}{x}+\innerp{A(x)}{y}=\conjugate{\innerp{A(y)}{x}}+\conjugate{\innerp{A(x)}{y}}=\innerp{x}{A(y)}+\innerp{y}{A(x)}.\tag{$\star$}
    \end{equation}
    又因
    \[\innerp{A(x-\ii y)}{x-\ii y}=\innerp{A(x)}{x}+\innerp{A(y)}{y}+\ii\innerp{A(x)}{y}-\ii\innerp{A(y)}{x}\in\mathbb{R},\]
    故 $\ii\innerp{A(x)}{y}-\ii\innerp{A(y)}{x}\in\mathbb{R}$, 故
    \[\ii(\innerp{A(x)}{y}-\innerp{A(y)}{x})=(-\ii)(\innerp{y}{A(x)}-\innerp{x}{A(y)}),\]
    从而
    \begin{equation}
        \innerp{A(x)}{y}-\innerp{A(y)}{x}=\innerp{x}{A(y)}-\innerp{y}{A(x)}.\tag{$\star\star$}
    \end{equation}
    结合 $(\star)(\star\star)$ 即得 $\innerp{A(x)}{y}=\innerp{x}{A(y)}$.

    再证明 $I-A$ 是正算子, 从而是自伴算子. 对任意 $x\in H$, 由 Cauchy-Schwarz 不等式有
    \[\innerp{A(x)}{x}\leq\|A(x)\|\|x\|\leq\|A\|\|x\|^2\leq\|x\|^2=\innerp{x}{x},\]
    即 $\innerp{(I-A)(x)}{x}\geq 0$, 故 $I-A$ 是正算子.

    原不等式 $\|x-A(x)\|^2\leq\|x\|^2-\|A(x)\|^2$ 等价于 $\innerp{A(x)}{(I-A)(x)}\geq 0$,
    故只需证后者即可, 由 $I-A$ 为自伴算子得
    \begin{align*}
        \innerp{A(x)}{(I-A)(x)}
        & =\innerp{(I-A)A(x)}{x} \\
        & =\innerp{(I-A)A(x)}{(I-A)(x)+A(x)} \\
        & =\innerp{(I-A)A(x)}{(I-A)(x)}+\innerp{(I-A)A(x)}{A(x)} \\
        & =\innerp{A(I-A)(x)}{(I-A)(x)}+\innerp{(I-A)A(x)}{A(x)} \\
        & \geq 0.
    \end{align*}
    最后用到了 $A$ 和 $I-A$ 皆为正算子, 证毕.

    (b) (i) 利用范数的三角不等式、Cauchy-Schwarz 不等式以及 (a) 中所得不等式得
    \begin{align*}
               & \|x-T(x)\|^2 \\
        ={}    & \|x-A_r(x)+A_r(x)-A_{r-1}A_r(x)+\cdots+A_2\cdots A_r(x)-A_1A_2\cdots A_r(x)\|^2 \\
        \leq{} & (\|x-A_r(x)\|+\|A_r(x)-A_{r-1}A_r(x)\|+\cdots+\|A_2\cdots A_r(x)-A_1A_2\cdots A_r(x)\|)^2 \\
        \leq{} & r\bigl(\|x-A_r(x)\|^2+\|A_r(x)-A_{r-1}A_r(x)\|^2+\cdots+\|A_2\cdots A_r(x)-A_1A_2\cdots A_r(x)\|^2\bigr) \\
        \leq{} & r\bigl(\|x\|^2-\|A_r(x)\|^2+\|A_r(x)\|^2-\|A_{r-1}A_r(x)\|^2+\cdots+\|A_2\cdots A_r(x)\|^2-\|T(x)\|^2\bigr) \\
        ={}    & r\bigl(\|x\|^2-\|T(x)\|^2\bigr).
    \end{align*}
    由
    \[\|x_n-T(x_n)\|^2\leq r\bigl(\|x_n\|^2-\|T(x_n)\|^2\bigr)\leq 2r\|x_n\|(\|x_n\|-\|T(x_n)\|)\]
    知当 $(x_n)$ 为有界序列且 $\|x_n\|-\|T(x_n)\|\to 0$ 时, 有
    \[\lim_{n\to\infty} \|x_n-T(x_n)\|=0.\]

    (ii) 当 $x\in N$ 时, $T(x)=P(x)=x$, 故 $\|T^n(x)-P(x)\|=\|x-x\|=0$;
    当 $x\in\closure{(I-T)(H)}$ 且 $x\ne 0$ 时, 必有 $\|T(x)\|<\|x\|$,
    否则 $\|T(x)\|=\|x\|$, 由 (i) 中结论知 $T(x)=x$, 则
    $x\in\ker(I-T)\cap\closure{(I-T)(H)}=\{0\}$, 矛盾. 因此存在 $0<\lambda<1$,
    使得 $\|T(x)\|\leq\lambda\|x\|$, 从而
    \[\|T^n(x)-P(x)\|=\|T^n(x)\|\leq\lambda^n\|x\|\to 0\quad n\to\infty.\]

    对于一般的 $x\in H$, 将其正交分解并利用三角不等式即得 $\lim\limits_{n\to\infty}\|T^n(x)-P(x)\|=0$.

    (iii) 首先容易验证
    \[\bigcap_{j=1}^r \ker(I-A_j)\subset N.\]
    下证反向包含关系, 任取 $x\in N$, 则 $T(x)=A_1A_2\cdots A_{r}(x)=x$, 故
    \[\|x\|=\|A_1A_2\cdots A_r(x)\|\leq\|A_r(x)\|\leq\|x\|,\]
    由上式知 $\|A_r(x)\|=\|x\|$, 结合 (a) 中结论得 $A_r(x)=x$, 依此下来可推得
    $A_i(x)=x$, $i=r-1,\cdots,1$. 从而
    \[x\in\bigcap_{i=1}^r \ker(I-A_j).\]
    由此便证明了
    \[N=\bigcup_{j=1}^r \ker(I-A_j).\qedhere\]
\end{proof}

\begin{remark}
    从 (a) 可以得到一个结论: 正算子必为自伴算子. 事实上, 我们也可以用极化恒等式证明此结论,
    但这里所说的极化恒等式是广义的极化恒等式, 设 $H$ 为 $\FC$ 上的向量空间,
    映射 $S:H\times H\to\FC$, $(x,y)\to S(x,y)$ 关于第一个变量是线性的,
    关于第二个变量是共轭线性的, 则我们有极化恒等式
    \[4S(x,y)=\sum_{k=0}^3 \ii^k S(x+\ii^k y,x+\ii^k y).\]
    证明很容易, 将上式右侧展开验证即可. 特别地, 取 $S$ 为内积, 则得到经典的极化恒等式
    \[4\innerp{x}{y}=\sum_{k=0}^3 \ii^k\|x+\ii^k y\|^2.\]
    如果假设 $S$ 关于第一个变量是共轭线性的而
    关于第二个变量是线性的, 则也有相应的极化恒等式
    \[4S(x,y)=\sum_{k=0}^3(-\ii)^k S(x+\ii^k y,x+\ii^k y).\]

    在本问题中, 定义 $\mathcal{S}(x,y)=\innerp{A(x)}{y}$ 和 $\mathcal{T}(x,y)=\innerp{x}{A(y)}$,
    则 $\mathcal{S}$ 和 $\mathcal{T}$ 都关于第一个变量为线性且关于第二个变量为共轭线性,
    由极化恒等式得
    \[4\mathcal{S}(x,y)=\sum_{k=0}^3 \ii^k \mathcal{S}(x+\ii^k y,x+\ii^k y),\]
    \[4\mathcal{T}(x,y)=\sum_{k=0}^3 \ii^k \mathcal{T}(x+\ii^k y,x+\ii^k y).\]
    即
    \[4\innerp{A(x)}{y}=\sum_{k=0}^3 \ii^k \innerp{A(x+\ii^k y)}{x+\ii^k y},\]
    \[4\innerp{x}{A(y)}=\sum_{k=0}^3 \ii^k \innerp{x+\ii^k y}{A(x+\ii^k y)}.\]
    结合 $A$ 为正算子即得 $\innerp{A(x)}{y}=\innerp{x}{A(y)}$.
\end{remark}