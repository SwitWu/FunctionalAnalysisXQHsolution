\setcounter{chapter}{7}
\chapter{Hahn-Banach 定理}
\thispagestyle{empty}


\begin{exercise}
    设 $1\leq p\leq\infty$, 考虑 $\FR^2$ 上的 $p$ 范数:
    \[\|(x_1,x_2)\|_p=\bigl(|x_1|^p+|x_2|^p\bigr)^{\frac{1}{p}},\;p<\infty;\quad \|(x_1,x_2)\|_{\infty}=\max\{|x_1|,|x_2|\}.\]
    设 $F=\FR\times\{0\}$, 即由 $e_1=(1,0)$ 生成的向量子空间, 并设 $f:F\to\FR$
    是线性泛函, 满足 $f(e_1)=1$.

    (a) 当 $\FR^2$ 上赋予 $\|\cdot\|_1$ 范数时, 确定 $f$ 从 $F$ 到 $\FR^2$ 的所有保范延拓.

    (b) 当 $\FR^2$ 上赋予 $\|\cdot\|_p$ 范数时, 考虑同样的问题.
\end{exercise}

\begin{proof}
    (a)首先 $\|f\|=\sup\limits_{x=te_1,t\neq 0}\frac{|f(x)|}{\|x\|_1}=\sup\limits_{x=te_1,t\neq0}\frac{|t|}{|t|}=1$,
    记 $e_2=(0,1)$, 则对任意 $x=x_1e_1+x_2e_2\in\FR^2$, 有
    \[\tilde{f}(x)=x_1f(e_1)+x_2\tilde{f}(e_2)=x_1+x_2\tilde{f}(e_2).\]
    则
    \[\|\tilde{f}\|=\sup_{x\in\FR^2,x\neq 0}\frac{|\tilde{f}(x)|}{\|x\|_1}=\sup_{x\in\FR^2,x\neq 0}\frac{|x_1+x_2\tilde{f}(e_2)|}{|x_1|+|x_2|}.\]
    要使得 $\|\tilde{f}\|=\|f\|=1$, 即
    \[\sup_{x\in\FR^2,x\neq 0}\frac{|x_1+x_2\tilde{f}(e_2)|}{|x_1|+|x_2|}=1.\]
    容易验证当且仅当 $|\tilde{f}(e_2)|\leq 1$ 时, 上式得以成立, 
    因此当 $\FR^2$ 上赋予 $\|\cdot\|_1$ 范数时, $f$ 从 $F$ 到 $\FR^2$ 的所有保范延拓为:
    \[\left\{\tilde{f}\colon\tilde{f}(x)=x_1+x_2\tilde{f}(e_2),|\tilde{f}(e_2)|\leq1\right\}.\]

    (b) 当 $1<p<\infty$ 时, 此时目标是:
    \[\|\tilde{f}\|=\sup_{x\in\FR^2,x\neq0}\frac{|x_1+x_2\tilde{f}(e_2)|}{\left(|x_1|^p+|x_2|^p\right)^{\frac{1}{p}}}=1.\]
    为叙述简便, 记 $t=|\tilde{f}(e_2)|$, 则$x_1\cdot x_2\tilde{f}(e_2)\geq 0$ 时
    \[\begin{split}\frac{|x_1+x_2\tilde{f}(e_2)|}{\left(|x_1|^p+|x_2|^p\right)^{\frac{1}{p}}}\leq 1
    &\Leftrightarrow \frac{|x_1|+|x_2|\cdot t}{\left(|x_1|^p+|x_2|^p\right)^{\frac{1}{p}}}\leq1\\
    &\Leftrightarrow (|x_1|+|x_2|\cdot t)^p\leq|x_1|^p+|x_2|^p(\mbox{不妨}|x_2|\neq0)\\
    &\Leftrightarrow\left(\frac{|x_1|}{|x_2|}+t\right)^p\leq\left(\frac{|x_1|}{|x_2|}\right)^p+1\\
    &\Leftrightarrow (\alpha+t)^p\leq\alpha^p+1(0\leq\alpha\leq\infty)\\
    &\Leftrightarrow t=0.
    \end{split}\]
    因此保范延拓为 $\left\{\tilde{f}:\tilde{f}(x)=x_1\right\}$.
    
    当 $p=\infty$ 时, 此时目标是:
    \[\sup_{x\in\FR^2,x\neq 0}\frac{|x_1+x_2\tilde{f}(e_2)|}{\max\{|x_1|,|x_2|\}}=1\Leftrightarrow\tilde{f}(e_2)=0.\]
    因此保范延拓为 $\left\{\tilde{f}:\tilde{f}(x)=x_1\right\}$.

    综上知当$\FR^2$上赋予$\|\cdot\|_p(1<p\leq\infty)$范数时,所有的保范延拓为$\left\{\tilde{f}:\tilde{f}(x)=x_1\right\}$.
\end{proof}



% \begin{exercise}
%     通过 $\FR^2$ 上的反例说明在几何形式的 Hahn-Banach 定理中, 一个凸子集是开集的条件是必要的.
% \end{exercise}

% \begin{proof}
% 将四个点$(\pm1,\pm1)$围成的正方形用$x$轴分成两部分,
% 上半部分去掉线段 $\{(x,y)|-1\leq x\leq0,y=0\}$, 得到区域 $A$,
% 下半部分去掉线段 $\{(x,y)|0\leq x\leq 1,y=0\}$, 得到区域$B$, 易知$A,B$无法被隔离.
% \end{proof}



% \begin{exercise}
%     设 $E$ 是数域 $\FK$ 上的赋范空间, $A\subset E$, 并设 $f:A\to\FK$
%     以及常数 $\lambda\geq 0$. 证明: 存在 $\widehat{f}\in E^*$, 使得
%     $\widehat{f}|_A=f\quad\text{且}\quad\|\widehat{f}\|\leq\lambda$
%     的充分必要条件是
%     \[\left|\sum_{k=1}^n \alpha_kf(a_k)\right|\leq\lambda\left\|\sum_{k=1}^n\alpha_ka_k\right\|,\forall n\in\FN^*,
%     \forall(a_1,\cdots,a_n)\in A^n,\forall (\alpha_1,\cdots,\alpha_n)\in\FK^n.\]
% \end{exercise}

% \begin{proof}
%     必要性显然,下面证明充分性. 由
%     \[\left|\sum_{k=1}^n\alpha_kf(a_k)\right|=\left|f\left(\sum_{k=1}^n\alpha_ka_k\right)\right|\leq\lambda\left\|\sum_{k=1}^n\alpha_ka_k\right\|.\]
%     知 $f$ 是连续线性泛函, 由 Hahn-Banach 定理知存在 $\hat{f}\in E^{*}$, 使得 $\hat{f}|_A=f,\|\hat{f}\|\leq\lambda$.
% \end{proof}



\begin{exercise}[4]
    设 $E$ 是 Hausdorff 拓扑向量空间, $A$ 是 $E$ 中包含原点的开凸集以及 $x_0\in E\setminus A$.

    (a) 证明: 存在 $f\in E^*$, 使得
    \[\Re f(x_0)=1,\;\text{且在\ }A\text{\ 上\ }\Re f<1.\]

    (b) 假设 $A$ 还是平衡的. 证明: 可以选择 $f\in E^*$, 使其满足
    \[f(x_0)=1,\;\text{且在\ }A\text{\ 上\ }|f|<1.\]
\end{exercise}

\begin{proof}
    (a) 由于 $\{x_0\}$ 为凸集, $A$ 为开凸集且二者不相交, 故由 Hahn-Banach 定理知存在
    $\widetilde{f}\in E^*$ 和 $\alpha>0$, 使得
    \[\Re\widetilde{f}(a)<\alpha\leq\Re\widetilde{f}(x_0),\quad\forall a\in A.\]
    令 $f=\frac{\widetilde{f}}{\Re\widetilde{f}(x_0)}$, 则 $\Re f(x_0)=1$ 且对任意 $a\in A$, 有
    \[\Re f(a)=\frac{\Re\widetilde{f}(a)}{\Re\widetilde{f}(x_0)}<\frac{\alpha}{\alpha}=1.\]

    (b) $\widetilde{f}$ 仍为 (a) 中所得有界线性泛函, 令 $f(x)=\frac{\widetilde{f}(x)}{\widetilde{f}(x_0)}$, 则
    $f(x_0)=1$, 且对任意 $a\in A$, 由 $A$ 平衡可得
    \begin{align*}
        |f(a)|
        & =\left|\frac{\widetilde{f}(a)}{\widetilde{f}(x_0)}\right|=\frac{|\widetilde{f}(a)|}{|\widetilde{f}(x_0)|}=\frac{\widetilde{f}(a)\sgn\widetilde{f}(a)}{|\widetilde{f}(x_0|} \\
        & =\frac{\widetilde{f}(a\sgn\widetilde{f}(a))}{|\widetilde{f}(x_0)|}\leq\frac{\Re\widetilde{f}(a\sgn\widetilde{f}(a))}{\alpha}<\frac{\alpha}{\alpha}=1.\qedhere
    \end{align*}
\end{proof}



\begin{exercise}
    设 $E$ 是 Hausdorff 局部凸空间, $A$ 是 $E$ 中包含原点的闭凸集以及 $x_0\in E\setminus A$.

    (a) 证明: 存在 $f\in E^*$, 使得
    \[\Re f(x_0)>1\quad\text{且}\quad\sup_{x\in A}\Re f(x)\leq 1.\]

    (b) 假设 $A$ 还是平衡的. 证明: 可以选择 $f$, 使其满足
    \[f(x_0)=1\quad\text{且}\quad\sup_{x\in A}|f(x)|\leq 1.\]
\end{exercise}

\begin{proof}
    (a) 由于 $A$是包含原点的闭凸集, $\{x_0\}$是紧集, 且二者不相交,
    故由 Hahn-Banach 定理知存在 $\widetilde{f}\in E^*$和常数 $\alpha>0$, 使得
    \[\sup_{x\in A}\Re\widetilde{f}(x)<\alpha<\Re\widetilde{f}(x_0).\]
    令 $f=\frac{\widetilde{f}}{\alpha}\in E^*$, 则
    \[\Re f(x_0)=\frac{\Re\widetilde{f}(x_0)}{\alpha}>1\quad\text{且}\quad\sup_{x\in A}\Re f(x)=\sup_{x\in A}\frac{\Re\widetilde{f}(x)}{\alpha}<1.\]
    (b) $\widetilde{f}$ 仍为 (a) 中所得有界线性泛函, 令 $f=\frac{\widetilde{f}}{\widetilde{f}(x_0)}$, 则
    $f(x_0)=1$, 且由 $A$ 平衡可得
    \begin{align*}
        \sup_{x\in A}|f(x)|
        & =\sup_{x\in A}\left|\frac{\widetilde{f}(x)}{\widetilde{f}(x_0)}\right|=\sup_{x\in A}\frac{|\widetilde{f}(x)|}{|\widetilde{f}(x_0)|} \\
        & <\frac{1}{\alpha}\sup_{x\in A}\Re\widetilde{f}(x\sgn\widetilde{f}(x))<\frac{\alpha}{\alpha}=1.\qedhere
    \end{align*}
\end{proof}
% 6.\textit{Proof}:(a)$\forall x\in\overline{\conv(A)},\exists(x_n)_{n\geq 1}\subset\conv(A),s.t.x_n\to x(n\to\infty)$,由凸包的定义知:\[x_n=\sum_{k=1}^{N_n}t_ka_k,a_k\in A,t_k\geq0,\sum_{k=1}^{N_n}t_k=1\]
% 取$f\in E^*$满足对任意$a\in A$,有$f(a)\leq1$,则
% \[f(x)=f\left(\lim_{n\to\infty}\sum_{k=1}^{N_n}t_ka_k\right)=\lim_{n\to\infty}\sum_{k=1}^{N_n}t_kf(a_k)\leq1\]
% 故$x\in\widehat{A}\Rightarrow\overline{\conv(A)}\subset\widehat{A}$\\
% (b)\\$(\Rightarrow)$当$\overline{\conv(A)}=\widehat{A}$时,因为显然$0\in\widehat{A}$,所以$0\in\overline{\conv(A)}$\\
% $(\Leftarrow)$因为$0\in\overline{\conv(A)}$,所以$\overline{\conv(A)}$是包含原点的闭凸集,任取$x_0\in E\setminus\overline{\conv(A)}$,由上一题结论知存在$f\in E^*$,使得
% \[f(x_0)>1\mbox{且}\sup_{x\in\overline{\conv(A)}}f(x)\leq 1\]
% 故$x_0\not\in\widehat{A}$,即$x_0\in\left(\widehat{A}\right)^c$,所以
% \[E\setminus\overline{\conv(A)}\subset\left(\widehat{A}\right)^c\Rightarrow\widehat{A}\subset\overline{\conv(A)}\]
% 结合$(a)$中结论得$\overline{\conv(A)}=\widehat{A}$\\\\
% 8.\textit{Proof}:(注:本题需要加上连续性的条件)\\
% (a)\begin{enumerate}[(i)]
% \item $G$是凸集:$\forall y^{(1)},y^{(2)}\in G,\exists x_1,x_2\in C$使得
% \[f_i(x_1)\leq y^{(1)}_i,f_i(x_2)\leq y^{(2)}_i,1\leq i\leq m\]
% 对于任意$0<\lambda<1$,由$C$是凸集知$\lambda x_1+(1-\lambda)x_2\in C$,又由$(f_i)$是凸函数得
% \[f_i(\lambda x_1+(1-\lambda)x_2)\leq\lambda f_i(x_1)+(1-\lambda)f_i(x_2)\leq\lambda y^{(1)}_i+(1-\lambda)y^{(2)}_i,1\leq i\leq m\]
% 故\[\lambda y^{(1)}+(1-\lambda)y^{(2)}\in G\]
% 因此$G$是凸集
% \item $G$是闭集:任取$G$中收敛序列$(y^{(n)})\to y$,需要证明$y\in G$,由$G$的定义知存在$(x_n)_{n\geq1}\subset C$,使得
% \[f_i(x_n)\leq y^{(n)}_i,1\leq i\leq m\]
% 因为$C$紧,故其有收敛子列$(x_{n_k})_{k\geq1}\to x\in C$,同时$f_i(x_{n_k})\leq y^{(n_k)}_i$,令$k\to\infty$,得
% \[\lim_{k\to\infty}f_i(x_{n_k})=f_i(x)\leq\lim_{k\to\infty}y^{(n_k)}_i=y_i\]
% 故$y\in G$,这就证明了$G$是闭集
% \end{enumerate}
% $S=\varnothing$意味着$\forall x\in C,\exists f_i,s.t.f_i(x)>0$,而
% \[G^c=\{y=(y_1,\cdots,y_m)\in\FR^m:\forall x\in C,\exists f_i,s.t.f_i(x)>y_i\}\]
% 故$S=\varnothing$可以表示为$0\in G^c$\\
% (b)显然\\\\


\begin{exercise}
  设 $E$ 是数域 $\FK$ 上的拓扑向量空间. 称 $E$ 的向量子空间 $H$ 是超平面,
  若有某个 $x_0\in E\setminus H$, 使得 $E = H + \FK x_0$.
  \begin{enumerate}[(a)]
    \item 证明: 若 $H$ 是超平面, 则对任意的 $x_0\in E\setminus H$, $E=H+\FK x_0$ 成立.
    \item 证明: 一个超平面或者是 $E$ 的稠密集, 或者是闭集.
    \item 证明: $H$ 是超平面当且仅当存在 $E$ 上的一个非零线性泛函 $f$, 使得 $H=\ker f$.
      因而 $H$ 是闭的等价于 $f$ 是连续的.
  \end{enumerate}
\end{exercise}

\begin{proof}
    (a) $\forall x_1\in E\setminus H,x_1=y_1+\lambda x_0,y_1\in H,\lambda\neq0$, 故对 $\forall x\in E$,
    \[x=y+kx_0=y+k\frac{x_1-y_1}{\lambda}=\left(y-\frac{k}{\lambda}y_1\right)+\frac{k}{\lambda}x_1\in H+\mathbb{K}x_1.\]
    因此
    \[E=H+\mathbb{K}x_1,\quad\forall x_1\in E\setminus H.\]

    (b) 由定理 7.1.6 知 $\closure{H}$ 是向量子空间, 又 $\dim (E\setminus H)=1$, 故只可能有两种情况:
    当 $\closure{H}=H$ 时, $H$ 为闭集; 当 $\closure{H}=E$ 时, $H$ 在 $E$ 中稠密.

    (c) \sufficient
    假设存在 $E$ 上的非零线性泛函 $f$, 使得 $H=\ker f$, 首先因 $f\not\equiv 0$,
    故存在 $x_0\in E\setminus\ker f$, 使得 $f(x_0)=1$, 则$\forall x\in E$, 有
    \[x=x-f(x)x_0+f(x)x_0.\]
    因为 $f(x-f(x)x_0)=f(x)-f(x)f(x_0)=0$, 所以 $x-f(x)x_0\in\ker f$,
    并且 $f(x)x_0\in\mathbb{K}x_0$, 又容易验证表示 $x=h+kx_0$, $h\in\ker f,k\in\mathbb{K}$ 是唯一的, 因此
    \[E=H+\mathbb{K}x_0.\]
    也即 $H$ 是超平面.

    \necessary
    因为 $H$ 是超平面, 所以存在 $x_0\in E\setminus H$, 使得 $E=H+\mathbb{K}x_0$, 
    故对于 $\forall x\in E$, $x=h+kx_0$, 定义泛函
    \[f:E\to\mathbb{K},x=h+kx_0\mapsto k.\]
    容易验证 $f$ 是合理定义的线性泛函且 $\ker f=H$.
    当 $f$ 连续时, 因为 $\{0\}\subset\mathbb{K}$ 是闭集, 所以 $H=\ker f=f^{-1}(0)$ 是闭集.
\end{proof}



\begin{exercise}
  设 $(X,\|\cdot\|_X)$ 是实赋范空间, $\closure{B}_X$ 表示该空间中的闭单位球.
  假设 $K\geq 1$, $C$ 是 $X$ 中闭凸对称子集 ($C$ 对称是指 $x\in C\Rightarrow -x\in C$), 且满足
  \[B_X\subset C\subset K\closure{B}_X.\]
  定义
  \[p(x) = \inf\Bigl\{\lambda>0\colon\frac{x}{\lambda}\in C\Bigr\},\;\forall x\in X.\]
  \begin{enumerate}[(a)]
    \item 证明: $p$ 是 $X$ 上和 $\|\cdot\|_X$ 等价的范数. 更确切地说, 证明:
      \[\frac{1}{K}\|x\|\leq p(x)\leq\|x\|,\;\forall x\in X.\]
    \item 设 $x\in X$. 证明:
      \begin{enumerate}[(i)]
          \item $x\in X\setminus C\Longleftrightarrow p(x)>1$.
          \item $x\in\mathring{C}\Longleftrightarrow p(x)<1$.
          \item $x\in\partial C\Longleftrightarrow p(x)=1$.
      \end{enumerate}
    \item 任取 $x\in\partial C$. 证明: 存在 $X$ 上的连续线性泛函 $f$, 使得 $f(x)=1$ 且在集合 $C$ 上, $|f|\leq 1$.
    \end{enumerate}
\end{exercise}

\begin{proof}
    (a) 任取 $\varepsilon>0$, 对于任意 $x,y\in X$, 有 $\frac{x}{p(x)+\varepsilon}\in C$, $\frac{y}{p(y)+\varepsilon}\in C$, 于是
    \[\frac{x + y}{p(x) + p(y) + 2\varepsilon} = \frac{p(x) + \varepsilon}{p(x) + p(y) + 2\varepsilon}\frac{x}{p(x) + \varepsilon} + \frac{p(y) + \varepsilon}{p(x) + p(y) + 2\varepsilon}\frac{y}{p(y) + \varepsilon}\in C,\]
    因此 $p(x+y)\leq p(x)+p(y)+2\varepsilon$, 由 $\varepsilon$ 的任意性得 $p(x+y)\leq p(x)+p(y)$.

    任取 $\lambda\in\FR$ 和 $x\in X$, 当 $\lambda=0$ 时, $p(\lambda x)=|\lambda|p(x)$
    显然成立, 当 $\lambda\neq 0$ 时, 有
    \[\frac{\lambda x}{p(\lambda x)+\varepsilon}\in C.\]
    由 $C$ 对称得
    \[\frac{x}{\frac{1}{|\lambda|}(p(\lambda x)+\varepsilon)}\in C.\]
    故 $p(x)\leq\frac{1}{|\lambda|}(p(\lambda x)+\varepsilon)\Rightarrow|\lambda|p(x)\leq p(\lambda x)$.

    又因 $\frac{x}{p(x)+\varepsilon}\in C$, 由 $C$ 对称可得 $\frac{\lambda x}{|\lambda|(p(x)+\varepsilon)}\in C$,
    故 $p(\lambda x)\leq |\lambda|(p(x)+\varepsilon)$, 从而$p(\lambda x)\leq|\lambda|p(x)$.
    因此 $p(\lambda x)=|\lambda|p(x)$.

    任取 $x\in X$, 有 $\frac{x}{\|x\|}\in\closure{B}_X\subset C$, 故 $p(x)\leq\|x\|$.
    又 $\frac{x}{p(x)+\varepsilon}\in C\subset K\closure{B}_X$, 故
    $\|\frac{x}{p(x)+\varepsilon}\|\leq K\Rightarrow \frac{1}{K}\|x\|\leq p(x)$.

    综上可知 $p$ 是在 $X$ 上和 $\|\cdot\|$ 等价的范数且满足
    \[\frac{1}{K}\|x\|\leq p(x)\leq\|x\|,\;\forall x\in X.\]

    (b) (i) \sufficient 因 $p(x)>1$, 故 $x=\frac{x}{1}\notin C$, 即 $x\in X\setminus C$.
    \necessary 因 $x\in X\setminus C$ 且 $X\setminus C$ 为开集,
    故存在 $\mu\in(0,1)$, 使得 $(1-\mu)x\in X\setminus C$, 即 $\frac{x}{1/(1-\mu)}\notin C$,
    因此 $p(x)\geq\frac{1}{1-\mu}>1$.

    (ii) \necessary 因 $x\in\mathring{C}$ 且 $\mathring{C}$ 为开集,
    故存在 $\mu>0$, 使得 $(1+\mu)x\in\mathring{C}\subset C$, 故 $p(x)\leq\frac{1}{1+\mu}<1$.
    \sufficient 因 $p(x)<1$, 故存在 $\lambda$ 使得 $p(x)<\lambda<1$,
    于是 $\frac{x}{\lambda}\in C\Rightarrow x\in\lambda C\subset\mathring{C}$.

    (iii) 由 (i)(ii) 即得 $x\in\partial C\Longleftrightarrow p(x)=1$.

    (c) 由教材推论 8.1.10 知存在 $X$ 上的连续线性泛函 $f$, 使得 $f(x)=1$ 且在集合 $C$ 上, $|f(x)|\leq p(x)\leq 1$.
\end{proof}



\begin{exercise}
    考虑空间 $\ell_{\infty}$ 和它的子空间 $F$:
    \[F=\bigl\{x\in\ell_{\infty}\colon \lim_{n\to\infty}m_n(x)\text{\ 存在}\bigr\},\quad\text{其中\ }m_n(x)=\frac{1}{n}\sum_{k=1}^n x_k.\]

    (a) 定义 $f:F\to\FR$ 为 $f(x)=\lim_{n\to\infty}m_n(x)$. 证明: $f\in F^*$.

    (b) 证明: 存在 $\ell_{\infty}$ 上的连续线性泛函 $m$ 满足下面的性质:
    \begin{enumerate}[(i)]
        \item $\liminf_{n\to\infty}x_n\leq m(x)\leq\limsup_{n\to\infty}x_n$, $\forall x\in\ell_{\infty}$.
        \item $m\circ\tau=m$, 这里 $\tau:\ell_{\infty}\to\ell_{\infty}$ 是右移算子, 即 $\tau(x)_n=x_{n+1}$.
    \end{enumerate}
\end{exercise}

\begin{proof}
    (a) 线性性 $f(\lambda x+y)=\lambda f(x)+f(y)$ 直接验证. 下证 $f$ 有界, 对任意 $n\geq 1$,
    \[|m_n(x)|\leq\frac{1}{n}\sum_{k=1}^n |x_k|\leq\|x\|_{\infty}.\]
    故对任意 $x\in F$, 有 $|f(x)|=\lim_{n\to\infty}|m_n(x)|\leq\|x\|_{\infty}$.

    (b)
\end{proof}

\begin{proof}
    (a) We can obtain it from definition easily that 
    \[ |m_n(x)| \leq \frac1n\sum_{k=1}^n |x_k|\leq \|x\|_\infty. \]
    Then $|f(x)| = |\lim m_n(x)|\leq \|x_\infty\|.$
    (b) Define $p: l^\infty\to \mathbb R, (x_n)\mapsto \overline\lim \frac1n|\sum_{k = 1}^nx_k|$, and we can obtain $p(x) \geq 0, p(x + y)\leq p(x) + p(y)$ and $|\lambda|p(x) = p(\lambda x)$. Since $p(x) \leq \|x\|_\infty$, $p$ is a continous seminorm. Therefore, $\exists m\in (l^\infty)^*$ s.t. $m = f$ on $F$ and $|m|\leq p$ on $l^\infty$. 
    
    Let $x^*$ denote $\overline\lim x_n$, and $x_*$ denote $\underline \lim x_n$. Set $e = (1, \cdots)\in F$, and then $\lrangle{m}{e} = 1$. We obtain that 
    \[ \langle m, x\rangle  = \langle m, x-x_*e\rangle + x_* \leq p(x-x_*e) + x_*.\]
    We claim that $p(x-x_*)\leq \overline\lim |x_n - x_*|$. Indeed, there exist $N>0$ such that $\forall n > N$, $|x_n - x_*| \leq \overline\lim |x_n - x_*| + \varepsilon$ for any $\varepsilon>0$. Thus 
    \[ \frac1n|\sum_{k=1}^n x_k - x_*|\leq \frac1n \sum_{k=1}^N |x_k - x_*| + \frac{n-N}{n}\overline\lim|x_n - x_*| + \varepsilon. \]
    Since $\exists n_k$ such that $\overline\lim |x_n - x_*|=\lim |x_{n_k} - x_*| = |\lim x_{n_k} - x_*| \leq x^*-x_*$,  we have 
    \[ \lrangle{m}{x}\leq p(x - x_*e) + x_*\leq x^* - x_* + x_* = x^*. \]
    
    Conversely, we have $\langle m, -x\rangle \leq -x_*$, and it follows that $x_*\leq \langle m, x\rangle \leq x^*$.

    Since
    \[ p(\tau x - x) = \overline\lim \frac1n |\sum_{k = 1}^nx_k - x_{k+1}| = \overline\lim\frac1n(|x_{n+1}-x_1|)\leq \overline\lim \frac{2\|x\|_\infty}{n} = 0, \]
    and then $|\langle m, \tau x - x\rangle|\leq p(\tau x - x) = 0$, $\langle m, \tau x\rangle = \langle m, x\rangle$. 
\end{proof}